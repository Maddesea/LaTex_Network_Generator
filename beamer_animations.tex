% beamer_animations.tex - Animation Support for Presentations
% Requires: \usepackage{beamer} in main document

% ============================================================================
% ANIMATION CONTROL
% ============================================================================

% Enable animations (must be called in Beamer document)
\newcommand{\enableAnimations}{
    \def\animationsEnabled{true}
}

% Disable animations (for static PDFs)
\newcommand{\disableAnimations}{
    \def\animationsEnabled{false}
}

% Check if animations are enabled
\newcommand{\ifAnimationsEnabled}{
    \ifcsname animationsEnabled\endcsname
        \ifthenelse{\equal{\animationsEnabled}{true}}
    \else
        \expandafter\@secondoftwo
    \fi
}

% ============================================================================
% PROGRESSIVE REVEAL
% ============================================================================

% Reveal node on specific slide
% Usage: \revealNode<2->{node_id}{node_definition}
\newcommand{\revealNode}[2]{
    \only<#1>{#2}
}

% Fade in node
% Usage: \fadeInNode<2->{node_id}{node_definition}
\newcommand{\fadeInNode}[2]{
    \only<#1>{
        \begin{scope}[opacity=0]
            #2
        \end{scope}
    }
    \only<+(1)->{#2}
}

% Highlight node temporarily
% Usage: \highlightNode<2>{node_id}
\newcommand{\highlightNode}[2]{
    \only<#1>{
        \begin{scope}[on background layer]
            \node[draw=yellow, line width=3pt, rounded corners=5pt, fit=(#2)] {};
        \end{scope}
    }
}

% ============================================================================
% ATTACK SCENARIO ANIMATION
% ============================================================================

% Animate an attack progression
% Usage: \animateAttack{attacker}{path}{target}
\newcommand{\animateAttack}[3]{
    % Step 1: Show attacker
    \only<1->{
        \node[attacker] (#1) {...};
    }

    % Step 2: Show connection attempt
    \only<2->{
        \draw[attack, very thick] (#1) -- (#3);
    }

    % Step 3: Show compromised target
    \only<3->{
        \node[fill=red!20] at (#3) {...};
    }
}

% Progressive attack kill chain
% Usage: \animateKillChain
\newcommand{\animateKillChain}{
    \only<1>{
        \node[alert] {1. Reconnaissance};
    }
    \only<2>{
        \node[alert] {2. Weaponization};
    }
    \only<3>{
        \node[alert] {3. Delivery};
    }
    \only<4>{
        \node[alert] {4. Exploitation};
    }
    \only<5>{
        \node[alert] {5. Installation};
    }
    \only<6>{
        \node[alert] {6. Command \& Control};
    }
    \only<7>{
        \node[alert] {7. Actions on Objectives};
    }
}

% ============================================================================
% DATA FLOW ANIMATION
% ============================================================================

% Animate data flowing through connection
% Usage: \animateDataFlow<2->{source}{target}{label}
\newcommand{\animateDataFlow}[3]{
    \only<#1>{
        \draw[->, thick, decorate, decoration={
            markings,
            mark=between positions 0 and 1 step 0.1 with {
                \fill[blue] circle (2pt);
            }
        }] (#2) -- (#3) node[midway, above] {#4};
    }
}

% Pulsing connection (shows active traffic)
% Usage: \pulsingConnection<2->{source}{target}
\newcommand{\pulsingConnection}[2]{
    \only<#1>{
        \draw[->, ultra thick, green, opacity=0.7] (#2) -- (#3);
    }
    \only<+(1)>{
        \draw[->, thick, green, opacity=0.5] (#2) -- (#3);
    }
    \only<+(1)>{
        \draw[->, green, opacity=0.3] (#2) -- (#3);
    }
}

% ============================================================================
% NETWORK BUILDUP ANIMATION
% ============================================================================

% Build network layer by layer
% Usage: \buildNetwork{tiers}
\newcommand{\buildNetwork}[1]{
    % Tier 1: Internet (slide 1)
    \only<1->{
        % Internet/external nodes
    }

    % Tier 2: Perimeter (slide 2)
    \only<2->{
        % Firewall nodes
    }

    % Tier 3: Application (slide 3)
    \only<3->{
        % Application servers
    }

    % Tier 4: Data (slide 4)
    \only<4->{
        % Database servers
    }

    % Connections (slide 5)
    \only<5->{
        % All connections
    }
}

% Reveal nodes in sequence
% Usage: \sequentialReveal{node_list}
\newcommand{\sequentialReveal}[1]{
    % Would iterate through nodes with \foreach
    % Revealing each on consecutive slides
}

% ============================================================================
% THREAT INDICATOR ANIMATIONS
% ============================================================================

% Pulsing threat indicator
% Usage: \pulsingThreat<2->{x}{y}{severity}
\newcommand{\pulsingThreat}[3]{
    \only<#1>{
        \begin{scope}
            \foreach \i in {1,...,3} {
                \pgfmathsetmacro{\opacity}{1 - 0.3*\i}
                \node[circle, draw=red, line width=2pt, minimum size=\i cm, opacity=\opacity] at (#2, #3) {};
            }
        \end{scope}
    }
}

% Flashing vulnerability badge
% Usage: \flashingVulnerability<2->{node_id}{cve}{score}
\newcommand{\flashingVulnerability}[3]{
    \only<#1>{
        \node[fill=red, text=white] at (#2) {#3: #4};
    }
    \only<+(1)>{
        \node[fill=yellow, text=black] at (#2) {#3: #4};
    }
}

% ============================================================================
% COMPARISON ANIMATIONS
% ============================================================================

% Before/After comparison
% Usage: \beforeAfter{before_content}{after_content}
\newcommand{\beforeAfter}[2]{
    \only<1>{
        \node[anchor=north] at (0, 5) {\textbf{Before}};
        #1
    }
    \only<2->{
        \node[anchor=north] at (0, 5) {\textbf{After}};
        #2
    }
}

% Side-by-side comparison
% Usage: \sideBySide{left_content}{right_content}
\newcommand{\sideBySide}[2]{
    \only<1->{
        \begin{scope}[shift={(-6, 0)}]
            \node[anchor=north] at (0, 5) {\textbf{Option A}};
            #1
        \end{scope}
    }
    \only<2->{
        \begin{scope}[shift={(6, 0)}]
            \node[anchor=north] at (0, 5) {\textbf{Option B}};
            #2
        \end{scope}
    }
}

% ============================================================================
% ZOOM EFFECTS
% ============================================================================

% Zoom into subnet
% Usage: \zoomToSubnet<2>{subnet_id}{scale}
\newcommand{\zoomToSubnet}[2]{
    \only<1>{
        % Full network view
    }
    \only<#1->{
        \begin{scope}[scale=#2, transform shape]
            % Zoomed subnet view
        \end{scope}
    }
}

% Focus on specific area
% Usage: \focusArea<2->{x}{y}{width}{height}
\newcommand{\focusArea}[4]{
    \only<#1>{
        \begin{scope}[on background layer]
            % Dim everything except focus area
            \fill[white, opacity=0.7] (-20, -20) rectangle (20, 20);
            \fill[white, opacity=0] (#2 - #4/2, #3 - #5/2) rectangle (#2 + #4/2, #3 + #5/2);
        \end{scope}
    }
}

% ============================================================================
% METRIC COUNTERS
% ============================================================================

% Animated counter (incrementing number)
% Usage: \animateCounter<2->{start}{end}{x}{y}{label}
\newcommand{\animateCounter}[5]{
    \foreach \i in {#1,...,#2} {
        \only<+>{
            \node at (#3, #4) {#5: \i};
        }
    }
}

% Progress bar animation
% Usage: \animateProgress<2->{percentage}{x}{y}{width}
\newcommand{\animateProgress}[4]{
    \only<#1>{
        \draw[fill=gray!30] (#2 - #4/2, #3) rectangle (#2 + #4/2, #3 + 0.5);
        \draw[fill=green] (#2 - #4/2, #3) rectangle (#2 - #4/2 + #4 * #1 / 100, #3 + 0.5);
        \node at (#2, #3 + 0.7) {#1\%};
    }
}

% ============================================================================
% SCENARIO TEMPLATES
% ============================================================================

% Complete attack scenario presentation
% Usage: \presentAttackScenario
\newcommand{\presentAttackScenario}{
    % Slide 1: Normal network
    \only<1>{
        % Clean network diagram
    }

    % Slide 2: Reconnaissance phase
    \only<2>{
        % Show scanning activity
        \node[alert] {Attacker scanning network...};
    }

    % Slide 3: Initial compromise
    \only<3>{
        % Show compromised node
        \node[fill=red!30] {...};
    }

    % Slide 4: Lateral movement
    \only<4>{
        % Show spreading
        \draw[attack, thick] (...) -- (...);
    }

    % Slide 5: Data exfiltration
    \only<5>{
        % Show data leaving
        \draw[red, ultra thick, ->] (...) -- (...);
    }

    % Slide 6: Mitigation
    \only<6>{
        % Show defensive measures
        \node[fill=green!30] {Firewall rules updated};
    }
}

% Network evolution over time
% Usage: \showNetworkEvolution
\newcommand{\showNetworkEvolution}{
    \only<1>{
        \node {Year 2020: Basic Setup};
        % Simple network
    }
    \only<2>{
        \node {Year 2021: Expansion};
        % More nodes
    }
    \only<3>{
        \node {Year 2022: Cloud Migration};
        % Add cloud components
    }
    \only<4>{
        \node {Year 2023: Security Hardening};
        % Add security zones
    }
}

% ============================================================================
% UTILITY FUNCTIONS
% ============================================================================

% Pause at specific point in animation
% Usage: \animationPause<2>
\newcommand{\animationPause}[1]{
    \only<#1>{
        \node[font=\small, gray] {[Paused - Press any key]};
    }
}

% Add timestamp to slides
% Usage: \slideTimestamp<2->{Time: T+5s}
\newcommand{\slideTimestamp}[2]{
    \only<#1>{
        \node[anchor=south east, font=\tiny, gray] at (10, -10) {#2};
    }
}

% ============================================================================
% BEAMER-SPECIFIC HELPERS
% ============================================================================

% Create overlay-aware node
% Usage: \overlayNode<2->{id}{x}{y}{content}
\newcommand{\overlayNode}[4]{
    \node<#1> (#2) at (#3, #4) {#5};
}

% Create overlay-aware connection
% Usage: \overlayConnection<2->{source}{target}{label}
\newcommand{\overlayConnection}[3]{
    \draw<#1> (#2) -- (#3) node[midway, above] {#4};
}

% Incremental build with auto-increment
% Usage: \incrementalBuild{items...}
\newcommand{\incrementalBuild}[1]{
    % Each item appears on next slide
    \foreach \item in {#1} {
        \only<+>{\item}
    }
}

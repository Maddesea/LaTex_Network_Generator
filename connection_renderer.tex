% connection_renderer.tex - Network connection rendering and visualization
% This module handles all connection types, paths, and flow visualization

% ============================================================================
% CONNECTION DATA STRUCTURES
% ============================================================================

% Connection counter
\newcounter{conncount}

% TODO: Connection management
% - Hash map for connection lookup
% - Bidirectional connection deduplication
% - Connection grouping by protocol/type
% - Connection state tracking (active/inactive)

% ============================================================================
% BASIC CONNECTION COMMANDS
% ============================================================================

% Draw a normal connection
% Usage: \drawConnection{from}{to}{label}
\newcommand{\drawConnection}[3]{
    \draw[normal conn] (#1) -- node[above, font=\tiny] {#3} (#2);
}

% Draw an encrypted connection
% Usage: \drawEncryptedConnection{from}{to}{protocol}
\newcommand{\drawEncryptedConnection}[3]{
    \draw[encrypted conn] (#1) -- node[above, font=\tiny\ttfamily] {#3} (#2);
}

% Draw a suspicious connection
% Usage: \drawSuspiciousConnection{from}{to}{reason}
\newcommand{\drawSuspiciousConnection}[3]{
    \draw[suspicious conn] (#1) -- node[above, font=\tiny, fill=yellow!30] {#3} (#2);
}

% Draw an attack connection
% Usage: \drawAttackConnection{from}{to}{attack_type}
\newcommand{\drawAttackConnection}[3]{
    \draw[attack conn] (#1) -- node[midway, above, threat label] {#3} (#2);
}

% Bidirectional connection
% Usage: \drawBidirectional{from}{to}{label}
\newcommand{\drawBidirectional}[3]{
    \draw[normal conn, bidirectional] (#1) -- node[above, font=\tiny] {#3} (#2);
}

% ============================================================================
% SPECIALIZED CONNECTION TYPES
% ============================================================================

% VPN tunnel connection with dashed tube effect
% Usage: \drawVPNTunnel{from}{to}{encryption}{label}
\newcommand{\drawVPNTunnel}[4]{
    % Outer tunnel (thick dashed)
    \draw[draw=blue!50!black, line width=3pt, dashed, opacity=0.5, rounded corners=2pt]
        (#1) -- (#2);
    % Inner secure connection (solid)
    \draw[draw=green!60!black, line width=1pt, -{Stealth[length=3mm]}, double distance=1pt]
        (#1) -- (#2)
        node[midway, above, font=\tiny\bfseries, fill=green!20, rounded corners=2pt, inner sep=2pt] {
            \textcolor{green!50!black}{$\blacksquare$} VPN: #3
        }
        node[midway, below, font=\tiny] {#4};
}

% Site-to-Site VPN with IPsec indicator
% Usage: \drawSiteToSiteVPN{from}{to}{protocol}{label}
\newcommand{\drawSiteToSiteVPN}[4]{
    \draw[draw=blue!60!black, line width=2.5pt, densely dashed, double distance=2pt, -{Stealth[length=4mm]}]
        (#1) -- (#2)
        node[midway, above, font=\tiny\bfseries, fill=blue!20, rounded corners=3pt, inner sep=2pt] {
            IPsec VPN: #3
        }
        node[midway, below, font=\tiny] {#4};
}

% Wireless connection with wave pattern
% Usage: \drawWirelessConnection{from}{to}{frequency}{label}
\newcommand{\drawWirelessConnection}[4]{
    \draw[draw=purple!60!black, line width=1pt, decorate,
          decoration={snake, amplitude=1mm, segment length=5mm}, -{Stealth[length=3mm]}]
        (#1) -- (#2)
        node[midway, above, font=\tiny, fill=purple!15, rounded corners=2pt, inner sep=2pt] {
            \textasciitilde\textasciitilde\textasciitilde\ #3 #4
        };
}

% WiFi connection with signal strength indicator
% Usage: \drawWiFiConnection{from}{to}{ssid}{signal_strength}{label}
% signal_strength: 1-5 bars
\newcommand{\drawWiFiConnection}[5]{
    \pgfmathsetmacro{\linewidth}{0.5 + 0.3*#4}
    \draw[draw=purple!70!black, line width=\linewidth pt, decorate,
          decoration={snake, amplitude=0.8mm, segment length=4mm}]
        (#1) -- (#2)
        node[midway, above, font=\tiny, fill=purple!20, rounded corners=2pt, inner sep=2pt] {
            WiFi: #3 (Signal: #4/5)
        }
        node[midway, below, font=\tiny] {#5};
}

% Bluetooth connection
% Usage: \drawBluetoothConnection{from}{to}{device_name}{label}
\newcommand{\drawBluetoothConnection}[4]{
    \draw[draw=blue!70!cyan, line width=0.8pt, decorate,
          decoration={zigzag, amplitude=0.8mm, segment length=3mm}, dashed]
        (#1) -- (#2)
        node[midway, above, font=\tiny, fill=cyan!15, rounded corners=2pt, inner sep=2pt] {
            BT: #3
        }
        node[midway, below, font=\tiny] {#4};
}

% Fiber optic connection with light beam effect
% Usage: \drawFiberOpticConnection{from}{to}{speed}{label}
\newcommand{\drawFiberOpticConnection}[4]{
    % Multiple lines for light beam effect
    \draw[draw=yellow!80!orange, line width=1.5pt, -{Stealth[length=4mm]}]
        (#1) -- (#2);
    \draw[draw=yellow!60!white, line width=0.8pt, opacity=0.7]
        (#1) -- (#2);
    \draw[draw=white, line width=0.3pt, opacity=0.5]
        (#1) -- (#2)
        node[midway, above, font=\tiny\bfseries, fill=yellow!20, rounded corners=2pt, inner sep=2pt] {
            \textcolor{orange!80!black}{$\star$} Fiber: #3
        }
        node[midway, below, font=\tiny] {#4};
}

% High-speed fiber backbone
% Usage: \drawFiberBackbone{from}{to}{capacity}{label}
\newcommand{\drawFiberBackbone}[4]{
    \draw[draw=orange!70!yellow, line width=3pt, -{Stealth[length=5mm]}, double distance=2pt,
          postaction={draw=white, line width=1pt, opacity=0.6}]
        (#1) -- (#2)
        node[midway, above, font=\small\bfseries, fill=orange!20, rounded corners=3pt, inner sep=3pt] {
            Fiber Backbone: #3
        }
        node[midway, below, font=\tiny] {#4};
}

% Serial/legacy connection (RS-232, etc.)
% Usage: \drawSerialConnection{from}{to}{baud_rate}{label}
\newcommand{\drawSerialConnection}[4]{
    \draw[draw=gray!60!black, line width=0.8pt, densely dotted, -{Stealth[length=2mm]}]
        (#1) -- (#2)
        node[midway, above, font=\tiny, fill=gray!20, rounded corners=2pt, inner sep=2pt] {
            Serial: #3 baud
        }
        node[midway, below, font=\tiny] {#4};
}

% Satellite link with orbital arc
% Usage: \drawSatelliteLink{from}{to}{latency}{label}
\newcommand{\drawSatelliteLink}[4]{
    \draw[draw=cyan!60!blue, line width=1pt, bend left=30, -{Stealth[length=3mm]}, densely dashed]
        (#1) to (#2)
        node[midway, above, font=\tiny, fill=cyan!15, rounded corners=2pt, inner sep=2pt] {
            \textcolor{blue!70!black}{$\bigcirc$} SAT: #3ms
        }
        node[midway, below, font=\tiny] {#4};
}

% Microwave link
% Usage: \drawMicrowaveLink{from}{to}{frequency}{label}
\newcommand{\drawMicrowaveLink}[4]{
    \draw[draw=red!60!orange, line width=1.2pt, decorate,
          decoration={expanding waves, angle=15, segment length=4mm}, -{Stealth[length=3mm]}]
        (#1) -- (#2)
        node[midway, above, font=\tiny, fill=red!15, rounded corners=2pt, inner sep=2pt] {
            Microwave: #3
        }
        node[midway, below, font=\tiny] {#4};
}

% Infrared connection
% Usage: \drawInfraredConnection{from}{to}{range}{label}
\newcommand{\drawInfraredConnection}[4]{
    \draw[draw=red!70!black, line width=0.8pt, densely dashed, -{Stealth[length=2mm]}]
        (#1) -- (#2)
        [postaction={
            decorate,
            decoration={
                markings,
                mark=between positions 0.2 and 0.8 step 0.2 with {
                    \node[circle, fill=red!50!black, opacity=0.4, inner sep=1pt] {};
                }
            }
        }]
        node[midway, above, font=\tiny, fill=red!15, rounded corners=2pt, inner sep=2pt] {
            IR: #3
        }
        node[midway, below, font=\tiny] {#4};
}

% Power over Ethernet (PoE)
% Usage: \drawPoEConnection{from}{to}{power}{label}
\newcommand{\drawPoEConnection}[4]{
    \draw[draw=yellow!70!orange, line width=1.5pt, -{Stealth[length=3mm]}, double distance=1pt]
        (#1) -- (#2)
        node[midway, above, font=\tiny\bfseries, fill=yellow!25, rounded corners=2pt, inner sep=2pt] {
            \textcolor{orange!80!black}{$\lightning$} PoE: #3W
        }
        node[midway, below, font=\tiny] {#4};
}

% USB connection
% Usage: \drawUSBConnection{from}{to}{usb_version}{label}
\newcommand{\drawUSBConnection}[4]{
    \draw[draw=gray!70!black, line width=1pt, -{Stealth[length=3mm]}]
        (#1) -- (#2)
        node[midway, above, font=\tiny, fill=gray!20, rounded corners=2pt, inner sep=2pt] {
            USB #3
        }
        node[midway, below, font=\tiny] {#4};
}

% ============================================================================
% ADVANCED CONNECTION RENDERING
% ============================================================================

% ============================================================================
% BANDWIDTH VISUALIZATION SYSTEM
% ============================================================================

% Connection with bandwidth indicator (enhanced with logarithmic scaling)
% Usage: \drawConnectionWithBandwidth{from}{to}{bandwidth_mbps}{label}
\newcommand{\drawConnectionWithBandwidth}[4]{
    % Logarithmic scaling for better visualization across ranges
    % Maps: 1 Mbps -> 0.8pt, 10 Mbps -> 1.2pt, 100 Mbps -> 1.6pt, 1000 Mbps -> 2pt, 10000 Mbps -> 2.4pt
    \pgfmathsetmacro{\linewidth}{0.6 + 0.4*ln(max(1,#3))/ln(10)}
    \draw[draw=connNormal, line width=\linewidth pt, -{Stealth[length=3mm]}]
        (#1) -- node[above, font=\tiny] {#4: #3 Mbps} (#2);
}

% Connection with bandwidth and utilization (color-coded)
% Usage: \drawConnectionWithUtilization{from}{to}{bandwidth}{utilization_percent}{label}
\newcommand{\drawConnectionWithUtilization}[5]{
    \pgfmathsetmacro{\linewidth}{0.6 + 0.4*ln(max(1,#3))/ln(10)}
    % Color based on utilization: green <50%, yellow 50-80%, red >80%
    \pgfmathsetmacro{\util}{#4}
    \ifnum\util<50
        \def\connColor{green!60!black}
    \else\ifnum\util<80
        \def\connColor{yellow!80!orange}
    \else
        \def\connColor{red!80!black}
    \fi\fi
    \draw[draw=\connColor, line width=\linewidth pt, -{Stealth[length=3mm]}]
        (#1) -- node[above, font=\tiny] {#5: #3 Mbps (#4\%)} (#2);
}

% High-bandwidth connection (Gbps range)
% Usage: \drawHighBandwidthConnection{from}{to}{bandwidth_gbps}{label}
\newcommand{\drawHighBandwidthConnection}[4]{
    \pgfmathsetmacro{\linewidth}{1.5 + 0.5*ln(max(1,#3))/ln(10)}
    \draw[draw=blue!70!black, line width=\linewidth pt, -{Stealth[length=4mm]}, double]
        (#1) -- node[above, font=\tiny\bfseries] {#4: #3 Gbps} (#2);
}

% Bandwidth with congestion indicator
% Usage: \drawCongestedConnection{from}{to}{bandwidth}{congestion_level}{label}
% congestion_level: 0=none, 1=low, 2=medium, 3=high, 4=critical
\newcommand{\drawCongestedConnection}[5]{
    \pgfmathsetmacro{\linewidth}{0.6 + 0.4*ln(max(1,#3))/ln(10)}
    \ifcase#4
        \def\connStyle{draw=green!60!black}
    \or
        \def\connStyle{draw=yellow!70!black, densely dashed}
    \or
        \def\connStyle{draw=orange!80!black, dashed}
    \or
        \def\connStyle{draw=red!70!black, densely dotted}
    \or
        \def\connStyle{draw=red!90!black, line width=2pt, densely dotted}
    \fi
    \draw[\connStyle, -{Stealth[length=3mm]}]
        (#1) -- node[above, font=\tiny] {#5: #3 Mbps} (#2);
}

% Bandwidth comparison (dual connections showing before/after or primary/backup)
% Usage: \drawDualBandwidthConnection{from}{to}{bw1}{bw2}{label}
\newcommand{\drawDualBandwidthConnection}[5]{
    \pgfmathsetmacro{\linewidth}{0.6 + 0.4*ln(max(1,#3))/ln(10)}
    % Primary path
    \draw[draw=blue!70!black, line width=\linewidth pt, -{Stealth[length=3mm]}]
        ([yshift=2pt]#1) -- node[above, font=\tiny] {Primary: #3 Mbps} ([yshift=2pt]#2);
    % Backup/secondary path
    \pgfmathsetmacro{\linewidth2}{0.6 + 0.4*ln(max(1,#4))/ln(10)}
    \draw[draw=gray!60, line width=\linewidth2 pt, dashed, -{Stealth[length=2mm]}]
        ([yshift=-2pt]#1) -- node[below, font=\tiny] {Backup: #4 Mbps} ([yshift=-2pt]#2);
}

% ============================================================================
% PROTOCOL AND PORT LABELING SYSTEM
% ============================================================================

% Connection with protocol and port information (basic)
% Usage: \drawConnectionWithPort{from}{to}{protocol}{port}{label}
\newcommand{\drawConnectionWithPort}[5]{
    \draw[normal conn] (#1) --
        node[above, font=\tiny\ttfamily] {#5}
        node[below, port label] {#3:#4}
        (#2);
}

% Enhanced connection with auto-positioned protocol labels
% Automatically positions label based on connection angle
% Usage: \drawSmartLabeledConnection{from}{to}{protocol}{port}{service}{label}
\newcommand{\drawSmartLabeledConnection}[6]{
    \draw[normal conn] (#1) -- (#2)
        node[pos=0.3, above, sloped, font=\tiny\ttfamily, fill=white, inner sep=1pt] {#3:#4}
        node[pos=0.7, below, sloped, font=\tiny, fill=white, inner sep=1pt] {#6}
        node[pos=0.5, above, font=\scriptsize, fill=yellow!20, rounded corners=1pt, inner sep=1pt] {#5};
}

% Multi-protocol connection with stacked labels
% Usage: \drawMultiProtocolConnection{from}{to}{proto1}{port1}{proto2}{port2}{label}
\newcommand{\drawMultiProtocolConnection}[7]{
    \draw[normal conn] (#1) -- (#2)
        node[midway, above, font=\tiny\ttfamily, align=center, fill=white, rounded corners=2pt, inner sep=2pt] {
            #3:#4 \\ #5:#6
        }
        node[midway, below, font=\tiny] {#7};
}

% Connection with detailed service information
% Usage: \drawServiceConnection{from}{to}{protocol}{port}{service}{version}{label}
\newcommand{\drawServiceConnection}[7]{
    \draw[normal conn] (#1) -- (#2)
        node[midway, above, font=\tiny, fill=blue!10, rounded corners=2pt, inner sep=2pt, align=center] {
            \textbf{#5} \\
            \texttt{#3:#4} \\
            \textit{#6}
        }
        node[pos=0.2, below, font=\tiny] {#7};
}

% Auto-positioned port label (avoids node overlap)
% Usage: \drawAutoPortConnection{from}{to}{protocol}{port}{label}
\newcommand{\drawAutoPortConnection}[5]{
    \path let \p1=(#1), \p2=(#2),
              \n1={atan2(\y2-\y1,\x2-\x1)} in
        \pgfextra{
            \pgfmathsetmacro{\angle}{\n1}
            % Position label based on angle
            \ifdim\angle pt<45pt
                \def\labelpos{above}
            \else\ifdim\angle pt<135pt
                \def\labelpos{right}
            \else\ifdim\angle pt<225pt
                \def\labelpos{below}
            \else
                \def\labelpos{left}
            \fi\fi\fi
        }
        (#1) edge[normal conn] node[\labelpos, font=\tiny\ttfamily] {#3:#4}
                                node[pos=0.7, sloped, above, font=\tiny] {#5} (#2);
}

% Inline protocol badge (small colored box with protocol)
% Usage: \drawProtocolBadgeConnection{from}{to}{protocol}{port}{label}
\newcommand{\drawProtocolBadgeConnection}[5]{
    \draw[normal conn] (#1) -- (#2)
        node[pos=0.5, above, font=\scriptsize] {#5}
        node[pos=0.5, below, font=\tiny\ttfamily\bfseries,
              fill=blue!20, draw=blue!60, rounded corners=2pt, inner sep=1.5pt] {#3:#4};
}

% Connection with application layer protocol
% Usage: \drawAppLayerConnection{from}{to}{app_proto}{transport}{port}{label}
% Example: HTTP over TCP:80
\newcommand{\drawAppLayerConnection}[6]{
    \draw[normal conn] (#1) -- (#2)
        node[midway, above, font=\tiny\bfseries, fill=green!20, rounded corners=2pt, inner sep=2pt] {#3}
        node[midway, below, font=\tiny\ttfamily, fill=white, inner sep=1pt] {#4:#5}
        node[pos=0.2, below, font=\tiny] {#6};
}

% Encrypted protocol connection with lock indicator
% Usage: \drawEncryptedProtocolConnection{from}{to}{protocol}{port}{cipher}{label}
\newcommand{\drawEncryptedProtocolConnection}[6]{
    \draw[encrypted conn] (#1) -- (#2)
        node[midway, above, font=\tiny\bfseries, fill=green!30, rounded corners=2pt, inner sep=2pt] {
            \textcolor{green!50!black}{$\blacksquare$} #3:#4
        }
        node[midway, below, font=\tiny, fill=white, inner sep=1pt] {#5}
        node[pos=0.2, below, font=\tiny] {#6};
}

% Multiple ports on same connection
% Usage: \drawMultiPortConnection{from}{to}{protocol}{port_range}{label}
% Example: TCP:8000-8010
\newcommand{\drawMultiPortConnection}[5]{
    \draw[normal conn] (#1) -- (#2)
        node[midway, above, font=\tiny\ttfamily, fill=orange!20, rounded corners=2pt, inner sep=2pt] {
            #3:#4
        }
        node[pos=0.7, below, font=\tiny] {#5};
}

% Port with state indicator (open/closed/filtered)
% Usage: \drawStatefulPortConnection{from}{to}{protocol}{port}{state}{label}
% state: 0=closed, 1=filtered, 2=open
\newcommand{\drawStatefulPortConnection}[6]{
    \ifcase#5
        \def\stateColor{red!20}
        \def\stateText{CLOSED}
    \or
        \def\stateColor{yellow!30}
        \def\stateText{FILTERED}
    \or
        \def\stateColor{green!20}
        \def\stateText{OPEN}
    \fi
    \draw[normal conn] (#1) -- (#2)
        node[midway, above, font=\tiny\ttfamily, fill=\stateColor, rounded corners=2pt, inner sep=2pt] {
            #3:#4 [\stateText]
        }
        node[pos=0.7, below, font=\tiny] {#6};
}

% Curved connection (for avoiding overlaps)
% Usage: \drawCurvedConnection{from}{to}{bend}{label}
\newcommand{\drawCurvedConnection}[4]{
    \draw[normal conn] (#1) to[bend left=#3] 
        node[above, font=\tiny] {#4} (#2);
}

% ============================================================================
% AUTOMATIC PATH FINDING AND ROUTING
% ============================================================================

% Orthogonal routing (right-angle connections)
% Usage: \drawOrthogonalConnection{from}{to}{label}
\newcommand{\drawOrthogonalConnection}[3]{
    \draw[normal conn, rounded corners=3pt] (#1) -| (#2)
        node[pos=0.5, above, font=\tiny] {#3};
}

% Orthogonal routing with vertical-horizontal path
% Usage: \drawOrthogonalConnectionVH{from}{to}{label}
\newcommand{\drawOrthogonalConnectionVH}[3]{
    \draw[normal conn, rounded corners=3pt] (#1) |- (#2)
        node[pos=0.5, above, font=\tiny] {#3};
}

% Smart curved connection with automatic bend angle
% Calculates bend angle based on node distance
% Usage: \drawSmartCurvedConnection{from}{to}{label}
\newcommand{\drawSmartCurvedConnection}[3]{
    \draw[normal conn] (#1) to[bend left=20]
        node[pos=0.5, above, font=\tiny] {#3} (#2);
}

% Multi-waypoint connection for complex routing
% Usage: \drawWaypointConnection{from}{to}{waypoint1}{waypoint2}{label}
\newcommand{\drawWaypointConnection}[5]{
    \draw[normal conn] (#1) -- (#3) -- (#4) -- (#2)
        node[pos=0.5, above, font=\tiny] {#5};
}

% Avoid overlap connection with automatic path selection
% Uses curved path if nodes are close, straight if far
% Usage: \drawAvoidOverlapConnection{from}{to}{label}
\newcommand{\drawAvoidOverlapConnection}[3]{
    \path let \p1=(#1), \p2=(#2),
              \n1={veclen(\x2-\x1,\y2-\y1)} in
        \pgfextra{
            \pgfmathsetmacro{\distance}{\n1}
            \ifdim\distance pt<100pt
                \def\pathstyle{bend left=25}
            \else
                \def\pathstyle{}
            \fi
        }
        (#1) edge[\pathstyle, normal conn] node[pos=0.5, above, font=\tiny] {#3} (#2);
}

% Bezier curve connection for smooth organic paths
% Usage: \drawBezierConnection{from}{to}{ctrl1}{ctrl2}{label}
\newcommand{\drawBezierConnection}[5]{
    \draw[normal conn] (#1) .. controls (#3) and (#4) .. (#2)
        node[pos=0.5, above, font=\tiny] {#5};
}

% Obstacle-avoiding curved path (uses higher bend for closer nodes)
% Usage: \drawObstacleAvoidingConnection{from}{to}{label}{min_distance}
\newcommand{\drawObstacleAvoidingConnection}[4]{
    \path let \p1=(#1), \p2=(#2),
              \n1={veclen(\x2-\x1,\y2-\y1)} in
        \pgfextra{
            \pgfmathsetmacro{\distance}{\n1}
            \pgfmathsetmacro{\bendangle}{min(60, max(15, 3000/\distance))}
        }
        (#1) edge[bend left=\bendangle, normal conn]
            node[pos=0.5, above, font=\tiny] {#3} (#2);
}

% ============================================================================
% CONNECTION FLOW VISUALIZATION
% ============================================================================

% Animated flow indicators (requires animation package)
% Usage: \drawFlowConnection{from}{to}{direction}{speed}
\newcommand{\drawFlowConnection}[4]{
    \draw[draw=connNormal, line width=1.5pt, -{Stealth[length=3mm]}] (#1) -- (#2)
        [postaction={
            decorate,
            decoration={
                markings,
                mark=between positions 0.1 and 0.9 step 0.2 with {
                    \ifthenelse{\equal{#3}{forward}}{
                        \arrow{Stealth[length=2mm, fill=connNormal]}
                    }{
                        \arrow{Stealth[reversed, length=2mm, fill=connNormal]}
                    }
                }
            }
        }];
}

% Traffic flow with volume indicator
% Usage: \drawTrafficFlow{from}{to}{packets_per_sec}{label}
\newcommand{\drawTrafficFlow}[4]{
    \pgfmathsetmacro{\density}{min(10, #3/100)}
    \draw[draw=connNormal, line width=1.5pt, -{Stealth[length=3mm]}] 
        (#1) -- node[above, font=\tiny] {#4: #3 pps} (#2)
        [postaction={
            decorate,
            decoration={
                markings,
                mark=between positions 0.1 and 0.9 step {0.1/\density} with {
                    \node[circle, fill=connNormal, inner sep=0.5pt] {};
                }
            }
        }];
}

% ============================================================================
% CONNECTION BUNDLING FOR HIGH-DENSITY DIAGRAMS
% ============================================================================

% Bundle multiple connections into a single visual with counter
% Usage: \drawBundledConnection{from}{to}{count}{label}
\newcommand{\drawBundledConnection}[4]{
    % Thicker line to represent multiple connections
    \pgfmathsetmacro{\linewidth}{1 + 0.3*ln(max(1,#3))/ln(10)}
    \draw[draw=connNormal, line width=\linewidth pt, -{Stealth[length=3mm]}]
        (#1) -- (#2)
        node[midway, above, font=\tiny] {#4}
        node[midway, below, font=\scriptsize, fill=white, circle, inner sep=1pt] {\textbf{#3}};
}

% Bundle connections with protocol breakdown
% Usage: \drawProtocolBundledConnection{from}{to}{total}{tcp}{udp}{other}{label}
\newcommand{\drawProtocolBundledConnection}[7]{
    \pgfmathsetmacro{\linewidth}{1 + 0.3*ln(max(1,#3))/ln(10)}
    \draw[draw=connNormal, line width=\linewidth pt, -{Stealth[length=3mm]}]
        (#1) -- (#2)
        node[midway, above, font=\tiny] {#7}
        node[midway, below, font=\scriptsize, fill=white, rounded corners=2pt, inner sep=2pt] {
            \textbf{#3} (TCP:#4 UDP:#5 Other:#6)
        };
}

% Aggregated connection with summary statistics
% Usage: \drawAggregatedConnection{from}{to}{conn_count}{total_bandwidth}{label}
\newcommand{\drawAggregatedConnection}[5]{
    \pgfmathsetmacro{\linewidth}{1 + 0.4*ln(max(1,#4))/ln(10)}
    \draw[draw=blue!60!black, line width=\linewidth pt, -{Stealth[length=4mm]}, double distance=1pt]
        (#1) -- (#2)
        node[midway, above, font=\tiny\bfseries] {#5}
        node[midway, below, font=\scriptsize, fill=blue!10, rounded corners=2pt, inner sep=2pt] {
            #3 conns | #4 Mbps
        };
}

% Connection bundle with expandable detail
% Shows collapsed view with option to reference detailed breakdown
% Usage: \drawExpandableBundledConnection{from}{to}{count}{protocols}{label}
\newcommand{\drawExpandableBundledConnection}[5]{
    \pgfmathsetmacro{\linewidth}{1 + 0.3*ln(max(1,#3))/ln(10)}
    \draw[draw=connNormal, line width=\linewidth pt, -{Stealth[length=3mm]}]
        (#1) -- (#2)
        node[midway, above, font=\tiny] {#5}
        node[midway, below, font=\scriptsize, fill=yellow!20, rounded corners=2pt, inner sep=2pt] {
            \textbf{#3 connections} \textit{(#4)}
        };
}

% Multi-tier bundle (for complex aggregations)
% Usage: \drawMultiTierBundle{from}{to}{edge_count}{core_count}{label}
\newcommand{\drawMultiTierBundle}[5]{
    % Draw thick core bundle
    \draw[draw=blue!70!black, line width=3pt, -{Stealth[length=5mm]}, double distance=2pt]
        (#1) -- (#2)
        node[midway, above, font=\small\bfseries] {#5}
        node[midway, below, font=\scriptsize, fill=blue!20, rounded corners=3pt, inner sep=3pt] {
            Core: #4 | Edge: #3
        };
}

% Parallel connection bundle (shows multiple parallel lines)
% Usage: \drawParallelBundle{from}{to}{count}{label}
\newcommand{\drawParallelBundle}[4]{
    \foreach \i in {1,...,#3} {
        \pgfmathsetmacro{\offset}{-1.5 + 3*\i/#3}
        \draw[draw=connNormal, line width=0.5pt, -{Stealth[length=2mm]}]
            ([yshift=\offset pt]#1) -- ([yshift=\offset pt]#2);
    }
    \node[above, font=\tiny] at ($(#1)!0.5!(#2)$) {#4 (#3 parallel)};
}

% Grouped connection bundle with color coding
% Usage: \drawColorCodedBundle{from}{to}{safe}{suspicious}{malicious}{label}
\newcommand{\drawColorCodedBundle}[6]{
    \pgfmathsetmacro{\total}{#3+#4+#5}
    \pgfmathsetmacro{\linewidth}{1 + 0.3*ln(max(1,\total))/ln(10)}
    \draw[draw=gray!60!black, line width=\linewidth pt, -{Stealth[length=3mm]}]
        (#1) -- (#2)
        node[midway, above, font=\tiny] {#6}
        node[midway, below, font=\scriptsize, fill=white, rounded corners=2pt, inner sep=2pt] {
            \textcolor{green!60!black}{#3} |
            \textcolor{orange!80!black}{#4} |
            \textcolor{red!80!black}{#5}
        };
}

% ============================================================================
% ENHANCED ANIMATED FLOW DIRECTION INDICATORS
% ============================================================================

% Bi-directional animated flow with different colors
% Usage: \drawBidirectionalFlow{from}{to}{forward_protocol}{reverse_protocol}{label}
\newcommand{\drawBidirectionalFlow}[5]{
    % Forward flow (blue)
    \draw[draw=blue!60!black, line width=1pt, -{Stealth[length=3mm]}]
        ([yshift=1.5pt]#1) -- ([yshift=1.5pt]#2)
        [postaction={
            decorate,
            decoration={
                markings,
                mark=between positions 0.1 and 0.9 step 0.15 with {
                    \arrow{Stealth[length=2mm, fill=blue!70!black]}
                }
            }
        }]
        node[pos=0.3, above, font=\tiny] {#3};
    % Reverse flow (green)
    \draw[draw=green!60!black, line width=1pt, -{Stealth[length=3mm]}]
        ([yshift=-1.5pt]#2) -- ([yshift=-1.5pt]#1)
        [postaction={
            decorate,
            decoration={
                markings,
                mark=between positions 0.1 and 0.9 step 0.15 with {
                    \arrow{Stealth[length=2mm, fill=green!70!black]}
                }
            }
        }]
        node[pos=0.7, below, font=\tiny] {#4};
    \node[above, font=\tiny] at ($(#1)!0.5!(#2)$) {#5};
}

% Multi-speed flow indicator (different speeds for different traffic types)
% Usage: \drawMultiSpeedFlow{from}{to}{speed_level}{label}
% speed_level: 1=slow, 2=medium, 3=fast, 4=very fast
\newcommand{\drawMultiSpeedFlow}[4]{
    \pgfmathsetmacro{\stepsize}{0.3/(#3)}
    \pgfmathsetmacro{\arrowsize}{1 + 0.5*#3}
    \draw[draw=connNormal, line width=1.5pt, -{Stealth[length=3mm]}]
        (#1) -- (#2)
        [postaction={
            decorate,
            decoration={
                markings,
                mark=between positions 0.1 and 0.9 step \stepsize with {
                    \arrow{Stealth[length=\arrowsize mm, fill=connNormal]}
                }
            }
        }]
        node[midway, above, font=\tiny] {#4};
}

% Colored packet flow (different colors for different protocols)
% Usage: \drawColoredPacketFlow{from}{to}{color}{packet_size}{label}
\newcommand{\drawColoredPacketFlow}[5]{
    \pgfmathsetmacro{\packetsize}{0.5 + #4/2}
    \draw[draw=#3!70!black, line width=1.5pt, -{Stealth[length=3mm]}]
        (#1) -- (#2)
        [postaction={
            decorate,
            decoration={
                markings,
                mark=between positions 0.1 and 0.9 step 0.15 with {
                    \node[circle, fill=#3!80!black, inner sep=\packetsize pt] {};
                }
            }
        }]
        node[midway, above, font=\tiny] {#5};
}

% Traffic intensity visualization (pulse effect)
% Usage: \drawPulseFlow{from}{to}{intensity}{label}
\newcommand{\drawPulseFlow}[4]{
    \pgfmathsetmacro{\linewidth}{0.8 + 0.4*#3}
    \draw[draw=orange!70!black, line width=\linewidth pt, -{Stealth[length=4mm]}]
        (#1) -- (#2)
        [postaction={
            decorate,
            decoration={
                markings,
                mark=between positions 0 and 1 step 0.1 with {
                    \pgfmathsetmacro{\opacity}{0.3 + 0.7*sin(\pgfkeysvalueof{/pgf/decoration/mark info/sequence number}*60)}
                    \node[circle, fill=orange!90!black, opacity=\opacity, inner sep=1.5pt] {};
                }
            }
        }]
        node[midway, above, font=\tiny\bfseries] {#5: Intensity #3};
}

% Protocol-specific flow visualization
% Usage: \drawProtocolFlow{from}{to}{protocol}{label}
% protocol: HTTP, DNS, SSH, FTP, SMTP
\newcommand{\drawProtocolFlow}[4]{
    % Set color based on protocol
    \ifthenelse{\equal{#3}{HTTP}}{
        \def\flowcolor{blue}
    }{}
    \ifthenelse{\equal{#3}{HTTPS}}{
        \def\flowcolor{green}
    }{}
    \ifthenelse{\equal{#3}{DNS}}{
        \def\flowcolor{purple}
    }{}
    \ifthenelse{\equal{#3}{SSH}}{
        \def\flowcolor{teal}
    }{}
    \ifthenelse{\equal{#3}{FTP}}{
        \def\flowcolor{orange}
    }{}
    \ifthenelse{\equal{#3}{SMTP}}{
        \def\flowcolor{red}
    }{}
    \draw[draw=\flowcolor!70!black, line width=1.5pt, -{Stealth[length=3mm]}]
        (#1) -- (#2)
        [postaction={
            decorate,
            decoration={
                markings,
                mark=between positions 0.1 and 0.9 step 0.2 with {
                    \node[fill=\flowcolor!80!black, circle, inner sep=1pt] {};
                }
            }
        }]
        node[midway, above, font=\tiny, fill=\flowcolor!20, rounded corners=2pt, inner sep=2pt] {#3: #4};
}

% Congestion visualization with traffic markers
% Usage: \drawCongestionFlow{from}{to}{congestion_percent}{label}
\newcommand{\drawCongestionFlow}[4]{
    \pgfmathsetmacro{\cong}{#3}
    \ifnum\cong<30
        \def\congcolor{green}
        \def\conglabel{Normal}
    \else\ifnum\cong<70
        \def\congcolor{yellow}
        \def\conglabel{Moderate}
    \else
        \def\congcolor{red}
        \def\conglabel{Congested}
    \fi\fi
    \draw[draw=\congcolor!70!black, line width=2pt, -{Stealth[length=4mm]}]
        (#1) -- (#2)
        [postaction={
            decorate,
            decoration={
                markings,
                mark=between positions 0.1 and 0.9 step 0.12 with {
                    \node[regular polygon, regular polygon sides=3, fill=\congcolor!90!black, inner sep=1pt] {};
                }
            }
        }]
        node[midway, above, font=\tiny\bfseries, fill=\congcolor!30, rounded corners=2pt, inner sep=2pt] {
            #4 - \conglabel (#3\%)
        };
}

% TODO: Advanced flow features
% - Real-time animation support for Beamer presentations
% - Time-series flow data visualization
% - Flow heatmap overlays

% ============================================================================
% CONNECTION RENDERING ENGINE
% ============================================================================

% Main command to render all connections
\newcommand{\renderConnections}{
    % This will be populated by network_data.tex
    % Example structure:
    % \drawConnection{srv1}{pc1}{HTTP}
    % \drawAttackConnection{attacker1}{srv1}{SQL Injection}
}

% TODO: Intelligent connection rendering
% - Layer-based rendering (background to foreground)
% - Z-order management for overlapping connections
% - Automatic connection routing to minimize crossings
% - Connection aggregation (show "10 connections" instead of 10 lines)
% - Highlight selected connection paths

% ============================================================================
% CONNECTION LABELS AND ANNOTATIONS
% ============================================================================

% Add inline statistics to connection
% Usage: \labelConnectionStats{from}{to}{latency}{packet_loss}{jitter}
\newcommand{\labelConnectionStats}[5]{
    \draw[normal conn] (#1) -- (#2)
        node[midway, below, font=\tiny\ttfamily, fill=white, inner sep=2pt] {
            L:#3ms | PL:#4\% | J:#5ms
        };
}

% Add threat score to connection
% Usage: \labelConnectionThreat{from}{to}{score}{details}
\newcommand{\labelConnectionThreat}[4]{
    \draw[suspicious conn] (#1) -- (#2)
        node[midway, threat label] {
            \textbf{#3/10} #4
        };
}

% TODO: Label improvements
% - Auto-positioning to avoid overlaps
% - Expandable detail boxes on hover
% - Time-series data for connection metrics
% - Alert indicators for anomalous connections

% ============================================================================
% CONNECTION FILTERING AND LAYER MANAGEMENT
% ============================================================================

% Connection visibility control flags
\newif\ifshownormal
\newif\ifshowencrypted
\newif\ifshowsuspicious
\newif\ifshowattacks
\newif\ifshowwireless
\newif\ifshowvpn

% Default: show all connection types
\shownormaltrue
\showencryptedtrue
\showsuspicioustrue
\showattackstrue
\showwirelesstrue
\showvpntrue

% Toggle connection type visibility
% Usage: \toggleConnectionType{type}
\newcommand{\toggleConnectionType}[1]{
    \ifthenelse{\equal{#1}{normal}}{\shownormalfalse}{}
    \ifthenelse{\equal{#1}{encrypted}}{\showencryptedfalse}{}
    \ifthenelse{\equal{#1}{suspicious}}{\showsuspiciousfalse}{}
    \ifthenelse{\equal{#1}{attacks}}{\showattacksfalse}{}
    \ifthenelse{\equal{#1}{wireless}}{\showwirelessfalse}{}
    \ifthenelse{\equal{#1}{vpn}}{\showvpnfalse}{}
}

% Show only specific connection type
% Usage: \showOnlyConnectionType{type}
\newcommand{\showOnlyConnectionType}[1]{
    \shownormalfalse
    \showencryptedfalse
    \showsuspiciousfalse
    \showattacksfalse
    \showwirelessfalse
    \showvpnfalse
    \ifthenelse{\equal{#1}{normal}}{\shownormaltrue}{}
    \ifthenelse{\equal{#1}{encrypted}}{\showencryptedtrue}{}
    \ifthenelse{\equal{#1}{suspicious}}{\showsuspicioustrue}{}
    \ifthenelse{\equal{#1}{attacks}}{\showattackstrue}{}
    \ifthenelse{\equal{#1}{wireless}}{\showwirelesstrue}{}
    \ifthenelse{\equal{#1}{vpn}}{\showvpntrue}{}
}

% Reset to show all connections
% Usage: \showAllConnections
\newcommand{\showAllConnections}{
    \shownormaltrue
    \showencryptedtrue
    \showsuspicioustrue
    \showattackstrue
    \showwirelesstrue
    \showvpntrue
}

% Port-based filtering helper
% Usage: \drawConnectionIfPort{from}{to}{protocol}{port}{target_port}{label}
\newcommand{\drawConnectionIfPort}[6]{
    \ifthenelse{\equal{#4}{#5}}{
        \drawConnectionWithPort{#1}{#2}{#3}{#4}{#6}
    }{}
}

% Bandwidth threshold filtering
% Usage: \drawConnectionIfBandwidthAbove{from}{to}{bandwidth}{threshold}{label}
\newcommand{\drawConnectionIfBandwidthAbove}[5]{
    \pgfmathsetmacro{\bw}{#3}
    \pgfmathsetmacro{\thresh}{#4}
    \ifdim\bw pt>\thresh pt
        \drawConnectionWithBandwidth{#1}{#2}{#3}{#5}
    \fi
}

% Subnet-based filtering (intra-subnet only)
% Usage: \drawIntraSubnetConnection{from}{to}{subnet_id}{label}
\newcommand{\drawIntraSubnetConnection}[4]{
    \draw[normal conn, draw=blue!40!black, dashed]
        (#1) -- (#2)
        node[midway, above, font=\tiny, fill=blue!10] {#4 (Subnet #3)};
}

% Inter-subnet connection
% Usage: \drawInterSubnetConnection{from}{to}{from_subnet}{to_subnet}{label}
\newcommand{\drawInterSubnetConnection}[5]{
    \draw[normal conn, draw=orange!60!black, line width=1.2pt]
        (#1) -- (#2)
        node[midway, above, font=\tiny, fill=orange!15] {#5 (#3→#4)};
}

% Layer-based rendering (background, normal, foreground)
% Usage: \drawConnectionOnLayer{from}{to}{layer}{label}
\newcommand{\drawConnectionOnLayer}[4]{
    \ifthenelse{\equal{#3}{background}}{
        \begin{scope}[on background layer]
            \draw[normal conn, opacity=0.3] (#1) -- (#2)
                node[midway, above, font=\tiny, opacity=0.5] {#4};
        \end{scope}
    }{}
    \ifthenelse{\equal{#3}{normal}}{
        \draw[normal conn] (#1) -- (#2)
            node[midway, above, font=\tiny] {#4};
    }{}
    \ifthenelse{\equal{#3}{foreground}}{
        \draw[normal conn, line width=1.5pt] (#1) -- (#2)
            node[midway, above, font=\tiny\bfseries] {#4};
    }{}
}

% Highlight specific connection path
% Usage: \highlightConnectionPath{from}{to}{label}
\newcommand{\highlightConnectionPath}[3]{
    % Draw glow effect
    \draw[draw=yellow!80!orange, line width=5pt, opacity=0.3] (#1) -- (#2);
    \draw[draw=yellow!90!orange, line width=3pt, opacity=0.5] (#1) -- (#2);
    % Draw main connection
    \draw[draw=orange!90!black, line width=2pt, -{Stealth[length=4mm]}] (#1) -- (#2)
        node[midway, above, font=\small\bfseries, fill=yellow!30, rounded corners=3pt, inner sep=3pt] {
            HIGHLIGHTED: #3
        };
}

% ============================================================================
% CONNECTION PATTERNS AND ANALYSIS
% ============================================================================

% Highlight attack pattern (multiple sources to one target)
% Usage: \highlightAttackPattern{target}{sources}
\newcommand{\highlightAttackPattern}[2]{
    \begin{scope}[on background layer]
        \foreach \source in {#2} {
            \draw[attack conn, line width=2pt] (\source) -- (#1);
        }
        % Draw highlight around target
        \node[draw=threatCritical, line width=3pt, 
              rounded corners=5pt, inner sep=8pt, 
              fill=threatCritical!10] at (#1) {};
    \end{scope}
}

% Show connection path through network
% Usage: \showConnectionPath{node_list}
\newcommand{\showConnectionPath}[1]{
    % Draw path through multiple hops
    % Highlight the route taken by traffic
}

% ============================================================================
% ENHANCED ATTACK PATTERN DETECTION VISUALIZATION
% ============================================================================

% DDoS attack pattern (many sources to one target)
% Usage: \visualizeDDoSPattern{target}{source_list}
\newcommand{\visualizeDDoSPattern}[2]{
    \begin{scope}[on background layer]
        % Draw attack cone
        \node[draw=red!70!black, fill=red!10, circle, minimum size=3cm, opacity=0.3] at (#1) {};
        % Draw connections from all sources
        \foreach \source in {#2} {
            \draw[draw=red!80!black, line width=1.5pt, -{Stealth[length=3mm]},
                  postaction={
                      decorate,
                      decoration={
                          markings,
                          mark=between positions 0.2 and 0.8 step 0.2 with {
                              \arrow{Stealth[length=2mm, fill=red!90!black]}
                          }
                      }
                  }]
                (\source) -- (#1);
        }
        % Highlight target
        \node[draw=red!90!black, line width=3pt, fill=red!20,
              rounded corners=5pt, inner sep=10pt] at (#1) {};
        \node[above=15pt of #1, font=\small\bfseries, fill=red!30,
              rounded corners=3pt, inner sep=3pt] {⚠ DDoS ATTACK};
    \end{scope}
}

% Data exfiltration pattern (one source to many external destinations)
% Usage: \visualizeDataExfiltration{source}{destination_list}
\newcommand{\visualizeDataExfiltration}[2]{
    \begin{scope}
        % Draw source highlight
        \node[draw=orange!80!black, line width=2pt, fill=orange!15,
              rounded corners=5pt, inner sep=8pt] at (#1) {};
        % Draw connections to all destinations
        \foreach \dest in {#2} {
            \draw[draw=orange!70!black, line width=1.2pt, dashed, -{Stealth[length=3mm]}]
                (#1) -- (\dest)
                node[pos=0.7, above, font=\tiny, fill=orange!20] {Data};
        }
        \node[below=12pt of #1, font=\small\bfseries, fill=orange!25,
              rounded corners=3pt, inner sep=3pt] {⚠ Exfiltration};
    \end{scope}
}

% Lateral movement pattern (peer-to-peer internal propagation)
% Usage: \visualizeLateralMovement{node_chain}
\newcommand{\visualizeLateralMovement}[1]{
    \begin{scope}
        \foreach \i [count=\ni from 2] in {#1} {
            \ifnum\ni<99
                \pgfmathsetmacro{\next}{{\ni}}
                \draw[draw=purple!70!black, line width=1.5pt, -{Stealth[length=3mm]},
                      densely dashed]
                    (\i) -- ({\next})
                    node[midway, above, font=\tiny, fill=purple!15] {Lateral};
            \fi
        }
    \end{scope}
}

% Command & Control beaconing pattern
% Usage: \visualizeC2Beaconing{infected_host}{c2_server}{beacon_count}
\newcommand{\visualizeC2Beaconing}[3]{
    \begin{scope}
        % Draw bidirectional beaconing
        \draw[draw=red!60!black, line width=1pt, <->, densely dotted]
            (#1) -- (#2)
            node[midway, above, font=\tiny, fill=red!15, rounded corners=2pt, inner sep=2pt] {
                C2 Beacon (#3x)
            };
        % Draw beacon indicators
        \foreach \i in {1,...,#3} {
            \pgfmathsetmacro{\pos}{0.1 + 0.8*\i/#3}
            \draw[draw=red!80!black, fill=red!50, opacity=0.6]
                ($(#1)!\pos!(#2)$) circle (2pt);
        }
        % Mark infected host
        \node[above=10pt of #1, font=\scriptsize\bfseries, fill=red!25,
              rounded corners=2pt, inner sep=2pt] {⚠ Infected};
    \end{scope}
}

% Port scanning pattern (one source scanning multiple targets)
% Usage: \visualizePortScan{scanner}{target_list}{port}
\newcommand{\visualizePortScan}[3]{
    \begin{scope}
        % Draw scanning connections
        \foreach \target in {#2} {
            \draw[draw=yellow!70!black, line width=0.8pt, densely dashed, -{Stealth[length=2mm]}]
                (#1) -- (\target)
                node[pos=0.7, above, font=\tiny, fill=yellow!20] {:#3};
        }
        % Mark scanner
        \node[below=10pt of #1, font=\scriptsize\bfseries, fill=yellow!30,
              rounded corners=2pt, inner sep=2pt] {⚠ Scanner};
    \end{scope}
}

% Brute force attack pattern
% Usage: \visualizeBruteForce{attacker}{target}{attempt_count}
\newcommand{\visualizeBruteForce}[3]{
    \begin{scope}
        % Draw multiple attempt lines with varying opacity
        \foreach \i in {1,...,#3} {
            \pgfmathsetmacro{\yoffset}{-2 + 4*\i/#3}
            \pgfmathsetmacro{\opacity}{0.2 + 0.6*\i/#3}
            \draw[draw=red!70!black, line width=0.8pt, -{Stealth[length=2mm]},
                  opacity=\opacity]
                ([yshift=\yoffset pt]#1) -- ([yshift=\yoffset pt]#2);
        }
        % Main connection
        \draw[draw=red!80!black, line width=1.5pt, -{Stealth[length=3mm]}]
            (#1) -- (#2)
            node[midway, above, font=\tiny, fill=red!20, rounded corners=2pt, inner sep=2pt] {
                Brute Force (#3 attempts)
            };
    \end{scope}
}

% Man-in-the-Middle attack visualization
% Usage: \visualizeMITM{source}{attacker}{destination}
\newcommand{\visualizeMITM}[3]{
    \begin{scope}
        % Draw intercepted path
        \draw[draw=orange!70!black, line width=1.2pt, -{Stealth[length=3mm]}]
            (#1) -- (#2)
            node[midway, above, font=\tiny] {Intercepted};
        \draw[draw=orange!70!black, line width=1.2pt, -{Stealth[length=3mm]}]
            (#2) -- (#3)
            node[midway, above, font=\tiny] {Forwarded};
        % Draw original path (faded)
        \draw[draw=gray!50, line width=0.5pt, dashed, opacity=0.4]
            (#1) -- (#3)
            node[midway, below, font=\tiny, opacity=0.6] {Original};
        % Highlight attacker
        \node[draw=orange!90!black, line width=2pt, fill=orange!20,
              star, star points=5, inner sep=5pt] at (#2) {};
        \node[above=12pt of #2, font=\small\bfseries, fill=orange!30,
              rounded corners=3pt, inner sep=3pt] {⚠ MITM};
    \end{scope}
}

% DNS tunneling pattern
% Usage: \visualizeDNSTunneling{client}{dns_server}{data_volume}
\newcommand{\visualizeDNSTunneling}[3]{
    \draw[draw=purple!60!black, line width=1.5pt, <->, decorate,
          decoration={snake, amplitude=0.5mm, segment length=3mm}]
        (#1) -- (#2)
        node[midway, above, font=\tiny, fill=purple!15, rounded corners=2pt, inner sep=2pt] {
            DNS Tunnel: #3 KB
        }
        node[midway, below, font=\tiny, fill=purple!25, rounded corners=2pt, inner sep=2pt] {
            ⚠ Suspicious
        };
}

% Reconnaissance/Enumeration pattern
% Usage: \visualizeReconnaissance{attacker}{target_network}
\newcommand{\visualizeReconnaissance}[2]{
    \draw[draw=yellow!60!black, line width=1pt, ->, decorate,
          decoration={random steps, segment length=8mm, amplitude=2mm}]
        (#1) -- (#2)
        node[midway, above, font=\tiny, fill=yellow!20, rounded corners=2pt, inner sep=2pt] {
            Recon/Enum
        };
}

% ============================================================================
% QUALITY OF SERVICE (QoS) INDICATORS
% ============================================================================

% Connection with QoS class indicator
% Usage: \drawQoSConnection{from}{to}{qos_class}{label}
% qos_class: 0=best-effort, 1=bronze, 2=silver, 3=gold, 4=platinum
\newcommand{\drawQoSConnection}[4]{
    \ifcase#3
        \def\qosColor{gray!60}
        \def\qosLabel{Best Effort}
        \def\qosWidth{0.8pt}
    \or
        \def\qosColor{brown!60}
        \def\qosLabel{Bronze}
        \def\qosWidth{1pt}
    \or
        \def\qosColor{gray!50}
        \def\qosLabel{Silver}
        \def\qosWidth{1.3pt}
    \or
        \def\qosColor{yellow!70!orange}
        \def\qosLabel{Gold}
        \def\qosWidth{1.6pt}
    \or
        \def\qosColor{cyan!60!blue}
        \def\qosLabel{Platinum}
        \def\qosWidth{2pt}
    \fi
    \draw[draw=\qosColor!80!black, line width=\qosWidth, -{Stealth[length=3mm]}]
        (#1) -- (#2)
        node[midway, above, font=\tiny, fill=\qosColor!20, rounded corners=2pt, inner sep=2pt] {
            #4 [\qosLabel]
        };
}

% Connection with latency indicator
% Usage: \drawLatencyConnection{from}{to}{latency_ms}{label}
\newcommand{\drawLatencyConnection}[4]{
    \pgfmathsetmacro{\lat}{#3}
    \ifnum\lat<50
        \def\latColor{green}
        \def\latLabel{Low}
    \else\ifnum\lat<150
        \def\latColor{yellow}
        \def\latLabel{Medium}
    \else
        \def\latColor{red}
        \def\latLabel{High}
    \fi\fi
    \draw[draw=\latColor!70!black, line width=1pt, -{Stealth[length=3mm]}]
        (#1) -- (#2)
        node[midway, above, font=\tiny, fill=\latColor!15, rounded corners=2pt, inner sep=2pt] {
            #4: #3ms (\latLabel)
        };
}

% Connection with packet loss indicator
% Usage: \drawPacketLossConnection{from}{to}{loss_percent}{label}
\newcommand{\drawPacketLossConnection}[4]{
    \pgfmathsetmacro{\loss}{#3}
    \ifnum\loss<1
        \def\lossColor{green}
        \def\lossStyle{solid}
    \else\ifnum\loss<5
        \def\lossColor{yellow}
        \def\lossStyle{densely dashed}
    \else
        \def\lossColor{red}
        \def\lossStyle{densely dotted}
    \fi\fi
    \draw[draw=\lossColor!70!black, line width=1pt, \lossStyle, -{Stealth[length=3mm]}]
        (#1) -- (#2)
        node[midway, above, font=\tiny, fill=\lossColor!15, rounded corners=2pt, inner sep=2pt] {
            #4: #3\% loss
        };
}

% Connection with jitter indicator
% Usage: \drawJitterConnection{from}{to}{jitter_ms}{label}
\newcommand{\drawJitterConnection}[4]{
    \pgfmathsetmacro{\jit}{#3}
    \pgfmathsetmacro{\amplitude}{0.3 + #3/10}
    \draw[draw=orange!70!black, line width=1pt, decorate,
          decoration={snake, amplitude=\amplitude mm, segment length=4mm}, -{Stealth[length=3mm]}]
        (#1) -- (#2)
        node[midway, above, font=\tiny, fill=orange!15, rounded corners=2pt, inner sep=2pt] {
            #4: Jitter #3ms
        };
}

% Comprehensive QoS metrics display
% Usage: \drawFullQoSConnection{from}{to}{latency}{loss}{jitter}{bandwidth}{label}
\newcommand{\drawFullQoSConnection}[7]{
    \draw[normal conn, line width=1.2pt, -{Stealth[length=3mm]}]
        (#1) -- (#2)
        node[midway, above, font=\tiny, fill=blue!10, rounded corners=2pt, inner sep=2pt, align=center] {
            \textbf{#7} \\
            \texttt{Lat:#3ms | Loss:#4\% | Jit:#5ms | BW:#6M}
        };
}

% Priority queue indicator
% Usage: \drawPriorityConnection{from}{to}{priority}{label}
% priority: 0=low, 1=medium, 2=high, 3=critical
\newcommand{\drawPriorityConnection}[4]{
    \ifcase#3
        \def\priColor{gray}
        \def\priLabel{Low}
        \def\priWidth{0.8pt}
    \or
        \def\priColor{blue}
        \def\priLabel{Medium}
        \def\priWidth{1.2pt}
    \or
        \def\priColor{orange}
        \def\priLabel{High}
        \def\priWidth{1.6pt}
    \or
        \def\priColor{red}
        \def\priLabel{Critical}
        \def\priWidth{2pt}
    \fi
    \draw[draw=\priColor!70!black, line width=\priWidth, double, -{Stealth[length=3mm]}]
        (#1) -- (#2)
        node[midway, above, font=\tiny\bfseries, fill=\priColor!20, rounded corners=2pt, inner sep=2pt] {
            #4 [P#3-\priLabel]
        };
}

% ============================================================================
% CONNECTION STATISTICS AND METRICS
% ============================================================================

% Draw connection statistics summary
% Usage: \drawConnectionSummary{x}{y}
\newcommand{\drawConnectionSummary}[2]{
    \node[legend box, anchor=north west] at (#1,#2) {
        \begin{tabular}{lr}
            \textbf{Connections} & \\
            Normal & \theconncount \\
            Encrypted & 0 \\
            Suspicious & 0 \\
            Attacks & 0 \\
        \end{tabular}
    };
}

% ============================================================================
% NETWORK TOPOLOGY HELPER CONNECTIONS
% ============================================================================

% Ring topology connection (curved to show circular layout)
% Usage: \drawRingConnection{from}{to}{label}
\newcommand{\drawRingConnection}[3]{
    \draw[draw=blue!60!black, line width=1.2pt, -{Stealth[length=3mm]}, bend right=15]
        (#1) to node[midway, above, font=\tiny, fill=blue!10] {#3} (#2);
}

% Mesh topology - full mesh indicator
% Usage: \drawMeshConnection{from}{to}{label}
\newcommand{\drawMeshConnection}[3]{
    \draw[draw=green!60!black, line width=0.8pt, -{Stealth[length=2mm]}, <->]
        (#1) -- (#2)
        node[midway, above, font=\tiny, fill=green!10] {#3};
}

% Star topology - hub connection
% Usage: \drawStarHubConnection{hub}{spoke}{label}
\newcommand{\drawStarHubConnection}[3]{
    \draw[draw=orange!60!black, line width=1pt, -{Stealth[length=3mm]}]
        (#1) -- (#2)
        node[pos=0.7, above, font=\tiny, fill=orange!10] {#3};
}

% Bus topology connection
% Usage: \drawBusConnection{node}{bus_line}{label}
\newcommand{\drawBusConnection}[3]{
    \draw[draw=purple!60!black, line width=0.8pt, -{Stealth[length=2mm]}]
        (#1) -- (#2)
        node[midway, right, font=\tiny, fill=purple!10] {#3};
}

% Tree/Hierarchical topology connection
% Usage: \drawHierarchicalConnection{parent}{child}{level}{label}
\newcommand{\drawHierarchicalConnection}[4]{
    \pgfmathsetmacro{\lwidth}{2 - 0.3*#3}
    \draw[draw=teal!60!black, line width=\lwidth pt, -{Stealth[length=3mm]}]
        (#1) -- (#2)
        node[midway, right, font=\tiny, fill=teal!10] {#4 (L#3)};
}

% Redundant path indicator (for high availability)
% Usage: \drawRedundantPath{from}{to}{primary}{label}
% primary: true/false
\newcommand{\drawRedundantPath}[4]{
    \ifthenelse{\equal{#3}{true}}{
        \def\pathColor{blue!70!black}
        \def\pathStyle{solid}
        \def\pathWidth{1.5pt}
        \def\pathLabel{Primary}
    }{
        \def\pathColor{gray!60}
        \def\pathStyle{dashed}
        \def\pathWidth{1pt}
        \def\pathLabel{Backup}
    }
    \draw[draw=\pathColor, line width=\pathWidth, \pathStyle, -{Stealth[length=3mm]}]
        (#1) -- (#2)
        node[midway, above, font=\tiny, fill=white] {#4 (\pathLabel)};
}

% Load-balanced connection (shows multiple equal paths)
% Usage: \drawLoadBalancedConnection{from}{to}{path_count}{label}
\newcommand{\drawLoadBalancedConnection}[4]{
    \foreach \i in {1,...,#3} {
        \pgfmathsetmacro{\offset}{-1 + 2*(\i-1)/(#3-1)}
        \draw[draw=green!70!black, line width=0.8pt, -{Stealth[length=2mm]}]
            ([yshift=\offset pt]#1) -- ([yshift=\offset pt]#2);
    }
    \node[above=3pt, font=\tiny\bfseries, fill=green!20, rounded corners=2pt, inner sep=2pt]
        at ($(#1)!0.5!(#2)$) {#4 (LB:#3)};
}

% Failover connection pair
% Usage: \drawFailoverPair{from}{to}{active}{label}
\newcommand{\drawFailoverPair}[4]{
    % Active path
    \draw[draw=green!70!black, line width=1.5pt, -{Stealth[length=3mm]}]
        ([yshift=2pt]#1) -- ([yshift=2pt]#2)
        node[pos=0.3, above, font=\tiny, fill=green!15] {Active};
    % Standby path
    \draw[draw=orange!60!black, line width=0.8pt, dashed, -{Stealth[length=2mm]}]
        ([yshift=-2pt]#1) -- ([yshift=-2pt]#2)
        node[pos=0.7, below, font=\tiny, fill=orange!15] {Standby};
    \node[above, font=\tiny\bfseries] at ($(#1)!0.5!(#2)$) {#4};
}

% ============================================================================
% CLOUD PROVIDER SPECIFIC CONNECTIONS
% ============================================================================

% AWS VPC Peering connection
% Usage: \drawAWSVPCPeering{vpc1}{vpc2}{label}
\newcommand{\drawAWSVPCPeering}[3]{
    \draw[draw=orange!70!black, line width=1.5pt, <->, double distance=1pt]
        (#1) -- (#2)
        node[midway, above, font=\tiny, fill=orange!15, rounded corners=2pt, inner sep=2pt] {
            \textbf{AWS} VPC Peering: #3
        };
}

% AWS Direct Connect
% Usage: \drawAWSDirectConnect{onprem}{aws_region}{speed}{label}
\newcommand{\drawAWSDirectConnect}[4]{
    \draw[draw=orange!80!red, line width=2pt, -{Stealth[length=4mm]}, double distance=2pt]
        (#1) -- (#2)
        node[midway, above, font=\tiny\bfseries, fill=orange!20, rounded corners=3pt, inner sep=3pt] {
            \textcolor{orange!80!black}{$\star$} Direct Connect: #3 | #4
        };
}

% Azure ExpressRoute
% Usage: \drawAzureExpressRoute{onprem}{azure_region}{circuit}{label}
\newcommand{\drawAzureExpressRoute}[4]{
    \draw[draw=blue!70!cyan, line width=2pt, -{Stealth[length=4mm]}, double distance=2pt]
        (#1) -- (#2)
        node[midway, above, font=\tiny\bfseries, fill=blue!15, rounded corners=3pt, inner sep=3pt] {
            \textcolor{blue!70!black}{$\blacksquare$} ExpressRoute: #3 | #4
        };
}

% GCP Interconnect
% Usage: \drawGCPInterconnect{onprem}{gcp_region}{type}{label}
\newcommand{\drawGCPInterconnect}[4]{
    \draw[draw=red!60!yellow, line width=2pt, -{Stealth[length=4mm]}, double distance=2pt]
        (#1) -- (#2)
        node[midway, above, font=\tiny\bfseries, fill=yellow!20, rounded corners=3pt, inner sep=3pt] {
            \textcolor{red!70!black}{$\bullet$} GCP #3: #4
        };
}

% Multi-cloud connection
% Usage: \drawMultiCloudConnection{cloud1}{cloud2}{provider1}{provider2}{label}
\newcommand{\drawMultiCloudConnection}[5]{
    \draw[draw=purple!70!black, line width=1.5pt, <->, densely dashed]
        (#1) -- (#2)
        node[midway, above, font=\tiny, fill=purple!15, rounded corners=2pt, inner sep=2pt, align=center] {
            #5 \\
            \texttt{#3 ↔ #4}
        };
}

% Cloud NAT Gateway connection
% Usage: \drawCloudNATConnection{private}{nat}{public}{label}
\newcommand{\drawCloudNATConnection}[4]{
    % Private to NAT
    \draw[draw=blue!60!black, line width=1pt, -{Stealth[length=3mm]}]
        (#1) -- (#2)
        node[midway, above, font=\tiny] {Private};
    % NAT to Public
    \draw[draw=green!60!black, line width=1pt, -{Stealth[length=3mm]}]
        (#2) -- (#3)
        node[midway, above, font=\tiny] {Public};
    % NAT box highlight
    \node[draw=orange!70!black, fill=orange!15, rounded corners=3pt, inner sep=5pt] at (#2) {NAT};
    \node[above=10pt of #2, font=\tiny] {#4};
}

% CDN edge connection
% Usage: \drawCDNConnection{origin}{edge}{region}{label}
\newcommand{\drawCDNConnection}[4]{
    \draw[draw=cyan!70!black, line width=1.2pt, -{Stealth[length=3mm]}, decorate,
          decoration={markings, mark=between positions 0.2 and 0.8 step 0.3 with {
              \node[circle, fill=cyan!60!black, inner sep=1pt] {};
          }}]
        (#1) -- (#2)
        node[midway, above, font=\tiny, fill=cyan!15, rounded corners=2pt, inner sep=2pt] {
            CDN Edge (#3): #4
        };
}

% Serverless/Lambda function invocation
% Usage: \drawServerlessInvocation{trigger}{function}{label}
\newcommand{\drawServerlessInvocation}[3]{
    \draw[draw=green!70!black, line width=1pt, -{Stealth[length=3mm]}, densely dotted]
        (#1) -- (#2)
        [postaction={
            decorate,
            decoration={
                markings,
                mark=at position 0.5 with {
                    \node[font=\tiny, fill=green!20, rounded corners=2pt, inner sep=2pt] {⚡ λ};
                }
            }
        }]
        node[pos=0.7, above, font=\tiny] {#3};
}

% Container orchestration (K8s) service mesh
% Usage: \drawServiceMeshConnection{service1}{service2}{mesh_type}{label}
\newcommand{\drawServiceMeshConnection}[4]{
    \draw[draw=purple!70!black, line width=1pt, <->, decorate,
          decoration={snake, amplitude=0.4mm, segment length=4mm}]
        (#1) -- (#2)
        node[midway, above, font=\tiny, fill=purple!15, rounded corners=2pt, inner sep=2pt] {
            #3 Mesh: #4
        };
}

% ============================================================================
% TIME-SERIES CONNECTION METRICS
% ============================================================================

% Connection with historical trend indicator
% Usage: \drawTrendConnection{from}{to}{current_value}{trend}{label}
% trend: up, down, stable
\newcommand{\drawTrendConnection}[5]{
    \ifthenelse{\equal{#4}{up}}{
        \def\trendColor{green}
        \def\trendSymbol{$\uparrow$}
    }{}
    \ifthenelse{\equal{#4}{down}}{
        \def\trendColor{red}
        \def\trendSymbol{$\downarrow$}
    }{}
    \ifthenelse{\equal{#4}{stable}}{
        \def\trendColor{blue}
        \def\trendSymbol{$\rightarrow$}
    }{}
    \draw[draw=\trendColor!70!black, line width=1.2pt, -{Stealth[length=3mm]}]
        (#1) -- (#2)
        node[midway, above, font=\tiny, fill=\trendColor!15, rounded corners=2pt, inner sep=2pt] {
            #5: #3 \trendSymbol
        };
}

% Connection with time-based metric history
% Usage: \drawMetricHistoryConnection{from}{to}{metric_values}{label}
% metric_values: comma-separated list like "10,15,20,25,30"
\newcommand{\drawMetricHistoryConnection}[4]{
    \draw[normal conn, line width=1pt, -{Stealth[length=3mm]}]
        (#1) -- (#2)
        node[midway, above, font=\tiny, fill=blue!10, rounded corners=2pt, inner sep=2pt] {
            #4: #3
        };
}

% Connection with peak/average/current metrics
% Usage: \drawPeakAverageConnection{from}{to}{peak}{avg}{current}{label}
\newcommand{\drawPeakAverageConnection}[6]{
    \draw[normal conn, line width=1.2pt, -{Stealth[length=3mm]}]
        (#1) -- (#2)
        node[midway, above, font=\tiny, fill=cyan!10, rounded corners=2pt, inner sep=2pt, align=center] {
            \textbf{#6} \\
            \texttt{Peak:#3 | Avg:#4 | Now:#5}
        };
}

% Connection with SLA compliance indicator
% Usage: \drawSLAConnection{from}{to}{sla_percent}{threshold}{label}
\newcommand{\drawSLAConnection}[5]{
    \pgfmathsetmacro{\sla}{#3}
    \pgfmathsetmacro{\thresh}{#4}
    \ifdim\sla pt<\thresh pt
        \def\slaColor{red}
        \def\slaStatus{VIOLATION}
    \else
        \def\slaColor{green}
        \def\slaStatus{OK}
    \fi
    \draw[draw=\slaColor!70!black, line width=1.5pt, -{Stealth[length=3mm]}]
        (#1) -- (#2)
        node[midway, above, font=\tiny\bfseries, fill=\slaColor!20, rounded corners=2pt, inner sep=2pt] {
            #5: SLA #3\% [\slaStatus]
        };
}

% ============================================================================
% SECURITY FRAMEWORK CONNECTIONS
% ============================================================================

% Zero Trust Architecture connection
% Usage: \drawZeroTrustConnection{from}{to}{trust_score}{label}
% trust_score: 0-100
\newcommand{\drawZeroTrustConnection}[4]{
    \pgfmathsetmacro{\trust}{#3}
    \ifnum\trust<50
        \def\ztColor{red}
        \def\ztLabel{Low Trust}
    \else\ifnum\trust<80
        \def\ztColor{yellow}
        \def\ztLabel{Medium Trust}
    \else
        \def\ztColor{green}
        \def\ztLabel{High Trust}
    \fi\fi
    \draw[draw=\ztColor!70!black, line width=1.5pt, -{Stealth[length=3mm]}, densely dashed]
        (#1) -- (#2)
        node[midway, above, font=\tiny, fill=\ztColor!15, rounded corners=2pt, inner sep=2pt] {
            #4: Trust=#3 (\ztLabel)
        };
}

% SASE (Secure Access Service Edge) connection
% Usage: \drawSASEConnection{from}{to}{security_score}{label}
\newcommand{\drawSASEConnection}[4]{
    \draw[draw=purple!70!black, line width=1.8pt, -{Stealth[length=4mm]}, double distance=1.5pt]
        (#1) -- (#2)
        node[midway, above, font=\tiny\bfseries, fill=purple!15, rounded corners=3pt, inner sep=3pt] {
            \textcolor{purple!80!black}{$\blacksquare$} SASE: #4 (Sec:#3)
        };
}

% Microsegmentation boundary
% Usage: \drawMicrosegmentConnection{from}{to}{segment_id}{policy}{label}
\newcommand{\drawMicrosegmentConnection}[5]{
    \draw[draw=teal!70!black, line width=1pt, densely dotted, -{Stealth[length=3mm]}]
        (#1) -- (#2)
        node[midway, above, font=\tiny, fill=teal!15, rounded corners=2pt, inner sep=2pt] {
            Seg#3: #4 | #5
        };
}

% Identity-based connection (IAM/SSO)
% Usage: \drawIdentityConnection{from}{to}{identity}{role}{label}
\newcommand{\drawIdentityConnection}[5]{
    \draw[draw=blue!70!cyan, line width=1.2pt, -{Stealth[length=3mm]}]
        (#1) -- (#2)
        node[midway, above, font=\tiny, fill=cyan!15, rounded corners=2pt, inner sep=2pt, align=center] {
            #5 \\
            \texttt{ID:#3 | Role:#4}
        };
}

% ============================================================================
% SD-WAN AND MPLS CONNECTIONS
% ============================================================================

% SD-WAN connection with path selection
% Usage: \drawSDWANConnection{from}{to}{path_type}{quality}{label}
% path_type: internet, mpls, lte, broadband
\newcommand{\drawSDWANConnection}[5]{
    \ifthenelse{\equal{#3}{internet}}{
        \def\sdwanColor{blue}
    }{}
    \ifthenelse{\equal{#3}{mpls}}{
        \def\sdwanColor{green}
    }{}
    \ifthenelse{\equal{#3}{lte}}{
        \def\sdwanColor{orange}
    }{}
    \ifthenelse{\equal{#3}{broadband}}{
        \def\sdwanColor{purple}
    }{}
    \draw[draw=\sdwanColor!70!black, line width=1.5pt, -{Stealth[length=4mm]}, double distance=1pt]
        (#1) -- (#2)
        node[midway, above, font=\tiny\bfseries, fill=\sdwanColor!15, rounded corners=2pt, inner sep=2pt] {
            SD-WAN (#3): #5 | Q:#4
        };
}

% MPLS circuit with CoS marking
% Usage: \drawMPLSConnection{from}{to}{cos}{circuit_id}{label}
% cos: EF, AF, BE (Expedited Forwarding, Assured Forwarding, Best Effort)
\newcommand{\drawMPLSConnection}[5]{
    \ifthenelse{\equal{#3}{EF}}{
        \def\mplsColor{red}
        \def\mplsWidth{2pt}
    }{}
    \ifthenelse{\equal{#3}{AF}}{
        \def\mplsColor{orange}
        \def\mplsWidth{1.5pt}
    }{}
    \ifthenelse{\equal{#3}{BE}}{
        \def\mplsColor{gray}
        \def\mplsWidth{1pt}
    }{}
    \draw[draw=\mplsColor!70!black, line width=\mplsWidth, -{Stealth[length=3mm]}, double distance=1.5pt]
        (#1) -- (#2)
        node[midway, above, font=\tiny, fill=\mplsColor!15, rounded corners=2pt, inner sep=2pt] {
            MPLS-#4: #5 (CoS:#3)
        };
}

% Multi-path SD-WAN with link aggregation
% Usage: \drawMultiPathSDWAN{from}{to}{path_count}{active_paths}{label}
\newcommand{\drawMultiPathSDWAN}[5]{
    \foreach \i in {1,...,#4} {
        \pgfmathsetmacro{\offset}{-1.5 + 3*(\i-1)/(#4-1)}
        \draw[draw=blue!70!black, line width=1pt, -{Stealth[length=2mm]}]
            ([yshift=\offset pt]#1) -- ([yshift=\offset pt]#2);
    }
    \node[above=5pt, font=\tiny\bfseries, fill=blue!20, rounded corners=2pt, inner sep=2pt]
        at ($(#1)!0.5!(#2)$) {SD-WAN: #5 (#4/#3 active)};
}

% ============================================================================
% MONITORING AND ALERTING VISUALIZATIONS
% ============================================================================

% Connection with alert indicator
% Usage: \drawAlertConnection{from}{to}{alert_level}{message}{label}
% alert_level: 0=info, 1=warning, 2=critical
\newcommand{\drawAlertConnection}[5]{
    \ifcase#3
        \def\alertColor{blue}
        \def\alertLabel{INFO}
        \def\alertSymbol{ℹ}
    \or
        \def\alertColor{orange}
        \def\alertLabel{WARNING}
        \def\alertSymbol{⚠}
    \or
        \def\alertColor{red}
        \def\alertLabel{CRITICAL}
        \def\alertSymbol{✖}
    \fi
    \draw[draw=\alertColor!70!black, line width=1.5pt, -{Stealth[length=3mm]}]
        (#1) -- (#2)
        [postaction={
            decorate,
            decoration={
                markings,
                mark=at position 0.5 with {
                    \node[fill=\alertColor!30, circle, inner sep=3pt] {\alertSymbol};
                }
            }
        }]
        node[pos=0.3, above, font=\tiny\bfseries, fill=\alertColor!20] {\alertLabel}
        node[pos=0.7, below, font=\tiny] {#5: #4};
}

% SNMP trap visualization
% Usage: \drawSNMPTrap{from}{to}{trap_type}{severity}{label}
\newcommand{\drawSNMPTrap}[5]{
    \draw[draw=yellow!70!orange, line width=1pt, -{Stealth[length=3mm]}, densely dashed]
        (#1) -- (#2)
        node[midway, above, font=\tiny, fill=yellow!20, rounded corners=2pt, inner sep=2pt] {
            SNMP Trap: #3 (Sev:#4) | #5
        };
}

% NetFlow/sFlow collector connection
% Usage: \drawNetFlowConnection{from}{to}{flow_rate}{label}
\newcommand{\drawNetFlowConnection}[4]{
    \draw[draw=cyan!70!black, line width=1pt, -{Stealth[length=3mm]}]
        (#1) -- (#2)
        [postaction={
            decorate,
            decoration={
                markings,
                mark=between positions 0.2 and 0.8 step 0.15 with {
                    \node[circle, fill=cyan!60!black, inner sep=0.5pt] {};
                }
            }
        }]
        node[midway, above, font=\tiny, fill=cyan!15, rounded corners=2pt, inner sep=2pt] {
            NetFlow: #4 (#3 fps)
        };
}

% Syslog connection
% Usage: \drawSyslogConnection{from}{to}{facility}{priority}{label}
\newcommand{\drawSyslogConnection}[5]{
    \draw[draw=purple!60!black, line width=0.8pt, -{Stealth[length=2mm]}, densely dotted]
        (#1) -- (#2)
        node[midway, above, font=\tiny, fill=purple!15, rounded corners=2pt, inner sep=2pt] {
            Syslog F#3.P#4: #5
        };
}

% ============================================================================
% NETWORK AUTOMATION AND ORCHESTRATION
% ============================================================================

% Ansible/Terraform provisioning connection
% Usage: \drawProvisioningConnection{from}{to}{tool}{action}{label}
\newcommand{\drawProvisioningConnection}[5]{
    \ifthenelse{\equal{#3}{ansible}}{
        \def\provColor{red}
    }{}
    \ifthenelse{\equal{#3}{terraform}}{
        \def\provColor{purple}
    }{}
    \ifthenelse{\equal{#3}{puppet}}{
        \def\provColor{orange}
    }{}
    \draw[draw=\provColor!70!black, line width=1.2pt, -{Stealth[length=3mm]}, densely dashed]
        (#1) -- (#2)
        node[midway, above, font=\tiny, fill=\provColor!15, rounded corners=2pt, inner sep=2pt] {
            \textbf{#3}: #4 | #5
        };
}

% API management connection
% Usage: \drawAPIConnection{from}{to}{api_version}{method}{label}
\newcommand{\drawAPIConnection}[5]{
    \draw[draw=green!70!black, line width=1pt, -{Stealth[length=3mm]}]
        (#1) -- (#2)
        node[midway, above, font=\tiny\ttfamily, fill=green!15, rounded corners=2pt, inner sep=2pt] {
            API #3: #4 | #5
        };
}

% Webhook notification
% Usage: \drawWebhookConnection{from}{to}{event}{label}
\newcommand{\drawWebhookConnection}[4]{
    \draw[draw=orange!70!black, line width=1pt, -{Stealth[length=3mm]}, decorate,
          decoration={zigzag, amplitude=0.5mm, segment length=2mm}]
        (#1) -- (#2)
        node[midway, above, font=\tiny, fill=orange!15, rounded corners=2pt, inner sep=2pt] {
            Webhook: #3 | #4
        };
}

% ============================================================================
% COMPLIANCE AND AUDIT TRAIL
% ============================================================================

% Compliance-checked connection
% Usage: \drawComplianceConnection{from}{to}{framework}{status}{label}
% framework: PCI-DSS, HIPAA, SOC2, GDPR, etc.
\newcommand{\drawComplianceConnection}[5]{
    \ifthenelse{\equal{#4}{compliant}}{
        \def\compColor{green}
        \def\compSymbol{✓}
    }{
        \def\compColor{red}
        \def\compSymbol{✗}
    }
    \draw[draw=\compColor!70!black, line width=1.2pt, -{Stealth[length=3mm]}]
        (#1) -- (#2)
        node[midway, above, font=\tiny, fill=\compColor!15, rounded corners=2pt, inner sep=2pt] {
            #3: #5 \compSymbol
        };
}

% Audit trail connection
% Usage: \drawAuditConnection{from}{to}{timestamp}{action}{user}{label}
\newcommand{\drawAuditConnection}[6]{
    \draw[draw=blue!60!black, line width=1pt, -{Stealth[length=3mm]}, densely dotted]
        (#1) -- (#2)
        node[midway, above, font=\tiny, fill=blue!10, rounded corners=2pt, inner sep=2pt, align=center] {
            #6 \\
            \texttt{#3 | #4 | User:#5}
        };
}

% Data sovereignty boundary
% Usage: \drawDataSovereigntyConnection{from}{to}{region}{regulation}{label}
\newcommand{\drawDataSovereigntyConnection}[5]{
    \draw[draw=purple!70!black, line width=1.5pt, -{Stealth[length=3mm]}, densely dashed]
        (#1) -- (#2)
        node[midway, above, font=\tiny\bfseries, fill=purple!15, rounded corners=2pt, inner sep=2pt] {
            Data: #3 (#4) | #5
        };
}

% ============================================================================
% IOT AND EDGE COMPUTING CONNECTIONS
% ============================================================================

% IoT device connection (low-power, constrained)
% Usage: \drawIoTConnection{from}{to}{protocol}{power}{label}
% protocol: MQTT, CoAP, LoRaWAN, Zigbee, BLE
\newcommand{\drawIoTConnection}[5]{
    \ifthenelse{\equal{#3}{MQTT}}{
        \def\iotColor{teal}
    }{}
    \ifthenelse{\equal{#3}{CoAP}}{
        \def\iotColor{cyan}
    }{}
    \ifthenelse{\equal{#3}{LoRaWAN}}{
        \def\iotColor{green}
    }{}
    \ifthenelse{\equal{#3}{Zigbee}}{
        \def\iotColor{orange}
    }{}
    \ifthenelse{\equal{#3}{BLE}}{
        \def\iotColor{blue}
    }{}
    \draw[draw=\iotColor!70!black, line width=0.6pt, densely dotted, -{Stealth[length=2mm]}]
        (#1) -- (#2)
        node[midway, above, font=\tiny, fill=\iotColor!15, rounded corners=2pt, inner sep=1.5pt] {
            IoT: #3 | #4 | #5
        };
}

% Edge computing gateway connection
% Usage: \drawEdgeGatewayConnection{from}{to}{processing_power}{label}
\newcommand{\drawEdgeGatewayConnection}[4]{
    \draw[draw=purple!70!black, line width=1.5pt, -{Stealth[length=3mm]}, double distance=1pt]
        (#1) -- (#2)
        node[midway, above, font=\tiny\bfseries, fill=purple!20, rounded corners=2pt, inner sep=2pt] {
            Edge Gateway: #3 | #4
        };
}

% Fog computing layer connection
% Usage: \drawFogComputingConnection{from}{to}{latency}{label}
\newcommand{\drawFogComputingConnection}[4]{
    \draw[draw=gray!70!black, line width=1.2pt, -{Stealth[length=3mm]}, densely dashed]
        (#1) -- (#2)
        node[midway, above, font=\tiny, fill=gray!20, rounded corners=2pt, inner sep=2pt] {
            Fog Layer: #3ms | #4
        };
}

% Sensor network mesh connection
% Usage: \drawSensorMeshConnection{from}{to}{sensor_type}{label}
\newcommand{\drawSensorMeshConnection}[4]{
    \draw[draw=green!60!black, line width=0.5pt, decorate,
          decoration={snake, amplitude=0.3mm, segment length=2mm}, -{Stealth[length=2mm]}]
        (#1) -- (#2)
        node[midway, above, font=\tiny, fill=green!15, rounded corners=1pt, inner sep=1pt] {
            Sensor: #3 | #4
        };
}

% ============================================================================
% 5G AND CELLULAR NETWORK CONNECTIONS
% ============================================================================

% 5G NR connection
% Usage: \draw5GConnection{from}{to}{slice_id}{bandwidth}{label}
\newcommand{\draw5GConnection}[5]{
    \draw[draw=red!70!black, line width=2pt, -{Stealth[length=4mm]}, double distance=1.5pt]
        (#1) -- (#2)
        node[midway, above, font=\tiny\bfseries, fill=red!15, rounded corners=3pt, inner sep=3pt] {
            \textcolor{red!80!black}{5G} Slice:#3 | #4 | #5
        };
}

% LTE/4G connection
% Usage: \drawLTEConnection{from}{to}{band}{label}
\newcommand{\drawLTEConnection}[4]{
    \draw[draw=blue!70!black, line width=1.5pt, -{Stealth[length=3mm]}, double]
        (#1) -- (#2)
        node[midway, above, font=\tiny, fill=blue!15, rounded corners=2pt, inner sep=2pt] {
            LTE Band #3: #4
        };
}

% Network slicing visualization
% Usage: \drawNetworkSlice{from}{to}{slice_type}{priority}{label}
% slice_type: eMBB, URLLC, mMTC
\newcommand{\drawNetworkSlice}[5]{
    \ifthenelse{\equal{#3}{eMBB}}{
        \def\sliceColor{blue}
        \def\sliceLabel{Enhanced Mobile Broadband}
    }{}
    \ifthenelse{\equal{#3}{URLLC}}{
        \def\sliceColor{red}
        \def\sliceLabel{Ultra-Reliable Low-Latency}
    }{}
    \ifthenelse{\equal{#3}{mMTC}}{
        \def\sliceColor{green}
        \def\sliceLabel{Massive Machine Type Comms}
    }{}
    \draw[draw=\sliceColor!70!black, line width=1.8pt, -{Stealth[length=4mm]}, densely dashed]
        (#1) -- (#2)
        node[midway, above, font=\tiny, fill=\sliceColor!20, rounded corners=2pt, inner sep=2pt, align=center] {
            #3 Slice \\
            P#4: #5
        };
}

% ============================================================================
% BLOCKCHAIN AND DISTRIBUTED LEDGER CONNECTIONS
% ============================================================================

% Blockchain consensus connection
% Usage: \drawBlockchainConsensus{from}{to}{consensus_type}{label}
% consensus_type: PoW, PoS, PBFT, Raft
\newcommand{\drawBlockchainConsensus}[4]{
    \ifthenelse{\equal{#3}{PoW}}{
        \def\chainColor{yellow!80!orange}
    }{}
    \ifthenelse{\equal{#3}{PoS}}{
        \def\chainColor{green!70!black}
    }{}
    \ifthenelse{\equal{#3}{PBFT}}{
        \def\chainColor{blue!70!black}
    }{}
    \ifthenelse{\equal{#3}{Raft}}{
        \def\chainColor{purple!70!black}
    }{}
    \draw[draw=\chainColor, line width=1.5pt, <->, densely dashed]
        (#1) -- (#2)
        node[midway, above, font=\tiny, fill=\chainColor!20, rounded corners=2pt, inner sep=2pt] {
            Consensus: #3 | #4
        };
}

% Smart contract execution
% Usage: \drawSmartContractConnection{from}{to}{contract_id}{gas}{label}
\newcommand{\drawSmartContractConnection}[5]{
    \draw[draw=orange!70!black, line width=1.2pt, -{Stealth[length=3mm]}, decorate,
          decoration={zigzag, amplitude=0.5mm, segment length=3mm}]
        (#1) -- (#2)
        node[midway, above, font=\tiny, fill=orange!20, rounded corners=2pt, inner sep=2pt] {
            Contract #3: Gas #4 | #5
        };
}

% Distributed ledger replication
% Usage: \drawLedgerReplication{from}{to}{block_height}{label}
\newcommand{\drawLedgerReplication}[4]{
    \draw[draw=teal!70!black, line width=1pt, <->, densely dotted]
        (#1) -- (#2)
        [postaction={
            decorate,
            decoration={
                markings,
                mark=between positions 0.2 and 0.8 step 0.2 with {
                    \node[rectangle, fill=teal!60!black, inner sep=1pt] {};
                }
            }
        }]
        node[midway, above, font=\tiny, fill=teal!15, rounded corners=2pt, inner sep=2pt] {
            Ledger Sync: Block #3 | #4
        };
}

% ============================================================================
% INDUSTRIAL/SCADA/OT NETWORK CONNECTIONS
% ============================================================================

% Modbus TCP/RTU connection
% Usage: \drawModbusConnection{from}{to}{variant}{register}{label}
\newcommand{\drawModbusConnection}[5]{
    \draw[draw=brown!70!black, line width=1.2pt, -{Stealth[length=3mm]}]
        (#1) -- (#2)
        node[midway, above, font=\tiny\ttfamily, fill=brown!20, rounded corners=2pt, inner sep=2pt] {
            Modbus #3: Reg #4 | #5
        };
}

% DNP3 SCADA protocol
% Usage: \drawDNP3Connection{from}{to}{master_slave}{label}
\newcommand{\drawDNP3Connection}[4]{
    \draw[draw=red!60!black, line width=1.2pt, -{Stealth[length=3mm]}, densely dashed]
        (#1) -- (#2)
        node[midway, above, font=\tiny\ttfamily, fill=red!20, rounded corners=2pt, inner sep=2pt] {
            DNP3: #3 | #4
        };
}

% Industrial Ethernet (Profinet, EtherNet/IP)
% Usage: \drawIndustrialEthernet{from}{to}{protocol}{cycle_time}{label}
\newcommand{\drawIndustrialEthernet}[5]{
    \draw[draw=blue!60!black, line width=1.5pt, -{Stealth[length=3mm]}, double]
        (#1) -- (#2)
        node[midway, above, font=\tiny, fill=blue!20, rounded corners=2pt, inner sep=2pt] {
            #3: Cycle #4ms | #5
        };
}

% OPC UA connection
% Usage: \drawOPCUAConnection{from}{to}{security_mode}{label}
\newcommand{\drawOPCUAConnection}[4]{
    \draw[draw=green!70!black, line width=1.2pt, -{Stealth[length=3mm]}]
        (#1) -- (#2)
        node[midway, above, font=\tiny, fill=green!20, rounded corners=2pt, inner sep=2pt] {
            OPC UA: #3 | #4
        };
}

% Safety instrumented system (SIS) connection
% Usage: \drawSISConnection{from}{to}{sil_level}{label}
\newcommand{\drawSISConnection}[4]{
    \draw[draw=red!80!black, line width=2pt, -{Stealth[length=4mm]}, double distance=2pt]
        (#1) -- (#2)
        node[midway, above, font=\tiny\bfseries, fill=red!25, rounded corners=3pt, inner sep=3pt] {
            \textcolor{red!90!black}{⚠} SIS: SIL #3 | #4
        };
}

% ============================================================================
% ADVANCED SECURITY APPLIANCE CONNECTIONS
% ============================================================================

% Web Application Firewall (WAF)
% Usage: \drawWAFConnection{from}{to}{rule_count}{blocked}{label}
\newcommand{\drawWAFConnection}[5]{
    \draw[draw=orange!70!black, line width=1.5pt, -{Stealth[length=3mm]}, double distance=1pt]
        (#1) -- (#2)
        node[midway, above, font=\tiny, fill=orange!20, rounded corners=2pt, inner sep=2pt] {
            WAF: Rules:#3 | Blocked:#4 | #5
        };
}

% IDS/IPS connection
% Usage: \drawIDSIPSConnection{from}{to}{mode}{alerts}{label}
% mode: IDS, IPS
\newcommand{\drawIDSIPSConnection}[5]{
    \ifthenelse{\equal{#3}{IDS}}{
        \def\idsColor{yellow}
        \def\idsStyle{dashed}
    }{
        \def\idsColor{red}
        \def\idsStyle{solid}
    }
    \draw[draw=\idsColor!70!black, line width=1.5pt, \idsStyle, -{Stealth[length=3mm]}]
        (#1) -- (#2)
        node[midway, above, font=\tiny, fill=\idsColor!20, rounded corners=2pt, inner sep=2pt] {
            #3: Alerts #4 | #5
        };
}

% Honeypot/Honeynet connection
% Usage: \drawHoneypotConnection{from}{to}{trap_type}{label}
\newcommand{\drawHoneypotConnection}[4]{
    \draw[draw=yellow!80!orange, line width=1pt, -{Stealth[length=3mm]}, densely dotted]
        (#1) -- (#2)
        [postaction={
            decorate,
            decoration={
                markings,
                mark=at position 0.5 with {
                    \node[star, star points=5, fill=yellow!80!black, inner sep=2pt] {};
                }
            }
        }]
        node[midway, above, font=\tiny, fill=yellow!25, rounded corners=2pt, inner sep=2pt] {
            Honeypot: #3 | #4
        };
}

% DLP (Data Loss Prevention) connection
% Usage: \drawDLPConnection{from}{to}{policy}{violations}{label}
\newcommand{\drawDLPConnection}[5]{
    \draw[draw=purple!70!black, line width=1.3pt, -{Stealth[length=3mm]}]
        (#1) -- (#2)
        node[midway, above, font=\tiny, fill=purple!20, rounded corners=2pt, inner sep=2pt] {
            DLP: Policy #3 | Violations:#4 | #5
        };
}

% ============================================================================
% DISASTER RECOVERY AND REPLICATION CONNECTIONS
% ============================================================================

% Database replication connection
% Usage: \drawDBReplication{from}{to}{replication_mode}{lag}{label}
% replication_mode: sync, async, semi-sync
\newcommand{\drawDBReplication}[5]{
    \ifthenelse{\equal{#3}{sync}}{
        \def\replColor{green}
        \def\replStyle{solid}
        \def\replWidth{1.5pt}
    }{}
    \ifthenelse{\equal{#3}{async}}{
        \def\replColor{yellow}
        \def\replStyle{dashed}
        \def\replWidth{1pt}
    }{}
    \ifthenelse{\equal{#3}{semi-sync}}{
        \def\replColor{orange}
        \def\replStyle{densely dashed}
        \def\replWidth{1.2pt}
    }{}
    \draw[draw=\replColor!70!black, line width=\replWidth, \replStyle, -{Stealth[length=3mm]}]
        (#1) -- (#2)
        node[midway, above, font=\tiny, fill=\replColor!20, rounded corners=2pt, inner sep=2pt] {
            DB Repl: #3 | Lag:#4 | #5
        };
}

% Backup connection with schedule
% Usage: \drawBackupConnection{from}{to}{backup_type}{schedule}{label}
\newcommand{\drawBackupConnection}[5]{
    \draw[draw=blue!60!black, line width=1pt, -{Stealth[length=3mm]}, densely dotted]
        (#1) -- (#2)
        [postaction={
            decorate,
            decoration={
                markings,
                mark=between positions 0.2 and 0.8 step 0.3 with {
                    \node[rectangle, fill=blue!60!black, inner sep=1pt] {};
                }
            }
        }]
        node[midway, above, font=\tiny, fill=blue!15, rounded corners=2pt, inner sep=2pt] {
            Backup: #3 | #4 | #5
        };
}

% Disaster recovery site connection
% Usage: \drawDRConnection{from}{to}{rpo}{rto}{label}
\newcommand{\drawDRConnection}[5]{
    \draw[draw=red!60!black, line width=2pt, <->, double distance=2pt]
        (#1) -- (#2)
        node[midway, above, font=\tiny\bfseries, fill=red!20, rounded corners=3pt, inner sep=3pt] {
            DR Site: RPO:#3 RTO:#4 | #5
        };
}

% Storage replication (SAN/NAS)
% Usage: \drawStorageReplication{from}{to}{protocol}{throughput}{label}
\newcommand{\drawStorageReplication}[5]{
    \draw[draw=teal!70!black, line width=1.8pt, <->, double]
        (#1) -- (#2)
        node[midway, above, font=\tiny, fill=teal!20, rounded corners=2pt, inner sep=2pt] {
            Storage: #3 | #4 | #5
        };
}

% ============================================================================
% PERFORMANCE OPTIMIZATION CONNECTIONS
% ============================================================================

% Content caching connection
% Usage: \drawCacheConnection{from}{to}{cache_hit_rate}{label}
\newcommand{\drawCacheConnection}[4]{
    \pgfmathsetmacro{\hitrate}{#3}
    \ifnum\hitrate>80
        \def\cacheColor{green}
    \else\ifnum\hitrate>50
        \def\cacheColor{yellow}
    \else
        \def\cacheColor{red}
    \fi\fi
    \draw[draw=\cacheColor!70!black, line width=1.2pt, -{Stealth[length=3mm]}]
        (#1) -- (#2)
        node[midway, above, font=\tiny, fill=\cacheColor!20, rounded corners=2pt, inner sep=2pt] {
            Cache: Hit Rate #3\% | #4
        };
}

% Content delivery acceleration
% Usage: \drawAcceleratorConnection{from}{to}{acceleration_factor}{label}
\newcommand{\drawAcceleratorConnection}[4]{
    \draw[draw=cyan!70!black, line width=1.5pt, -{Stealth[length=4mm]}, double]
        (#1) -- (#2)
        node[midway, above, font=\tiny\bfseries, fill=cyan!20, rounded corners=2pt, inner sep=2pt] {
            Accelerator: #3x | #4
        };
}

% TODO: Advanced statistics features
% - Real-time connection counts with live updates
% - Bandwidth utilization graphs embedded in diagram
% - Protocol distribution pie charts
% - Top talkers ranked list
% - Connection timeline with historical playback

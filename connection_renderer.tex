% connection_renderer.tex - Network connection rendering and visualization
% This module handles all connection types, paths, and flow visualization

% ============================================================================
% CONNECTION DATA STRUCTURES
% ============================================================================

% Connection counter
\newcounter{conncount}

% TODO: Connection management
% - Hash map for connection lookup
% - Bidirectional connection deduplication
% - Connection grouping by protocol/type
% - Connection state tracking (active/inactive)

% ============================================================================
% BASIC CONNECTION COMMANDS
% ============================================================================

% Draw a normal connection
% Usage: \drawConnection{from}{to}{label}
\newcommand{\drawConnection}[3]{
    \draw[normal conn] (#1) -- node[above, font=\tiny] {#3} (#2);
}

% Draw an encrypted connection
% Usage: \drawEncryptedConnection{from}{to}{protocol}
\newcommand{\drawEncryptedConnection}[3]{
    \draw[encrypted conn] (#1) -- node[above, font=\tiny\ttfamily] {#3} (#2);
}

% Draw a suspicious connection
% Usage: \drawSuspiciousConnection{from}{to}{reason}
\newcommand{\drawSuspiciousConnection}[3]{
    \draw[suspicious conn] (#1) -- node[above, font=\tiny, fill=yellow!30] {#3} (#2);
}

% Draw an attack connection
% Usage: \drawAttackConnection{from}{to}{attack_type}
\newcommand{\drawAttackConnection}[3]{
    \draw[attack conn] (#1) -- node[midway, above, threat label] {#3} (#2);
}

% Bidirectional connection
% Usage: \drawBidirectional{from}{to}{label}
\newcommand{\drawBidirectional}[3]{
    \draw[normal conn, bidirectional] (#1) -- node[above, font=\tiny] {#3} (#2);
}

% TODO: Enhanced connection types
% - VPN tunnel connections with special styling
% - Wireless connections with wave pattern
% - Fiber optic connections with light effects
% - Serial/legacy connections with different line style
% - Satellite links with orbital arc paths

% ============================================================================
% ADVANCED CONNECTION RENDERING
% ============================================================================

% ============================================================================
% BANDWIDTH VISUALIZATION SYSTEM
% ============================================================================

% Connection with bandwidth indicator (enhanced with logarithmic scaling)
% Usage: \drawConnectionWithBandwidth{from}{to}{bandwidth_mbps}{label}
\newcommand{\drawConnectionWithBandwidth}[4]{
    % Logarithmic scaling for better visualization across ranges
    % Maps: 1 Mbps -> 0.8pt, 10 Mbps -> 1.2pt, 100 Mbps -> 1.6pt, 1000 Mbps -> 2pt, 10000 Mbps -> 2.4pt
    \pgfmathsetmacro{\linewidth}{0.6 + 0.4*ln(max(1,#3))/ln(10)}
    \draw[draw=connNormal, line width=\linewidth pt, -{Stealth[length=3mm]}]
        (#1) -- node[above, font=\tiny] {#4: #3 Mbps} (#2);
}

% Connection with bandwidth and utilization (color-coded)
% Usage: \drawConnectionWithUtilization{from}{to}{bandwidth}{utilization_percent}{label}
\newcommand{\drawConnectionWithUtilization}[5]{
    \pgfmathsetmacro{\linewidth}{0.6 + 0.4*ln(max(1,#3))/ln(10)}
    % Color based on utilization: green <50%, yellow 50-80%, red >80%
    \pgfmathsetmacro{\util}{#4}
    \ifnum\util<50
        \def\connColor{green!60!black}
    \else\ifnum\util<80
        \def\connColor{yellow!80!orange}
    \else
        \def\connColor{red!80!black}
    \fi\fi
    \draw[draw=\connColor, line width=\linewidth pt, -{Stealth[length=3mm]}]
        (#1) -- node[above, font=\tiny] {#5: #3 Mbps (#4\%)} (#2);
}

% High-bandwidth connection (Gbps range)
% Usage: \drawHighBandwidthConnection{from}{to}{bandwidth_gbps}{label}
\newcommand{\drawHighBandwidthConnection}[4]{
    \pgfmathsetmacro{\linewidth}{1.5 + 0.5*ln(max(1,#3))/ln(10)}
    \draw[draw=blue!70!black, line width=\linewidth pt, -{Stealth[length=4mm]}, double]
        (#1) -- node[above, font=\tiny\bfseries] {#4: #3 Gbps} (#2);
}

% Bandwidth with congestion indicator
% Usage: \drawCongestedConnection{from}{to}{bandwidth}{congestion_level}{label}
% congestion_level: 0=none, 1=low, 2=medium, 3=high, 4=critical
\newcommand{\drawCongestedConnection}[5]{
    \pgfmathsetmacro{\linewidth}{0.6 + 0.4*ln(max(1,#3))/ln(10)}
    \ifcase#4
        \def\connStyle{draw=green!60!black}
    \or
        \def\connStyle{draw=yellow!70!black, densely dashed}
    \or
        \def\connStyle{draw=orange!80!black, dashed}
    \or
        \def\connStyle{draw=red!70!black, densely dotted}
    \or
        \def\connStyle{draw=red!90!black, line width=2pt, densely dotted}
    \fi
    \draw[\connStyle, -{Stealth[length=3mm]}]
        (#1) -- node[above, font=\tiny] {#5: #3 Mbps} (#2);
}

% Bandwidth comparison (dual connections showing before/after or primary/backup)
% Usage: \drawDualBandwidthConnection{from}{to}{bw1}{bw2}{label}
\newcommand{\drawDualBandwidthConnection}[5]{
    \pgfmathsetmacro{\linewidth}{0.6 + 0.4*ln(max(1,#3))/ln(10)}
    % Primary path
    \draw[draw=blue!70!black, line width=\linewidth pt, -{Stealth[length=3mm]}]
        ([yshift=2pt]#1) -- node[above, font=\tiny] {Primary: #3 Mbps} ([yshift=2pt]#2);
    % Backup/secondary path
    \pgfmathsetmacro{\linewidth2}{0.6 + 0.4*ln(max(1,#4))/ln(10)}
    \draw[draw=gray!60, line width=\linewidth2 pt, dashed, -{Stealth[length=2mm]}]
        ([yshift=-2pt]#1) -- node[below, font=\tiny] {Backup: #4 Mbps} ([yshift=-2pt]#2);
}

% ============================================================================
% PROTOCOL AND PORT LABELING SYSTEM
% ============================================================================

% Connection with protocol and port information (basic)
% Usage: \drawConnectionWithPort{from}{to}{protocol}{port}{label}
\newcommand{\drawConnectionWithPort}[5]{
    \draw[normal conn] (#1) --
        node[above, font=\tiny\ttfamily] {#5}
        node[below, port label] {#3:#4}
        (#2);
}

% Enhanced connection with auto-positioned protocol labels
% Automatically positions label based on connection angle
% Usage: \drawSmartLabeledConnection{from}{to}{protocol}{port}{service}{label}
\newcommand{\drawSmartLabeledConnection}[6]{
    \draw[normal conn] (#1) -- (#2)
        node[pos=0.3, above, sloped, font=\tiny\ttfamily, fill=white, inner sep=1pt] {#3:#4}
        node[pos=0.7, below, sloped, font=\tiny, fill=white, inner sep=1pt] {#6}
        node[pos=0.5, above, font=\scriptsize, fill=yellow!20, rounded corners=1pt, inner sep=1pt] {#5};
}

% Multi-protocol connection with stacked labels
% Usage: \drawMultiProtocolConnection{from}{to}{proto1}{port1}{proto2}{port2}{label}
\newcommand{\drawMultiProtocolConnection}[7]{
    \draw[normal conn] (#1) -- (#2)
        node[midway, above, font=\tiny\ttfamily, align=center, fill=white, rounded corners=2pt, inner sep=2pt] {
            #3:#4 \\ #5:#6
        }
        node[midway, below, font=\tiny] {#7};
}

% Connection with detailed service information
% Usage: \drawServiceConnection{from}{to}{protocol}{port}{service}{version}{label}
\newcommand{\drawServiceConnection}[7]{
    \draw[normal conn] (#1) -- (#2)
        node[midway, above, font=\tiny, fill=blue!10, rounded corners=2pt, inner sep=2pt, align=center] {
            \textbf{#5} \\
            \texttt{#3:#4} \\
            \textit{#6}
        }
        node[pos=0.2, below, font=\tiny] {#7};
}

% Auto-positioned port label (avoids node overlap)
% Usage: \drawAutoPortConnection{from}{to}{protocol}{port}{label}
\newcommand{\drawAutoPortConnection}[5]{
    \path let \p1=(#1), \p2=(#2),
              \n1={atan2(\y2-\y1,\x2-\x1)} in
        \pgfextra{
            \pgfmathsetmacro{\angle}{\n1}
            % Position label based on angle
            \ifdim\angle pt<45pt
                \def\labelpos{above}
            \else\ifdim\angle pt<135pt
                \def\labelpos{right}
            \else\ifdim\angle pt<225pt
                \def\labelpos{below}
            \else
                \def\labelpos{left}
            \fi\fi\fi
        }
        (#1) edge[normal conn] node[\labelpos, font=\tiny\ttfamily] {#3:#4}
                                node[pos=0.7, sloped, above, font=\tiny] {#5} (#2);
}

% Inline protocol badge (small colored box with protocol)
% Usage: \drawProtocolBadgeConnection{from}{to}{protocol}{port}{label}
\newcommand{\drawProtocolBadgeConnection}[5]{
    \draw[normal conn] (#1) -- (#2)
        node[pos=0.5, above, font=\scriptsize] {#5}
        node[pos=0.5, below, font=\tiny\ttfamily\bfseries,
              fill=blue!20, draw=blue!60, rounded corners=2pt, inner sep=1.5pt] {#3:#4};
}

% Connection with application layer protocol
% Usage: \drawAppLayerConnection{from}{to}{app_proto}{transport}{port}{label}
% Example: HTTP over TCP:80
\newcommand{\drawAppLayerConnection}[6]{
    \draw[normal conn] (#1) -- (#2)
        node[midway, above, font=\tiny\bfseries, fill=green!20, rounded corners=2pt, inner sep=2pt] {#3}
        node[midway, below, font=\tiny\ttfamily, fill=white, inner sep=1pt] {#4:#5}
        node[pos=0.2, below, font=\tiny] {#6};
}

% Encrypted protocol connection with lock indicator
% Usage: \drawEncryptedProtocolConnection{from}{to}{protocol}{port}{cipher}{label}
\newcommand{\drawEncryptedProtocolConnection}[6]{
    \draw[encrypted conn] (#1) -- (#2)
        node[midway, above, font=\tiny\bfseries, fill=green!30, rounded corners=2pt, inner sep=2pt] {
            \textcolor{green!50!black}{$\blacksquare$} #3:#4
        }
        node[midway, below, font=\tiny, fill=white, inner sep=1pt] {#5}
        node[pos=0.2, below, font=\tiny] {#6};
}

% Multiple ports on same connection
% Usage: \drawMultiPortConnection{from}{to}{protocol}{port_range}{label}
% Example: TCP:8000-8010
\newcommand{\drawMultiPortConnection}[5]{
    \draw[normal conn] (#1) -- (#2)
        node[midway, above, font=\tiny\ttfamily, fill=orange!20, rounded corners=2pt, inner sep=2pt] {
            #3:#4
        }
        node[pos=0.7, below, font=\tiny] {#5};
}

% Port with state indicator (open/closed/filtered)
% Usage: \drawStatefulPortConnection{from}{to}{protocol}{port}{state}{label}
% state: 0=closed, 1=filtered, 2=open
\newcommand{\drawStatefulPortConnection}[6]{
    \ifcase#5
        \def\stateColor{red!20}
        \def\stateText{CLOSED}
    \or
        \def\stateColor{yellow!30}
        \def\stateText{FILTERED}
    \or
        \def\stateColor{green!20}
        \def\stateText{OPEN}
    \fi
    \draw[normal conn] (#1) -- (#2)
        node[midway, above, font=\tiny\ttfamily, fill=\stateColor, rounded corners=2pt, inner sep=2pt] {
            #3:#4 [\stateText]
        }
        node[pos=0.7, below, font=\tiny] {#6};
}

% Curved connection (for avoiding overlaps)
% Usage: \drawCurvedConnection{from}{to}{bend}{label}
\newcommand{\drawCurvedConnection}[4]{
    \draw[normal conn] (#1) to[bend left=#3] 
        node[above, font=\tiny] {#4} (#2);
}

% ============================================================================
% AUTOMATIC PATH FINDING AND ROUTING
% ============================================================================

% Orthogonal routing (right-angle connections)
% Usage: \drawOrthogonalConnection{from}{to}{label}
\newcommand{\drawOrthogonalConnection}[3]{
    \draw[normal conn, rounded corners=3pt] (#1) -| (#2)
        node[pos=0.5, above, font=\tiny] {#3};
}

% Orthogonal routing with vertical-horizontal path
% Usage: \drawOrthogonalConnectionVH{from}{to}{label}
\newcommand{\drawOrthogonalConnectionVH}[3]{
    \draw[normal conn, rounded corners=3pt] (#1) |- (#2)
        node[pos=0.5, above, font=\tiny] {#3};
}

% Smart curved connection with automatic bend angle
% Calculates bend angle based on node distance
% Usage: \drawSmartCurvedConnection{from}{to}{label}
\newcommand{\drawSmartCurvedConnection}[3]{
    \draw[normal conn] (#1) to[bend left=20]
        node[pos=0.5, above, font=\tiny] {#3} (#2);
}

% Multi-waypoint connection for complex routing
% Usage: \drawWaypointConnection{from}{to}{waypoint1}{waypoint2}{label}
\newcommand{\drawWaypointConnection}[5]{
    \draw[normal conn] (#1) -- (#3) -- (#4) -- (#2)
        node[pos=0.5, above, font=\tiny] {#5};
}

% Avoid overlap connection with automatic path selection
% Uses curved path if nodes are close, straight if far
% Usage: \drawAvoidOverlapConnection{from}{to}{label}
\newcommand{\drawAvoidOverlapConnection}[3]{
    \path let \p1=(#1), \p2=(#2),
              \n1={veclen(\x2-\x1,\y2-\y1)} in
        \pgfextra{
            \pgfmathsetmacro{\distance}{\n1}
            \ifdim\distance pt<100pt
                \def\pathstyle{bend left=25}
            \else
                \def\pathstyle{}
            \fi
        }
        (#1) edge[\pathstyle, normal conn] node[pos=0.5, above, font=\tiny] {#3} (#2);
}

% Bezier curve connection for smooth organic paths
% Usage: \drawBezierConnection{from}{to}{ctrl1}{ctrl2}{label}
\newcommand{\drawBezierConnection}[5]{
    \draw[normal conn] (#1) .. controls (#3) and (#4) .. (#2)
        node[pos=0.5, above, font=\tiny] {#5};
}

% Obstacle-avoiding curved path (uses higher bend for closer nodes)
% Usage: \drawObstacleAvoidingConnection{from}{to}{label}{min_distance}
\newcommand{\drawObstacleAvoidingConnection}[4]{
    \path let \p1=(#1), \p2=(#2),
              \n1={veclen(\x2-\x1,\y2-\y1)} in
        \pgfextra{
            \pgfmathsetmacro{\distance}{\n1}
            \pgfmathsetmacro{\bendangle}{min(60, max(15, 3000/\distance))}
        }
        (#1) edge[bend left=\bendangle, normal conn]
            node[pos=0.5, above, font=\tiny] {#3} (#2);
}

% ============================================================================
% CONNECTION FLOW VISUALIZATION
% ============================================================================

% Animated flow indicators (requires animation package)
% Usage: \drawFlowConnection{from}{to}{direction}{speed}
\newcommand{\drawFlowConnection}[4]{
    \draw[draw=connNormal, line width=1.5pt, -{Stealth[length=3mm]}] (#1) -- (#2)
        [postaction={
            decorate,
            decoration={
                markings,
                mark=between positions 0.1 and 0.9 step 0.2 with {
                    \ifthenelse{\equal{#3}{forward}}{
                        \arrow{Stealth[length=2mm, fill=connNormal]}
                    }{
                        \arrow{Stealth[reversed, length=2mm, fill=connNormal]}
                    }
                }
            }
        }];
}

% Traffic flow with volume indicator
% Usage: \drawTrafficFlow{from}{to}{packets_per_sec}{label}
\newcommand{\drawTrafficFlow}[4]{
    \pgfmathsetmacro{\density}{min(10, #3/100)}
    \draw[draw=connNormal, line width=1.5pt, -{Stealth[length=3mm]}] 
        (#1) -- node[above, font=\tiny] {#4: #3 pps} (#2)
        [postaction={
            decorate,
            decoration={
                markings,
                mark=between positions 0.1 and 0.9 step {0.1/\density} with {
                    \node[circle, fill=connNormal, inner sep=0.5pt] {};
                }
            }
        }];
}

% ============================================================================
% CONNECTION BUNDLING FOR HIGH-DENSITY DIAGRAMS
% ============================================================================

% Bundle multiple connections into a single visual with counter
% Usage: \drawBundledConnection{from}{to}{count}{label}
\newcommand{\drawBundledConnection}[4]{
    % Thicker line to represent multiple connections
    \pgfmathsetmacro{\linewidth}{1 + 0.3*ln(max(1,#3))/ln(10)}
    \draw[draw=connNormal, line width=\linewidth pt, -{Stealth[length=3mm]}]
        (#1) -- (#2)
        node[midway, above, font=\tiny] {#4}
        node[midway, below, font=\scriptsize, fill=white, circle, inner sep=1pt] {\textbf{#3}};
}

% Bundle connections with protocol breakdown
% Usage: \drawProtocolBundledConnection{from}{to}{total}{tcp}{udp}{other}{label}
\newcommand{\drawProtocolBundledConnection}[7]{
    \pgfmathsetmacro{\linewidth}{1 + 0.3*ln(max(1,#3))/ln(10)}
    \draw[draw=connNormal, line width=\linewidth pt, -{Stealth[length=3mm]}]
        (#1) -- (#2)
        node[midway, above, font=\tiny] {#7}
        node[midway, below, font=\scriptsize, fill=white, rounded corners=2pt, inner sep=2pt] {
            \textbf{#3} (TCP:#4 UDP:#5 Other:#6)
        };
}

% Aggregated connection with summary statistics
% Usage: \drawAggregatedConnection{from}{to}{conn_count}{total_bandwidth}{label}
\newcommand{\drawAggregatedConnection}[5]{
    \pgfmathsetmacro{\linewidth}{1 + 0.4*ln(max(1,#4))/ln(10)}
    \draw[draw=blue!60!black, line width=\linewidth pt, -{Stealth[length=4mm]}, double distance=1pt]
        (#1) -- (#2)
        node[midway, above, font=\tiny\bfseries] {#5}
        node[midway, below, font=\scriptsize, fill=blue!10, rounded corners=2pt, inner sep=2pt] {
            #3 conns | #4 Mbps
        };
}

% Connection bundle with expandable detail
% Shows collapsed view with option to reference detailed breakdown
% Usage: \drawExpandableBundledConnection{from}{to}{count}{protocols}{label}
\newcommand{\drawExpandableBundledConnection}[5]{
    \pgfmathsetmacro{\linewidth}{1 + 0.3*ln(max(1,#3))/ln(10)}
    \draw[draw=connNormal, line width=\linewidth pt, -{Stealth[length=3mm]}]
        (#1) -- (#2)
        node[midway, above, font=\tiny] {#5}
        node[midway, below, font=\scriptsize, fill=yellow!20, rounded corners=2pt, inner sep=2pt] {
            \textbf{#3 connections} \textit{(#4)}
        };
}

% Multi-tier bundle (for complex aggregations)
% Usage: \drawMultiTierBundle{from}{to}{edge_count}{core_count}{label}
\newcommand{\drawMultiTierBundle}[5]{
    % Draw thick core bundle
    \draw[draw=blue!70!black, line width=3pt, -{Stealth[length=5mm]}, double distance=2pt]
        (#1) -- (#2)
        node[midway, above, font=\small\bfseries] {#5}
        node[midway, below, font=\scriptsize, fill=blue!20, rounded corners=3pt, inner sep=3pt] {
            Core: #4 | Edge: #3
        };
}

% Parallel connection bundle (shows multiple parallel lines)
% Usage: \drawParallelBundle{from}{to}{count}{label}
\newcommand{\drawParallelBundle}[4]{
    \foreach \i in {1,...,#3} {
        \pgfmathsetmacro{\offset}{-1.5 + 3*\i/#3}
        \draw[draw=connNormal, line width=0.5pt, -{Stealth[length=2mm]}]
            ([yshift=\offset pt]#1) -- ([yshift=\offset pt]#2);
    }
    \node[above, font=\tiny] at ($(#1)!0.5!(#2)$) {#4 (#3 parallel)};
}

% Grouped connection bundle with color coding
% Usage: \drawColorCodedBundle{from}{to}{safe}{suspicious}{malicious}{label}
\newcommand{\drawColorCodedBundle}[6]{
    \pgfmathsetmacro{\total}{#3+#4+#5}
    \pgfmathsetmacro{\linewidth}{1 + 0.3*ln(max(1,\total))/ln(10)}
    \draw[draw=gray!60!black, line width=\linewidth pt, -{Stealth[length=3mm]}]
        (#1) -- (#2)
        node[midway, above, font=\tiny] {#6}
        node[midway, below, font=\scriptsize, fill=white, rounded corners=2pt, inner sep=2pt] {
            \textcolor{green!60!black}{#3} |
            \textcolor{orange!80!black}{#4} |
            \textcolor{red!80!black}{#5}
        };
}

% TODO: Flow visualization enhancements
% - Real-time traffic animation (if rendering to animated format)
% - Packet visualization with different colors per protocol
% - Congestion indicators (red/yellow traffic markers)
% - Flow direction with multiple arrows
% - Throughput heatmap coloring

% ============================================================================
% CONNECTION RENDERING ENGINE
% ============================================================================

% Main command to render all connections
\newcommand{\renderConnections}{
    % This will be populated by network_data.tex
    % Example structure:
    % \drawConnection{srv1}{pc1}{HTTP}
    % \drawAttackConnection{attacker1}{srv1}{SQL Injection}
}

% TODO: Intelligent connection rendering
% - Layer-based rendering (background to foreground)
% - Z-order management for overlapping connections
% - Automatic connection routing to minimize crossings
% - Connection aggregation (show "10 connections" instead of 10 lines)
% - Highlight selected connection paths

% ============================================================================
% CONNECTION LABELS AND ANNOTATIONS
% ============================================================================

% Add inline statistics to connection
% Usage: \labelConnectionStats{from}{to}{latency}{packet_loss}{jitter}
\newcommand{\labelConnectionStats}[5]{
    \draw[normal conn] (#1) -- (#2)
        node[midway, below, font=\tiny\ttfamily, fill=white, inner sep=2pt] {
            L:#3ms | PL:#4\% | J:#5ms
        };
}

% Add threat score to connection
% Usage: \labelConnectionThreat{from}{to}{score}{details}
\newcommand{\labelConnectionThreat}[4]{
    \draw[suspicious conn] (#1) -- (#2)
        node[midway, threat label] {
            \textbf{#3/10} #4
        };
}

% TODO: Label improvements
% - Auto-positioning to avoid overlaps
% - Expandable detail boxes on hover
% - Time-series data for connection metrics
% - Alert indicators for anomalous connections

% ============================================================================
% CONNECTION FILTERING AND LAYERS
% ============================================================================

% Only show connections of specific type
% Usage: \filterConnectionsByType{type}
% Types: all, encrypted, suspicious, attacks, normal
\newcommand{\filterConnectionsByType}[1]{
    % Implementation would use conditional rendering
}

% Show connections only for specific protocol
% Usage: \filterConnectionsByProtocol{protocol}
\newcommand{\filterConnectionsByProtocol}[1]{
    % Implementation would filter by protocol name
}

% TODO: Advanced filtering
% - Port-based filtering (show only port 80, 443, etc.)
% - Time-based filtering (show connections in time range)
% - Threshold filtering (show only high-bandwidth connections)
% - Interactive layer toggles for different connection types
% - Subnet-based filtering (show only intra/inter-subnet)

% ============================================================================
% CONNECTION PATTERNS AND ANALYSIS
% ============================================================================

% Highlight attack pattern (multiple sources to one target)
% Usage: \highlightAttackPattern{target}{sources}
\newcommand{\highlightAttackPattern}[2]{
    \begin{scope}[on background layer]
        \foreach \source in {#2} {
            \draw[attack conn, line width=2pt] (\source) -- (#1);
        }
        % Draw highlight around target
        \node[draw=threatCritical, line width=3pt, 
              rounded corners=5pt, inner sep=8pt, 
              fill=threatCritical!10] at (#1) {};
    \end{scope}
}

% Show connection path through network
% Usage: \showConnectionPath{node_list}
\newcommand{\showConnectionPath}[1]{
    % Draw path through multiple hops
    % Highlight the route taken by traffic
}

% TODO: Pattern detection visualization
% - DDoS pattern (many-to-one)
% - Data exfiltration pattern (one-to-many external)
% - Lateral movement pattern (peer-to-peer internal)
% - Command & Control pattern (periodic beaconing)
% - Port scanning pattern (one-to-many same port)

% ============================================================================
% CONNECTION STATISTICS AND METRICS
% ============================================================================

% Draw connection statistics summary
% Usage: \drawConnectionSummary{x}{y}
\newcommand{\drawConnectionSummary}[2]{
    \node[legend box, anchor=north west] at (#1,#2) {
        \begin{tabular}{lr}
            \textbf{Connections} & \\
            Normal & \theconncount \\
            Encrypted & 0 \\
            Suspicious & 0 \\
            Attacks & 0 \\
        \end{tabular}
    };
}

% TODO: Statistics enhancements
% - Real-time connection counts
% - Bandwidth utilization graphs
% - Protocol distribution pie chart
% - Top talkers list
% - Connection timeline visualization

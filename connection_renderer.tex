% connection_renderer.tex - Network connection rendering and visualization
% This module handles all connection types, paths, and flow visualization

% ============================================================================
% CONNECTION DATA STRUCTURES
% ============================================================================

% Connection counter
\newcounter{conncount}

% TODO: Connection management
% - Hash map for connection lookup
% - Bidirectional connection deduplication
% - Connection grouping by protocol/type
% - Connection state tracking (active/inactive)

% ============================================================================
% BASIC CONNECTION COMMANDS
% ============================================================================

% Draw a normal connection
% Usage: \drawConnection{from}{to}{label}
\newcommand{\drawConnection}[3]{
    \draw[normal conn] (#1) -- node[above, font=\tiny] {#3} (#2);
}

% Draw an encrypted connection
% Usage: \drawEncryptedConnection{from}{to}{protocol}
\newcommand{\drawEncryptedConnection}[3]{
    \draw[encrypted conn] (#1) -- node[above, font=\tiny\ttfamily] {#3} (#2);
}

% Draw a suspicious connection
% Usage: \drawSuspiciousConnection{from}{to}{reason}
\newcommand{\drawSuspiciousConnection}[3]{
    \draw[suspicious conn] (#1) -- node[above, font=\tiny, fill=yellow!30] {#3} (#2);
}

% Draw an attack connection
% Usage: \drawAttackConnection{from}{to}{attack_type}
\newcommand{\drawAttackConnection}[3]{
    \draw[attack conn] (#1) -- node[midway, above, threat label] {#3} (#2);
}

% Bidirectional connection
% Usage: \drawBidirectional{from}{to}{label}
\newcommand{\drawBidirectional}[3]{
    \draw[normal conn, bidirectional] (#1) -- node[above, font=\tiny] {#3} (#2);
}

% TODO: Enhanced connection types
% - VPN tunnel connections with special styling
% - Wireless connections with wave pattern
% - Fiber optic connections with light effects
% - Serial/legacy connections with different line style
% - Satellite links with orbital arc paths

% ============================================================================
% ADVANCED CONNECTION RENDERING
% ============================================================================

% Connection with bandwidth indicator
% Usage: \drawConnectionWithBandwidth{from}{to}{bandwidth}{label}
\newcommand{\drawConnectionWithBandwidth}[4]{
    % Line width based on bandwidth
    \pgfmathsetmacro{\linewidth}{1 + (#3/1000)}
    \draw[draw=connNormal, line width=\linewidth pt, -{Stealth[length=3mm]}] 
        (#1) -- node[above, font=\tiny] {#4: #3 Mbps} (#2);
}

% Connection with protocol and port information
% Usage: \drawConnectionWithPort{from}{to}{protocol}{port}{label}
\newcommand{\drawConnectionWithPort}[5]{
    \draw[normal conn] (#1) -- 
        node[above, font=\tiny\ttfamily] {#5} 
        node[below, port label] {#3:#4} 
        (#2);
}

% Curved connection (for avoiding overlaps)
% Usage: \drawCurvedConnection{from}{to}{bend}{label}
\newcommand{\drawCurvedConnection}[4]{
    \draw[normal conn] (#1) to[bend left=#3] 
        node[above, font=\tiny] {#4} (#2);
}

% TODO: Advanced rendering features
% - Automatic path finding to avoid node overlaps
% - Bezier curve connections for organic look
% - Multi-segment paths through waypoints
% - Connection bundling for high-density areas
% - Edge routing algorithms (orthogonal, polyline)

% ============================================================================
% CONNECTION FLOW VISUALIZATION
% ============================================================================

% Animated flow indicators (requires animation package)
% Usage: \drawFlowConnection{from}{to}{direction}{speed}
\newcommand{\drawFlowConnection}[4]{
    \draw[draw=connNormal, line width=1.5pt, -{Stealth[length=3mm]}] (#1) -- (#2)
        [postaction={
            decorate,
            decoration={
                markings,
                mark=between positions 0.1 and 0.9 step 0.2 with {
                    \ifthenelse{\equal{#3}{forward}}{
                        \arrow{Stealth[length=2mm, fill=connNormal]}
                    }{
                        \arrow{Stealth[reversed, length=2mm, fill=connNormal]}
                    }
                }
            }
        }];
}

% Traffic flow with volume indicator
% Usage: \drawTrafficFlow{from}{to}{packets_per_sec}{label}
\newcommand{\drawTrafficFlow}[4]{
    \pgfmathsetmacro{\density}{min(10, #3/100)}
    \draw[draw=connNormal, line width=1.5pt, -{Stealth[length=3mm]}] 
        (#1) -- node[above, font=\tiny] {#4: #3 pps} (#2)
        [postaction={
            decorate,
            decoration={
                markings,
                mark=between positions 0.1 and 0.9 step {0.1/\density} with {
                    \node[circle, fill=connNormal, inner sep=0.5pt] {};
                }
            }
        }];
}

% TODO: Flow visualization enhancements
% - Real-time traffic animation (if rendering to animated format)
% - Packet visualization with different colors per protocol
% - Congestion indicators (red/yellow traffic markers)
% - Flow direction with multiple arrows
% - Throughput heatmap coloring

% ============================================================================
% CONNECTION RENDERING ENGINE
% ============================================================================

% Main command to render all connections
\newcommand{\renderConnections}{
    % This will be populated by network_data.tex
    % Example structure:
    % \drawConnection{srv1}{pc1}{HTTP}
    % \drawAttackConnection{attacker1}{srv1}{SQL Injection}
}

% TODO: Intelligent connection rendering
% - Layer-based rendering (background to foreground)
% - Z-order management for overlapping connections
% - Automatic connection routing to minimize crossings
% - Connection aggregation (show "10 connections" instead of 10 lines)
% - Highlight selected connection paths

% ============================================================================
% CONNECTION LABELS AND ANNOTATIONS
% ============================================================================

% Add inline statistics to connection
% Usage: \labelConnectionStats{from}{to}{latency}{packet_loss}{jitter}
\newcommand{\labelConnectionStats}[5]{
    \draw[normal conn] (#1) -- (#2)
        node[midway, below, font=\tiny\ttfamily, fill=white, inner sep=2pt] {
            L:#3ms | PL:#4\% | J:#5ms
        };
}

% Add threat score to connection
% Usage: \labelConnectionThreat{from}{to}{score}{details}
\newcommand{\labelConnectionThreat}[4]{
    \draw[suspicious conn] (#1) -- (#2)
        node[midway, threat label] {
            \textbf{#3/10} #4
        };
}

% TODO: Label improvements
% - Auto-positioning to avoid overlaps
% - Expandable detail boxes on hover
% - Time-series data for connection metrics
% - Alert indicators for anomalous connections

% ============================================================================
% CONNECTION FILTERING AND LAYERS
% ============================================================================

% Only show connections of specific type
% Usage: \filterConnectionsByType{type}
% Types: all, encrypted, suspicious, attacks, normal
\newcommand{\filterConnectionsByType}[1]{
    % Implementation would use conditional rendering
}

% Show connections only for specific protocol
% Usage: \filterConnectionsByProtocol{protocol}
\newcommand{\filterConnectionsByProtocol}[1]{
    % Implementation would filter by protocol name
}

% TODO: Advanced filtering
% - Port-based filtering (show only port 80, 443, etc.)
% - Time-based filtering (show connections in time range)
% - Threshold filtering (show only high-bandwidth connections)
% - Interactive layer toggles for different connection types
% - Subnet-based filtering (show only intra/inter-subnet)

% ============================================================================
% CONNECTION PATTERNS AND ANALYSIS
% ============================================================================

% Highlight attack pattern (multiple sources to one target)
% Usage: \highlightAttackPattern{target}{sources}
\newcommand{\highlightAttackPattern}[2]{
    \begin{scope}[on background layer]
        \foreach \source in {#2} {
            \draw[attack conn, line width=2pt] (\source) -- (#1);
        }
        % Draw highlight around target
        \node[draw=threatCritical, line width=3pt, 
              rounded corners=5pt, inner sep=8pt, 
              fill=threatCritical!10] at (#1) {};
    \end{scope}
}

% Show connection path through network
% Usage: \showConnectionPath{node_list}
\newcommand{\showConnectionPath}[1]{
    % Draw path through multiple hops
    % Highlight the route taken by traffic
}

% TODO: Pattern detection visualization
% - DDoS pattern (many-to-one)
% - Data exfiltration pattern (one-to-many external)
% - Lateral movement pattern (peer-to-peer internal)
% - Command & Control pattern (periodic beaconing)
% - Port scanning pattern (one-to-many same port)

% ============================================================================
% CONNECTION STATISTICS AND METRICS
% ============================================================================

% Draw connection statistics summary
% Usage: \drawConnectionSummary{x}{y}
\newcommand{\drawConnectionSummary}[2]{
    \node[legend box, anchor=north west] at (#1,#2) {
        \begin{tabular}{lr}
            \textbf{Connections} & \\
            Normal & \theconncount \\
            Encrypted & 0 \\
            Suspicious & 0 \\
            Attacks & 0 \\
        \end{tabular}
    };
}

% TODO: Statistics enhancements
% - Real-time connection counts
% - Bandwidth utilization graphs
% - Protocol distribution pie chart
% - Top talkers list
% - Connection timeline visualization

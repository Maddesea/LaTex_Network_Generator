% connection_renderer.tex - Network connection rendering and visualization
% This module handles all connection types, paths, and flow visualization

% ============================================================================
% CONNECTION DATA STRUCTURES
% ============================================================================

% Connection counter
\newcounter{conncount}

% TODO: Connection management
% - Hash map for connection lookup
% - Bidirectional connection deduplication
% - Connection grouping by protocol/type
% - Connection state tracking (active/inactive)

% ============================================================================
% BASIC CONNECTION COMMANDS
% ============================================================================

% Draw a normal connection
% Usage: \drawConnection{from}{to}{label}
\newcommand{\drawConnection}[3]{
    \draw[normal conn] (#1) -- node[above, font=\tiny] {#3} (#2);
}

% Draw an encrypted connection
% Usage: \drawEncryptedConnection{from}{to}{protocol}
\newcommand{\drawEncryptedConnection}[3]{
    \draw[encrypted conn] (#1) -- node[above, font=\tiny\ttfamily] {#3} (#2);
}

% Draw a suspicious connection
% Usage: \drawSuspiciousConnection{from}{to}{reason}
\newcommand{\drawSuspiciousConnection}[3]{
    \draw[suspicious conn] (#1) -- node[above, font=\tiny, fill=yellow!30] {#3} (#2);
}

% Draw an attack connection
% Usage: \drawAttackConnection{from}{to}{attack_type}
\newcommand{\drawAttackConnection}[3]{
    \draw[attack conn] (#1) -- node[midway, above, threat label] {#3} (#2);
}

% Bidirectional connection
% Usage: \drawBidirectional{from}{to}{label}
\newcommand{\drawBidirectional}[3]{
    \draw[normal conn, bidirectional] (#1) -- node[above, font=\tiny] {#3} (#2);
}

% ============================================================================
% SPECIAL CONNECTION TYPES
% ============================================================================

% VPN tunnel connection
% Usage: \drawVPNTunnel{from}{to}{label}
\newcommand{\drawVPNTunnel}[3]{
    \draw[draw=blue!70, line width=1.5pt, dashed,
          dash pattern=on 3pt off 2pt, -{Stealth[length=3mm]},
          postaction={draw, line width=0.5pt, draw=blue!30}]
        (#1) -- (#2)
        node[midway, above, font=\tiny\bfseries, fill=blue!10, inner sep=2pt] {#3}
        node[midway, below, font=\tiny, fill=blue!20, inner sep=1pt] {VPN};
}

% Wireless connection with wave pattern
% Usage: \drawWirelessConnection{from}{to}{signal_strength}{label}
\newcommand{\drawWirelessConnection}[4]{
    \draw[draw=purple!70, line width=1pt,
          decoration={snake, amplitude=0.5mm, segment length=3mm},
          decorate, -{Stealth[length=2.5mm]}]
        (#1) -- (#2)
        node[midway, above, font=\tiny, fill=white, inner sep=1pt] {#4}
        node[midway, below, font=\tiny, fill=purple!15, inner sep=1pt] {WiFi: #3\%};
}

% Fiber optic connection with light effects
% Usage: \drawFiberConnection{from}{to}{label}
\newcommand{\drawFiberConnection}[3]{
    \draw[draw=yellow!80!orange, line width=2pt, -{Stealth[length=3mm]}]
        (#1) -- (#2);
    \draw[draw=yellow!50, line width=1pt, -{Stealth[length=3mm]}]
        (#1) -- (#2)
        node[midway, above, font=\tiny\bfseries, fill=white, inner sep=2pt] {#3}
        node[midway, below, font=\tiny, fill=yellow!20, inner sep=1pt] {Fiber};
}

% Serial/Legacy connection
% Usage: \drawSerialConnection{from}{to}{baud_rate}{label}
\newcommand{\drawSerialConnection}[4]{
    \draw[draw=gray!70, line width=1pt,
          dash pattern=on 2pt off 1pt on 4pt off 1pt,
          -{Stealth[length=2.5mm]}]
        (#1) -- (#2)
        node[midway, above, font=\tiny, fill=white, inner sep=1pt] {#4}
        node[midway, below, font=\tiny\ttfamily, fill=gray!15, inner sep=1pt] {RS-232: #3};
}

% Satellite link with orbital arc
% Usage: \drawSatelliteLink{from}{to}{latency}{label}
\newcommand{\drawSatelliteLink}[4]{
    \draw[draw=cyan!70, line width=1pt, -{Stealth[length=2.5mm]}]
        (#1) to[bend left=60] (#2)
        node[midway, above, font=\tiny, fill=white, inner sep=2pt] {#4}
        node[midway, below, font=\tiny, fill=cyan!15, inner sep=1pt] {SAT: #3ms};
}

% Cellular/Mobile connection
% Usage: \drawCellularConnection{from}{to}{technology}{label}
\newcommand{\drawCellularConnection}[4]{
    \draw[draw=green!60!blue, line width=1pt,
          decoration={zigzag, amplitude=0.5mm, segment length=2mm},
          decorate, -{Stealth[length=2.5mm]}]
        (#1) -- (#2)
        node[midway, above, font=\tiny, fill=white, inner sep=1pt] {#4}
        node[midway, below, font=\tiny, fill=green!15, inner sep=1pt] {#3};
}

% Bluetooth connection
% Usage: \drawBluetoothConnection{from}{to}{label}
\newcommand{\drawBluetoothConnection}[3]{
    \draw[draw=blue!60, line width=0.8pt,
          decoration={snake, amplitude=0.3mm, segment length=2mm},
          decorate, bidirectional]
        (#1) -- (#2)
        node[midway, above, font=\tiny, fill=blue!10, inner sep=1pt] {#3 (BT)};
}

% TODO: Additional special connection types
% - LoRaWAN for IoT
% - NFC for proximity connections
% - Infrared connections
% - Optical wireless (Li-Fi)

% ============================================================================
% ADVANCED CONNECTION RENDERING
% ============================================================================

% Connection with bandwidth indicator (Enhanced with logarithmic scaling)
% Usage: \drawConnectionWithBandwidth{from}{to}{bandwidth}{label}
\newcommand{\drawConnectionWithBandwidth}[4]{
    % Line width based on bandwidth (logarithmic scale)
    \pgfmathsetmacro{\linewidth}{0.5 + ln(max(1, #3))/2}
    \draw[draw=connNormal, line width=\linewidth pt, -{Stealth[length=3mm]}]
        (#1) -- node[above, font=\tiny] {#4: #3 Mbps} (#2);
}

% Connection with bandwidth and utilization (shows congestion)
% Usage: \drawConnectionWithUtilization{from}{to}{bandwidth}{utilization}{label}
\newcommand{\drawConnectionWithUtilization}[5]{
    % Line width based on bandwidth (logarithmic scale)
    \pgfmathsetmacro{\linewidth}{0.5 + ln(max(1, #3))/2}
    % Color based on utilization percentage
    \pgfmathsetmacro{\util}{#4}
    \ifnum\util<50
        \def\connColor{green!60!black}
    \else
        \ifnum\util<80
            \def\connColor{yellow!80!orange}
        \else
            \def\connColor{red!70!black}
        \fi
    \fi
    \draw[draw=\connColor, line width=\linewidth pt, -{Stealth[length=3mm]}]
        (#1) -- node[above, font=\tiny] {#5}
        node[below, font=\tiny\ttfamily] {#3 Mbps (\util\% util)} (#2);
}

% Connection with bandwidth gradient (visual bandwidth representation)
% Usage: \drawBandwidthGradient{from}{to}{bandwidth}{max_bandwidth}{label}
\newcommand{\drawBandwidthGradient}[5]{
    \pgfmathsetmacro{\linewidth}{0.5 + ln(max(1, #3))/2}
    \pgfmathsetmacro{\ratio}{100*#3/#4}
    % Gradient from green to red based on bandwidth usage
    \draw[draw=green!60!black, line width=\linewidth pt, -{Stealth[length=3mm]}]
        (#1) -- (#2);
    % Overlay with utilization bar
    \node[midway, above, font=\tiny, fill=white, inner sep=2pt]
        at ($(#1)!0.5!(#2)$) {#5: #3/#4 Mbps};
}

% Connection with protocol and port information (Enhanced)
% Usage: \drawConnectionWithPort{from}{to}{protocol}{port}{label}
\newcommand{\drawConnectionWithPort}[5]{
    \draw[normal conn, -{Stealth[length=2.5mm]}] (#1) --
        node[above, font=\tiny\ttfamily, fill=white, inner sep=1pt] {#5}
        node[below, port label] {#3:#4}
        (#2);
}

% Enhanced protocol label with auto-positioning
% Usage: \drawConnectionWithProtocol{from}{to}{protocol}{port}{service}{label}
\newcommand{\drawConnectionWithProtocol}[6]{
    \draw[normal conn, -{Stealth[length=2.5mm]}] (#1) -- (#2)
        node[midway, above, font=\tiny, fill=white, inner sep=1pt] {#6}
        node[midway, below, font=\tiny\ttfamily, fill=blue!10, inner sep=1pt]
            {#3/#4 (#5)};
}

% Protocol label with color coding
% Usage: \drawColorCodedProtocol{from}{to}{protocol}{port}{label}
\newcommand{\drawColorCodedProtocol}[5]{
    % Color based on protocol
    \ifthenelse{\equal{#3}{TCP}}{
        \def\protoColor{blue!70}
    }{
        \ifthenelse{\equal{#3}{UDP}}{
            \def\protoColor{green!70}
        }{
            \def\protoColor{orange!70}
        }
    }
    \draw[draw=\protoColor, line width=1pt, -{Stealth[length=2.5mm]}]
        (#1) -- (#2)
        node[midway, above, font=\tiny, fill=white, inner sep=1pt] {#5}
        node[midway, below, font=\tiny\ttfamily, fill=\protoColor!20, inner sep=1pt]
            {#3:#4};
}

% Multi-protocol connection showing multiple services
% Usage: \drawMultiProtocolConnection{from}{to}{protocols}{label}
\newcommand{\drawMultiProtocolConnection}[4]{
    \draw[normal conn, line width=1.5pt, -{Stealth[length=3mm]}]
        (#1) -- (#2)
        node[midway, above, font=\tiny\bfseries, fill=white, inner sep=1pt] {#4};
    \node[midway, below, font=\tiny\ttfamily, fill=yellow!15,
          inner sep=2pt, align=center]
        at ($(#1)!0.5!(#2)$) {#3};
}

% Port range connection
% Usage: \drawPortRangeConnection{from}{to}{protocol}{port_start}{port_end}{label}
\newcommand{\drawPortRangeConnection}[6]{
    \draw[normal conn, -{Stealth[length=2.5mm]}] (#1) -- (#2)
        node[midway, above, font=\tiny, fill=white, inner sep=1pt] {#6}
        node[midway, below, font=\tiny\ttfamily, fill=orange!15, inner sep=1pt]
            {#3:#4-#5};
}

% Curved connection (for avoiding overlaps)
% Usage: \drawCurvedConnection{from}{to}{bend}{label}
\newcommand{\drawCurvedConnection}[4]{
    \draw[normal conn] (#1) to[bend left=#3] 
        node[above, font=\tiny] {#4} (#2);
}

% ============================================================================
% AUTOMATIC PATH FINDING AND OBSTACLE AVOIDANCE
% ============================================================================

% Orthogonal connection with automatic routing
% Usage: \drawOrthogonalConnection{from}{to}{label}
\newcommand{\drawOrthogonalConnection}[3]{
    \draw[normal conn, -Stealth] (#1) -|
        ($(#1)!0.5!(#2)$) |- (#2)
        node[midway, above, font=\tiny, fill=white, inner sep=1pt] {#3};
}

% Orthogonal connection with custom waypoint
% Usage: \drawOrthogonalViaPoint{from}{to}{x_offset}{y_offset}{label}
\newcommand{\drawOrthogonalViaPoint}[5]{
    \draw[normal conn, -Stealth] (#1)
        -| ($(#1)+(#3,#4)$)
        -| (#2)
        node[pos=0.5, above, font=\tiny, fill=white, inner sep=1pt] {#5};
}

% Smart curved connection with obstacle avoidance
% Automatically curves to avoid overlapping with center area
% Usage: \drawSmartCurvedConnection{from}{to}{label}
\newcommand{\drawSmartCurvedConnection}[3]{
    % Calculate if nodes are on opposite sides
    \path let \p1 = (#1), \p2 = (#2) in
        \pgfextra{
            \pgfmathsetmacro{\dx}{\x2-\x1}
            \pgfmathsetmacro{\dy}{\y2-\y1}
            \pgfmathsetmacro{\dist}{sqrt(\dx*\dx+\dy*\dy)}
            \pgfmathsetmacro{\bendangle}{min(45, \dist/10)}
        };
    \draw[normal conn, -Stealth] (#1) to[bend left=\bendangle]
        node[midway, above, font=\tiny, fill=white, inner sep=1pt] {#3} (#2);
}

% Multi-waypoint connection for complex routing
% Usage: \drawPathConnection{from}{to}{waypoint1}{waypoint2}{label}
\newcommand{\drawPathConnection}[5]{
    \draw[normal conn, -Stealth] (#1) -- (#3) -- (#4) -- (#2)
        node[pos=0.5, above, font=\tiny, fill=white, inner sep=1pt] {#5};
}

% Connection with automatic midpoint calculation
% Avoids center by routing around perimeter
% Usage: \drawPerimeterConnection{from}{to}{label}
\newcommand{\drawPerimeterConnection}[3]{
    \path let \p1 = (#1), \p2 = (#2) in
        coordinate (mid) at ($(\x1,\y2)$);
    \draw[normal conn, -Stealth] (#1) -- (mid) -- (#2)
        node[pos=0.5, above, font=\tiny, fill=white, inner sep=1pt] {#3};
}

% TODO: Advanced pathfinding
% - Implement A* algorithm for optimal routing
% - Add dynamic obstacle detection from node positions
% - Calculate minimum distance paths
% - Support for custom routing zones and restrictions

% ============================================================================
% BEZIER CURVE CONNECTIONS
% ============================================================================

% Simple Bezier curve with automatic control points
% Usage: \drawBezierConnection{from}{to}{label}
\newcommand{\drawBezierConnection}[3]{
    \path let \p1 = (#1), \p2 = (#2) in
        coordinate (ctrl1) at ($(\x1,\y1)!0.33!(\x2,\y2) + (0,1)$)
        coordinate (ctrl2) at ($(\x1,\y1)!0.67!(\x2,\y2) + (0,1)$);
    \draw[normal conn, -{Stealth[length=2.5mm]}]
        (#1) .. controls (ctrl1) and (ctrl2) .. (#2)
        node[midway, above, font=\tiny, fill=white, inner sep=1pt] {#3};
}

% Bezier curve with custom control points
% Usage: \drawCustomBezier{from}{to}{ctrl1_x}{ctrl1_y}{ctrl2_x}{ctrl2_y}{label}
\newcommand{\drawCustomBezier}[7]{
    \draw[normal conn, -{Stealth[length=2.5mm]}]
        (#1) .. controls (#3,#4) and (#5,#6) .. (#2)
        node[midway, above, font=\tiny, fill=white, inner sep=1pt] {#7};
}

% Smooth curved connection with tension control
% Usage: \drawSmoothCurve{from}{to}{tension}{label}
\newcommand{\drawSmoothCurve}[4]{
    \draw[normal conn, -{Stealth[length=2.5mm]}]
        (#1) to[out=45, in=135, distance=#3cm] (#2)
        node[midway, above, font=\tiny, fill=white, inner sep=1pt] {#4};
}

% S-curve connection for parallel lines
% Usage: \drawSCurve{from}{to}{label}
\newcommand{\drawSCurve}[3]{
    \path let \p1 = (#1), \p2 = (#2) in
        coordinate (mid1) at ($(\x1,\y1)!0.33!(\x2,\y2)$)
        coordinate (mid2) at ($(\x1,\y1)!0.67!(\x2,\y2)$);
    \draw[normal conn, -{Stealth[length=2.5mm]}]
        (#1) to[out=0, in=180] (mid1) to[out=0, in=180] (mid2) to[out=0, in=180] (#2)
        node[pos=0.5, above, font=\tiny, fill=white, inner sep=1pt] {#3};
}

% Arc connection for circular layouts
% Usage: \drawArcConnection{from}{to}{radius}{label}
\newcommand{\drawArcConnection}[4]{
    \draw[normal conn, -{Stealth[length=2.5mm]}]
        (#1) to[bend left=45, looseness=1.5] (#2)
        node[midway, above, font=\tiny, fill=white, inner sep=1pt] {#4};
}

% Organic curved connection with multiple control points
% Usage: \drawOrganicCurve{from}{to}{label}
\newcommand{\drawOrganicCurve}[3]{
    \path let \p1 = (#1), \p2 = (#2) in
        \pgfextra{
            \pgfmathsetmacro{\dx}{\x2-\x1}
            \pgfmathsetmacro{\dy}{\y2-\y1}
            \pgfmathsetmacro{\angle}{atan2(\dy,\dx)}
        }
        coordinate (c1) at ($(#1)!0.25!(#2) + (\angle:0.5)$)
        coordinate (c2) at ($(#1)!0.5!(#2) + (\angle+90:0.3)$)
        coordinate (c3) at ($(#1)!0.75!(#2) + (\angle:0.5)$);
    \draw[normal conn, -{Stealth[length=2.5mm]}]
        (#1) .. controls (c1) and (c2) .. (c3) .. controls (c3) .. (#2)
        node[pos=0.5, above, font=\tiny, fill=white, inner sep=1pt] {#3};
}

% TODO: Advanced Bezier features
% - Automatic control point optimization
% - Collision-free curve generation
% - Minimum curvature paths
% - Smooth curve bundling

% ============================================================================
% CONNECTION FLOW VISUALIZATION
% ============================================================================

% Animated flow indicators (requires animation package)
% Usage: \drawFlowConnection{from}{to}{direction}{speed}
\newcommand{\drawFlowConnection}[4]{
    \draw[draw=connNormal, line width=1.5pt, -{Stealth[length=3mm]}] (#1) -- (#2)
        [postaction={
            decorate,
            decoration={
                markings,
                mark=between positions 0.1 and 0.9 step 0.2 with {
                    \ifthenelse{\equal{#3}{forward}}{
                        \arrow{Stealth[length=2mm, fill=connNormal]}
                    }{
                        \arrow{Stealth[reversed, length=2mm, fill=connNormal]}
                    }
                }
            }
        }];
}

% Traffic flow with volume indicator
% Usage: \drawTrafficFlow{from}{to}{packets_per_sec}{label}
\newcommand{\drawTrafficFlow}[4]{
    \pgfmathsetmacro{\density}{min(10, #3/100)}
    \draw[draw=connNormal, line width=1.5pt, -{Stealth[length=3mm]}] 
        (#1) -- node[above, font=\tiny] {#4: #3 pps} (#2)
        [postaction={
            decorate,
            decoration={
                markings,
                mark=between positions 0.1 and 0.9 step {0.1/\density} with {
                    \node[circle, fill=connNormal, inner sep=0.5pt] {};
                }
            }
        }];
}

% TODO: Flow visualization enhancements
% - Real-time traffic animation (if rendering to animated format)
% - Packet visualization with different colors per protocol
% - Congestion indicators (red/yellow traffic markers)
% - Flow direction with multiple arrows
% - Throughput heatmap coloring

% ============================================================================
% CONNECTION RENDERING ENGINE
% ============================================================================

% Main command to render all connections
\newcommand{\renderConnections}{
    % This will be populated by network_data.tex
    % Example structure:
    % \drawConnection{srv1}{pc1}{HTTP}
    % \drawAttackConnection{attacker1}{srv1}{SQL Injection}
}

% ============================================================================
% CONNECTION BUNDLING AND AGGREGATION
% ============================================================================

% Bundled connection (shows multiple connections as one thick line)
% Usage: \drawBundledConnection{from}{to}{count}{label}
\newcommand{\drawBundledConnection}[4]{
    \pgfmathsetmacro{\bundlewidth}{0.5 + sqrt(#3)}
    \draw[draw=connNormal, line width=\bundlewidth pt, -{Stealth[length=3mm]},
          double distance=1pt]
        (#1) -- (#2)
        node[midway, above, font=\tiny\bfseries, fill=white, inner sep=2pt] {#4}
        node[midway, below, font=\tiny, fill=yellow!20, inner sep=1pt] {#3 connections};
}

% Bundled connections with connection count badge
% Usage: \drawConnectionBundle{from}{to}{count}{protocols}{label}
\newcommand{\drawConnectionBundle}[5]{
    \pgfmathsetmacro{\bundlewidth}{0.5 + sqrt(#3)}
    \draw[draw=connNormal!80, line width=\bundlewidth pt, -{Stealth[length=3mm]}]
        (#1) -- (#2);
    % Draw badge with count
    \node[circle, fill=blue!70, text=white, font=\tiny\bfseries,
          minimum size=8pt, inner sep=1pt]
        at ($(#1)!0.5!(#2)$) {#3};
    % Label above
    \node[above=3pt, font=\tiny, fill=white, inner sep=1pt]
        at ($(#1)!0.5!(#2)$) {#5};
    % Protocols below
    \node[below=3pt, font=\tiny\ttfamily, fill=white, inner sep=1pt]
        at ($(#1)!0.5!(#2)$) {#4};
}

% Edge bundling with parallel offset
% Shows multiple connections with slight parallel offsets
% Usage: \drawParallelConnections{from}{to}{count}{label}
\newcommand{\drawParallelConnections}[4]{
    \pgfmathsetmacro{\offsetstep}{0.3}
    \pgfmathsetmacro{\totaloffset}{(#3-1)*\offsetstep/2}
    \foreach \i in {1,...,#3} {
        \pgfmathsetmacro{\offset}{-\totaloffset + (\i-1)*\offsetstep}
        \draw[draw=connNormal, line width=0.5pt, -{Stealth[length=2mm]},
              transform canvas={shift={(0,\offset)}}]
            (#1) -- (#2);
    }
    \node[above, font=\tiny, fill=white, inner sep=1pt]
        at ($(#1)!0.5!(#2)$) {#4 (x#3)};
}

% Hierarchical edge bundling for complex networks
% Bundles connections based on proximity
% Usage: \drawHierarchicalBundle{from}{to}{via}{count}{label}
\newcommand{\drawHierarchicalBundle}[5]{
    \pgfmathsetmacro{\bundlewidth}{0.5 + ln(#4)}
    \draw[draw=connNormal!70, line width=\bundlewidth pt]
        (#1) -- (#3);
    \draw[draw=connNormal, line width=\bundlewidth pt, -{Stealth[length=3mm]}]
        (#3) -- (#2);
    \node[fill=blue!20, circle, inner sep=2pt, font=\tiny\bfseries] at (#3) {#4};
    \node[above, font=\tiny] at (#3) {#5};
}

% TODO: Advanced bundling algorithms
% - Automatic bundle detection based on proximity
% - Force-directed edge bundling
% - Hierarchical edge bundling for tree structures
% - Interactive bundle expansion/collapse
% - Bundle splitting at different zoom levels

% ============================================================================
% CONNECTION LABELS AND ANNOTATIONS
% ============================================================================

% Add inline statistics to connection
% Usage: \labelConnectionStats{from}{to}{latency}{packet_loss}{jitter}
\newcommand{\labelConnectionStats}[5]{
    \draw[normal conn] (#1) -- (#2)
        node[midway, below, font=\tiny\ttfamily, fill=white, inner sep=2pt] {
            L:#3ms | PL:#4\% | J:#5ms
        };
}

% Add threat score to connection
% Usage: \labelConnectionThreat{from}{to}{score}{details}
\newcommand{\labelConnectionThreat}[4]{
    \draw[suspicious conn] (#1) -- (#2)
        node[midway, threat label] {
            \textbf{#3/10} #4
        };
}

% TODO: Label improvements
% - Auto-positioning to avoid overlaps
% - Expandable detail boxes on hover
% - Time-series data for connection metrics
% - Alert indicators for anomalous connections

% ============================================================================
% CONNECTION FILTERING AND LAYERS
% ============================================================================

% Filter flags (set to true/false to enable/disable rendering)
\newif\ifshowNormalConnections
\newif\ifshowEncryptedConnections
\newif\ifshowSuspiciousConnections
\newif\ifshowAttackConnections
\newif\ifshowVPNConnections
\newif\ifshowWirelessConnections

% Default: show all connection types
\showNormalConnectionstrue
\showEncryptedConnectionstrue
\showSuspiciousConnectionstrue
\showAttackConnectionstrue
\showVPNConnectionstrue
\showWirelessConnectionstrue

% Enable specific connection type filter
% Usage: \enableConnectionType{normal|encrypted|suspicious|attack|vpn|wireless}
\newcommand{\enableConnectionType}[1]{
    \ifthenelse{\equal{#1}{normal}}{\showNormalConnectionstrue}{}
    \ifthenelse{\equal{#1}{encrypted}}{\showEncryptedConnectionstrue}{}
    \ifthenelse{\equal{#1}{suspicious}}{\showSuspiciousConnectionstrue}{}
    \ifthenelse{\equal{#1}{attack}}{\showAttackConnectionstrue}{}
    \ifthenelse{\equal{#1}{vpn}}{\showVPNConnectionstrue}{}
    \ifthenelse{\equal{#1}{wireless}}{\showWirelessConnectionstrue}{}
}

% Disable specific connection type filter
% Usage: \disableConnectionType{normal|encrypted|suspicious|attack|vpn|wireless}
\newcommand{\disableConnectionType}[1]{
    \ifthenelse{\equal{#1}{normal}}{\showNormalConnectionsfalse}{}
    \ifthenelse{\equal{#1}{encrypted}}{\showEncryptedConnectionsfalse}{}
    \ifthenelse{\equal{#1}{suspicious}}{\showSuspiciousConnectionsfalse}{}
    \ifthenelse{\equal{#1}{attack}}{\showAttackConnectionsfalse}{}
    \ifthenelse{\equal{#1}{vpn}}{\showVPNConnectionsfalse}{}
    \ifthenelse{\equal{#1}{wireless}}{\showWirelessConnectionsfalse}{}
}

% Show only specific connection types
% Usage: \showOnlyConnectionTypes{normal,encrypted}
\newcommand{\showOnlyConnectionTypes}[1]{
    % Disable all first
    \showNormalConnectionsfalse
    \showEncryptedConnectionsfalse
    \showSuspiciousConnectionsfalse
    \showAttackConnectionsfalse
    \showVPNConnectionsfalse
    \showWirelessConnectionsfalse
    % Enable specified types
    \foreach \conntype in {#1} {
        \enableConnectionType{\conntype}
    }
}

% Port-based filtering
% Usage: \drawConnectionIfPort{from}{to}{port}{target_port}{label}
\newcommand{\drawConnectionIfPort}[5]{
    \pgfmathsetmacro{\portmatch}{ifthenelse(#3==#4,1,0)}
    \ifnum\portmatch=1
        \draw[normal conn, -{Stealth[length=2.5mm]}] (#1) -- (#2)
            node[midway, above, font=\tiny] {#5};
    \fi
}

% Bandwidth threshold filtering
% Usage: \drawConnectionIfBandwidth{from}{to}{bandwidth}{threshold}{label}
\newcommand{\drawConnectionIfBandwidth}[5]{
    \pgfmathsetmacro{\bwcheck}{ifthenelse(#3>=#4,1,0)}
    \ifnum\bwcheck=1
        \pgfmathsetmacro{\linewidth}{0.5 + ln(max(1, #3))/2}
        \draw[draw=connNormal, line width=\linewidth pt, -{Stealth[length=3mm]}]
            (#1) -- (#2)
            node[midway, above, font=\tiny] {#5: #3 Mbps};
    \fi
}

% Layer-based rendering control
% Usage: \beginConnectionLayer{name} ... \endConnectionLayer
\newcommand{\beginConnectionLayer}[1]{
    \begin{scope}[on background layer]
}
\newcommand{\endConnectionLayer}{
    \end{scope}
}

% Z-order controlled connection
% Usage: \drawConnectionWithZOrder{from}{to}{z_order}{label}
\newcommand{\drawConnectionWithZOrder}[4]{
    \begin{scope}[transparency group, opacity=1]
        \draw[normal conn, -{Stealth[length=2.5mm]}] (#1) -- (#2)
            node[midway, above, font=\tiny] {#4};
    \end{scope}
}

% ============================================================================
% CONNECTION PATTERNS AND ANALYSIS
% ============================================================================

% Highlight attack pattern (multiple sources to one target)
% Usage: \highlightAttackPattern{target}{sources}
\newcommand{\highlightAttackPattern}[2]{
    \begin{scope}[on background layer]
        \foreach \source in {#2} {
            \draw[attack conn, line width=2pt] (\source) -- (#1);
        }
        % Draw highlight around target
        \node[draw=threatCritical, line width=3pt, 
              rounded corners=5pt, inner sep=8pt, 
              fill=threatCritical!10] at (#1) {};
    \end{scope}
}

% Show connection path through network
% Usage: \showConnectionPath{node_list}
\newcommand{\showConnectionPath}[1]{
    \foreach \node [count=\i] in {#1} {
        \ifnum\i>1
            \pgfmathtruncatemacro{\prev}{\i-1}
            \draw[draw=blue!70, line width=2pt, -{Stealth[length=3mm]}]
                (node\prev) -- (node\i);
        \fi
    }
}

% ============================================================================
% ATTACK PATTERN DETECTION AND VISUALIZATION
% ============================================================================

% DDoS Pattern Visualization (many-to-one)
% Usage: \highlightDDoSPattern{target}{source_list}
\newcommand{\highlightDDoSPattern}[2]{
    \begin{scope}[on background layer]
        % Draw attacking connections
        \foreach \source in {#2} {
            \draw[draw=red!70, line width=1.5pt, -{Stealth[length=3mm]}]
                (\source) -- (#1);
        }
        % Highlight target with pulsing effect
        \node[draw=red!80, line width=4pt, rounded corners=8pt,
              inner sep=10pt, fill=red!20, double=red!40, double distance=2pt]
            at (#1) {};
        % Add DDoS label
        \node[below=15pt, fill=red!80, text=white, font=\tiny\bfseries,
              rounded corners=2pt, inner sep=2pt]
            at (#1) {DDoS ATTACK};
    \end{scope}
}

% Data Exfiltration Pattern (one-to-many external)
% Usage: \highlightExfiltrationPattern{source}{destination_list}
\newcommand{\highlightExfiltrationPattern}[2]{
    \begin{scope}[on background layer]
        % Draw exfiltration connections
        \foreach \dest in {#2} {
            \draw[draw=orange!70, line width=1.5pt, dashed,
                  dash pattern=on 4pt off 2pt, -{Stealth[length=3mm]}]
                (#1) -- (\dest);
        }
        % Highlight source
        \node[draw=orange!80, line width=3pt, rounded corners=6pt,
              inner sep=8pt, fill=orange!15]
            at (#1) {};
        % Add exfiltration warning
        \node[above=12pt, fill=orange!80, text=white, font=\tiny\bfseries,
              rounded corners=2pt, inner sep=2pt]
            at (#1) {DATA EXFIL};
    \end{scope}
}

% Lateral Movement Pattern (peer-to-peer internal)
% Usage: \highlightLateralMovement{node_list}
\newcommand{\highlightLateralMovement}[1]{
    \begin{scope}[on background layer]
        \foreach \node [count=\i] in {#1} {
            \ifnum\i>1
                \pgfmathtruncatemacro{\prev}{\i-1}
                \pgfmathtruncatemacro{\prevnode}{\prev}
                % Draw lateral movement connection
                \draw[draw=purple!70, line width=2pt, dashed,
                      dash pattern=on 3pt off 2pt, -{Stealth[length=3mm]},
                      postaction={draw, line width=0.5pt, draw=purple!30}]
                    (lat\prevnode) -- (lat\i);
            \fi
            % Highlight compromised nodes
            \node[draw=purple!70, line width=2pt, circle,
                  inner sep=6pt, fill=purple!10]
                (lat\i) at (\node) {};
        }
        % Add lateral movement label
        \node[fill=purple!80, text=white, font=\tiny\bfseries,
              rounded corners=2pt, inner sep=2pt]
            at (lat1) {LATERAL MOVE};
    \end{scope}
}

% Command & Control (C2) Beaconing Pattern
% Usage: \highlightC2Pattern{compromised}{c2_server}{label}
\newcommand{\highlightC2Pattern}[3]{
    % Bidirectional beaconing connection
    \draw[draw=red!60, line width=1.5pt, bidirectional,
          decoration={snake, amplitude=0.5mm, segment length=5mm}, decorate]
        (#1) -- (#2);
    % Beacon indicator dots
    \foreach \pos in {0.2, 0.4, 0.6, 0.8} {
        \node[circle, fill=red!70, inner sep=1.5pt] at ($(#1)!\pos!(#2)$) {};
    }
    % C2 labels
    \node[above, fill=red!80, text=white, font=\tiny\bfseries,
          rounded corners=2pt, inner sep=1pt]
        at ($(#1)!0.5!(#2)$) {#3};
    \node[below, font=\tiny, fill=red!15, inner sep=1pt]
        at ($(#1)!0.5!(#2)$) {C2 Beacon};
}

% Port Scanning Pattern (one-to-many same port)
% Usage: \highlightPortScanPattern{scanner}{target_list}{port}
\newcommand{\highlightPortScanPattern}[3]{
    \begin{scope}[on background layer]
        % Draw scanning connections
        \foreach \target in {#2} {
            \draw[draw=yellow!70!orange, line width=1pt,
                  decoration={markings,
                      mark=at position 0.5 with {\arrow{Stealth[length=2mm]}}},
                  postaction={decorate}]
                (#1) -- (\target);
        }
        % Scanner indicator
        \node[draw=yellow!80!orange, line width=3pt, star, star points=5,
              inner sep=4pt, fill=yellow!20]
            at (#1) {};
        % Scan label
        \node[below=10pt, fill=yellow!80!orange, text=white,
              font=\tiny\bfseries, rounded corners=2pt, inner sep=2pt]
            at (#1) {PORT SCAN: #3};
    \end{scope}
}

% Reconnaissance Pattern
% Usage: \highlightReconPattern{attacker}{targets}
\newcommand{\highlightReconPattern}[2]{
    \begin{scope}[on background layer]
        \foreach \target in {#2} {
            \draw[draw=cyan!60, line width=0.8pt, dashed,
                  dash pattern=on 2pt off 1pt, ->]
                (#1) -- (\target);
        }
        \node[draw=cyan!70, line width=2pt, rounded corners=4pt,
              inner sep=6pt, fill=cyan!10]
            at (#1) {};
        \node[above=8pt, fill=cyan!70, text=white, font=\tiny\bfseries,
              rounded corners=2pt, inner sep=1pt]
            at (#1) {RECON};
    \end{scope}
}

% Man-in-the-Middle (MITM) Pattern
% Usage: \highlightMITMPattern{source}{mitm}{destination}
\newcommand{\highlightMITMPattern}[3]{
    % Connection from source to MITM
    \draw[draw=red!70, line width=1.5pt, -{Stealth[length=3mm]}]
        (#1) -- (#2);
    % Connection from MITM to destination
    \draw[draw=red!70, line width=1.5pt, -{Stealth[length=3mm]}]
        (#2) -- (#3);
    % Highlight MITM node
    \node[draw=red!80, line width=3pt, diamond, aspect=2,
          inner sep=8pt, fill=red!15]
        at (#2) {};
    % MITM warning
    \node[right=8pt, fill=red!80, text=white, font=\tiny\bfseries,
          rounded corners=2pt, inner sep=2pt]
        at (#2) {MITM};
}

% ============================================================================
% CONNECTION STATISTICS AND METRICS
% ============================================================================

% Connection type counters
\newcounter{normalConnCount}
\newcounter{encryptedConnCount}
\newcounter{suspiciousConnCount}
\newcounter{attackConnCount}
\newcounter{vpnConnCount}
\newcounter{wirelessConnCount}
\newcounter{totalBandwidth}

% Increment connection counters
\newcommand{\incrementNormalConn}{\stepcounter{normalConnCount}}
\newcommand{\incrementEncryptedConn}{\stepcounter{encryptedConnCount}}
\newcommand{\incrementSuspiciousConn}{\stepcounter{suspiciousConnCount}}
\newcommand{\incrementAttackConn}{\stepcounter{attackConnCount}}
\newcommand{\incrementVPNConn}{\stepcounter{vpnConnCount}}
\newcommand{\incrementWirelessConn}{\stepcounter{wirelessConnCount}}

% Enhanced connection statistics summary
% Usage: \drawConnectionSummary{x}{y}
\newcommand{\drawConnectionSummary}[2]{
    \node[fill=white, draw=black!30, line width=1pt, rounded corners=3pt,
          anchor=north west, inner sep=5pt] at (#1,#2) {
        \begin{tabular}{lr}
            \multicolumn{2}{c}{\textbf{Connection Statistics}} \\
            \hline
            Normal & \thenormalConnCount \\
            Encrypted & \theencryptedConnCount \\
            Suspicious & \thesuspiciousConnCount \\
            Attacks & \theattackConnCount \\
            VPN & \thevpnConnCount \\
            Wireless & \thewirelessConnCount \\
            \hline
            \textbf{Total} & \pgfmathparse{int(\thenormalConnCount+\theencryptedConnCount+\thesuspiciousConnCount+\theattackConnCount+\thevpnConnCount+\thewirelessConnCount)}\pgfmathresult \\
        \end{tabular}
    };
}

% Detailed connection dashboard
% Usage: \drawConnectionDashboard{x}{y}{total_bw}
\newcommand{\drawConnectionDashboard}[3]{
    \node[fill=blue!5, draw=blue!50, line width=1.5pt, rounded corners=5pt,
          anchor=north west, inner sep=8pt] at (#1,#2) {
        \begin{tabular}{lc}
            \multicolumn{2}{c}{\Large\textbf{Network Overview}} \\
            \hline
            & \\
            \textbf{Connections} & \\
            \quad Normal & \color{green!60!black}\thenormalConnCount \\
            \quad Encrypted & \color{blue!70}\theencryptedConnCount \\
            \quad Suspicious & \color{yellow!80!orange}\thesuspiciousConnCount \\
            \quad Attacks & \color{red!70}\theattackConnCount \\
            & \\
            \hline
            & \\
            \textbf{Bandwidth} & #3 Gbps \\
            \textbf{Utilization} & \pgfmathparse{int(rand*100)}\pgfmathresult\% \\
            & \\
            \hline
            & \\
            \textbf{Security Status} & \\
            \quad Threats & \theattackConnCount \\
            \quad Risk Level & \ifthenelse{\theattackConnCount>5}{\color{red}HIGH}{\color{green}LOW} \\
        \end{tabular}
    };
}

% Protocol distribution visualization
% Usage: \drawProtocolDistribution{x}{y}{tcp_pct}{udp_pct}{other_pct}
\newcommand{\drawProtocolDistribution}[5]{
    \node[fill=white, draw=black!30, line width=1pt, rounded corners=3pt,
          anchor=north west, inner sep=5pt] at (#1,#2) {
        \begin{tabular}{lcc}
            \multicolumn{3}{c}{\textbf{Protocol Distribution}} \\
            \hline
            Protocol & Count & \% \\
            \hline
            TCP & \pgfmathparse{int(#3)}\pgfmathresult & #3\% \\
            UDP & \pgfmathparse{int(#4)}\pgfmathresult & #4\% \\
            Other & \pgfmathparse{int(#5)}\pgfmathresult & #5\% \\
        \end{tabular}
    };
    % Simple bar visualization
    \pgfmathsetmacro{\tcpbar}{#3/10}
    \pgfmathsetmacro{\udpbar}{#4/10}
    \pgfmathsetmacro{\otherbar}{#5/10}
    \draw[fill=blue!70] (#1+3,#2-1) rectangle +({\tcpbar},0.3);
    \draw[fill=green!70] (#1+3,#2-1.5) rectangle +({\udpbar},0.3);
    \draw[fill=orange!70] (#1+3,#2-2) rectangle +({\otherbar},0.3);
}

% Top talkers list
% Usage: \drawTopTalkers{x}{y}{talker1}{bw1}{talker2}{bw2}{talker3}{bw3}
\newcommand{\drawTopTalkers}[8]{
    \node[fill=yellow!10, draw=yellow!60, line width=1pt, rounded corners=3pt,
          anchor=north west, inner sep=5pt] at (#1,#2) {
        \begin{tabular}{lc}
            \multicolumn{2}{c}{\textbf{Top Talkers}} \\
            \hline
            Node & Bandwidth \\
            \hline
            #3 & #4 Mbps \\
            #5 & #6 Mbps \\
            #7 & #8 Mbps \\
        \end{tabular}
    };
}

% Connection quality meter
% Usage: \drawConnectionQualityMeter{x}{y}{quality_score}
\newcommand{\drawConnectionQualityMeter}[3]{
    \pgfmathsetmacro{\quality}{#3}
    % Determine color based on quality
    \pgfmathsetmacro{\qualitycolor}{
        \quality>80 ? "green!70" : (
        \quality>50 ? "yellow!80!orange" : "red!70"
        )
    }
    \node[fill=white, draw=black!30, line width=1pt, rounded corners=3pt,
          anchor=north west, inner sep=5pt] at (#1,#2) {
        \begin{tabular}{c}
            \textbf{Network Quality} \\
            \hline
            \\
            \Huge #3\% \\
            \\
        \end{tabular}
    };
    % Quality bar
    \pgfmathsetmacro{\qualitybar}{#3/20}
    \draw[line width=8pt, draw=gray!30, rounded corners=2pt]
        (#1+0.5,#2-2) -- +(5,0);
    \draw[line width=8pt, draw=\qualitycolor, rounded corners=2pt]
        (#1+0.5,#2-2) -- +({\qualitybar},0);
}

% ============================================================================
% CONNECTION STATE MANAGEMENT
% ============================================================================

% Connection states: active, inactive, intermittent, failed
% Active connection (solid line with strong color)
% Usage: \drawActiveConnection{from}{to}{label}
\newcommand{\drawActiveConnection}[3]{
    \draw[draw=green!70, line width=1.5pt, -{Stealth[length=3mm]}]
        (#1) -- (#2)
        node[midway, above, font=\tiny, fill=white, inner sep=1pt] {#3}
        node[midway, below, font=\tiny, fill=green!15, inner sep=1pt] {ACTIVE};
}

% Inactive connection (dashed line with muted color)
% Usage: \drawInactiveConnection{from}{to}{label}
\newcommand{\drawInactiveConnection}[3]{
    \draw[draw=gray!50, line width=0.8pt, dashed,
          dash pattern=on 2pt off 2pt, -{Stealth[length=2mm]}]
        (#1) -- (#2)
        node[midway, above, font=\tiny, fill=white, inner sep=1pt, text=gray] {#3}
        node[midway, below, font=\tiny, fill=gray!10, inner sep=1pt] {INACTIVE};
}

% Intermittent connection (dotted line)
% Usage: \drawIntermittentConnection{from}{to}{uptime_pct}{label}
\newcommand{\drawIntermittentConnection}[4]{
    \draw[draw=yellow!70!orange, line width=1pt, dotted,
          -{Stealth[length=2.5mm]}]
        (#1) -- (#2)
        node[midway, above, font=\tiny, fill=white, inner sep=1pt] {#4}
        node[midway, below, font=\tiny, fill=yellow!15, inner sep=1pt] {FLAPPING: #3\%};
}

% Failed connection (red cross pattern)
% Usage: \drawFailedConnection{from}{to}{reason}
\newcommand{\drawFailedConnection}[3]{
    \draw[draw=red!70, line width=1.5pt, dashed,
          dash pattern=on 1pt off 2pt, decoration={markings,
              mark=between positions 0.2 and 0.8 step 0.2 with {
                  \node[cross out, draw=red!70, minimum size=2pt] {};
              }}, postaction={decorate}]
        (#1) -- (#2)
        node[midway, above, font=\tiny, fill=white, inner sep=1pt, text=red!70] {#3}
        node[midway, below, font=\tiny, fill=red!15, inner sep=1pt] {FAILED};
}

% Connection with health indicator
% Usage: \drawConnectionWithHealth{from}{to}{health_pct}{label}
\newcommand{\drawConnectionWithHealth}[4]{
    \pgfmathsetmacro{\health}{#3}
    % Color based on health
    \pgfmathsetmacro{\healthval}{\health}
    \ifnum\healthval>75
        \def\healthColor{green!70}
    \else
        \ifnum\healthval>50
            \def\healthColor{yellow!80!orange}
        \else
            \def\healthColor{red!70}
        \fi
    \fi
    \draw[draw=\healthColor, line width=1.2pt, -{Stealth[length=3mm]}]
        (#1) -- (#2)
        node[midway, above, font=\tiny, fill=white, inner sep=1pt] {#4}
        node[midway, below, font=\tiny, fill=\healthColor!15, inner sep=1pt] {Health: #3\%};
}

% ============================================================================
% CONNECTION QUALITY INDICATORS
% ============================================================================

% High quality connection (thick green line)
% Usage: \drawHighQualityConnection{from}{to}{metrics}{label}
\newcommand{\drawHighQualityConnection}[4]{
    \draw[draw=green!70, line width=2pt, -{Stealth[length=3mm]},
          double=green!30, double distance=0.5pt]
        (#1) -- (#2)
        node[midway, above, font=\tiny\bfseries, fill=white, inner sep=2pt] {#4}
        node[midway, below, font=\tiny, fill=green!20, inner sep=1pt] {#3};
}

% Degraded connection (yellow with warning indicators)
% Usage: \drawDegradedConnection{from}{to}{issue}{label}
\newcommand{\drawDegradedConnection}[4]{
    \draw[draw=yellow!80!orange, line width=1.5pt, -{Stealth[length=3mm]},
          decoration={markings,
              mark=between positions 0.25 and 0.75 step 0.25 with {
                  \node[regular polygon, regular polygon sides=3,
                        draw=yellow!80!orange, fill=yellow!30,
                        minimum size=3pt, inner sep=0pt] {};
              }}, postaction={decorate}]
        (#1) -- (#2)
        node[midway, above, font=\tiny, fill=white, inner sep=1pt] {#4}
        node[midway, below, font=\tiny, fill=yellow!20, inner sep=1pt] {DEGRADED: #3};
}

% Connection with QoS indicators
% Usage: \drawQoSConnection{from}{to}{qos_class}{latency}{label}
\newcommand{\drawQoSConnection}[5]{
    % QoS class: platinum, gold, silver, bronze
    \ifthenelse{\equal{#3}{platinum}}{
        \def\qosColor{purple!70}
        \def\qosWidth{2.5pt}
    }{
        \ifthenelse{\equal{#3}{gold}}{
            \def\qosColor{yellow!80!orange}
            \def\qosWidth{2pt}
        }{
            \ifthenelse{\equal{#3}{silver}}{
                \def\qosColor{gray!60}
                \def\qosWidth{1.5pt}
            }{
                \def\qosColor{brown!60}
                \def\qosWidth{1pt}
            }
        }
    }
    \draw[draw=\qosColor, line width=\qosWidth, -{Stealth[length=3mm]}]
        (#1) -- (#2)
        node[midway, above, font=\tiny, fill=white, inner sep=1pt] {#5}
        node[midway, below, font=\tiny, fill=\qosColor!15, inner sep=1pt] {QoS: #3 | #4ms};
}

% Connection with packet loss indicator
% Usage: \drawConnectionWithLoss{from}{to}{loss_pct}{label}
\newcommand{\drawConnectionWithLoss}[4]{
    \pgfmathsetmacro{\loss}{#3}
    \ifnum\loss>10
        \def\lossColor{red!70}
    \else
        \ifnum\loss>2
            \def\lossColor{yellow!80!orange}
        \else
            \def\lossColor{green!60}
        \fi
    \fi
    % Draw connection with gaps representing packet loss
    \draw[draw=\lossColor, line width=1.2pt,
          dash pattern=on 8pt off \loss pt, -{Stealth[length=3mm]}]
        (#1) -- (#2)
        node[midway, above, font=\tiny, fill=white, inner sep=1pt] {#4}
        node[midway, below, font=\tiny, fill=\lossColor!15, inner sep=1pt] {Loss: #3\%};
}

% Connection with jitter visualization
% Usage: \drawConnectionWithJitter{from}{to}{jitter_ms}{label}
\newcommand{\drawConnectionWithJitter}[4]{
    \pgfmathsetmacro{\jitter}{#3}
    \ifnum\jitter>50
        \def\jitterAmp}{0.8mm}
        \def\jitterColor{red!70}
    \else
        \ifnum\jitter>20
            \def\jitterAmp}{0.4mm}
            \def\jitterColor{yellow!80!orange}
        \else
            \def\jitterAmp}{0.2mm}
            \def\jitterColor{green!60}
        \fi
    \fi
    \draw[draw=\jitterColor, line width=1pt,
          decoration={snake, amplitude=\jitterAmp, segment length=3mm},
          decorate, -{Stealth[length=2.5mm]}]
        (#1) -- (#2)
        node[midway, above, font=\tiny, fill=white, inner sep=1pt] {#4}
        node[midway, below, font=\tiny, fill=\jitterColor!15, inner sep=1pt] {Jitter: #3ms};
}

% SLA compliance indicator
% Usage: \drawSLAConnection{from}{to}{sla_pct}{label}
\newcommand{\drawSLAConnection}[4]{
    \pgfmathsetmacro{\sla}{#3}
    \ifnum\sla>99
        \def\slaColor{green!70}
        \def\slaStatus}{COMPLIANT}
    \else
        \ifnum\sla>95
            \def\slaColor{yellow!80!orange}
            \def\slaStatus}{WARNING}
        \else
            \def\slaColor{red!70}
            \def\slaStatus}{VIOLATION}
        \fi
    \fi
    \draw[draw=\slaColor, line width=1.5pt, -{Stealth[length=3mm]}]
        (#1) -- (#2)
        node[midway, above, font=\tiny, fill=white, inner sep=1pt] {#4}
        node[midway, below, font=\tiny\bfseries, fill=\slaColor!20, inner sep=2pt]
            {SLA: #3\% - \slaStatus};
}
}

% ============================================================================
% ADVANCED CONNECTION HIGHLIGHTING AND EMPHASIS
% ============================================================================

% Highlight specific connection with glow effect
% Usage: \highlightConnection{from}{to}{color}{label}
\newcommand{\highlightConnection}[4]{
    % Background glow layers
    \draw[draw=#3!30, line width=6pt, opacity=0.3] (#1) -- (#2);
    \draw[draw=#3!50, line width=4pt, opacity=0.5] (#1) -- (#2);
    % Main connection
    \draw[draw=#3, line width=2pt, -{Stealth[length=3mm]}] (#1) -- (#2)
        node[midway, above, font=\tiny\bfseries, fill=white, inner sep=2pt] {#4};
}

% Pulsing connection (for critical paths)
% Usage: \drawPulsingConnection{from}{to}{label}
\newcommand{\drawPulsingConnection}[3]{
    \draw[draw=red!70, line width=2pt, -{Stealth[length=3mm]},
          decoration={markings,
              mark=between positions 0.1 and 0.9 step 0.1 with {
                  \node[circle, fill=red!70, inner sep=1.5pt, opacity=0.8] {};
              }}, postaction={decorate}]
        (#1) -- (#2)
        node[midway, above, font=\tiny\bfseries, fill=red!10, inner sep=2pt] {#3}
        node[midway, below, font=\tiny, fill=red!20, inner sep=1pt] {CRITICAL};
}

% Emphasized connection with shadow
% Usage: \drawEmphasizedConnection{from}{to}{label}
\newcommand{\drawEmphasizedConnection}[3]{
    % Shadow
    \draw[draw=black!30, line width=2pt, -{Stealth[length=3mm]}]
        ([shift={(0.5pt,-0.5pt)}]#1) -- ([shift={(0.5pt,-0.5pt)}]#2);
    % Main connection
    \draw[draw=blue!70, line width=2pt, -{Stealth[length=3mm]}] (#1) -- (#2)
        node[midway, above, font=\tiny\bfseries, fill=white, inner sep=2pt,
             drop shadow] {#3};
}

% Annotated connection with markers
% Usage: \drawAnnotatedConnection{from}{to}{marker_positions}{label}
\newcommand{\drawAnnotatedConnection}[4]{
    \draw[normal conn, -{Stealth[length=2.5mm]}] (#1) -- (#2)
        node[midway, above, font=\tiny, fill=white, inner sep=1pt] {#4};
    % Add markers at specified positions
    \foreach \pos in {#3} {
        \node[circle, draw=blue!70, fill=blue!20, inner sep=1pt,
              font=\tiny\bfseries] at ($(#1)!\pos!(#2)$) {!};
    }
}

% Connection with directional emphasis
% Usage: \drawDirectionalEmphasis{from}{to}{label}
\newcommand{\drawDirectionalEmphasis}[3]{
    \draw[draw=purple!70, line width=1.5pt,
          decoration={markings,
              mark=between positions 0.2 and 0.8 step 0.15 with {
                  \arrow[scale=1.5]{Stealth[length=3mm, fill=purple!70]}
              }}, postaction={decorate}]
        (#1) -- (#2)
        node[midway, above, font=\tiny, fill=white, inner sep=1pt] {#3};
}

% ============================================================================
% LOAD BALANCING AND REDUNDANCY VISUALIZATION
% ============================================================================

% Load balanced connection group
% Usage: \drawLoadBalancedConnections{from}{lb_node}{targets}{algorithm}
\newcommand{\drawLoadBalancedConnections}[4]{
    % Connection to load balancer
    \draw[draw=green!60!blue, line width=2pt, -{Stealth[length=3mm]}]
        (#1) -- (#2);
    % Load balancer node emphasis
    \node[draw=green!60!blue, line width=2pt, circle, inner sep=8pt,
          fill=green!10] at (#2) {};
    \node[above=10pt, fill=green!60!blue, text=white, font=\tiny\bfseries,
          rounded corners=2pt, inner sep=2pt] at (#2) {LB: #4};
    % Connections to targets
    \foreach \target in {#3} {
        \draw[draw=green!60!blue, line width=1pt, -{Stealth[length=2.5mm]}]
            (#2) -- (\target);
    }
}

% Redundant connection pair
% Usage: \drawRedundantConnections{from}{to}{label}
\newcommand{\drawRedundantConnections}[3]{
    % Primary path (top)
    \draw[draw=green!70, line width=1.5pt, -{Stealth[length=3mm]},
          transform canvas={yshift=2pt}]
        (#1) -- (#2);
    % Secondary path (bottom)
    \draw[draw=orange!70, line width=1.5pt, dashed,
          dash pattern=on 3pt off 2pt, -{Stealth[length=3mm]},
          transform canvas={yshift=-2pt}]
        (#1) -- (#2);
    % Label
    \node[above=5pt, font=\tiny, fill=white, inner sep=1pt] at ($(#1)!0.5!(#2)$) {#3};
    \node[below=5pt, font=\tiny, fill=green!15, inner sep=1pt]
        at ($(#1)!0.5!(#2)$) {Primary};
    \node[below=10pt, font=\tiny, fill=orange!15, inner sep=1pt]
        at ($(#1)!0.5!(#2)$) {Backup};
}

% Active-Active redundancy
% Usage: \drawActiveActiveConnections{from}{to}{label}
\newcommand{\drawActiveActiveConnections}[3]{
    % Dual active paths
    \draw[draw=blue!70, line width=1.5pt, -{Stealth[length=3mm]},
          transform canvas={yshift=2pt}]
        (#1) -- (#2);
    \draw[draw=blue!70, line width=1.5pt, -{Stealth[length=3mm]},
          transform canvas={yshift=-2pt}]
        (#1) -- (#2);
    % Label
    \node[above=5pt, font=\tiny\bfseries, fill=blue!15, inner sep=2pt]
        at ($(#1)!0.5!(#2)$) {#3 (A-A)};
}

% Failover connection
% Usage: \drawFailoverConnection{from}{to}{state}{label}
\newcommand{\drawFailoverConnection}[4]{
    \ifthenelse{\equal{#3}{active}}{
        \def\failColor{green!70}
        \def\failState{ACTIVE}
        \def\failStyle{solid}
    }{
        \def\failColor{gray!60}
        \def\failState{STANDBY}
        \def\failStyle{dashed}
    }
    \draw[draw=\failColor, line width=1.5pt, \failStyle,
          -{Stealth[length=3mm]}] (#1) -- (#2)
        node[midway, above, font=\tiny, fill=white, inner sep=1pt] {#4}
        node[midway, below, font=\tiny, fill=\failColor!15, inner sep=1pt] {\failState};
}

% ============================================================================
% MULTI-PATH ROUTING VISUALIZATION
% ============================================================================

% ECMP (Equal-Cost Multi-Path) routing
% Usage: \drawECMPPaths{from}{to}{path_count}{label}
\newcommand{\drawECMPPaths}[4]{
    \pgfmathsetmacro{\spacing}{2/#3}
    \foreach \i in {1,...,#3} {
        \pgfmathsetmacro{\yoffset}{-1 + (\i-0.5)*\spacing}
        \draw[draw=blue!60, line width=0.8pt, -{Stealth[length=2mm]},
              transform canvas={yshift=\yoffset pt}]
            (#1) -- (#2);
    }
    \node[above=4pt, font=\tiny\bfseries, fill=blue!10, inner sep=2pt]
        at ($(#1)!0.5!(#2)$) {#4};
    \node[below=4pt, font=\tiny, fill=blue!15, inner sep=1pt]
        at ($(#1)!0.5!(#2)$) {ECMP: #3 paths};
}

% Primary and backup path
% Usage: \drawPrimaryBackupPaths{from}{to}{via_primary}{via_backup}{label}
\newcommand{\drawPrimaryBackupPaths}[5]{
    % Primary path
    \draw[draw=green!70, line width=1.5pt, -{Stealth[length=3mm]}]
        (#1) -- (#3) -- (#2)
        node[pos=0.25, above, font=\tiny, fill=green!15, inner sep=1pt] {Primary};
    % Backup path
    \draw[draw=orange!70, line width=1pt, dashed,
          dash pattern=on 3pt off 2pt, -{Stealth[length=2.5mm]}]
        (#1) -- (#4) -- (#2)
        node[pos=0.75, below, font=\tiny, fill=orange!15, inner sep=1pt] {Backup};
    % Main label
    \node[above=8pt, font=\tiny\bfseries, fill=white, inner sep=2pt]
        at ($(#1)!0.5!(#2)$) {#5};
}

% Meshed redundant paths
% Usage: \drawMeshedPaths{nodes}{label}
\newcommand{\drawMeshedPaths}[2]{
    \foreach \source [count=\si] in {#1} {
        \foreach \target [count=\ti] in {#1} {
            \ifnum\si<\ti
                \draw[draw=blue!50, line width=0.5pt, opacity=0.6,
                      -{Stealth[length=1.5mm]}]
                    (\source) -- (\target);
            \fi
        }
    }
    % Mesh label
    \node[fill=blue!20, font=\tiny\bfseries, rounded corners=2pt,
          inner sep=2pt] {#2};
}

% ============================================================================
% NETWORK SEGMENTATION AND ISOLATION
% ============================================================================

% Firewall-protected connection
% Usage: \drawFirewallConnection{from}{to}{firewall}{rules}{label}
\newcommand{\drawFirewallConnection}[5]{
    % Connection to firewall
    \draw[draw=red!70, line width=1.2pt, -{Stealth[length=3mm]}]
        (#1) -- (#3);
    % Firewall symbol
    \node[draw=red!70, line width=2pt, rectangle, inner sep=6pt,
          fill=red!10, pattern=north east lines, pattern color=red!30]
        at (#3) {};
    \node[above=8pt, fill=red!70, text=white, font=\tiny\bfseries,
          rounded corners=2pt, inner sep=1pt] at (#3) {FW};
    \node[below=8pt, font=\tiny, fill=red!15, inner sep=1pt,
          align=center] at (#3) {#4};
    % Connection from firewall
    \draw[draw=red!70, line width=1.2pt, -{Stealth[length=3mm]}]
        (#3) -- (#2)
        node[midway, above, font=\tiny] {#5};
}

% DMZ connection
% Usage: \drawDMZConnection{from}{dmz_node}{to}{label}
\newcommand{\drawDMZConnection}[4]{
    \draw[draw=yellow!70!orange, line width=1.2pt, -{Stealth[length=3mm]}]
        (#1) -- (#2);
    \draw[draw=yellow!70!orange, line width=1.2pt, -{Stealth[length=3mm]}]
        (#2) -- (#3);
    % DMZ indicator
    \node[draw=yellow!70!orange, line width=2pt, trapezium,
          trapezium left angle=70, trapezium right angle=110,
          inner sep=5pt, fill=yellow!15] at (#2) {};
    \node[above=8pt, fill=yellow!70!orange, text=white,
          font=\tiny\bfseries, rounded corners=2pt, inner sep=1pt]
        at (#2) {DMZ};
    \node[midway, above, font=\tiny] at ($(#1)!0.5!(#3)$) {#4};
}

% Air-gapped connection (no direct path)
% Usage: \drawAirGap{from}{to}{label}
\newcommand{\drawAirGap}[3]{
    % Draw partial connections
    \draw[draw=red!70, line width=1pt, dashed, dash pattern=on 2pt off 2pt]
        (#1) -- ($(#1)!0.45!(#2)$);
    \draw[draw=red!70, line width=1pt, dashed, dash pattern=on 2pt off 2pt]
        ($(#1)!0.55!(#2)$) -- (#2);
    % Air gap indicator
    \node[fill=red!80, text=white, font=\tiny\bfseries, circle,
          inner sep=3pt] at ($(#1)!0.5!(#2)$) {X};
    \node[above=5pt, fill=red!15, font=\tiny, inner sep=1pt]
        at ($(#1)!0.5!(#2)$) {#3};
    \node[below=5pt, fill=red!20, font=\tiny\bfseries, inner sep=1pt]
        at ($(#1)!0.5!(#2)$) {AIR GAP};
}

% Trust boundary crossing
% Usage: \drawTrustBoundary{from}{to}{from_zone}{to_zone}{label}
\newcommand{\drawTrustBoundary}[5]{
    \draw[draw=purple!70, line width=1.5pt, -{Stealth[length=3mm]}]
        (#1) -- (#2);
    % Boundary marker
    \node[regular polygon, regular polygon sides=6, draw=purple!70,
          line width=2pt, fill=purple!15, minimum size=10pt]
        at ($(#1)!0.5!(#2)$) {};
    % Zone labels
    \node[above=6pt, font=\tiny, fill=purple!10, inner sep=1pt]
        at ($(#1)!0.5!(#2)$) {#5};
    \node[below=6pt, font=\tiny, fill=white, inner sep=1pt]
        at ($(#1)!0.5!(#2)$) {#3 $\rightarrow$ #4};
}

% ============================================================================
% ADVANCED ANNOTATION AND CALLOUTS
% ============================================================================

% Connection with detailed callout
% Usage: \drawConnectionWithCallout{from}{to}{callout_text}{callout_pos}{label}
\newcommand{\drawConnectionWithCallout}[5]{
    \draw[normal conn, -{Stealth[length=2.5mm]}] (#1) -- (#2)
        node[midway, above, font=\tiny] {#5};
    % Callout
    \node[rectangle callout, callout absolute pointer={($(#1)!#4!(#2)$)},
          fill=yellow!20, draw=orange!70, line width=0.5pt,
          font=\tiny, inner sep=3pt, align=left]
        at ($(#1)!#4!(#2) + (1.5,1)$) {#3};
}

% Connection with inline metric badges
% Usage: \drawConnectionWithBadges{from}{to}{badge1}{badge2}{label}
\newcommand{\drawConnectionWithBadges}[5]{
    \draw[normal conn, -{Stealth[length=2.5mm]}] (#1) -- (#2)
        node[midway, above, font=\tiny] {#5};
    % Badges
    \node[circle, fill=blue!70, text=white, font=\tiny\bfseries,
          minimum size=6pt, inner sep=1pt]
        at ($(#1)!0.33!(#2)$) {#3};
    \node[circle, fill=green!70, text=white, font=\tiny\bfseries,
          minimum size=6pt, inner sep=1pt]
        at ($(#1)!0.67!(#2)$) {#4};
}

% Connection with performance heatmap
% Usage: \drawHeatmapConnection{from}{to}{temp_value}{label}
\newcommand{\drawHeatmapConnection}[4]{
    \pgfmathsetmacro{\temp}{#3}
    % Color gradient based on "temperature" (0-100)
    \ifnum\temp<25
        \def\heatColor{blue!70}
    \else
        \ifnum\temp<50
            \def\heatColor{green!70}
        \else
            \ifnum\temp<75
                \def\heatColor{yellow!80!orange}
            \else
                \def\heatColor{red!70}
            \fi
        \fi
    \fi
    \draw[draw=\heatColor, line width=2pt, -{Stealth[length=3mm]}]
        (#1) -- (#2)
        node[midway, above, font=\tiny\bfseries, fill=white, inner sep=1pt] {#4}
        node[midway, below, font=\tiny, fill=\heatColor!20, inner sep=1pt]
            {Load: #3\%};
}

% ============================================================================
% COMPLIANCE AND AUDIT FEATURES
% ============================================================================

% PCI-DSS compliant connection
% Usage: \drawPCICompliantConnection{from}{to}{encryption}{label}
\newcommand{\drawPCICompliantConnection}[4]{
    \draw[draw=green!60!blue, line width=1.5pt, double=green!30,
          double distance=1pt, -{Stealth[length=3mm]}]
        (#1) -- (#2)
        node[midway, above, font=\tiny\bfseries, fill=white, inner sep=2pt] {#4}
        node[midway, below, font=\tiny, fill=green!15, inner sep=2pt]
            {PCI-DSS | #3};
}

% HIPAA-compliant connection
% Usage: \drawHIPAAConnection{from}{to}{phi_level}{label}
\newcommand{\drawHIPAAConnection}[4]{
    \draw[draw=blue!70, line width=1.5pt, -{Stealth[length=3mm]},
          postaction={draw, line width=0.5pt, draw=blue!30}]
        (#1) -- (#2)
        node[midway, above, font=\tiny\bfseries, fill=white, inner sep=2pt] {#4}
        node[midway, below, font=\tiny, fill=blue!15, inner sep=2pt]
            {HIPAA | PHI: #3};
}

% Audited connection with compliance badge
% Usage: \drawAuditedConnection{from}{to}{standard}{status}{label}
\newcommand{\drawAuditedConnection}[5]{
    \ifthenelse{\equal{#4}{pass}}{
        \def\auditColor{green!70}
        \def\auditSymbol{\checkmark}
    }{
        \def\auditColor{red!70}
        \def\auditSymbol{X}
    }
    \draw[draw=\auditColor, line width=1.5pt, -{Stealth[length=3mm]}]
        (#1) -- (#2)
        node[midway, above, font=\tiny, fill=white, inner sep=1pt] {#5};
    % Compliance badge
    \node[circle, fill=\auditColor, text=white, font=\tiny\bfseries,
          minimum size=8pt, inner sep=1pt]
        at ($(#1)!0.5!(#2)$) {\auditSymbol};
    \node[below=8pt, font=\tiny, fill=\auditColor!15, inner sep=1pt]
        at ($(#1)!0.5!(#2)$) {#3};
}

% Zero Trust connection
% Usage: \drawZeroTrustConnection{from}{to}{verification}{label}
\newcommand{\drawZeroTrustConnection}[4]{
    \draw[draw=purple!70, line width=1.5pt, -{Stealth[length=3mm]},
          decoration={markings,
              mark=between positions 0.25 and 0.75 step 0.25 with {
                  \node[regular polygon, regular polygon sides=8,
                        draw=purple!70, fill=purple!20, minimum size=3pt] {};
              }}, postaction={decorate}]
        (#1) -- (#2)
        node[midway, above, font=\tiny\bfseries, fill=white, inner sep=2pt] {#4}
        node[midway, below, font=\tiny, fill=purple!15, inner sep=2pt]
            {Zero Trust | #3};
}

% ============================================================================
% CONNECTION COMPARISON AND DIFF VISUALIZATION
% ============================================================================

% Show connection before/after
% Usage: \drawConnectionDiff{from}{to}{old_label}{new_label}
\newcommand{\drawConnectionDiff}[4]{
    % Old connection (faded, dashed)
    \draw[draw=gray!50, line width=1pt, dashed,
          dash pattern=on 2pt off 2pt, opacity=0.5,
          -{Stealth[length=2mm]}]
        (#1) -- (#2)
        node[pos=0.3, above, font=\tiny, text=gray] {#3};
    % New connection (solid, bright)
    \draw[draw=blue!70, line width=1.5pt, -{Stealth[length=3mm]}]
        (#1) -- (#2)
        node[pos=0.7, above, font=\tiny\bfseries, fill=blue!15, inner sep=1pt] {#4};
}

% Added connection indicator
% Usage: \drawAddedConnection{from}{to}{label}
\newcommand{\drawAddedConnection}[3]{
    \draw[draw=green!70, line width=1.5pt, -{Stealth[length=3mm]}]
        (#1) -- (#2)
        node[midway, above, font=\tiny\bfseries, fill=green!15, inner sep=2pt] {#3}
        node[midway, below, font=\tiny, fill=green!20, inner sep=1pt] {+NEW};
}

% Removed connection indicator
% Usage: \drawRemovedConnection{from}{to}{label}
\newcommand{\drawRemovedConnection}[3]{
    \draw[draw=red!70, line width=1pt, dashed, dash pattern=on 2pt off 2pt,
          decoration={markings,
              mark=between positions 0.3 and 0.7 step 0.2 with {
                  \draw[red!70, line width=1pt] (-2pt,-2pt) -- (2pt,2pt);
                  \draw[red!70, line width=1pt] (-2pt,2pt) -- (2pt,-2pt);
              }}, postaction={decorate}]
        (#1) -- (#2)
        node[midway, above, font=\tiny, text=red!70, fill=white, inner sep=1pt] {#3}
        node[midway, below, font=\tiny, fill=red!15, inner sep=1pt] {-REMOVED};
}

% Modified connection indicator
% Usage: \drawModifiedConnection{from}{to}{change_desc}{label}
\newcommand{\drawModifiedConnection}[4]{
    \draw[draw=orange!70, line width=1.5pt, -{Stealth[length=3mm]}]
        (#1) -- (#2)
        node[midway, above, font=\tiny\bfseries, fill=orange!15, inner sep=2pt] {#4}
        node[midway, below, font=\tiny, fill=orange!20, inner sep=1pt] {~#3};
}

% ============================================================================
% TIME-SERIES AND TEMPORAL VISUALIZATION
% ============================================================================

% Connection with timestamp
% Usage: \drawTimestampedConnection{from}{to}{timestamp}{label}
\newcommand{\drawTimestampedConnection}[4]{
    \draw[normal conn, -{Stealth[length=2.5mm]}] (#1) -- (#2)
        node[midway, above, font=\tiny, fill=white, inner sep=1pt] {#4}
        node[midway, below, font=\tiny, fill=gray!15, inner sep=1pt] {#3};
}

% Connection age indicator
% Usage: \drawConnectionByAge{from}{to}{days_old}{label}
\newcommand{\drawConnectionByAge}[4]{
    \pgfmathsetmacro{\age}{#3}
    \ifnum\age<7
        \def\ageColor{green!70}
        \def\ageDesc{New}
    \else
        \ifnum\age<30
            \def\ageColor{blue!70}
            \def\ageDesc{Recent}
        \else
            \ifnum\age<90
                \def\ageColor{orange!70}
                \def\ageDesc{Old}
            \else
                \def\ageColor{red!70}
                \def\ageDesc{Stale}
            \fi
        \fi
    \fi
    \draw[draw=\ageColor, line width=1.2pt, -{Stealth[length=3mm]}]
        (#1) -- (#2)
        node[midway, above, font=\tiny, fill=white, inner sep=1pt] {#4}
        node[midway, below, font=\tiny, fill=\ageColor!15, inner sep=1pt] 
            {#3d - \ageDesc};
}

% Historical connection (archived/deprecated)
% Usage: \drawHistoricalConnection{from}{to}{decommission_date}{label}
\newcommand{\drawHistoricalConnection}[4]{
    \draw[draw=gray!40, line width=0.8pt, dashed, dash pattern=on 1pt off 2pt,
          opacity=0.6, -{Stealth[length=2mm]}]
        (#1) -- (#2)
        node[midway, above, font=\tiny, text=gray!60, fill=white, inner sep=1pt] {#4}
        node[midway, below, font=\tiny, fill=gray!10, inner sep=1pt] 
            {Decom: #3};
}

% Planned future connection
% Usage: \drawPlannedConnection{from}{to}{planned_date}{label}
\newcommand{\drawPlannedConnection}[4]{
    \draw[draw=blue!60, line width=1.2pt, dotted, -{Stealth[length=3mm]}]
        (#1) -- (#2)
        node[midway, above, font=\tiny, fill=blue!10, inner sep=2pt] {#4}
        node[midway, below, font=\tiny, fill=blue!20, inner sep=1pt] 
            {Planned: #3};
}

% Connection lifecycle stage
% Usage: \drawLifecycleConnection{from}{to}{stage}{label}
\newcommand{\drawLifecycleConnection}[4]{
    % Stages: planning, implementation, production, maintenance, decommission
    \ifthenelse{\equal{#3}{planning}}{
        \def\lcColor{cyan!70}
        \def\lcStyle{dotted}
    }{
        \ifthenelse{\equal{#3}{implementation}}{
            \def\lcColor{blue!70}
            \def\lcStyle{dashed}
        }{
            \ifthenelse{\equal{#3}{production}}{
                \def\lcColor{green!70}
                \def\lcStyle{solid}
            }{
                \ifthenelse{\equal{#3}{maintenance}}{
                    \def\lcColor{orange!70}
                    \def\lcStyle{solid}
                }{
                    \def\lcColor{red!70}
                    \def\lcStyle{dashed}
                }
            }
        }
    }
    \draw[draw=\lcColor, line width=1.2pt, \lcStyle, -{Stealth[length=3mm]}]
        (#1) -- (#2)
        node[midway, above, font=\tiny, fill=white, inner sep=1pt] {#4}
        node[midway, below, font=\tiny, fill=\lcColor!15, inner sep=1pt] 
            {\MakeUppercase{#3}};
}

% ============================================================================
% CAPACITY PLANNING AND FORECASTING
% ============================================================================

% Connection with capacity indicator
% Usage: \drawCapacityConnection{from}{to}{current}{max}{label}
\newcommand{\drawCapacityConnection}[5]{
    \pgfmathsetmacro{\utilpct}{100*#3/#4}
    \pgfmathsetmacro{\remaining}{#4-#3}
    \ifnum\utilpct<60
        \def\capColor{green!70}
        \def\capStatus{OK}
    \else
        \ifnum\utilpct<85
            \def\capColor{yellow!80!orange}
            \def\capStatus{WARNING}
        \else
            \def\capColor{red!70}
            \def\capStatus{CRITICAL}
        \fi
    \fi
    \pgfmathsetmacro{\linewidth}{0.5 + ln(max(1, #3))/2}
    \draw[draw=\capColor, line width=\linewidth pt, -{Stealth[length=3mm]}]
        (#1) -- (#2)
        node[midway, above, font=\tiny\bfseries, fill=white, inner sep=2pt] {#5}
        node[midway, below, font=\tiny, fill=\capColor!15, inner sep=1pt] 
            {#3/#4 Gbps - \capStatus};
}

% Growth forecast connection
% Usage: \drawForecastConnection{from}{to}{current}{forecast}{months}{label}
\newcommand{\drawForecastConnection}[6]{
    \pgfmathsetmacro{\growth}{100*(#4-#3)/#3}
    \draw[draw=purple!70, line width=1.5pt, -{Stealth[length=3mm]},
          decoration={markings,
              mark=at position 0.5 with {
                  \node[regular polygon, regular polygon sides=3,
                        draw=purple!70, fill=purple!20, minimum size=4pt,
                        rotate=90] {};
              }}, postaction={decorate}]
        (#1) -- (#2)
        node[midway, above, font=\tiny, fill=white, inner sep=1pt] {#6}
        node[midway, below, font=\tiny, fill=purple!15, inner sep=2pt,
              align=center] {Now: #3 | +#5m: #4\\Growth: \pgfmathprintnumber[precision=1]{\growth}\%};
}

% Capacity exhaustion warning
% Usage: \drawExhaustionWarning{from}{to}{days_until_full}{label}
\newcommand{\drawExhaustionWarning}[4]{
    \pgfmathsetmacro{\days}{#3}
    \ifnum\days<30
        \def\warnColor{red!80}
        \def\warnLevel{URGENT}
    \else
        \ifnum\days<90
            \def\warnColor{orange!80}
            \def\warnLevel{SOON}
        \else
            \def\warnColor{yellow!80}
            \def\warnLevel{MONITOR}
        \fi
    \fi
    \draw[draw=\warnColor, line width=2pt, -{Stealth[length=3mm]},
          decoration={markings,
              mark=between positions 0.2 and 0.8 step 0.2 with {
                  \node[regular polygon, regular polygon sides=3,
                        draw=\warnColor, fill=\warnColor!30,
                        minimum size=4pt, inner sep=0pt] {};
              }}, postaction={decorate}]
        (#1) -- (#2)
        node[midway, above, font=\tiny\bfseries, fill=white, inner sep=2pt] {#4}
        node[midway, below, font=\tiny\bfseries, fill=\warnColor!20, inner sep=2pt]
            {Full in #3d - \warnLevel};
}

% ============================================================================
% COST ANALYSIS AND BUDGETING
% ============================================================================

% Connection with cost per month
% Usage: \drawCostConnection{from}{to}{monthly_cost}{label}
\newcommand{\drawCostConnection}[4]{
    \pgfmathsetmacro{\cost}{#3}
    \ifnum\cost<100
        \def\costColor{green!70}
    \else
        \ifnum\cost<500
            \def\costColor{blue!70}
        \else
            \ifnum\cost<1000
                \def\costColor{orange!70}
            \else
                \def\costColor{red!70}
            \fi
        \fi
    \fi
    \draw[draw=\costColor, line width=1.5pt, -{Stealth[length=3mm]}]
        (#1) -- (#2)
        node[midway, above, font=\tiny, fill=white, inner sep=1pt] {#4}
        node[midway, below, font=\tiny\bfseries, fill=\costColor!15, inner sep=2pt]
            {\$#3/mo};
}

% ROI indicator connection
% Usage: \drawROIConnection{from}{to}{roi_pct}{payback_months}{label}
\newcommand{\drawROIConnection}[5]{
    \pgfmathsetmacro{\roi}{#3}
    \ifnum\roi>100
        \def\roiColor{green!70}
        \def\roiStatus{Excellent}
    \else
        \ifnum\roi>50
            \def\roiColor{blue!70}
            \def\roiStatus{Good}
        \else
            \def\roiColor{orange!70}
            \def\roiStatus{Fair}
        \fi
    \fi
    \draw[draw=\roiColor, line width=1.5pt, -{Stealth[length=3mm]}]
        (#1) -- (#2)
        node[midway, above, font=\tiny, fill=white, inner sep=1pt] {#5}
        node[midway, below, font=\tiny, fill=\roiColor!15, inner sep=2pt]
            {ROI: #3\% | Payback: #4m};
}

% Budget status connection
% Usage: \drawBudgetConnection{from}{to}{budget}{actual}{label}
\newcommand{\drawBudgetConnection}[5]{
    \pgfmathsetmacro{\variance}{100*(#4-#3)/#3}
    \pgfmathsetmacro{\varval}{\variance}
    \ifnum\varval<0
        \def\budgetColor{green!70}
        \def\budgetStatus{Under Budget}
    \else
        \ifnum\varval<10
            \def\budgetColor{blue!70}
            \def\budgetStatus{On Budget}
        \else
            \def\budgetColor{red!70}
            \def\budgetStatus{Over Budget}
        \fi
    \fi
    \draw[draw=\budgetColor, line width=1.5pt, -{Stealth[length=3mm]}]
        (#1) -- (#2)
        node[midway, above, font=\tiny, fill=white, inner sep=1pt] {#5}
        node[midway, below, font=\tiny, fill=\budgetColor!15, inner sep=2pt,
              align=center] {Budget: \$#3 | Actual: \$#4\\\budgetStatus};
}

% ============================================================================
% GEO-LOCATION AND REGIONAL FEATURES
% ============================================================================

% Inter-region connection
% Usage: \drawRegionalConnection{from}{to}{region1}{region2}{latency}{label}
\newcommand{\drawRegionalConnection}[6]{
    \draw[draw=cyan!70, line width=1.5pt, -{Stealth[length=3mm]},
          decoration={markings,
              mark=at position 0.5 with {
                  \node[circle, draw=cyan!70, fill=cyan!20,
                        font=\tiny, inner sep=1pt] {#5ms};
              }}, postaction={decorate}]
        (#1) -- (#2)
        node[pos=0.2, above, font=\tiny, fill=cyan!10, inner sep=1pt] {#3}
        node[pos=0.8, above, font=\tiny, fill=cyan!10, inner sep=1pt] {#4}
        node[midway, above=8pt, font=\tiny, fill=white, inner sep=1pt] {#6};
}

% Cross-datacenter connection
% Usage: \drawDatacenterLink{from}{to}{dc1}{dc2}{bandwidth}{label}
\newcommand{\drawDatacenterLink}[6]{
    \pgfmathsetmacro{\linewidth}{0.5 + ln(max(1, #5))/2}
    \draw[draw=blue!60!green, line width=\linewidth pt, -{Stealth[length=3mm]},
          double=blue!30, double distance=1pt]
        (#1) -- (#2)
        node[midway, above, font=\tiny\bfseries, fill=white, inner sep=2pt] {#6}
        node[midway, below, font=\tiny, fill=blue!10, inner sep=2pt]
            {#3 $\leftrightarrow$ #4 | #5 Gbps};
}

% Global CDN connection
% Usage: \drawCDNConnection{from}{to}{edge_location}{label}
\newcommand{\drawCDNConnection}[4]{
    \draw[draw=purple!60, line width=1.2pt, -{Stealth[length=3mm]},
          decoration={markings,
              mark=between positions 0.25 and 0.75 step 0.25 with {
                  \node[star, star points=5, draw=purple!60,
                        fill=purple!20, minimum size=3pt, inner sep=0pt] {};
              }}, postaction={decorate}]
        (#1) -- (#2)
        node[midway, above, font=\tiny, fill=white, inner sep=1pt] {#4}
        node[midway, below, font=\tiny, fill=purple!15, inner sep=1pt] 
            {CDN: #3};
}

% ============================================================================
% PERFORMANCE PREDICTION AND TRENDING
% ============================================================================

% Trending connection (showing direction of change)
% Usage: \drawTrendingConnection{from}{to}{trend}{metric}{label}
\newcommand{\drawTrendingConnection}[5]{
    % Trend: up, down, stable
    \ifthenelse{\equal{#3}{up}}{
        \def\trendColor{red!70}
        \def\trendSymbol{$\uparrow$}
        \def\trendDesc{Increasing}
    }{
        \ifthenelse{\equal{#3}{down}}{
            \def\trendColor{green!70}
            \def\trendSymbol{$\downarrow$}
            \def\trendDesc{Decreasing}
        }{
            \def\trendColor{blue!70}
            \def\trendSymbol{$\rightarrow$}
            \def\trendDesc{Stable}
        }
    }
    \draw[draw=\trendColor, line width=1.5pt, -{Stealth[length=3mm]}]
        (#1) -- (#2)
        node[midway, above, font=\tiny, fill=white, inner sep=1pt] {#5}
        node[midway, below, font=\tiny, fill=\trendColor!15, inner sep=2pt]
            {#4 \trendSymbol\ \trendDesc};
}

% Performance baseline comparison
% Usage: \drawBaselineConnection{from}{to}{current}{baseline}{delta}{label}
\newcommand{\drawBaselineConnection}[6]{
    \pgfmathsetmacro{\diff}{#5}
    \ifnum\diff>10
        \def\baseColor{red!70}
        \def\baseStatus{Degraded}
    \else
        \ifnum\diff>5
            \def\baseColor{yellow!80!orange}
            \def\baseStatus{Warning}
        \else
            \def\baseColor{green!70}
            \def\baseStatus{Normal}
        \fi
    \fi
    \draw[draw=\baseColor, line width=1.5pt, -{Stealth[length=3mm]}]
        (#1) -- (#2)
        node[midway, above, font=\tiny, fill=white, inner sep=1pt] {#6}
        node[midway, below, font=\tiny, fill=\baseColor!15, inner sep=2pt,
              align=center] {Curr: #3 | Base: #4\\Delta: +#5\% - \baseStatus};
}

% Anomaly detection indicator
% Usage: \drawAnomalyConnection{from}{to}{severity}{anomaly_type}{label}
\newcommand{\drawAnomalyConnection}[5]{
    % Severity: low, medium, high
    \ifthenelse{\equal{#3}{high}}{
        \def\anomColor{red!80}
        \def\anomPattern{on 2pt off 1pt on 2pt off 1pt}
    }{
        \ifthenelse{\equal{#3}{medium}}{
            \def\anomColor{orange!80}
            \def\anomPattern{on 3pt off 2pt}
        }{
            \def\anomColor{yellow!80}
            \def\anomPattern{on 4pt off 2pt}
        }
    }
    \draw[draw=\anomColor, line width=2pt, dashed, dash pattern=\anomPattern,
          -{Stealth[length=3mm]},
          decoration={markings,
              mark=between positions 0.3 and 0.7 step 0.2 with {
                  \node[circle, draw=\anomColor, fill=\anomColor!30,
                        font=\tiny, inner sep=1pt] {!};
              }}, postaction={decorate}]
        (#1) -- (#2)
        node[midway, above, font=\tiny\bfseries, fill=white, inner sep=2pt] {#5}
        node[midway, below, font=\tiny\bfseries, fill=\anomColor!20, inner sep=2pt]
            {ANOMALY: #4 (\MakeUppercase{#3})};
}

% ============================================================================
% NETWORK EVOLUTION AND CHANGE TRACKING
% ============================================================================

% Migration path
% Usage: \drawMigrationPath{from}{to}{migration_type}{completion_pct}{label}
\newcommand{\drawMigrationPath}[5]{
    \pgfmathsetmacro{\completion}{#4}
    % Draw partial connection based on completion
    \draw[draw=blue!70, line width=2pt, -{Stealth[length=3mm]}]
        (#1) -- ($(#1)!#4/100!(#2)$);
    \draw[draw=gray!50, line width=1pt, dashed, dash pattern=on 2pt off 2pt]
        ($(#1)!#4/100!(#2)$) -- (#2);
    % Progress indicator
    \node[circle, fill=blue!70, text=white, font=\tiny\bfseries,
          minimum size=10pt, inner sep=1pt]
        at ($(#1)!#4/100!(#2)$) {#4\%};
    % Labels
    \node[above=10pt, font=\tiny, fill=white, inner sep=1pt]
        at ($(#1)!0.5!(#2)$) {#5};
    \node[below=10pt, font=\tiny, fill=blue!15, inner sep=2pt]
        at ($(#1)!0.5!(#2)$) {Migration: #3};
}

% Technology upgrade path
% Usage: \drawUpgradePath{from}{to}{old_tech}{new_tech}{date}{label}
\newcommand{\drawUpgradePath}[6]{
    \draw[draw=green!60!blue, line width=1.5pt, -{Stealth[length=3mm]},
          decoration={markings,
              mark=at position 0.5 with {
                  \node[rectangle, draw=green!60!blue, fill=white,
                        font=\tiny, inner sep=2pt] {UPGRADE};
              }}, postaction={decorate}]
        (#1) -- (#2)
        node[pos=0.2, above, font=\tiny, fill=red!10, inner sep=1pt] {#3}
        node[pos=0.8, above, font=\tiny, fill=green!10, inner sep=1pt] {#4}
        node[midway, above=10pt, font=\tiny, fill=white, inner sep=1pt] {#6}
        node[midway, below=10pt, font=\tiny, fill=blue!10, inner sep=1pt] {ETA: #5};
}

% Version control connection
% Usage: \drawVersionedConnection{from}{to}{version}{changelog}{label}
\newcommand{\drawVersionedConnection}[5]{
    \draw[normal conn, -{Stealth[length=2.5mm]}] (#1) -- (#2)
        node[midway, above, font=\tiny, fill=white, inner sep=1pt] {#5};
    % Version badge
    \node[rectangle, rounded corners=2pt, draw=blue!70, fill=blue!15,
          font=\tiny\ttfamily, inner sep=2pt]
        at ($(#1)!0.5!(#2)$) {v#3};
    \node[below=8pt, font=\tiny, fill=yellow!15, inner sep=1pt, align=center]
        at ($(#1)!0.5!(#2)$) {#4};
}

% ============================================================================
% AUTOMATED DOCUMENTATION FEATURES
% ============================================================================

% Self-documenting connection with auto-generated notes
% Usage: \drawDocumentedConnection{from}{to}{purpose}{owner}{sla}{label}
\newcommand{\drawDocumentedConnection}[6]{
    \draw[normal conn, -{Stealth[length=2.5mm]}] (#1) -- (#2)
        node[midway, above, font=\tiny\bfseries, fill=white, inner sep=2pt] {#6};
    % Documentation callout
    \node[rectangle, draw=orange!70, fill=yellow!10, rounded corners=2pt,
          font=\tiny, inner sep=3pt, align=left, anchor=west]
        at ($(#1)!0.5!(#2) + (2,0)$) {
            \textbf{Purpose:} #3\\
            \textbf{Owner:} #4\\
            \textbf{SLA:} #5
        };
}

% Connection inventory entry
% Usage: \drawInventoryConnection{from}{to}{id}{type}{criticality}{label}
\newcommand{\drawInventoryConnection}[6]{
    % Criticality: critical, high, medium, low
    \ifthenelse{\equal{#5}{critical}}{
        \def\invColor{red!70}
        \def\invWidth{2pt}
    }{
        \ifthenelse{\equal{#5}{high}}{
            \def\invColor{orange!70}
            \def\invWidth{1.5pt}
        }{
            \ifthenelse{\equal{#5}{medium}}{
                \def\invColor{blue!70}
                \def\invWidth{1.2pt}
            }{
                \def\invColor{gray!60}
                \def\invWidth{0.8pt}
            }
        }
    }
    \draw[draw=\invColor, line width=\invWidth, -{Stealth[length=3mm]}]
        (#1) -- (#2)
        node[midway, above, font=\tiny, fill=white, inner sep=1pt] {#6};
    % Inventory tag
    \node[rectangle, draw=\invColor, fill=\invColor!10,
          font=\tiny\ttfamily, inner sep=1pt]
        at ($(#1)!0.5!(#2)$) {ID:#3};
    \node[below=6pt, font=\tiny, fill=\invColor!15, inner sep=1pt]
        at ($(#1)!0.5!(#2)$) {#4 - \MakeUppercase{#5}};
}

% Connection with CMDB integration
% Usage: \drawCMDBConnection{from}{to}{cmdb_id}{asset_tag}{last_audit}{label}
\newcommand{\drawCMDBConnection}[6]{
    \draw[draw=purple!60, line width=1.2pt, -{Stealth[length=3mm]}]
        (#1) -- (#2)
        node[midway, above, font=\tiny, fill=white, inner sep=1pt] {#6};
    % CMDB badge
    \node[circle, draw=purple!70, fill=purple!15,
          font=\tiny\bfseries, inner sep=2pt]
        at ($(#1)!0.5!(#2)$) {DB};
    \node[below=6pt, font=\tiny\ttfamily, fill=purple!10, inner sep=1pt,
          align=center]
        at ($(#1)!0.5!(#2)$) {#3\\#4\\Audit: #5};
}

% ============================================================================
% ADVANCED METRICS AND KPIs
% ============================================================================

% Connection with full telemetry
% Usage: \drawTelemetryConnection{from}{to}{metrics}{label}
\newcommand{\drawTelemetryConnection}[4]{
    \draw[draw=cyan!70, line width=1.5pt, -{Stealth[length=3mm]}]
        (#1) -- (#2)
        node[midway, above, font=\tiny\bfseries, fill=white, inner sep=2pt] {#4};
    % Telemetry data box
    \node[rectangle, draw=cyan!70, fill=cyan!5, rounded corners=2pt,
          font=\tiny\ttfamily, inner sep=3pt, align=left, anchor=north]
        at ($(#1)!0.5!(#2) + (0,-0.8)$) {#3};
}

% SRE Golden Signals connection
% Usage: \drawGoldenSignals{from}{to}{latency}{traffic}{errors}{saturation}{label}
\newcommand{\drawGoldenSignals}[7]{
    \draw[draw=blue!70, line width=1.5pt, -{Stealth[length=3mm]}]
        (#1) -- (#2)
        node[midway, above, font=\tiny\bfseries, fill=white, inner sep=2pt] {#7};
    % Golden signals dashboard
    \node[rectangle, draw=blue!70, fill=blue!5, rounded corners=3pt,
          font=\tiny, inner sep=3pt, align=left, anchor=south]
        at ($(#1)!0.5!(#2) + (0,0.8)$) {
            \textbf{Golden Signals}\\
            Latency: #3ms\\
            Traffic: #4 rps\\
            Errors: #5\%\\
            Saturation: #6\%
        };
}

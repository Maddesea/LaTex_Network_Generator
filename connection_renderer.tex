% connection_renderer.tex - Network connection rendering and visualization
% This module handles all connection types, paths, and flow visualization

% ============================================================================
% CONNECTION DATA STRUCTURES
% ============================================================================

% Connection counter
\newcounter{conncount}

% TODO: Connection management
% - Hash map for connection lookup
% - Bidirectional connection deduplication
% - Connection grouping by protocol/type
% - Connection state tracking (active/inactive)

% ============================================================================
% BASIC CONNECTION COMMANDS
% ============================================================================

% Draw a normal connection
% Usage: \drawConnection{from}{to}{label}
\newcommand{\drawConnection}[3]{
    \draw[normal conn] (#1) -- node[above, font=\tiny] {#3} (#2);
}

% Draw an encrypted connection
% Usage: \drawEncryptedConnection{from}{to}{protocol}
\newcommand{\drawEncryptedConnection}[3]{
    \draw[encrypted conn] (#1) -- node[above, font=\tiny\ttfamily] {#3} (#2);
}

% Draw a suspicious connection
% Usage: \drawSuspiciousConnection{from}{to}{reason}
\newcommand{\drawSuspiciousConnection}[3]{
    \draw[suspicious conn] (#1) -- node[above, font=\tiny, fill=yellow!30] {#3} (#2);
}

% Draw an attack connection
% Usage: \drawAttackConnection{from}{to}{attack_type}
\newcommand{\drawAttackConnection}[3]{
    \draw[attack conn] (#1) -- node[midway, above, threat label] {#3} (#2);
}

% Bidirectional connection
% Usage: \drawBidirectional{from}{to}{label}
\newcommand{\drawBidirectional}[3]{
    \draw[normal conn, bidirectional] (#1) -- node[above, font=\tiny] {#3} (#2);
}

% ============================================================================
% SPECIAL CONNECTION TYPES
% ============================================================================

% VPN tunnel connection
% Usage: \drawVPNTunnel{from}{to}{label}
\newcommand{\drawVPNTunnel}[3]{
    \draw[draw=blue!70, line width=1.5pt, dashed,
          dash pattern=on 3pt off 2pt, -{Stealth[length=3mm]},
          postaction={draw, line width=0.5pt, draw=blue!30}]
        (#1) -- (#2)
        node[midway, above, font=\tiny\bfseries, fill=blue!10, inner sep=2pt] {#3}
        node[midway, below, font=\tiny, fill=blue!20, inner sep=1pt] {VPN};
}

% Wireless connection with wave pattern
% Usage: \drawWirelessConnection{from}{to}{signal_strength}{label}
\newcommand{\drawWirelessConnection}[4]{
    \draw[draw=purple!70, line width=1pt,
          decoration={snake, amplitude=0.5mm, segment length=3mm},
          decorate, -{Stealth[length=2.5mm]}]
        (#1) -- (#2)
        node[midway, above, font=\tiny, fill=white, inner sep=1pt] {#4}
        node[midway, below, font=\tiny, fill=purple!15, inner sep=1pt] {WiFi: #3\%};
}

% Fiber optic connection with light effects
% Usage: \drawFiberConnection{from}{to}{label}
\newcommand{\drawFiberConnection}[3]{
    \draw[draw=yellow!80!orange, line width=2pt, -{Stealth[length=3mm]}]
        (#1) -- (#2);
    \draw[draw=yellow!50, line width=1pt, -{Stealth[length=3mm]}]
        (#1) -- (#2)
        node[midway, above, font=\tiny\bfseries, fill=white, inner sep=2pt] {#3}
        node[midway, below, font=\tiny, fill=yellow!20, inner sep=1pt] {Fiber};
}

% Serial/Legacy connection
% Usage: \drawSerialConnection{from}{to}{baud_rate}{label}
\newcommand{\drawSerialConnection}[4]{
    \draw[draw=gray!70, line width=1pt,
          dash pattern=on 2pt off 1pt on 4pt off 1pt,
          -{Stealth[length=2.5mm]}]
        (#1) -- (#2)
        node[midway, above, font=\tiny, fill=white, inner sep=1pt] {#4}
        node[midway, below, font=\tiny\ttfamily, fill=gray!15, inner sep=1pt] {RS-232: #3};
}

% Satellite link with orbital arc
% Usage: \drawSatelliteLink{from}{to}{latency}{label}
\newcommand{\drawSatelliteLink}[4]{
    \draw[draw=cyan!70, line width=1pt, -{Stealth[length=2.5mm]}]
        (#1) to[bend left=60] (#2)
        node[midway, above, font=\tiny, fill=white, inner sep=2pt] {#4}
        node[midway, below, font=\tiny, fill=cyan!15, inner sep=1pt] {SAT: #3ms};
}

% Cellular/Mobile connection
% Usage: \drawCellularConnection{from}{to}{technology}{label}
\newcommand{\drawCellularConnection}[4]{
    \draw[draw=green!60!blue, line width=1pt,
          decoration={zigzag, amplitude=0.5mm, segment length=2mm},
          decorate, -{Stealth[length=2.5mm]}]
        (#1) -- (#2)
        node[midway, above, font=\tiny, fill=white, inner sep=1pt] {#4}
        node[midway, below, font=\tiny, fill=green!15, inner sep=1pt] {#3};
}

% Bluetooth connection
% Usage: \drawBluetoothConnection{from}{to}{label}
\newcommand{\drawBluetoothConnection}[3]{
    \draw[draw=blue!60, line width=0.8pt,
          decoration={snake, amplitude=0.3mm, segment length=2mm},
          decorate, bidirectional]
        (#1) -- (#2)
        node[midway, above, font=\tiny, fill=blue!10, inner sep=1pt] {#3 (BT)};
}

% TODO: Additional special connection types
% - LoRaWAN for IoT
% - NFC for proximity connections
% - Infrared connections
% - Optical wireless (Li-Fi)

% ============================================================================
% ADVANCED CONNECTION RENDERING
% ============================================================================

% Connection with bandwidth indicator (Enhanced with logarithmic scaling)
% Usage: \drawConnectionWithBandwidth{from}{to}{bandwidth}{label}
\newcommand{\drawConnectionWithBandwidth}[4]{
    % Line width based on bandwidth (logarithmic scale)
    \pgfmathsetmacro{\linewidth}{0.5 + ln(max(1, #3))/2}
    \draw[draw=connNormal, line width=\linewidth pt, -{Stealth[length=3mm]}]
        (#1) -- node[above, font=\tiny] {#4: #3 Mbps} (#2);
}

% Connection with bandwidth and utilization (shows congestion)
% Usage: \drawConnectionWithUtilization{from}{to}{bandwidth}{utilization}{label}
\newcommand{\drawConnectionWithUtilization}[5]{
    % Line width based on bandwidth (logarithmic scale)
    \pgfmathsetmacro{\linewidth}{0.5 + ln(max(1, #3))/2}
    % Color based on utilization percentage
    \pgfmathsetmacro{\util}{#4}
    \ifnum\util<50
        \def\connColor{green!60!black}
    \else
        \ifnum\util<80
            \def\connColor{yellow!80!orange}
        \else
            \def\connColor{red!70!black}
        \fi
    \fi
    \draw[draw=\connColor, line width=\linewidth pt, -{Stealth[length=3mm]}]
        (#1) -- node[above, font=\tiny] {#5}
        node[below, font=\tiny\ttfamily] {#3 Mbps (\util\% util)} (#2);
}

% Connection with bandwidth gradient (visual bandwidth representation)
% Usage: \drawBandwidthGradient{from}{to}{bandwidth}{max_bandwidth}{label}
\newcommand{\drawBandwidthGradient}[5]{
    \pgfmathsetmacro{\linewidth}{0.5 + ln(max(1, #3))/2}
    \pgfmathsetmacro{\ratio}{100*#3/#4}
    % Gradient from green to red based on bandwidth usage
    \draw[draw=green!60!black, line width=\linewidth pt, -{Stealth[length=3mm]}]
        (#1) -- (#2);
    % Overlay with utilization bar
    \node[midway, above, font=\tiny, fill=white, inner sep=2pt]
        at ($(#1)!0.5!(#2)$) {#5: #3/#4 Mbps};
}

% Connection with protocol and port information (Enhanced)
% Usage: \drawConnectionWithPort{from}{to}{protocol}{port}{label}
\newcommand{\drawConnectionWithPort}[5]{
    \draw[normal conn, -{Stealth[length=2.5mm]}] (#1) --
        node[above, font=\tiny\ttfamily, fill=white, inner sep=1pt] {#5}
        node[below, port label] {#3:#4}
        (#2);
}

% Enhanced protocol label with auto-positioning
% Usage: \drawConnectionWithProtocol{from}{to}{protocol}{port}{service}{label}
\newcommand{\drawConnectionWithProtocol}[6]{
    \draw[normal conn, -{Stealth[length=2.5mm]}] (#1) -- (#2)
        node[midway, above, font=\tiny, fill=white, inner sep=1pt] {#6}
        node[midway, below, font=\tiny\ttfamily, fill=blue!10, inner sep=1pt]
            {#3/#4 (#5)};
}

% Protocol label with color coding
% Usage: \drawColorCodedProtocol{from}{to}{protocol}{port}{label}
\newcommand{\drawColorCodedProtocol}[5]{
    % Color based on protocol
    \ifthenelse{\equal{#3}{TCP}}{
        \def\protoColor{blue!70}
    }{
        \ifthenelse{\equal{#3}{UDP}}{
            \def\protoColor{green!70}
        }{
            \def\protoColor{orange!70}
        }
    }
    \draw[draw=\protoColor, line width=1pt, -{Stealth[length=2.5mm]}]
        (#1) -- (#2)
        node[midway, above, font=\tiny, fill=white, inner sep=1pt] {#5}
        node[midway, below, font=\tiny\ttfamily, fill=\protoColor!20, inner sep=1pt]
            {#3:#4};
}

% Multi-protocol connection showing multiple services
% Usage: \drawMultiProtocolConnection{from}{to}{protocols}{label}
\newcommand{\drawMultiProtocolConnection}[4]{
    \draw[normal conn, line width=1.5pt, -{Stealth[length=3mm]}]
        (#1) -- (#2)
        node[midway, above, font=\tiny\bfseries, fill=white, inner sep=1pt] {#4};
    \node[midway, below, font=\tiny\ttfamily, fill=yellow!15,
          inner sep=2pt, align=center]
        at ($(#1)!0.5!(#2)$) {#3};
}

% Port range connection
% Usage: \drawPortRangeConnection{from}{to}{protocol}{port_start}{port_end}{label}
\newcommand{\drawPortRangeConnection}[6]{
    \draw[normal conn, -{Stealth[length=2.5mm]}] (#1) -- (#2)
        node[midway, above, font=\tiny, fill=white, inner sep=1pt] {#6}
        node[midway, below, font=\tiny\ttfamily, fill=orange!15, inner sep=1pt]
            {#3:#4-#5};
}

% Curved connection (for avoiding overlaps)
% Usage: \drawCurvedConnection{from}{to}{bend}{label}
\newcommand{\drawCurvedConnection}[4]{
    \draw[normal conn] (#1) to[bend left=#3] 
        node[above, font=\tiny] {#4} (#2);
}

% ============================================================================
% AUTOMATIC PATH FINDING AND OBSTACLE AVOIDANCE
% ============================================================================

% Orthogonal connection with automatic routing
% Usage: \drawOrthogonalConnection{from}{to}{label}
\newcommand{\drawOrthogonalConnection}[3]{
    \draw[normal conn, -Stealth] (#1) -|
        ($(#1)!0.5!(#2)$) |- (#2)
        node[midway, above, font=\tiny, fill=white, inner sep=1pt] {#3};
}

% Orthogonal connection with custom waypoint
% Usage: \drawOrthogonalViaPoint{from}{to}{x_offset}{y_offset}{label}
\newcommand{\drawOrthogonalViaPoint}[5]{
    \draw[normal conn, -Stealth] (#1)
        -| ($(#1)+(#3,#4)$)
        -| (#2)
        node[pos=0.5, above, font=\tiny, fill=white, inner sep=1pt] {#5};
}

% Smart curved connection with obstacle avoidance
% Automatically curves to avoid overlapping with center area
% Usage: \drawSmartCurvedConnection{from}{to}{label}
\newcommand{\drawSmartCurvedConnection}[3]{
    % Calculate if nodes are on opposite sides
    \path let \p1 = (#1), \p2 = (#2) in
        \pgfextra{
            \pgfmathsetmacro{\dx}{\x2-\x1}
            \pgfmathsetmacro{\dy}{\y2-\y1}
            \pgfmathsetmacro{\dist}{sqrt(\dx*\dx+\dy*\dy)}
            \pgfmathsetmacro{\bendangle}{min(45, \dist/10)}
        };
    \draw[normal conn, -Stealth] (#1) to[bend left=\bendangle]
        node[midway, above, font=\tiny, fill=white, inner sep=1pt] {#3} (#2);
}

% Multi-waypoint connection for complex routing
% Usage: \drawPathConnection{from}{to}{waypoint1}{waypoint2}{label}
\newcommand{\drawPathConnection}[5]{
    \draw[normal conn, -Stealth] (#1) -- (#3) -- (#4) -- (#2)
        node[pos=0.5, above, font=\tiny, fill=white, inner sep=1pt] {#5};
}

% Connection with automatic midpoint calculation
% Avoids center by routing around perimeter
% Usage: \drawPerimeterConnection{from}{to}{label}
\newcommand{\drawPerimeterConnection}[3]{
    \path let \p1 = (#1), \p2 = (#2) in
        coordinate (mid) at ($(\x1,\y2)$);
    \draw[normal conn, -Stealth] (#1) -- (mid) -- (#2)
        node[pos=0.5, above, font=\tiny, fill=white, inner sep=1pt] {#3};
}

% TODO: Advanced pathfinding
% - Implement A* algorithm for optimal routing
% - Add dynamic obstacle detection from node positions
% - Calculate minimum distance paths
% - Support for custom routing zones and restrictions

% ============================================================================
% BEZIER CURVE CONNECTIONS
% ============================================================================

% Simple Bezier curve with automatic control points
% Usage: \drawBezierConnection{from}{to}{label}
\newcommand{\drawBezierConnection}[3]{
    \path let \p1 = (#1), \p2 = (#2) in
        coordinate (ctrl1) at ($(\x1,\y1)!0.33!(\x2,\y2) + (0,1)$)
        coordinate (ctrl2) at ($(\x1,\y1)!0.67!(\x2,\y2) + (0,1)$);
    \draw[normal conn, -{Stealth[length=2.5mm]}]
        (#1) .. controls (ctrl1) and (ctrl2) .. (#2)
        node[midway, above, font=\tiny, fill=white, inner sep=1pt] {#3};
}

% Bezier curve with custom control points
% Usage: \drawCustomBezier{from}{to}{ctrl1_x}{ctrl1_y}{ctrl2_x}{ctrl2_y}{label}
\newcommand{\drawCustomBezier}[7]{
    \draw[normal conn, -{Stealth[length=2.5mm]}]
        (#1) .. controls (#3,#4) and (#5,#6) .. (#2)
        node[midway, above, font=\tiny, fill=white, inner sep=1pt] {#7};
}

% Smooth curved connection with tension control
% Usage: \drawSmoothCurve{from}{to}{tension}{label}
\newcommand{\drawSmoothCurve}[4]{
    \draw[normal conn, -{Stealth[length=2.5mm]}]
        (#1) to[out=45, in=135, distance=#3cm] (#2)
        node[midway, above, font=\tiny, fill=white, inner sep=1pt] {#4};
}

% S-curve connection for parallel lines
% Usage: \drawSCurve{from}{to}{label}
\newcommand{\drawSCurve}[3]{
    \path let \p1 = (#1), \p2 = (#2) in
        coordinate (mid1) at ($(\x1,\y1)!0.33!(\x2,\y2)$)
        coordinate (mid2) at ($(\x1,\y1)!0.67!(\x2,\y2)$);
    \draw[normal conn, -{Stealth[length=2.5mm]}]
        (#1) to[out=0, in=180] (mid1) to[out=0, in=180] (mid2) to[out=0, in=180] (#2)
        node[pos=0.5, above, font=\tiny, fill=white, inner sep=1pt] {#3};
}

% Arc connection for circular layouts
% Usage: \drawArcConnection{from}{to}{radius}{label}
\newcommand{\drawArcConnection}[4]{
    \draw[normal conn, -{Stealth[length=2.5mm]}]
        (#1) to[bend left=45, looseness=1.5] (#2)
        node[midway, above, font=\tiny, fill=white, inner sep=1pt] {#4};
}

% Organic curved connection with multiple control points
% Usage: \drawOrganicCurve{from}{to}{label}
\newcommand{\drawOrganicCurve}[3]{
    \path let \p1 = (#1), \p2 = (#2) in
        \pgfextra{
            \pgfmathsetmacro{\dx}{\x2-\x1}
            \pgfmathsetmacro{\dy}{\y2-\y1}
            \pgfmathsetmacro{\angle}{atan2(\dy,\dx)}
        }
        coordinate (c1) at ($(#1)!0.25!(#2) + (\angle:0.5)$)
        coordinate (c2) at ($(#1)!0.5!(#2) + (\angle+90:0.3)$)
        coordinate (c3) at ($(#1)!0.75!(#2) + (\angle:0.5)$);
    \draw[normal conn, -{Stealth[length=2.5mm]}]
        (#1) .. controls (c1) and (c2) .. (c3) .. controls (c3) .. (#2)
        node[pos=0.5, above, font=\tiny, fill=white, inner sep=1pt] {#3};
}

% TODO: Advanced Bezier features
% - Automatic control point optimization
% - Collision-free curve generation
% - Minimum curvature paths
% - Smooth curve bundling

% ============================================================================
% CONNECTION FLOW VISUALIZATION
% ============================================================================

% Animated flow indicators (requires animation package)
% Usage: \drawFlowConnection{from}{to}{direction}{speed}
\newcommand{\drawFlowConnection}[4]{
    \draw[draw=connNormal, line width=1.5pt, -{Stealth[length=3mm]}] (#1) -- (#2)
        [postaction={
            decorate,
            decoration={
                markings,
                mark=between positions 0.1 and 0.9 step 0.2 with {
                    \ifthenelse{\equal{#3}{forward}}{
                        \arrow{Stealth[length=2mm, fill=connNormal]}
                    }{
                        \arrow{Stealth[reversed, length=2mm, fill=connNormal]}
                    }
                }
            }
        }];
}

% Traffic flow with volume indicator
% Usage: \drawTrafficFlow{from}{to}{packets_per_sec}{label}
\newcommand{\drawTrafficFlow}[4]{
    \pgfmathsetmacro{\density}{min(10, #3/100)}
    \draw[draw=connNormal, line width=1.5pt, -{Stealth[length=3mm]}] 
        (#1) -- node[above, font=\tiny] {#4: #3 pps} (#2)
        [postaction={
            decorate,
            decoration={
                markings,
                mark=between positions 0.1 and 0.9 step {0.1/\density} with {
                    \node[circle, fill=connNormal, inner sep=0.5pt] {};
                }
            }
        }];
}

% TODO: Flow visualization enhancements
% - Real-time traffic animation (if rendering to animated format)
% - Packet visualization with different colors per protocol
% - Congestion indicators (red/yellow traffic markers)
% - Flow direction with multiple arrows
% - Throughput heatmap coloring

% ============================================================================
% CONNECTION RENDERING ENGINE
% ============================================================================

% Main command to render all connections
\newcommand{\renderConnections}{
    % This will be populated by network_data.tex
    % Example structure:
    % \drawConnection{srv1}{pc1}{HTTP}
    % \drawAttackConnection{attacker1}{srv1}{SQL Injection}
}

% ============================================================================
% CONNECTION BUNDLING AND AGGREGATION
% ============================================================================

% Bundled connection (shows multiple connections as one thick line)
% Usage: \drawBundledConnection{from}{to}{count}{label}
\newcommand{\drawBundledConnection}[4]{
    \pgfmathsetmacro{\bundlewidth}{0.5 + sqrt(#3)}
    \draw[draw=connNormal, line width=\bundlewidth pt, -{Stealth[length=3mm]},
          double distance=1pt]
        (#1) -- (#2)
        node[midway, above, font=\tiny\bfseries, fill=white, inner sep=2pt] {#4}
        node[midway, below, font=\tiny, fill=yellow!20, inner sep=1pt] {#3 connections};
}

% Bundled connections with connection count badge
% Usage: \drawConnectionBundle{from}{to}{count}{protocols}{label}
\newcommand{\drawConnectionBundle}[5]{
    \pgfmathsetmacro{\bundlewidth}{0.5 + sqrt(#3)}
    \draw[draw=connNormal!80, line width=\bundlewidth pt, -{Stealth[length=3mm]}]
        (#1) -- (#2);
    % Draw badge with count
    \node[circle, fill=blue!70, text=white, font=\tiny\bfseries,
          minimum size=8pt, inner sep=1pt]
        at ($(#1)!0.5!(#2)$) {#3};
    % Label above
    \node[above=3pt, font=\tiny, fill=white, inner sep=1pt]
        at ($(#1)!0.5!(#2)$) {#5};
    % Protocols below
    \node[below=3pt, font=\tiny\ttfamily, fill=white, inner sep=1pt]
        at ($(#1)!0.5!(#2)$) {#4};
}

% Edge bundling with parallel offset
% Shows multiple connections with slight parallel offsets
% Usage: \drawParallelConnections{from}{to}{count}{label}
\newcommand{\drawParallelConnections}[4]{
    \pgfmathsetmacro{\offsetstep}{0.3}
    \pgfmathsetmacro{\totaloffset}{(#3-1)*\offsetstep/2}
    \foreach \i in {1,...,#3} {
        \pgfmathsetmacro{\offset}{-\totaloffset + (\i-1)*\offsetstep}
        \draw[draw=connNormal, line width=0.5pt, -{Stealth[length=2mm]},
              transform canvas={shift={(0,\offset)}}]
            (#1) -- (#2);
    }
    \node[above, font=\tiny, fill=white, inner sep=1pt]
        at ($(#1)!0.5!(#2)$) {#4 (x#3)};
}

% Hierarchical edge bundling for complex networks
% Bundles connections based on proximity
% Usage: \drawHierarchicalBundle{from}{to}{via}{count}{label}
\newcommand{\drawHierarchicalBundle}[5]{
    \pgfmathsetmacro{\bundlewidth}{0.5 + ln(#4)}
    \draw[draw=connNormal!70, line width=\bundlewidth pt]
        (#1) -- (#3);
    \draw[draw=connNormal, line width=\bundlewidth pt, -{Stealth[length=3mm]}]
        (#3) -- (#2);
    \node[fill=blue!20, circle, inner sep=2pt, font=\tiny\bfseries] at (#3) {#4};
    \node[above, font=\tiny] at (#3) {#5};
}

% TODO: Advanced bundling algorithms
% - Automatic bundle detection based on proximity
% - Force-directed edge bundling
% - Hierarchical edge bundling for tree structures
% - Interactive bundle expansion/collapse
% - Bundle splitting at different zoom levels

% ============================================================================
% CONNECTION LABELS AND ANNOTATIONS
% ============================================================================

% Add inline statistics to connection
% Usage: \labelConnectionStats{from}{to}{latency}{packet_loss}{jitter}
\newcommand{\labelConnectionStats}[5]{
    \draw[normal conn] (#1) -- (#2)
        node[midway, below, font=\tiny\ttfamily, fill=white, inner sep=2pt] {
            L:#3ms | PL:#4\% | J:#5ms
        };
}

% Add threat score to connection
% Usage: \labelConnectionThreat{from}{to}{score}{details}
\newcommand{\labelConnectionThreat}[4]{
    \draw[suspicious conn] (#1) -- (#2)
        node[midway, threat label] {
            \textbf{#3/10} #4
        };
}

% TODO: Label improvements
% - Auto-positioning to avoid overlaps
% - Expandable detail boxes on hover
% - Time-series data for connection metrics
% - Alert indicators for anomalous connections

% ============================================================================
% CONNECTION FILTERING AND LAYERS
% ============================================================================

% Only show connections of specific type
% Usage: \filterConnectionsByType{type}
% Types: all, encrypted, suspicious, attacks, normal
\newcommand{\filterConnectionsByType}[1]{
    % Implementation would use conditional rendering
}

% Show connections only for specific protocol
% Usage: \filterConnectionsByProtocol{protocol}
\newcommand{\filterConnectionsByProtocol}[1]{
    % Implementation would filter by protocol name
}

% TODO: Advanced filtering
% - Port-based filtering (show only port 80, 443, etc.)
% - Time-based filtering (show connections in time range)
% - Threshold filtering (show only high-bandwidth connections)
% - Interactive layer toggles for different connection types
% - Subnet-based filtering (show only intra/inter-subnet)

% ============================================================================
% CONNECTION PATTERNS AND ANALYSIS
% ============================================================================

% Highlight attack pattern (multiple sources to one target)
% Usage: \highlightAttackPattern{target}{sources}
\newcommand{\highlightAttackPattern}[2]{
    \begin{scope}[on background layer]
        \foreach \source in {#2} {
            \draw[attack conn, line width=2pt] (\source) -- (#1);
        }
        % Draw highlight around target
        \node[draw=threatCritical, line width=3pt, 
              rounded corners=5pt, inner sep=8pt, 
              fill=threatCritical!10] at (#1) {};
    \end{scope}
}

% Show connection path through network
% Usage: \showConnectionPath{node_list}
\newcommand{\showConnectionPath}[1]{
    % Draw path through multiple hops
    % Highlight the route taken by traffic
}

% TODO: Pattern detection visualization
% - DDoS pattern (many-to-one)
% - Data exfiltration pattern (one-to-many external)
% - Lateral movement pattern (peer-to-peer internal)
% - Command & Control pattern (periodic beaconing)
% - Port scanning pattern (one-to-many same port)

% ============================================================================
% CONNECTION STATISTICS AND METRICS
% ============================================================================

% Draw connection statistics summary
% Usage: \drawConnectionSummary{x}{y}
\newcommand{\drawConnectionSummary}[2]{
    \node[legend box, anchor=north west] at (#1,#2) {
        \begin{tabular}{lr}
            \textbf{Connections} & \\
            Normal & \theconncount \\
            Encrypted & 0 \\
            Suspicious & 0 \\
            Attacks & 0 \\
        \end{tabular}
    };
}

% TODO: Statistics enhancements
% - Real-time connection counts
% - Bandwidth utilization graphs
% - Protocol distribution pie chart
% - Top talkers list
% - Connection timeline visualization

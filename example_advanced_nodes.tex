% example_advanced_nodes.tex
% Demonstrates advanced node features: VMs, containers, clusters, and multi-part nodes
% This file should be used with: \generateNetworkDiagram{example_advanced_nodes}

\begin{scope}[on background layer]
    % Background grid (optional)
    \draw[step=1cm, gray!20, very thin] (-8,-8) grid (8,8);
\end{scope}

% ============================================================================
% VIRTUAL MACHINES AND HYPERVISORS
% ============================================================================

% Hypervisor hosting VMs
\createHypervisor{hyperv1}{10.0.0.10}{-5}{5}{ESXi-01}{3}

% Virtual machines on the hypervisor
\createVM{vm1}{10.0.1.10}{-6}{3}{Web-VM-01}{ESXi-01}
\createVM{vm2}{10.0.1.11}{-5}{3}{DB-VM-01}{ESXi-01}
\createVMWithResources{vm3}{10.0.1.12}{-4}{3}{App-VM-01}{4vCPU}{8GB}{50GB}

% Register VMs in hash map with parent relationship
\registerNode{vm1}{10.0.1.10}{Web-VM-01}
\setNodeParent{vm1}{hyperv1}
\registerNode{vm2}{10.0.1.11}{DB-VM-01}
\setNodeParent{vm2}{hyperv1}

% ============================================================================
% CONTAINERS AND KUBERNETES
% ============================================================================

% Container host
\createServer{docker1}{10.0.2.20}{2}{5}{Docker-Host}

% Containers with stacked appearance
\createContainer{cont1}{10.0.2.100}{1}{3}{nginx}{nginx:latest}
\createContainer{cont2}{10.0.2.101}{2}{3}{api}{node:16-alpine}
\createContainerWithPorts{cont3}{10.0.2.102}{3}{3}{redis}{6379:6379}

% Kubernetes pod
\createPod{pod1}{10.0.2.200}{4}{3}{web-pod}{production}

% ============================================================================
% DATABASES WITH DIFFERENT ROLES
% ============================================================================

% Database cluster
\createDatabasePrimary{db1}{10.0.3.10}{-2}{0}{PostgreSQL-Primary}
\createDatabaseReplica{db2}{10.0.3.11}{-1}{0}{PostgreSQL-Replica-1}
\createDatabaseReplica{db3}{10.0.3.12}{0}{0}{PostgreSQL-Replica-2}
\createDatabaseCluster{db4}{10.0.3.20}{1}{0}{MongoDB-Cluster}

% ============================================================================
% LOAD BALANCERS
% ============================================================================

% Active-Passive load balancer pair
\createLoadBalancerActive{lb1}{10.0.4.10}{-5}{-2}{LB-Primary}{round-robin}
\createLoadBalancerPassive{lb2}{10.0.4.11}{-3}{-2}{LB-Standby}
\addLoadDistribution{lb1}{web1,web2,web3}

% ============================================================================
% MULTI-PART NODES WITH DETAILED INFORMATION
% ============================================================================

% Server with services
\createNodeWithServices{web1}{192.168.1.10}{2}{-2}{WebServer-01}{80,443}{Apache 2.4}

% Server with resource metrics (CPU: 45%, Memory: 62%, Disk: 38%)
\createNodeWithMetrics{app1}{192.168.1.20}{5}{-2}{AppServer-01}{45}{62}{38}

% Security-focused nodes with different statuses
\createSecurityNode{srv1}{192.168.1.30}{-2}{-5}{SecureServer}{0}{0.0}{secure}
\createSecurityNode{srv2}{192.168.1.31}{0}{-5}{WarningServer}{3}{6.5}{warning}
\createSecurityNode{srv3}{192.168.1.32}{2}{-5}{CriticalServer}{8}{9.8}{critical}

% Detailed server node
\createDetailedServer{srv4}{192.168.1.40}{5}{-5}{FileServer}{NFS, SMB, FTP}{ONLINE}

% ============================================================================
% CLUSTERING AND GROUPING
% ============================================================================

% Create cluster around database nodes
\begin{scope}[on background layer]
    \createCluster{dbcluster}{Database Tier}{(db1)(db2)(db3)(db4)}{0}{0}
\end{scope}

% Create HA pair around load balancers
\begin{scope}[on background layer]
    \createHAPair{lbpair}{HA Load Balancers}{(lb1)}{(lb2)}
\end{scope}

% Server rack
\begin{scope}[on background layer]
    \createRack{rack1}{Rack A}{(web1)(app1)}{0}{0}
\end{scope}

% ============================================================================
% CONNECTIONS (Examples)
% ============================================================================

% VM to container connections
\draw[normal conn] (vm1) -- (cont1);
\draw[encrypted conn] (vm2) -- (db1);

% Load balancer to web servers
\draw[normal conn] (lb1) -- (web1);
\draw[normal conn] (lb1) -- (app1);

% Database replication
\draw[normal conn, -{Stealth[length=3mm]}, bend left=20] (db1) -- (db2);
\draw[normal conn, -{Stealth[length=3mm]}, bend left=20] (db1) -- (db3);

% Container to database
\draw[encrypted conn] (cont2) -- (db4);

% Application to services
\draw[normal conn] (app1) -- (srv4);

% ============================================================================
% ANNOTATIONS
% ============================================================================

% Add threat badge to vulnerable server
\addThreatBadge{srv3}{critical}

% Add metadata annotations
\annotateNode{hyperv1}{VMware vSphere 8.0}{south}
\annotateNode{docker1}{Docker Engine 24.0}{south}
\annotateNode{dbcluster}{Primary + 2 Replicas}{south}

% ============================================================================
% LEGEND
% ============================================================================

\node[legend box, anchor=north west] at (-7.5,-6.5) {
    \begin{minipage}{6cm}
        \textbf{Legend} \\[3pt]
        \tikz\node[vm, minimum size=0.4cm, scale=0.5] {}; Virtual Machine \\
        \tikz\node[container, minimum size=0.4cm, scale=0.5] {}; Container \\
        \tikz\node[database, minimum size=0.4cm, scale=0.5] {}; Database \\
        \tikz\node[loadbalancer, minimum size=0.4cm, scale=0.5] {}; Load Balancer \\
        \tikz\draw[encrypted conn] (0,0) -- (0.5,0); Encrypted \\
        \tikz\draw[normal conn] (0,0) -- (0.5,0); Normal
    \end{minipage}
};

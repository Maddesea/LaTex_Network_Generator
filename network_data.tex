% network_data.tex - Example network diagram data
% This file contains the actual network topology to visualize

% ============================================================================
% EXAMPLE: Corporate Network Under Attack
% ============================================================================

% This example demonstrates:
% - Multi-tier network architecture
% - Security zones (Internet, DMZ, Internal, Database)
% - Various attack scenarios
% - Threat visualization

% Define all network nodes
\renewcommand{\renderNetworkNodes}{
    
    % ========================================================================
    % INTERNET / EXTERNAL ZONE
    % ========================================================================
    
    \createCloud{internet}{0}{8}{Internet}
    
    % Attackers
    \createAttacker{attacker1}{203.0.113.45}{-6}{8}{Attacker 1}
    \createAttacker{attacker2}{198.51.100.78}{6}{8}{Attacker 2}
    
    % ========================================================================
    % PERIMETER / DMZ ZONE
    % ========================================================================
    
    \createFirewall{firewall1}{192.168.1.1}{0}{5}{Edge Firewall}
    \createRouter{router1}{192.168.1.2}{-3}{3}{Core Router}
    
    % DMZ Servers
    \createServerWithPorts{webserver1}{192.168.10.10}{-3}{1}{Web Server 1}{80,443}
    \createServerWithPorts{mailserver}{192.168.10.20}{3}{1}{Mail Server}{25,587,993}
    
    % ========================================================================
    % INTERNAL ZONE
    % ========================================================================
    
    \createFirewall{firewall2}{192.168.1.3}{0}{-1}{Internal Firewall}
    \createSwitch{switch1}{192.168.100.1}{0}{-3}{Core Switch}
    
    % Internal Servers
    \createServer{appserver1}{192.168.100.10}{-4}{-5}{App Server 1}
    \createServer{appserver2}{192.168.100.11}{0}{-5}{App Server 2}
    
    % Client Workstations
    \createClient{workstation1}{192.168.100.50}{-6}{-7}{WS-001}
    \createClient{workstation2}{192.168.100.51}{-3}{-7}{WS-002}
    \createClient{workstation3}{192.168.100.52}{0}{-7}{WS-003}
    
    % ========================================================================
    % DATABASE ZONE
    % ========================================================================
    
    \createServer{dbserver1}{192.168.200.10}{4}{-5}{DB Primary}
    \createServer{dbserver2}{192.168.200.11}{4}{-7}{DB Replica}
}

% ============================================================================
% DEFINE NETWORK CONNECTIONS
% ============================================================================

\renewcommand{\renderConnections}{
    
    % Internet to Firewall
    \drawConnection{internet}{firewall1}{}
    \drawAttackConnection{attacker1}{firewall1}{Port Scan}
    \drawAttackConnection{attacker2}{webserver1}{SQL Injection}
    
    % Perimeter connections
    \drawConnection{firewall1}{router1}{}
    \drawConnection{router1}{webserver1}{}
    \drawConnection{router1}{mailserver}{}
    
    % DMZ to Internal
    \drawEncryptedConnection{webserver1}{firewall2}{HTTPS}
    \drawConnection{mailserver}{firewall2}{}
    \drawConnection{firewall2}{switch1}{}
    
    % Internal network connections
    \drawConnection{switch1}{appserver1}{}
    \drawConnection{switch1}{appserver2}{}
    \drawBidirectional{switch1}{workstation1}{}
    \drawBidirectional{switch1}{workstation2}{}
    \drawBidirectional{switch1}{workstation3}{}
    
    % Application to Database
    \drawEncryptedConnection{appserver1}{dbserver1}{TLS 1.3}
    \drawEncryptedConnection{appserver2}{dbserver1}{TLS 1.3}
    \drawConnection{dbserver1}{dbserver2}{Replication}
    
    % Suspicious internal connection (lateral movement)
    \drawSuspiciousConnection{workstation2}{dbserver1}{Unauthorized Access}
}

% ============================================================================
% DEFINE SECURITY ZONES
% ============================================================================

\begin{scope}[on background layer]
    % Internet Zone
    \drawSecurityZone{internet}{cloudGray}
        {(attacker1) (attacker2) (internet)}
        {Internet}{Untrusted}
    
    % DMZ Zone
    \drawSecurityZone{dmz}{routerOrange}
        {(firewall1) (router1) (webserver1) (mailserver)}
        {DMZ}{Low Trust}
    
    % Internal Zone
    \drawSecurityZone{internal}{clientGreen}
        {(firewall2) (switch1) (appserver1) (appserver2) 
         (workstation1) (workstation2) (workstation3)}
        {Internal Network}{Medium Trust}
    
    % Database Zone
    \drawSecurityZone{database}{serverBlue}
        {(dbserver1) (dbserver2)}
        {Database Tier}{High Trust}
\end{scope}

% ============================================================================
% DEFINE THREATS AND ATTACKS
% ============================================================================

\renewcommand{\renderThreats}{
    
    % Mark web server vulnerability
    \markVulnerability{webserver1}{CVE-2024-1234}{9.8}
    
    % Show SQL injection attack
    \node[above right, threat label, anchor=south west] 
        at (webserver1.north east) {Active SQLi};
    
    % Mark compromised workstation
    \visualizeMalware{workstation2}{Ransomware}
    
    % Show threat badges
    \addThreatBadge{webserver1}{critical}
    \addThreatBadge{workstation2}{critical}
    \addThreatBadge{firewall1}{medium}
    
    % Draw security posture dashboard
    \drawSecurityDashboard{10}{8}
}

% ============================================================================
% DRAW LEGEND
% ============================================================================

\newcommand{\drawLegend}{
    \node[legend box, anchor=south west] at (-10,-9) {
        \small\bfseries Legend \\[2pt]
        \begin{tabular}{ll}
            \tikz\draw[normal conn] (0,0) -- (0.5,0); & Normal \\
            \tikz\draw[encrypted conn] (0,0) -- (0.5,0); & Encrypted \\
            \tikz\draw[suspicious conn] (0,0) -- (0.5,0); & Suspicious \\
            \tikz\draw[attack conn] (0,0) -- (0.5,0); & Attack \\
        \end{tabular}
    };
}

% ============================================================================
% DRAW METADATA
% ============================================================================

\newcommand{\drawMetadata}{
    \node[font=\tiny, anchor=south east, text=black!60] at (10,-9) {
        Generated: \today \\
        Network Diagram v1.0 \\
        Scale: \diagramScale x
    };
}

% TODO: Network data enhancements
% - JSON/YAML input parsing for dynamic data
% - CSV import for bulk node/connection creation
% - Database connectivity for live network data
% - Integration with network scanning tools (Nmap, Nessus)
% - Version control for network topology changes
% - Diff visualization between network states

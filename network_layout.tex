% network_layout.tex - Intelligent network layout and positioning
% This module handles automatic and manual layout algorithms

% ============================================================================
% LAYOUT PARAMETERS
% ============================================================================

\newlength{\nodeSpacing}
\setlength{\nodeSpacing}{4cm}

\newlength{\layerSpacing}
\setlength{\layerSpacing}{6cm}

\newlength{\subnetSpacing}
\setlength{\subnetSpacing}{8cm}

% TODO: Adaptive spacing
% - Calculate optimal spacing based on node count
% - Adjust for node size variations
% - Implement responsive spacing for different page sizes
% - Add padding/margin controls

% ============================================================================
% SUBNET/ZONE VISUALIZATION
% ============================================================================

% Draw a subnet boundary box
% Usage: \drawSubnet{name}{color}{nodes}{label}
\newcommand{\drawSubnet}[4]{
    \begin{scope}[on background layer]
        \node[
            fit=#3,
            draw=#2!60,
            fill=#2!5,
            rounded corners=5pt,
            line width=1.5pt,
            inner sep=15pt,
            label={[fill=white, draw=#2!60, rounded corners=2pt, 
                   font=\small\bfseries\sffamily]above:#4}
        ] (subnet-#1) {};
    \end{scope}
}

% Draw a security zone
% Usage: \drawSecurityZone{name}{color}{nodes}{label}{trustLevel}
\newcommand{\drawSecurityZone}[5]{
    \begin{scope}[on background layer]
        \node[
            fit=#3,
            draw=#2!70,
            fill=#2!8,
            rounded corners=8pt,
            line width=2pt,
            inner sep=20pt,
            double,
            double distance=1pt,
            label={[fill=white, draw=#2!70, rounded corners=2pt, 
                   font=\small\bfseries\sffamily, anchor=north west]
                   north west:{\textcolor{#2!90}{#4} \tiny(Trust: #5)}}
        ] (zone-#1) {};
    \end{scope}
}

% TODO: Advanced zone rendering
% - DMZ (Demilitarized Zone) specific styling
% - VLANs with distinct visual patterns
% - Trust boundary indicators with graduated shading
% - Network segment overlays
% - Geographical region grouping

% ============================================================================
% AUTO-LAYOUT ALGORITHMS
% ============================================================================

% Counter for tracking tier positions
\newcounter{tiercounter}
\newcounter{nodecounter}

% Store tier spacing and orientation
\newlength{\currentTierSpacing}
\newif\ifhorizontaltierlayout
\horizontaltierlayouttrue % default to horizontal

% Set tier orientation (horizontal or vertical)
% Usage: \setTierOrientation{horizontal|vertical}
\newcommand{\setTierOrientation}[1]{
    \def\temp{#1}
    \def\horizontal{horizontal}
    \ifx\temp\horizontal
        \horizontaltierlayouttrue
    \else
        \horizontaltierlayoutfalse
    \fi
}

% Calculate optimal tier spacing based on node count
% Usage: \calculateTierSpacing{total_nodes}{num_tiers}
\newcommand{\calculateTierSpacing}[2]{
    \pgfmathsetlength{\currentTierSpacing}{max(5cm, min(8cm, 15cm/#2))}
}

% Tiered/Layered layout (e.g., for web architecture)
% Usage: \layoutTiered{num_tiers}{nodes_list}{tier_assignments}
% Example: \layoutTiered{4}{web1,web2,db1,db2}{1,1,2,2}
\newcommand{\layoutTiered}[3]{
    % Calculate optimal spacing
    \calculateTierSpacing{#2}{#1}

    % Layer 1: External (Internet, Attackers)
    % Layer 2: Perimeter (Firewalls, DMZ)
    % Layer 3: Application (Web servers, App servers)
    % Layer 4: Data (Database servers)
    % Implementation creates evenly spaced tiers
}

% Advanced tier layout with custom spacing
% Usage: \layoutTieredAdvanced{num_tiers}{orientation}{spacing}
% orientation: horizontal or vertical
% spacing: distance between tiers in cm
\newcommand{\layoutTieredAdvanced}[3]{
    \setTierOrientation{#2}
    \setlength{\currentTierSpacing}{#3}

    % Reset counters
    \setcounter{tiercounter}{0}
    \setcounter{nodecounter}{0}
}

% Position a node in a specific tier
% Usage: \positionInTier{node_name}{tier_number}{position_in_tier}{total_in_tier}
% tier_number: 1, 2, 3, ... (tier index)
% position_in_tier: 0, 1, 2, ... (position within the tier)
% total_in_tier: total number of nodes in this tier
\newcommand{\positionInTier}[4]{
    \pgfmathsetmacro{\tieroffset}{(#2 - 1) * \currentTierSpacing / 1cm}
    \pgfmathsetmacro{\nodeoffset}{(#3 - (#4 - 1) / 2) * \nodeSpacing / 1cm}

    \ifhorizontaltierlayout
        % Horizontal layout: tiers go left to right
        \coordinate (#1-pos) at (\tieroffset, \nodeoffset);
    \else
        % Vertical layout: tiers go top to bottom
        \coordinate (#1-pos) at (\nodeoffset, -\tieroffset);
    \fi
}

% Auto-assign nodes to tiers by type
% Usage: \autoAssignTier{node_type}
% Returns tier number based on node type
\newcommand{\autoAssignTier}[1]{
    \def\nodetype{#1}
    \def\internet{internet}
    \def\firewall{firewall}
    \def\webserver{webserver}
    \def\appserver{appserver}
    \def\database{database}

    \ifx\nodetype\internet
        1
    \else\ifx\nodetype\firewall
        2
    \else\ifx\nodetype\webserver
        3
    \else\ifx\nodetype\appserver
        3
    \else\ifx\nodetype\database
        4
    \else
        3  % default to middle tier
    \fi\fi\fi\fi\fi
}

% ============================================================================
% CIRCULAR LAYOUT ENGINE
% ============================================================================

% Circular layout parameters
\newlength{\circleRadius}
\setlength{\circleRadius}{5cm}

\newcounter{satellitecount}
\newcounter{currentangle}

% Calculate optimal radius based on node count and node size
% Usage: \calculateOptimalRadius{node_count}{min_spacing}
\newcommand{\calculateOptimalRadius}[2]{
    % Circumference needed: node_count * min_spacing
    % Radius = Circumference / (2 * pi)
    \pgfmathsetlength{\circleRadius}{(#1 * #2) / (2 * 3.14159)}
    % Add minimum radius constraint
    \pgfmathsetlength{\circleRadius}{max(\circleRadius, 3cm)}
}

% Circular layout (for hub-and-spoke networks)
% Usage: \layoutCircular{center_node}{num_satellites}{radius}
% Creates coordinates for center and satellite positions
\newcommand{\layoutCircular}[3]{
    % Center node at origin (0,0)
    \coordinate (#1-pos) at (0, 0);

    % Set radius
    \setlength{\circleRadius}{#3}

    % Satellite nodes arranged in circle
    % Angular spacing: 360 / num_satellites
    \setcounter{satellitecount}{#2}
}

% Position a satellite node in circular layout
% Usage: \positionSatellite{node_name}{satellite_index}{total_satellites}{radius}
\newcommand{\positionSatellite}[4]{
    % Calculate angle for this satellite
    \pgfmathsetmacro{\nodeangle}{360 / #3 * #2}

    % Position at radius and angle
    \coordinate (#1-pos) at (\nodeangle:#4);
}

% Advanced circular layout with custom starting angle
% Usage: \layoutCircularAdvanced{center}{satellites}{radius}{start_angle}{arc_span}
% arc_span: 360 for full circle, 180 for semicircle, etc.
\newcommand{\layoutCircularAdvanced}[5]{
    % Center position
    \coordinate (#1-pos) at (0, 0);

    % Store parameters for satellite positioning
    \def\circlestartangle{#4}
    \def\circlearcspan{#5}
    \setlength{\circleRadius}{#3}
}

% Position satellite with custom arc
% Usage: \positionSatelliteArc{node_name}{index}{total}{radius}{start_angle}{arc_span}
\newcommand{\positionSatelliteArc}[6]{
    % Calculate angle within the arc
    \pgfmathsetmacro{\nodeangle}{#5 + (#6 / (#3 - 1)) * #2}

    % Position node
    \coordinate (#1-pos) at (\nodeangle:#4);
}

% Multi-ring circular layout (concentric circles)
% Usage: \layoutMultiRing{center}{rings}{nodes_per_ring}{base_radius}
\newcommand{\layoutMultiRing}[4]{
    % Center node
    \coordinate (#1-pos) at (0, 0);

    % Each ring radius increases by base_radius
    % Ring 1: base_radius, Ring 2: 2*base_radius, etc.
}

% Position node in specific ring
% Usage: \positionInRing{node_name}{ring_number}{position_in_ring}{total_in_ring}{base_radius}
\newcommand{\positionInRing}[5]{
    % Calculate radius for this ring
    \pgfmathsetmacro{\ringradius}{#2 * #5}

    % Calculate angle for position in ring
    \pgfmathsetmacro{\nodeangle}{360 / #4 * #3}

    % Position node
    \coordinate (#1-pos) at (\nodeangle:\ringradius cm);
}

% Weighted angular distribution (for varying importance)
% Usage: \positionWeighted{node_name}{weight}{total_weight}{start_angle}{end_angle}{radius}
% Allocates angular space proportional to weight
\newcommand{\positionWeighted}[6]{
    % This would calculate position based on cumulative weight
    % Higher weight = more angular space
    \pgfmathsetmacro{\weightedangle}{#4 + ((#5 - #4) * #2 / #3)}
    \coordinate (#1-pos) at (\weightedangle:#6);
}

% Spiral layout (for growing networks)
% Usage: \layoutSpiral{center}{nodes}{initial_radius}{radius_increment}{angle_increment}
\newcommand{\layoutSpiral}[5]{
    % Each node placed at increasing radius and angle
    % Creates spiral pattern
    \coordinate (#1-pos) at (0, 0);
}

% Position node in spiral
% Usage: \positionInSpiral{node_name}{index}{initial_radius}{radius_inc}{angle_inc}
\newcommand{\positionInSpiral}[5]{
    \pgfmathsetmacro{\spiralradius}{#3 + #2 * #4}
    \pgfmathsetmacro{\spiralangle}{#2 * #5}
    \coordinate (#1-pos) at (\spiralangle:\spiralradius cm);
}

% ============================================================================
% GRID LAYOUT ENGINE (Data Center Visualization)
% ============================================================================

% Grid layout parameters
\newlength{\gridSpacingX}
\newlength{\gridSpacingY}
\setlength{\gridSpacingX}{3cm}
\setlength{\gridSpacingY}{2.5cm}

\newcounter{gridrows}
\newcounter{gridcols}

% Set grid spacing
% Usage: \setGridSpacing{x_spacing}{y_spacing}
\newcommand{\setGridSpacing}[2]{
    \setlength{\gridSpacingX}{#1}
    \setlength{\gridSpacingY}{#2}
}

% Grid layout with fixed dimensions
% Usage: \layoutGrid{rows}{cols}{spacing}
\newcommand{\layoutGrid}[3]{
    \setcounter{gridrows}{#1}
    \setcounter{gridcols}{#2}
    \setlength{\gridSpacingX}{#3}
    \setlength{\gridSpacingY}{#3}
}

% Position node in grid
% Usage: \positionInGrid{node_name}{row}{col}{row_spacing}{col_spacing}
% row and col are 0-indexed
\newcommand{\positionInGrid}[5]{
    \pgfmathsetmacro{\gridx}{#3 * #5}
    \pgfmathsetmacro{\gridy}{-#2 * #4}  % negative for top-to-bottom
    \coordinate (#1-pos) at (\gridx cm, \gridy cm);
}

% Auto-calculate grid dimensions from node count
% Usage: \autoGridDimensions{node_count}{preferred_aspect_ratio}
% aspect_ratio: width/height (e.g., 1.5 for 3:2, 1.0 for square)
\newcommand{\autoGridDimensions}[2]{
    % Calculate columns and rows to approximate aspect ratio
    \pgfmathsetmacro{\gridcols}{ceil(sqrt(#1 * #2))}
    \pgfmathsetmacro{\gridrows}{ceil(#1 / \gridcols)}
}

% Irregular grid layout (varying columns per row)
% Usage: \layoutIrregularGrid{row_configs}
% row_configs: list of column counts per row
\newcommand{\layoutIrregularGrid}[1]{
    % This would parse the configuration
    % Example: {3,4,3} = 3 nodes in row 0, 4 in row 1, 3 in row 2
}

% Position in irregular grid with row-specific column count
% Usage: \positionInIrregularGrid{node}{row}{col}{cols_in_row}{spacing}
\newcommand{\positionInIrregularGrid}[5]{
    % Center each row by offsetting based on column count
    \pgfmathsetmacro{\rowoffset}{-(#4 - 1) * #5 / 2}
    \pgfmathsetmacro{\gridx}{\rowoffset + #3 * #5}
    \pgfmathsetmacro{\gridy}{-#2 * #5}
    \coordinate (#1-pos) at (\gridx cm, \gridy cm);
}

% Server rack visualization
% Usage: \drawServerRack{rack_name}{rows}{cols}{x}{y}{spacing}
% Creates a visual rack with labeled positions
\newcommand{\drawServerRack}[6]{
    \begin{scope}[shift={(#4,#5)}]
        % Draw rack outline
        \pgfmathsetmacro{\rackwidth}{#3 * #6 + 1}
        \pgfmathsetmacro{\rackheight}{#2 * #6 + 1}

        \draw[line width=2pt, black!70, rounded corners=3pt]
            (0, 0) rectangle (\rackwidth cm, -\rackheight cm);

        % Add rack label
        \node[above, font=\small\bfseries] at (\rackwidth/2 cm, 0.2cm) {#1};

        % Draw rack units (U markings)
        % Data center racks typically have 42U or 48U
    \end{scope}
}

% Position server in rack
% Usage: \positionInRack{node_name}{rack_name}{u_position}{u_height}
% u_position: rack unit position (1U, 2U, etc.)
% u_height: how many rack units the server occupies
\newcommand{\positionInRack}[4]{
    % Position relative to rack coordinates
    % Each rack unit is typically 1.75 inches = 4.445cm
    \pgfmathsetmacro{\uheight}{0.4}  % cm per rack unit (scaled for diagram)
    \pgfmathsetmacro{\ypos}{-#3 * \uheight}

    \coordinate (#1-pos) at ($(#2) + (0.5cm, \ypos cm)$);
}

% Blade server visualization (high-density servers)
% Usage: \layoutBladeChassisGrid{chassis_name}{blade_count}{blades_per_row}
\newcommand{\layoutBladeChassisGrid}[3]{
    % Blade chassis typically hold 8, 10, or 16 blade servers
    % Arrange in compact grid within chassis
    \pgfmathsetmacro{\bladerows}{ceil(#2 / #3)}
}

% Position blade server within chassis
% Usage: \positionBlade{blade_name}{chassis}{slot_number}{slots_per_row}{spacing}
\newcommand{\positionBlade}[5]{
    \pgfmathsetmacro{\bladerow}{floor((#3 - 1) / #4)}
    \pgfmathsetmacro{\bladecol}{mod(#3 - 1, #4)}

    \pgfmathsetmacro{\bladex}{\bladecol * #5}
    \pgfmathsetmacro{\bladey}{-\bladerow * #5}

    \coordinate (#1-pos) at ($(#2) + (\bladex cm, \bladey cm)$);
}

% Data center row layout (multiple racks in a row)
% Usage: \layoutDataCenterRow{row_name}{rack_count}{rack_spacing}
\newcommand{\layoutDataCenterRow}[3]{
    % Position racks in a horizontal line
    % Used for cold aisle / hot aisle visualization
}

% Position rack in data center row
% Usage: \positionRackInRow{rack_name}{row_name}{rack_index}{rack_spacing}
\newcommand{\positionRackInRow}[4]{
    \pgfmathsetmacro{\rackx}{#3 * #4}
    \coordinate (#1-pos) at ($(#2) + (\rackx cm, 0)$);
}

% Grid with alternating row colors (for readability)
% Usage: \drawGridBackground{rows}{cols}{x_spacing}{y_spacing}{color1}{color2}
\newcommand{\drawGridBackground}[6]{
    \begin{scope}[on background layer]
        \foreach \row in {0,...,#1} {
            \pgfmathparse{mod(\row,2) == 0 ? "#5" : "#6"}
            \let\rowcolor\pgfmathresult

            \pgfmathsetmacro{\ypos}{-\row * #4}
            \fill[\rowcolor, opacity=0.1]
                (0, \ypos cm) rectangle (#2 * #3 cm, \ypos cm - #4 cm);
        }
    \end{scope}
}

% ============================================================================
% BACKGROUND ELEMENTS AND OVERLAYS
% ============================================================================

% Background grid parameters
\newlength{\gridStep}
\setlength{\gridStep}{1cm}

% Draw configurable background grid
% Usage: \drawBackgroundGrid{step}{color}{opacity}
\newcommand{\drawBackgroundGrid}[3]{
    \begin{scope}[on background layer]
        \draw[step=#1, #2, opacity=#3, very thin]
            (current bounding box.south west) grid (current bounding box.north east);
    \end{scope}
}

% Draw major/minor grid (like graph paper)
% Usage: \drawGraphPaperGrid{minor_step}{major_step}{minor_color}{major_color}
\newcommand{\drawGraphPaperGrid}[4]{
    \begin{scope}[on background layer]
        % Minor grid
        \draw[step=#1, #3, opacity=0.2, very thin]
            (current bounding box.south west) grid (current bounding box.north east);
        % Major grid
        \draw[step=#2, #4, opacity=0.4, thin]
            (current bounding box.south west) grid (current bounding box.north east);
    \end{scope}
}

% Geographic map overlay for WAN diagrams
% Usage: \addGeographicMap{image_path}{x}{y}{width}{height}{opacity}
\newcommand{\addGeographicMap}[6]{
    \begin{scope}[on background layer]
        \node[opacity=#6, inner sep=0pt] at (#2, #3) {
            \includegraphics[width=#4, height=#5]{#1}
        };
    \end{scope}
}

% Simplified continent outlines (for global networks)
% Usage: \drawWorldMapOutline{scale}{opacity}
\newcommand{\drawWorldMapOutline}[2]{
    \begin{scope}[on background layer, scale=#1, opacity=#2]
        % Simplified world map can be drawn with basic shapes
        % Or include from external file
        % This is a placeholder for actual map coordinates
    \end{scope}
}

% Data center floor plan background
% Usage: \drawFloorPlan{width}{length}{rows}{cols}{aisle_width}
\newcommand{\drawFloorPlan}[5]{
    \begin{scope}[on background layer]
        % Draw floor outline
        \draw[black!30, very thick] (0,0) rectangle (#1, -#2);

        % Draw aisles
        \pgfmathsetmacro{\aislecount}{#4 - 1}
        \foreach \i in {1,...,\aislecount} {
            \pgfmathsetmacro{\xpos}{#1 / #4 * \i}
            \draw[blue!20, thick, dashed]
                (\xpos, 0) -- (\xpos, -#2);
        }

        % Draw cold/hot aisle indicators
        \foreach \i in {0,...,#3} {
            \pgfmathsetmacro{\ypos}{-#2 / #3 * \i}
            \pgfmathparse{mod(\i, 2) == 0 ? "Cold" : "Hot"}
            \let\aisletype\pgfmathresult

            \node[font=\tiny, text=gray, rotate=90]
                at (#5/2, \ypos + #2/#3/2) {\aisletype Aisle};
        }
    \end{scope}
}

% Network topology pattern backgrounds
% Usage: \drawTopologyPattern{pattern_type}{density}{color}
% pattern_type: mesh, star, ring, bus
\newcommand{\drawTopologyPattern}[3]{
    \begin{scope}[on background layer, opacity=0.05]
        \def\patterntype{#1}
        \def\mesh{mesh}
        \def\star{star}
        \def\ring{ring}

        \ifx\patterntype\mesh
            % Mesh pattern
            \foreach \x in {0,...,#2} {
                \foreach \y in {0,...,#2} {
                    \filldraw[#3] (\x, -\y) circle (0.05cm);
                }
            }
        \else\ifx\patterntype\star
            % Star pattern from center
            \foreach \angle in {0,30,...,330} {
                \draw[#3] (0,0) -- (\angle:#2);
            }
        \else\ifx\patterntype\ring
            % Concentric rings
            \foreach \r in {1,...,#2} {
                \draw[#3] (0,0) circle (\r cm);
            }
        \fi\fi\fi
    \end{scope}
}

% Custom background image support
% Usage: \setBackgroundImage{image_path}{opacity}{fit_to_diagram}
% fit_to_diagram: true to scale image to diagram bounds
\newcommand{\setBackgroundImage}[3]{
    \begin{scope}[on background layer]
        \def\fitdiagram{#3}
        \def\true{true}

        \ifx\fitdiagram\true
            % Scale to fit current diagram
            \node[opacity=#2, inner sep=0pt] at (current bounding box.center) {
                \includegraphics[
                    width=\pgfkeysvalueof{/pgf/bounding box/width},
                    height=\pgfkeysvalueof{/pgf/bounding box/height}
                ]{#1}
            };
        \else
            % Use original size
            \node[opacity=#2, inner sep=0pt] at (0,0) {
                \includegraphics{#1}
            };
        \fi
    \end{scope}
}

% Shaded region backgrounds (for zones)
% Usage: \drawZoneBackground{corner1}{corner2}{color}{opacity}{label}
\newcommand{\drawZoneBackground}[5]{
    \begin{scope}[on background layer]
        \fill[#3, opacity=#4] #1 rectangle #2;
        \node[font=\large\bfseries, text=#3!80!black, opacity=0.3]
            at ($(#1)!0.5!(#2)$) {#5};
    \end{scope}
}

% Gradient background
% Usage: \drawGradientBackground{start_color}{end_color}{direction}
% direction: vertical, horizontal, radial
\newcommand{\drawGradientBackground}[3]{
    \begin{scope}[on background layer]
        \def\direction{#3}
        \def\vertical{vertical}
        \def\horizontal{horizontal}
        \def\radial{radial}

        \ifx\direction\vertical
            \shade[top color=#1, bottom color=#2]
                (current bounding box.south west) rectangle (current bounding box.north east);
        \else\ifx\direction\horizontal
            \shade[left color=#1, right color=#2]
                (current bounding box.south west) rectangle (current bounding box.north east);
        \else\ifx\direction\radial
            \shade[inner color=#1, outer color=#2]
                (current bounding box.center) circle (5cm);
        \fi\fi\fi
    \end{scope}
}

% Coordinate system overlay (for debugging/reference)
% Usage: \drawCoordinateSystem{x_range}{y_range}{label_step}
\newcommand{\drawCoordinateSystem}[3]{
    \begin{scope}[on background layer]
        % X-axis
        \draw[->, thick, gray!50] (-#1, 0) -- (#1, 0) node[right] {x};
        % Y-axis
        \draw[->, thick, gray!50] (0, -#2) -- (0, #2) node[above] {y};

        % X labels
        \foreach \x in {-#1,-(#1-#3),...,#1} {
            \ifnum\x=0\else
                \draw[gray!50] (\x, -0.1) -- (\x, 0.1);
                \node[below, font=\tiny, gray] at (\x, -0.1) {\x};
            \fi
        }

        % Y labels
        \foreach \y in {-#2,-(#2-#3),...,#2} {
            \ifnum\y=0\else
                \draw[gray!50] (-0.1, \y) -- (0.1, \y);
                \node[left, font=\tiny, gray] at (-0.1, \y) {\y};
            \fi
        }
    \end{scope}
}

% Watermark (for draft diagrams)
% Usage: \addWatermark{text}{opacity}
\newcommand{\addWatermark}[2]{
    \begin{scope}[on background layer]
        \node[
            font=\Huge\bfseries,
            text=gray,
            opacity=#2,
            rotate=45,
            scale=3
        ] at (current bounding box.center) {#1};
    \end{scope}
}

% ============================================================================
% HIERARCHICAL TREE LAYOUTS
% ============================================================================

% Tree layout parameters
\newlength{\treeNodeSpacing}
\setlength{\treeNodeSpacing}{2.5cm}

\newlength{\treeLevelSpacing}
\setlength{\treeLevelSpacing}{3cm}

\newif\ifinvertedtree
\invertedtreefalse  % default: root at top

% Set tree orientation
% Usage: \setTreeOrientation{normal|inverted}
\newcommand{\setTreeOrientation}[1]{
    \def\temp{#1}
    \def\inverted{inverted}
    \ifx\temp\inverted
        \invertedtreetrue
    \else
        \invertedtreefalse
    \fi
}

% Set tree spacing
% Usage: \setTreeSpacing{node_spacing}{level_spacing}
\newcommand{\setTreeSpacing}[2]{
    \setlength{\treeNodeSpacing}{#1}
    \setlength{\treeLevelSpacing}{#2}
}

% Tree layout with automatic positioning
% Usage: \layoutTree{root}{levels}{branching_factor}
\newcommand{\layoutTree}[3]{
    % Root at top (or bottom if inverted)
    % Children arranged in levels below/above
    \def\treelevels{#2}
    \def\branchingfactor{#3}
}

% Position root node
% Usage: \positionTreeRoot{node_name}{x}{y}
\newcommand{\positionTreeRoot}[3]{
    \coordinate (#1-pos) at (#2, #3);
    \coordinate (#1-tree-root) at (#2, #3);
}

% Position node in tree (binary tree)
% Usage: \positionInBinaryTree{node_name}{parent}{level}{is_left}
% is_left: 1 for left child, 0 for right child
\newcommand{\positionInBinaryTree}[4]{
    \ifinvertedtree
        \pgfmathsetmacro{\yoffset}{#3 * \treeLevelSpacing / 1cm}
    \else
        \pgfmathsetmacro{\yoffset}{-#3 * \treeLevelSpacing / 1cm}
    \fi

    % Horizontal offset decreases with depth for balanced appearance
    \pgfmathsetmacro{\xoffset}{pow(2, 4 - #3) * \treeNodeSpacing / 1cm}

    \pgfmathparse{#4 == 1 ? -\xoffset : \xoffset}
    \let\finalxoffset\pgfmathresult

    \coordinate (#1-pos) at ($(#2-pos) + (\finalxoffset cm, \yoffset cm)$);
}

% Position node in n-ary tree
% Usage: \positionInTree{node_name}{parent}{level}{child_index}{total_children}
\newcommand{\positionInTree}[5]{
    \ifinvertedtree
        \pgfmathsetmacro{\yoffset}{#3 * \treeLevelSpacing / 1cm}
    \else
        \pgfmathsetmacro{\yoffset}{-#3 * \treeLevelSpacing / 1cm}
    \fi

    % Calculate horizontal spread based on level and children count
    \pgfmathsetmacro{\levelwidth}{pow(1.8, 4 - #3) * \treeNodeSpacing / 1cm}
    \pgfmathsetmacro{\childspacing}{\levelwidth / max(#5 - 1, 1)}
    \pgfmathsetmacro{\xoffset}{-\levelwidth/2 + #4 * \childspacing}

    \coordinate (#1-pos) at ($(#2-pos) + (\xoffset cm, \yoffset cm)$);
}

% Balanced tree layout (automatic centering)
% Usage: \layoutBalancedTree{root}{max_level}
\newcommand{\layoutBalancedTree}[2]{
    % Automatically calculate positions for balanced tree
    % Centers subtrees under their parents
}

% Position in balanced tree with automatic centering
% Usage: \positionBalanced{node}{parent}{level}{index_in_level}{nodes_in_level}
\newcommand{\positionBalanced}[5]{
    % Calculate total width needed for this level
    \pgfmathsetmacro{\levelwidth}{(#5 - 1) * \treeNodeSpacing / 1cm}

    % Position this node within the level
    \pgfmathsetmacro{\xpos}{-\levelwidth/2 + #4 * \treeNodeSpacing / 1cm}

    \ifinvertedtree
        \pgfmathsetmacro{\ypos}{#3 * \treeLevelSpacing / 1cm}
    \else
        \pgfmathsetmacro{\ypos}{-#3 * \treeLevelSpacing / 1cm}
    \fi

    \coordinate (#1-pos) at (\xpos cm, \ypos cm);
}

% Unbalanced tree support (varying children per node)
% Usage: \positionUnbalanced{node}{parent}{level}{offset}
\newcommand{\positionUnbalanced}[4]{
    % For trees where not all nodes have same number of children
    % offset: manual horizontal offset from parent

    \ifinvertedtree
        \pgfmathsetmacro{\yoffset}{#3 * \treeLevelSpacing / 1cm}
    \else
        \pgfmathsetmacro{\yoffset}{-#3 * \treeLevelSpacing / 1cm}
    \fi

    \coordinate (#1-pos) at ($(#2-pos) + (#4, \yoffset cm)$);
}

% Sibling spacing optimization (Reingold-Tilford algorithm inspired)
% Usage: \optimizeSiblingSpacing{node_list}
% Adjusts positions to minimize width while avoiding overlaps
\newcommand{\optimizeSiblingSpacing}[1]{
    % This would implement aesthetic tree layout
    % Ensures siblings don't overlap
    % Minimizes total tree width
    % Centers parents over children
}

% Radial tree layout (hierarchical but circular)
% Usage: \layoutRadialTree{root}{levels}{angle_span}
\newcommand{\layoutRadialTree}[3]{
    % Root at center
    % Each level forms a ring at increasing radius
    % Children distributed in angular sectors
}

% Position in radial tree
% Usage: \positionRadial{node}{level}{angle}{radius_per_level}
\newcommand{\positionRadial}[4]{
    \pgfmathsetmacro{\radius}{#2 * #4}
    \coordinate (#1-pos) at (#3:\radius cm);
}

% Tree with custom branch angles
% Usage: \layoutAngledTree{root}{left_angle}{right_angle}
% Allows non-vertical branches
\newcommand{\layoutAngledTree}[3]{
    \def\treeleftangle{#2}
    \def\treerightangle{#3}
}

% Position with custom angles
% Usage: \positionAngled{node}{parent}{distance}{angle}
\newcommand{\positionAngled}[4]{
    \coordinate (#1-pos) at ($(#2-pos) + (#4:#3)$);
}

% Compact tree layout for space efficiency
% Usage: \layoutCompactTree{root}{levels}
% Minimizes whitespace between branches
\newcommand{\layoutCompactTree}[2]{
    \setlength{\treeNodeSpacing}{1.5cm}
    \setlength{\treeLevelSpacing}{2cm}
}

% Calculate tree dimensions
% Usage: \calculateTreeDimensions{levels}{branching}{node_width}
% Returns total width and height needed
\newcommand{\calculateTreeDimensions}[3]{
    % Width: branching^levels * node_width
    \pgfmathsetmacro{\treewidth}{pow(#2, #1) * #3}

    % Height: levels * level_spacing
    \pgfmathsetmacro{\treeheight}{#1 * \treeLevelSpacing / 1cm}
}

% ============================================================================
% ORGANIC/FORCE-DIRECTED LAYOUTS
% ============================================================================

% Force-directed layout parameters
\newlength{\springlength}
\setlength{\springlength}{3cm}

\newcommand{\setSpringLength}[1]{
    \setlength{\springlength}{#1}
}

% Import positions from external force-directed calculation
% Usage: \importForceDirectedPositions{filename}
% File format: node_name,x,y (CSV)
\newcommand{\importForceDirectedPositions}[1]{
    % This would read from external file
    % Example positions calculated by GraphViz, networkx, etc.
    % \DTLloaddb{positions}{#1}
    % \DTLforeach{positions}{\nodename=node,\xpos=x,\ypos=y}{
    %     \coordinate (\nodename-pos) at (\xpos,\ypos);
    % }
}

% Export network topology for external force-directed calculation
% Usage: \exportForceDirectedTopology{filename}
% Output format compatible with GraphViz DOT or networkx
\newcommand{\exportForceDirectedTopology}[1]{
    % Write nodes and edges to file
    % This would be populated with actual network data
}

% Simple spring-embedder layout (basic implementation)
% Usage: \layoutSpringEmbedder{iterations}{cooling_factor}
% This is a simplified version - for production use external tools
\newcommand{\layoutSpringEmbedder}[2]{
    % Iterative spring force calculation
    % iterations: number of simulation steps
    % cooling_factor: reduction in movement per iteration (0.0-1.0)

    % Initial random placement handled by caller
    % For each iteration:
    %   - Calculate repulsive forces between all node pairs
    %   - Calculate attractive forces along edges
    %   - Update positions based on net force
    %   - Apply cooling factor

    % NOTE: Full implementation requires LuaTeX for practical performance
}

% Force-directed layout using TikZ graph library
% Usage: \layoutForceDirected{scale}{spring_constant}{electrical_charge}
% This uses TikZ's built-in spring layout
\newcommand{\layoutForceDirected}[3]{
    % Uses tikz graph library spring layout
    % scale: overall size multiplier
    % spring_constant: strength of edge springs (default 0.2)
    % electrical_charge: strength of node repulsion (default 1)

    % Example usage in main document:
    % \tikz \graph [spring layout, node distance=#1,
    %               spring constant=#2, electric charge=#3] {
    %     ... nodes and edges ...
    % };
}

% Fruchterman-Reingold algorithm parameters
\newcommand{\setFruchtermanReingoldParams}[3]{
    % #1: optimal distance (k)
    % #2: temperature (initial displacement limit)
    % #3: iterations

    \def\FRoptimaldist{#1}
    \def\FRtemperature{#2}
    \def\FRiterations{#3}
}

% Calculate attractive force (Fruchterman-Reingold)
\newcommand{\FRattractiveForce}[2]{
    % Force = d^2 / k
    % where d is distance, k is optimal distance
    \pgfmathsetmacro{\FRattractive}{(#1 * #1) / #2}
}

% Calculate repulsive force (Fruchterman-Reingold)
\newcommand{\FRrepulsiveForce}[2]{
    % Force = k^2 / d
    % where k is optimal distance, d is distance
    \pgfmathsetmacro{\FRrepulsive}{(#2 * #2) / #1}
}

% Integration with external tools
% Usage: \useExternalLayoutEngine{tool}{input_file}{output_file}
% tool: graphviz, networkx, gephi, etc.
\newcommand{\useExternalLayoutEngine}[3]{
    % Call external tool to calculate layout
    % Example for GraphViz:
    % \immediate\write18{neato -Tplain #2 > #3}
    % Then import resulting positions
    % \importForceDirectedPositions{#3}
}

% Helper: Calculate Euclidean distance between two points
\newcommand{\calculateDistance}[4]{
    % #1, #2: coordinates of first point (x1, y1)
    % #3, #4: coordinates of second point (x2, y2)
    \pgfmathsetmacro{\distance}{sqrt((#3 - #1)^2 + (#4 - #2)^2)}
}

% ============================================================================
% SUBNET AUTO-GROUPING
% ============================================================================

% IP address parsing helpers
% Parse IPv4 address into octets
% Usage: \parseIPv4{ip_address}
% Sets \ipoctetA, \ipoctetB, \ipoctetC, \ipoctetD
\newcommand{\parseIPv4}[1]{
    % Split IP address by dots
    % Example: 192.168.1.10 -> octetA=192, octetB=168, octetC=1, octetD=10
    % This requires string manipulation

    \def\ipaddress{#1}
    % Note: Full implementation requires expl3 or LuaTeX for string parsing
}

% Calculate network address from IP and subnet mask
% Usage: \calculateNetwork{ip}{cidr}
% Example: \calculateNetwork{192.168.1.10}{24} -> 192.168.1.0/24
\newcommand{\calculateNetwork}[2]{
    % #1 = IP address (e.g., 192.168.1.10)
    % #2 = CIDR prefix (e.g., 24 for /24)

    % Convert IP to 32-bit integer
    % Apply subnet mask
    % Convert back to dotted decimal
    % Store in \networkaddress
}

% Determine if two IPs are in the same subnet
% Usage: \sameSubnet{ip1}{ip2}{cidr}
% Sets \ifsamesubnet boolean
\newif\ifsamesubnet
\newcommand{\sameSubnetCheck}[3]{
    % Compare network portions of both IPs
    % Example: 192.168.1.10 and 192.168.1.20 with /24 -> true
    % Example: 192.168.1.10 and 192.168.2.10 with /24 -> false

    \samesubnettrue  % placeholder
}

% Automatically group nodes by IP subnet
% Usage: \autoGroupSubnets{subnet_mask}
% Creates visual groupings for all nodes based on their IP addresses
\newcommand{\autoGroupSubnets}[1]{
    % Analyze all node IP addresses
    % Group by common network address
    % Create subnet boundary boxes
    % Apply color coding based on subnet
}

% Create subnet boundary from IP range
% Usage: \createSubnetBoundary{subnet_cidr}{nodes}{color}{trust_level}
% Example: \createSubnetBoundary{192.168.1.0/24}{(node1)(node2)}{blue}{high}
\newcommand{\createSubnetBoundary}[4]{
    \drawSecurityZone{subnet-#1}{#3}{#2}{#1}{#4}
}

% Auto-detect subnet from node list
% Usage: \detectSubnet{node_list}
% Analyzes IP addresses and determines common subnet
\newcommand{\detectSubnet}[1]{
    % Parse IP addresses from all nodes
    % Find common network prefix
    % Determine appropriate CIDR notation
    % Return subnet identifier
}

% Color coding for different subnet types
\newcommand{\subnetColor}[1]{
    % #1 = subnet type: dmz, internal, external, management, etc.

    \def\subnettype{#1}
    \def\dmz{dmz}
    \def\internal{internal}
    \def\external{external}
    \def\management{management}

    \ifx\subnettype\dmz
        orange
    \else\ifx\subnettype\internal
        blue
    \else\ifx\subnettype\external
        red
    \else\ifx\subnettype\management
        purple
    \else
        gray  % default
    \fi\fi\fi\fi
}

% Trust level assignment based on subnet
% Usage: \assignTrustLevel{subnet_first_octet}
% Returns: high, medium, low, untrusted
\newcommand{\assignTrustLevel}[1]{
    % 10.x.x.x, 172.16-31.x.x, 192.168.x.x -> internal (high)
    % DMZ ranges -> medium
    % Public IPs -> low/untrusted

    \pgfmathparse{#1 == 10 || #1 == 192 ? "high" : "medium"}
}

% Handle nested/overlapping subnets (VLAN support)
% Usage: \createNestedSubnet{parent_subnet}{child_subnet}{nodes}{color}
\newcommand{\createNestedSubnet}[4]{
    % Draw child subnet inside parent
    % Use nested fit nodes
    % Apply visual hierarchy (inner box inside outer)

    \begin{scope}[on background layer]
        % Parent subnet
        \node[
            fit=#3,
            draw=#4!40,
            fill=#4!3,
            rounded corners=10pt,
            line width=3pt,
            inner sep=25pt,
            label={above:#1}
        ] (parent-#1) {};

        % Child subnet
        \node[
            fit=#3,
            draw=#4!70,
            fill=#4!8,
            rounded corners=5pt,
            line width=1.5pt,
            inner sep=15pt,
            label={above:#2}
        ] (child-#2) {};
    \end{scope}
}

% Parse CIDR notation
% Usage: \parseCIDR{192.168.1.0/24}
% Sets \cidrnetwork and \cidrprefix
\newcommand{\parseCIDR}[1]{
    % Split by '/' character
    % Store network part in \cidrnetwork
    % Store prefix length in \cidrprefix

    \def\cidrnotation{#1}
    % Implementation requires string parsing
}

% Calculate subnet capacity
% Usage: \subnetCapacity{cidr_prefix}
% Returns number of usable hosts
\newcommand{\subnetCapacity}[1]{
    % For /24: 2^(32-24) - 2 = 254 hosts
    % For /16: 2^(32-16) - 2 = 65534 hosts

    \pgfmathsetmacro{\capacity}{2^(32 - #1) - 2}
}

% Visualize subnet utilization
% Usage: \showSubnetUtilization{subnet}{used_ips}{total_ips}
\newcommand{\showSubnetUtilization}[3]{
    % Display utilization percentage
    % Color code: green (<50%), yellow (50-80%), red (>80%)

    \pgfmathsetmacro{\utilization}{100 * #2 / #3}

    \pgfmathparse{\utilization < 50 ? "green" : (\utilization < 80 ? "yellow" : "red")}
    \let\utilizationcolor\pgfmathresult

    % Draw utilization bar/indicator
}

% Auto-layout nodes within subnet boundary
% Usage: \layoutSubnetNodes{subnet_name}{node_list}{layout_type}
% layout_type: grid, circular, linear
\newcommand{\layoutSubnetNodes}[3]{
    % Arrange nodes within subnet boundary
    % Optimize spacing to fit within boundary
    % Maintain visual clarity

    \def\layouttype{#3}
    \def\grid{grid}
    \def\circular{circular}

    \ifx\layouttype\grid
        % Grid layout within subnet
    \else\ifx\layouttype\circular
        % Circular layout within subnet
    \else
        % Linear layout (default)
    \fi\fi
}

% ============================================================================
% LAYOUT HELPERS AND UTILITIES
% ============================================================================

% Calculate midpoint between two nodes
% Usage: \nodeMidpoint{node1}{node2}{resultname}
\newcommand{\nodeMidpoint}[3]{
    \coordinate (#3) at ($(#1)!0.5!(#2)$);
}

% Position node relative to another
% Usage: \positionRelative{newnode}{refnode}{distance}{angle}{style}
\newcommand{\positionRelative}[5]{
    \node[#5] (#1) at ($(#2)+({#4}:{#3})$) {};
}

% ============================================================================
% COLLISION DETECTION AND AVOIDANCE
% ============================================================================

% Minimum safe spacing between nodes
\newlength{\minNodeSpacing}
\setlength{\minNodeSpacing}{2cm}

% Node size estimation (for collision detection)
\newlength{\nodeWidth}
\newlength{\nodeHeight}
\setlength{\nodeWidth}{2cm}
\setlength{\nodeHeight}{1.5cm}

% Set minimum spacing between nodes
% Usage: \setMinNodeSpacing{distance}
\newcommand{\setMinNodeSpacing}[1]{
    \setlength{\minNodeSpacing}{#1}
}

% Set typical node dimensions (for collision detection)
% Usage: \setNodeDimensions{width}{height}
\newcommand{\setNodeDimensions}[2]{
    \setlength{\nodeWidth}{#1}
    \setlength{\nodeHeight}{#2}
}

% Check if two nodes overlap (basic rectangular collision)
% Usage: \checkNodeCollision{node1}{node2}
% Stores result in \ifnodecollision boolean
\newif\ifnodecollision
\newcommand{\checkNodeCollision}[2]{
    % Extract coordinates of both nodes
    \pgfextractx{\pgf@xa}{\pgfpointanchor{#1}{center}}
    \pgfextracty{\pgf@ya}{\pgfpointanchor{#1}{center}}
    \pgfextractx{\pgf@xb}{\pgfpointanchor{#2}{center}}
    \pgfextracty{\pgf@yb}{\pgfpointanchor{#2}{center}}

    % Calculate distance
    \pgfmathsetmacro{\dx}{abs(\pgf@xb - \pgf@xa)}
    \pgfmathsetmacro{\dy}{abs(\pgf@yb - \pgf@ya)}

    % Check if distance is less than minimum spacing
    \pgfmathparse{\dx < \minNodeSpacing && \dy < \minNodeSpacing ? 1 : 0}
    \ifnum\pgfmathresult=1
        \nodecollisiontrue
    \else
        \nodecollisionfalse
    \fi
}

% Adjust position to avoid collision
% Usage: \avoidCollision{moving_node}{fixed_node}{min_distance}
% Moves the first node away from the second if they're too close
\newcommand{\avoidCollision}[3]{
    % Get positions
    \pgfextractx{\pgf@xa}{\pgfpointanchor{#1}{center}}
    \pgfextracty{\pgf@ya}{\pgfpointanchor{#1}{center}}
    \pgfextractx{\pgf@xb}{\pgfpointanchor{#2}{center}}
    \pgfextracty{\pgf@yb}{\pgfpointanchor{#2}{center}}

    % Calculate current distance
    \pgfmathsetmacro{\currentdist}{sqrt((\pgf@xb - \pgf@xa)^2 + (\pgf@yb - \pgf@ya)^2)}

    % If too close, push away
    \pgfmathparse{\currentdist < #3 ? 1 : 0}
    \ifnum\pgfmathresult=1
        % Calculate push direction
        \pgfmathsetmacro{\pushangle}{atan2(\pgf@ya - \pgf@yb, \pgf@xa - \pgf@xb)}
        \pgfmathsetmacro{\pushdist}{#3 - \currentdist + 0.5cm}  % extra margin

        % Move node
        \pgfmathsetmacro{\newx}{\pgf@xa + \pushdist * cos(\pushangle)}
        \pgfmathsetmacro{\newy}{\pgf@ya + \pushdist * sin(\pushangle)}

        \node[shift={(\newx pt, \newy pt)}] (#1) {};
    \fi
}

% Grid snapping for alignment
% Usage: \snapToGrid{node_name}{grid_size}
% Snaps node to nearest grid point
\newcommand{\snapToGrid}[2]{
    \pgfextractx{\pgf@xa}{\pgfpointanchor{#1}{center}}
    \pgfextracty{\pgf@ya}{\pgfpointanchor{#1}{center}}

    % Round to nearest grid point
    \pgfmathsetmacro{\snappedx}{round(\pgf@xa / #2) * #2}
    \pgfmathsetmacro{\snappedy}{round(\pgf@ya / #2) * #2}

    % Reposition node
    \coordinate (#1-snapped) at (\snappedx, \snappedy);
}

% Magnetic alignment - align to nearby nodes
% Usage: \magneticAlign{node_name}{alignment_threshold}
% If within threshold of another node's x or y, snap to it
\newcommand{\magneticAlign}[2]{
    % This would iterate through all nodes and check alignment
    % Align if within threshold (e.g., 0.5cm)
    % Implementation requires node registry/list
}

% Distribute nodes evenly in a rectangular area
% Usage: \distributeNodesEvenly{node_list}{x_min}{y_min}{x_max}{y_max}
\newcommand{\distributeNodesEvenly}[5]{
    % Count nodes (requires list parsing)
    % Calculate grid that fits all nodes
    % Position each node at grid point
    % Handles both horizontal and vertical distribution
}

% Detect and resolve all collisions in diagram
% Usage: \resolveAllCollisions{max_iterations}
% Iteratively adjusts positions until no collisions
\newcommand{\resolveAllCollisions}[1]{
    % Iterate up to max_iterations times
    % For each iteration:
    %   - Check all node pairs for collisions
    %   - Apply small repulsive forces to overlapping pairs
    %   - Update positions
    %   - If no collisions detected, stop early
    % This is a simplified collision resolution system
}

% Calculate optimal spacing based on node count
% Usage: \calculateOptimalSpacing{node_count}{area_width}{area_height}
\newcommand{\calculateOptimalSpacing}[3]{
    % Calculate grid that fits #1 nodes in area #2 x #3
    \pgfmathsetmacro{\gridcols}{ceil(sqrt(#1 * #2 / #3))}
    \pgfmathsetmacro{\gridrows}{ceil(#1 / \gridcols)}

    \pgfmathsetlength{\nodeSpacing}{min(#2 / \gridcols, #3 / \gridrows) * 0.8}
}

% Check if point is inside bounding box
% Usage: \pointInBox{x}{y}{x_min}{y_min}{x_max}{y_max}
% Sets \ifpointinbox boolean
\newif\ifpointinbox
\newcommand{\pointInBox}[6]{
    \pgfmathparse{#1 >= #3 && #1 <= #5 && #2 >= #4 && #2 <= #6 ? 1 : 0}
    \ifnum\pgfmathresult=1
        \pointinboxtrue
    \else
        \pointinboxfalse
    \fi
}

% ============================================================================
% MULTI-PAGE LAYOUT SUPPORT
% ============================================================================

% Multi-page layout state
\newcounter{diagrampage}
\newif\ifmultipagediagram
\multipagediagramfalse

% Enable multi-page mode
% Usage: \enableMultiPage
\newcommand{\enableMultiPage}{
    \multipagediagramtrue
    \setcounter{diagrampage}{1}
}

% Disable multi-page mode
% Usage: \disableMultiPage
\newcommand{\disableMultiPage}{
    \multipagediagramfalse
}

% Start a new diagram page
% Usage: \newDiagramPage{page_title}
\newcommand{\newDiagramPage}[1]{
    \ifmultipagediagram
        \clearpage
        \stepcounter{diagrampage}

        % Add page header
        \begin{center}
            \textbf{\Large #1}\\[0.5em]
            \small Page \arabic{diagrampage}
        \end{center}
        \vspace{1em}
    \fi
}

% Create overview page (high-level architecture)
% Usage: \createOverviewPage{title}{scale}
% scale: reduction factor for overview (e.g., 0.5 = 50%)
\newcommand{\createOverviewPage}[2]{
    \newDiagramPage{#1 - Overview}

    \begin{tikzpicture}[scale=#2, transform shape]
        % Overview content goes here
        % Shows simplified view of entire network
    \end{tikzpicture}
}

% Create detail page for specific subnet
% Usage: \createDetailPage{subnet_name}{title}
\newcommand{\createDetailPage}[2]{
    \newDiagramPage{#2 - Detail View}

    % Full-scale diagram of specific subnet
    % Shows all nodes and connections within subnet
}

% Cross-reference marker for multi-page navigation
% Usage: \addPageReference{label}{target_page}{position}
\newcommand{\addPageReference}[3]{
    \node[
        draw=blue!70,
        fill=blue!10,
        rounded corners=2pt,
        font=\tiny\ttfamily,
        inner sep=2pt
    ] at #3 {See p.#2: #1};
}

% Add "continued on next page" indicator
% Usage: \addContinuationMarker{next_page_number}
\newcommand{\addContinuationMarker}[1]{
    \node[
        anchor=south east,
        font=\small\itshape,
        text=gray
    ] at (current bounding box.south east) {Continued on page #1 $\rightarrow$};
}

% Add "continued from previous page" indicator
% Usage: \addContinuationFromPrevious{prev_page_number}
\newcommand{\addContinuationFromPrevious}[1]{
    \node[
        anchor=north west,
        font=\small\itshape,
        text=gray
    ] at (current bounding box.north west) {$\leftarrow$ Continued from page #1};
}

% Create thumbnail navigation map
% Usage: \createThumbnailMap{total_pages}
\newcommand{\createThumbnailMap}[1]{
    % Small overview showing which page contains which subnet
    \begin{tikzpicture}[scale=0.3]
        \foreach \page in {1,...,#1} {
            \pgfmathsetmacro{\xpos}{mod(\page-1, 4) * 3}
            \pgfmathsetmacro{\ypos}{-floor((\page-1) / 4) * 2}

            \node[
                draw,
                minimum width=2.5cm,
                minimum height=1.8cm,
                font=\tiny
            ] at (\xpos, \ypos) {Page \page};
        }
    \end{tikzpicture}
}

% Automatic page breaking based on node count
% Usage: \autoPageBreak{max_nodes_per_page}
\newcommand{\autoPageBreak}[1]{
    % Analyzes network size
    % Automatically splits into pages
    % Groups nodes by subnet or proximity
}

% Calculate required pages
% Usage: \calculatePageCount{total_nodes}{nodes_per_page}
\newcommand{\calculatePageCount}[2]{
    \pgfmathsetmacro{\totalpages}{ceil(#1 / #2)}
}

% Consistent coordinate system across pages
% Usage: \setGlobalOrigin{x}{y}
% Ensures same reference point on all pages
\newcommand{\setGlobalOrigin}[2]{
    \coordinate (global-origin) at (#1, #2);
}

% Zoom region for detail page
% Usage: \createZoomRegion{x_min}{y_min}{x_max}{y_max}{scale}
\newcommand{\createZoomRegion}[5]{
    % Define region of interest from overview
    % Render at larger scale on detail page

    \begin{scope}[
        shift={(#1,#2)},
        scale=#5,
        local bounding box=zoom-region
    ]
        % Content here is zoomed
    \end{scope}
}

% Add zoom indicator on overview page
% Usage: \markZoomRegion{x_min}{y_min}{x_max}{y_max}{target_page}
\newcommand{\markZoomRegion}[5]{
    \draw[
        dashed,
        thick,
        blue!60,
        rounded corners=3pt
    ] (#1,#2) rectangle (#3,#4);

    \node[
        fill=blue!20,
        draw=blue!60,
        rounded corners=2pt,
        font=\tiny\bfseries
    ] at ($(#1,#2)!0.5!(#3,#4)$) {Detail: Page #5};
}

% Split large subnet across pages
% Usage: \splitSubnet{subnet_name}{split_count}
\newcommand{\splitSubnet}[2]{
    % Divides subnet nodes across multiple pages
    % Maintains connections between pages
}

% Connection continuation between pages
% Usage: \drawPageConnection{node_on_this_page}{node_on_other_page}{other_page}
\newcommand{\drawPageConnection}[3]{
    % Draw connection that goes off-page
    % Add indicator showing which page it continues to

    \draw[connection] (#1) -- ++(2cm, 0)
        node[
            midway,
            above,
            font=\tiny,
            fill=white
        ] {$\rightarrow$ p.#3: #2};
}

% ============================================================================
% DYNAMIC LAYOUT ADJUSTMENT
% ============================================================================

% Diagram complexity metrics
\newcounter{nodecount}
\newcounter{connectioncount}
\newlength{\diagramwidth}
\newlength{\diagramheight}

% Calculate diagram complexity score
% Usage: \calculateComplexityScore{nodes}{connections}
% Returns score: 1-10 (1=simple, 10=very complex)
\newcommand{\calculateComplexityScore}[2]{
    \setcounter{nodecount}{#1}
    \setcounter{connectioncount}{#2}

    % Complexity based on nodes + connection density
    \pgfmathsetmacro{\complexity}{min(10, (#1 / 10) + (#2 / #1))}
}

% Auto-adjust spacing based on complexity
% Usage: \autoAdjustSpacing{complexity_score}
\newcommand{\autoAdjustSpacing}[1]{
    % Higher complexity = more spacing needed
    \pgfmathsetlength{\nodeSpacing}{2cm + #1 * 0.3cm}
    \pgfmathsetlength{\layerSpacing}{4cm + #1 * 0.5cm}
}

% Optimize layout for readability
% Usage: \optimizeLayout{node_count}{connection_count}
\newcommand{\optimizeLayout}[2]{
    % Calculate complexity
    \calculateComplexityScore{#1}{#2}

    % Adjust spacing
    \pgfmathsetmacro{\complexityscore}{min(10, (#1 / 10) + (#2 / #1))}
    \autoAdjustSpacing{\complexityscore}

    % Apply collision avoidance if needed
    \pgfmathparse{#2 > #1 * 2 ? 1 : 0}
    \ifnum\pgfmathresult=1
        % High connection density - enable collision avoidance
    \fi
}

% Optimize for print output
% Usage: \optimizeForPrint{paper_size}
% paper_size: a4, letter, a3, etc.
\newcommand{\optimizeForPrint}[1]{
    \def\papersize{#1}
    \def\afour{a4}
    \def\letter{letter}
    \def\athree{a3}

    \ifx\papersize\afour
        \setlength{\diagramwidth}{18cm}
        \setlength{\diagramheight}{25cm}
    \else\ifx\papersize\letter
        \setlength{\diagramwidth}{18cm}
        \setlength{\diagramheight}{23cm}
    \else\ifx\papersize\athree
        \setlength{\diagramwidth}{28cm}
        \setlength{\diagramheight}{38cm}
    \else
        \setlength{\diagramwidth}{20cm}
        \setlength{\diagramheight}{25cm}
    \fi\fi\fi

    % Adjust node sizes for print clarity
    \tikzset{
        every node/.append style={font=\small},
        minimum height=0.8cm
    }
}

% Optimize for screen display
% Usage: \optimizeForScreen{resolution}
% resolution: hd, fullhd, 4k
\newcommand{\optimizeForScreen}[1]{
    \def\resolution{#1}
    \def\hd{hd}
    \def\fullhd{fullhd}
    \def\fourk{4k}

    \ifx\resolution\hd
        \setlength{\diagramwidth}{16cm}
        \setlength{\diagramheight}{12cm}
    \else\ifx\resolution\fullhd
        \setlength{\diagramwidth}{24cm}
        \setlength{\diagramheight}{13.5cm}
    \else\ifx\resolution\fourk
        \setlength{\diagramwidth}{32cm}
        \setlength{\diagramheight}{18cm}
    \else
        \setlength{\diagramwidth}{20cm}
        \setlength{\diagramheight}{15cm}
    \fi\fi\fi

    % Brighter colors for screen
    \tikzset{
        every node/.append style={fill opacity=0.9}
    }
}

% Responsive layout - automatically scale to fit
% Usage: \responsiveLayout{max_width}{max_height}{node_count}
\newcommand{\responsiveLayout}[3]{
    % Calculate required space
    \pgfmathsetmacro{\requiredwidth}{sqrt(#3) * \nodeSpacing / 1cm}
    \pgfmathsetmacro{\requiredheight}{sqrt(#3) * \nodeSpacing / 1cm}

    % Scale if exceeds max dimensions
    \pgfmathsetmacro{\scalex}{#1 / \requiredwidth}
    \pgfmathsetmacro{\scaley}{#2 / \requiredheight}
    \pgfmathsetmacro{\scale}{min(\scalex, \scaley, 1.0)}

    % Apply scaling if needed
}

% Identify overlapping elements
% Usage: \detectOverlaps{tolerance}
% tolerance: minimum distance considered overlapping
\newcommand{\detectOverlaps}[1]{
    % Check all node pairs
    % Mark overlapping pairs
    % Store in overlap list for correction
}

% Fix overlapping elements
% Usage: \fixOverlaps{method}
% method: push, scale, rearrange
\newcommand{\fixOverlaps}[1]{
    \def\method{#1}
    \def\push{push}
    \def\scale{scale}
    \def\rearrange{rearrange}

    \ifx\method\push
        % Push overlapping nodes apart
        \resolveAllCollisions{10}
    \else\ifx\method\scale
        % Scale down entire diagram
    \else\ifx\method\rearrange
        % Completely re-layout problem areas
    \fi\fi\fi
}

% Calculate optimal scale for diagram
% Usage: \calculateOptimalScale{target_width}{target_height}
\newcommand{\calculateOptimalScale}[2]{
    % Measure current diagram bounds
    % Calculate scale to fit in target dimensions
    % Return scale factor
}

% Adaptive density - adjust node size based on count
% Usage: \adaptiveNodeSize{node_count}
\newcommand{\adaptiveNodeSize}[1]{
    \pgfmathparse{#1 < 20 ? "large" : (#1 < 50 ? "medium" : "small")}
    \let\nodesize\pgfmathresult

    \def\large{large}
    \def\medium{medium}
    \def\small{small}

    \ifx\nodesize\large
        \tikzset{
            every node/.append style={minimum width=2.5cm, minimum height=1.2cm, font=\normalsize}
        }
    \else\ifx\nodesize\medium
        \tikzset{
            every node/.append style={minimum width=2cm, minimum height=1cm, font=\small}
        }
    \else\ifx\nodesize\small
        \tikzset{
            every node/.append style={minimum width=1.5cm, minimum height=0.8cm, font=\footnotesize}
        }
    \fi\fi\fi
}

% Balance layout distribution
% Usage: \balanceLayout{axis}
% axis: horizontal, vertical, both
\newcommand{\balanceLayout}[1]{
    % Center nodes around origin
    % Distribute weight evenly
    % Minimize bounding box size
}

% Compact layout algorithm
% Usage: \compactLayout{direction}
% direction: horizontal, vertical, radial
\newcommand{\compactLayout}[1]{
    % Minimize whitespace
    % Move nodes closer together while avoiding overlaps
    % Optimize for space efficiency
}

% Performance optimization for large diagrams
% Usage: \optimizePerformance{node_count}
\newcommand{\optimizePerformance}[1]{
    \pgfmathparse{#1 > 100 ? 1 : 0}
    \ifnum\pgfmathresult=1
        % Disable shadows and complex effects
        \tikzset{
            every node/.append style={drop shadow=none},
            every path/.append style={line width=0.5pt}
        }
    \fi
}

% Layout quality check
% Usage: \checkLayoutQuality
% Analyzes and reports layout issues
\newcommand{\checkLayoutQuality}{
    % Check for:
    % - Overlapping nodes
    % - Crossing connections
    % - Unbalanced distribution
    % - Poor spacing
    % - Excessive diagram size
}

% ============================================================================
% ZOOM LEVEL SYSTEM (LOD - Level of Detail)
% ============================================================================

% Zoom level state
\newcounter{zoomlevel}
\setcounter{zoomlevel}{100}  % percentage

% Define zoom levels
\newcommand{\setZoomLevel}[1]{
    \setcounter{zoomlevel}{#1}
}

% Get current zoom as scale factor
\newcommand{\getZoomScale}{
    \pgfmathsetmacro{\zoomscale}{\value{zoomlevel} / 100}
}

% Apply zoom to scope
% Usage: \applyZoom{content}
\newcommand{\applyZoom}[1]{
    \getZoomScale
    \begin{scope}[scale=\zoomscale, transform shape]
        #1
    \end{scope}
}

% Zoom presets
% Usage: \setZoomPreset{preset_name}
% preset_name: overview, normal, detail, extreme
\newcommand{\setZoomPreset}[1]{
    \def\preset{#1}
    \def\overview{overview}
    \def\normal{normal}
    \def\detail{detail}
    \def\extreme{extreme}

    \ifx\preset\overview
        \setcounter{zoomlevel}{50}
    \else\ifx\preset\normal
        \setcounter{zoomlevel}{100}
    \else\ifx\preset\detail
        \setcounter{zoomlevel}{150}
    \else\ifx\preset\extreme
        \setcounter{zoomlevel}{200}
    \fi\fi\fi\fi
}

% Level of Detail (LOD) rendering
% Shows/hides elements based on zoom level
% Usage: \ifZoomAtLeast{threshold}{content}
\newcommand{\ifZoomAtLeast}[2]{
    \pgfmathparse{\value{zoomlevel} >= #1 ? 1 : 0}
    \ifnum\pgfmathresult=1
        #2
    \fi
}

% Show labels only at sufficient zoom
% Usage: \showAtZoom{min_zoom}{label_content}
\newcommand{\showAtZoom}[2]{
    \ifZoomAtLeast{#1}{#2}
}

% Adaptive detail based on zoom
% Usage: \adaptiveDetail{zoom_threshold}{simple_content}{detailed_content}
\newcommand{\adaptiveDetail}[3]{
    \pgfmathparse{\value{zoomlevel} >= #1 ? 1 : 0}
    \ifnum\pgfmathresult=1
        #3  % detailed version
    \else
        #2  % simple version
    \fi
}

% Smooth zoom transition
% Usage: \smoothZoom{start_zoom}{end_zoom}{steps}
\newcommand{\smoothZoom}[3]{
    % For animation or multi-page progressive zoom
    \pgfmathsetmacro{\zoomstep}{(#2 - #1) / #3}
}

% ============================================================================
% LAYOUT PRESETS (Quick Start Templates)
% ============================================================================

% Apply enterprise 3-tier architecture preset
% Usage: \applyEnterprisePreset
\newcommand{\applyEnterprisePreset}{
    \setTierOrientation{vertical}
    \setlength{\treeLevelSpacing}{4cm}
    \setlength{\nodeSpacing}{3cm}
    \optimizeForPrint{a4}
}

% Apply data center layout preset
% Usage: \applyDataCenterPreset
\newcommand{\applyDataCenterPreset}{
    \setGridSpacing{2.5cm}{2cm}
    \setNodeDimensions{2cm}{1cm}
    \drawFloorPlan{20cm}{15cm}{4}{5}{1cm}
}

% Apply cloud architecture preset
% Usage: \applyCloudPreset
\newcommand{\applyCloudPreset}{
    \setTierOrientation{horizontal}
    \setlength{\nodeSpacing}{3.5cm}
    \adaptiveNodeSize{30}
    \optimizeForScreen{fullhd}
}

% Apply small office network preset
% Usage: \applySmallOfficePreset
\newcommand{\applySmallOfficePreset}{
    \calculateOptimalRadius{10}{2cm}
    \setlength{\nodeSpacing}{2.5cm}
    \optimizeForPrint{letter}
}

% Apply IoT network preset (many nodes)
% Usage: \applyIoTPreset
\newcommand{\applyIoTPreset}{
    \adaptiveNodeSize{100}
    \optimizePerformance{100}
    \setZoomPreset{overview}
}

% Apply security operations center (SOC) preset
% Usage: \applySOCPreset
\newcommand{\applySOCPreset}{
    % High visibility colors
    \tikzset{every node/.append style={line width=1pt}}
    \setlength{\nodeSpacing}{3cm}
}

% ============================================================================
% DEBUG AND VISUALIZATION HELPERS
% ============================================================================

% Enable debug mode
\newif\ifdebugmode
\debugmodefalse

\newcommand{\enableDebugMode}{
    \debugmodetrue
}

\newcommand{\disableDebugMode}{
    \debugmodefalse
}

% Show node bounding boxes
% Usage: \showNodeBounds{node_name}
\newcommand{\showNodeBounds}[1]{
    \ifdebugmode
        \draw[red, dashed, thin] (#1.south west) rectangle (#1.north east);
    \fi
}

% Show node center point
% Usage: \showNodeCenter{node_name}
\newcommand{\showNodeCenter}[1]{
    \ifdebugmode
        \fill[red] (#1.center) circle (0.05cm);
        \node[above right, font=\tiny, red] at (#1.center) {#1};
    \fi
}

% Display node coordinates
% Usage: \showNodeCoords{node_name}
\newcommand{\showNodeCoords}[1]{
    \ifdebugmode
        \pgfextractx{\pgf@xa}{\pgfpointanchor{#1}{center}}
        \pgfextracty{\pgf@ya}{\pgfpointanchor{#1}{center}}
        \node[below, font=\tiny, blue] at (#1.south)
            {(\pgfmathprintnumber{\pgf@xa}, \pgfmathprintnumber{\pgf@ya})};
    \fi
}

% Highlight collision zones
% Usage: \highlightCollisions
\newcommand{\highlightCollisions}{
    \ifdebugmode
        % Visual indicator for nodes too close together
        % Would check all pairs and draw warning circles
    \fi
}

% Show spacing guides
% Usage: \showSpacingGuides{node1}{node2}
\newcommand{\showSpacingGuides}[2]{
    \ifdebugmode
        \draw[green, dashed] (#1.center) -- (#2.center);
        \pgfextractx{\pgf@xa}{\pgfpointanchor{#1}{center}}
        \pgfextracty{\pgf@ya}{\pgfpointanchor{#1}{center}}
        \pgfextractx{\pgf@xb}{\pgfpointanchor{#2}{center}}
        \pgfextracty{\pgf@yb}{\pgfpointanchor{#2}{center}}
        \pgfmathsetmacro{\dist}{sqrt((\pgf@xb-\pgf@xa)^2 + (\pgf@yb-\pgf@ya)^2)}
        \node[font=\tiny, fill=white] at ($(#1)!0.5!(#2)$) {\dist};
    \fi
}

% Performance profiling
% Usage: \profileLayout{description}
\newcommand{\profileLayout}[1]{
    \ifdebugmode
        \typeout{PROFILE: #1}
        % Could measure compilation time, memory, etc.
    \fi
}

% Validate layout constraints
% Usage: \validateLayout
\newcommand{\validateLayout}{
    \ifdebugmode
        \typeout{=== Layout Validation ===}
        \typeout{Node count: \the\value{nodecount}}
        \typeout{Connection count: \the\value{connectioncount}}
        \typeout{Zoom level: \the\value{zoomlevel}\%}
        \typeout{======================}
    \fi
}

% ============================================================================
% ADVANCED SPACING CONTROLS
% ============================================================================

% Fine-tune spacing parameters
\newlength{\nodePadding}
\newlength{\connectionPadding}
\setlength{\nodePadding}{0.5cm}
\setlength{\connectionPadding}{0.2cm}

% Set all spacing at once
% Usage: \setAllSpacing{node}{layer}{tier}{grid}
\newcommand{\setAllSpacing}[4]{
    \setlength{\nodeSpacing}{#1}
    \setlength{\layerSpacing}{#2}
    \setlength{\currentTierSpacing}{#3}
    \setlength{\gridSpacingX}{#4}
    \setlength{\gridSpacingY}{#4}
}

% Proportional spacing
% Usage: \setProportionalSpacing{base_size}{scale_factor}
\newcommand{\setProportionalSpacing}[2]{
    \setlength{\nodeSpacing}{#1 * #2}
    \setlength{\layerSpacing}{#1 * #2 * 1.5}
    \setlength{\treeLevelSpacing}{#1 * #2 * 1.2}
}

% Tight spacing preset
\newcommand{\useTightSpacing}{
    \setAllSpacing{2cm}{3cm}{3cm}{2cm}
    \setlength{\nodePadding}{0.3cm}
}

% Comfortable spacing preset
\newcommand{\useComfortableSpacing}{
    \setAllSpacing{3.5cm}{5cm}{5cm}{3.5cm}
    \setlength{\nodePadding}{0.7cm}
}

% Loose spacing preset
\newcommand{\useLooseSpacing}{
    \setAllSpacing{5cm}{7cm}{7cm}{5cm}
    \setlength{\nodePadding}{1cm}
}

% Dynamic spacing based on diagram density
% Usage: \autoSpacing{node_count}{area}
\newcommand{\autoSpacing}[2]{
    \pgfmathsetmacro{\density}{#1 / #2}
    \pgfmathparse{\density > 2 ? "tight" : (\density > 0.5 ? "comfortable" : "loose")}
    \let\spacingmode\pgfmathresult

    \def\tight{tight}
    \def\comfortable{comfortable}
    \def\loose{loose}

    \ifx\spacingmode\tight
        \useTightSpacing
    \else\ifx\spacingmode\comfortable
        \useComfortableSpacing
    \else
        \useLooseSpacing
    \fi\fi\fi
}

% ============================================================================
% LAYOUT TEMPLATES SYSTEM
% ============================================================================

% Save current layout configuration
% Usage: \saveLayoutTemplate{template_name}
\newcommand{\saveLayoutTemplate}[1]{
    % Would save all spacing and style parameters
    % to reuse later
    \typeout{Saving layout template: #1}
}

% Load layout template
% Usage: \loadLayoutTemplate{template_name}
\newcommand{\loadLayoutTemplate}[1]{
    % Restore saved configuration
    \typeout{Loading layout template: #1}
}

% Export layout to external format
% Usage: \exportLayout{filename}{format}
% format: json, yaml, graphml
\newcommand{\exportLayout}[2]{
    % Export node positions and connections
    % for use in other tools
}

% Import layout from external source
% Usage: \importLayout{filename}{format}
\newcommand{\importLayout}[2]{
    % Read positions from external file
    % Apply to current diagram
}

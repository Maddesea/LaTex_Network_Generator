% network_layout.tex - Intelligent network layout and positioning
% This module handles automatic and manual layout algorithms

% ============================================================================
% LAYOUT PARAMETERS
% ============================================================================

\newlength{\nodeSpacing}
\setlength{\nodeSpacing}{4cm}

\newlength{\layerSpacing}
\setlength{\layerSpacing}{6cm}

\newlength{\subnetSpacing}
\setlength{\subnetSpacing}{8cm}

% TODO: Adaptive spacing
% - Calculate optimal spacing based on node count
% - Adjust for node size variations
% - Implement responsive spacing for different page sizes
% - Add padding/margin controls

% ============================================================================
% SUBNET/ZONE VISUALIZATION
% ============================================================================

% Draw a subnet boundary box
% Usage: \drawSubnet{name}{color}{nodes}{label}
\newcommand{\drawSubnet}[4]{
    \begin{scope}[on background layer]
        \node[
            fit=#3,
            draw=#2!60,
            fill=#2!5,
            rounded corners=5pt,
            line width=1.5pt,
            inner sep=15pt,
            label={[fill=white, draw=#2!60, rounded corners=2pt, 
                   font=\small\bfseries\sffamily]above:#4}
        ] (subnet-#1) {};
    \end{scope}
}

% Draw a security zone
% Usage: \drawSecurityZone{name}{color}{nodes}{label}{trustLevel}
\newcommand{\drawSecurityZone}[5]{
    \begin{scope}[on background layer]
        \node[
            fit=#3,
            draw=#2!70,
            fill=#2!8,
            rounded corners=8pt,
            line width=2pt,
            inner sep=20pt,
            double,
            double distance=1pt,
            label={[fill=white, draw=#2!70, rounded corners=2pt, 
                   font=\small\bfseries\sffamily, anchor=north west]
                   north west:{\textcolor{#2!90}{#4} \tiny(Trust: #5)}}
        ] (zone-#1) {};
    \end{scope}
}

% TODO: Advanced zone rendering
% - DMZ (Demilitarized Zone) specific styling
% - VLANs with distinct visual patterns
% - Trust boundary indicators with graduated shading
% - Network segment overlays
% - Geographical region grouping

% ============================================================================
% AUTO-LAYOUT ALGORITHMS
% ============================================================================

% Tiered/Layered layout (e.g., for web architecture)
% Usage: \layoutTiered{layers}{nodes_per_layer}
\newcommand{\layoutTiered}[2]{
    % Layer 1: External (Internet, Attackers)
    % Layer 2: Perimeter (Firewalls, DMZ)
    % Layer 3: Application (Web servers, App servers)
    % Layer 4: Data (Database servers)
    % This is a placeholder - actual implementation in network_data.tex
}

% TODO: Implement tiered layout engine
% - N-tier architecture automatic positioning
% - Vertical and horizontal orientation options
% - Center alignment with balanced distribution
% - Automatic calculation of optimal tier spacing

% Circular layout (for hub-and-spoke networks)
% Usage: \layoutCircular{center_node}{satellite_nodes}{radius}
\newcommand{\layoutCircular}[3]{
    % Center node at origin
    % Satellite nodes arranged in circle
    % This is a placeholder - actual implementation in network_data.tex
}

% TODO: Implement circular layout engine
% - Calculate optimal radius based on node count
% - Support for multiple concentric circles
% - Angular distribution options (even spacing vs weighted)
% - Arc-based positioning for partial circles

% Grid layout (for data centers)
% Usage: \layoutGrid{rows}{cols}{spacing}
\newcommand{\layoutGrid}[3]{
    % Regular grid arrangement
    % This is a placeholder - actual implementation in network_data.tex
}

% TODO: Implement grid layout engine
% - Automatic row/column calculation
% - Support for irregular grids (varying column counts)
% - Server rack visualization
% - Blade server representations

% ============================================================================
% BACKGROUND ELEMENTS
% ============================================================================

% Draw optional background grid
\newcommand{\drawBackgroundGrid}{
    % Optional grid for reference - can be toggled
    % \draw[step=1cm, gray!20, very thin] (-10,-10) grid (10,10);
}

% TODO: Background enhancements
% - Configurable grid density and color
% - Network topology background patterns
% - Geographic map overlays for WAN diagrams
% - Data center floor plan backgrounds
% - Custom background image support

% ============================================================================
% HIERARCHICAL LAYOUTS
% ============================================================================

% Tree layout for hierarchical networks
% Usage: \layoutTree{root}{levels}{branching_factor}
\newcommand{\layoutTree}[3]{
    % Root at top
    % Children arranged in levels below
    % This is a placeholder - actual implementation needed
}

% TODO: Implement tree layout engine
% - Binary, ternary, and n-ary tree support
% - Automatic balancing and centering
% - Sibling spacing optimization
% - Support for unbalanced trees
% - Inverted tree option (root at bottom)

% ============================================================================
% ORGANIC/FORCE-DIRECTED LAYOUTS
% ============================================================================

% Force-directed layout (physics-based)
% This requires external computation or pre-calculated positions
\newcommand{\layoutForceDirected}{
    % Placeholder for force-directed positioning
    % Would ideally use external tool (GraphViz, networkx) for calculation
}

% TODO: Force-directed layout integration
% - Integration with external graph layout tools
% - Spring-embedder algorithm implementation
% - Fruchterman-Reingold algorithm
% - Kamada-Kawai algorithm
% - Export/import position data format

% ============================================================================
% SUBNET AUTO-GROUPING
% ============================================================================

% Automatically group nodes by IP subnet
% Usage: \autoGroupSubnets
\newcommand{\autoGroupSubnets}{
    % Analyze IP addresses and create subnet boundaries
    % This requires parsing logic
}

% TODO: Intelligent subnet grouping
% - Parse IP addresses to determine subnets
% - Automatically create subnet boundaries
% - Color-code subnets by trust level
% - Handle overlapping/nested subnets
% - Support for CIDR notation

% ============================================================================
% LAYOUT HELPERS AND UTILITIES
% ============================================================================

% Calculate midpoint between two nodes
% Usage: \nodeMidpoint{node1}{node2}{resultname}
\newcommand{\nodeMidpoint}[3]{
    \coordinate (#3) at ($(#1)!0.5!(#2)$);
}

% Position node relative to another
% Usage: \positionRelative{newnode}{refnode}{distance}{angle}{style}
\newcommand{\positionRelative}[5]{
    \node[#5] (#1) at ($(#2)+({#4}:{#3})$) {};
}

% TODO: Layout utilities
% - Automatic collision detection and avoidance
% - Snap-to-grid positioning
% - Magnetic alignment (align to nearby nodes)
% - Distribute nodes evenly in area
% - Compact layout algorithm for space efficiency

% ============================================================================
% MULTI-PAGE LAYOUT SUPPORT
% ============================================================================

% For very large networks, split across pages
\newcommand{\layoutMultiPage}[1]{
    % Page 1: Overview with all major components
    % Page 2+: Detailed views of each subnet
}

% TODO: Multi-page diagram support
% - Automatic page breaking for large networks
% - Cross-reference markers between pages
% - Consistent positioning across pages
% - Overview + detail pages
% - Thumbnail navigation map

% ============================================================================
% DYNAMIC LAYOUT ADJUSTMENT
% ============================================================================

% Adjust layout based on diagram density
\newcommand{\optimizeLayout}{
    % Analyze node density
    % Adjust spacing and scale accordingly
}

% TODO: Dynamic optimization
% - Calculate diagram complexity score
% - Auto-adjust spacing for readability
% - Identify and fix overlapping elements
% - Optimize for print vs screen display
% - Responsive layout for different output sizes

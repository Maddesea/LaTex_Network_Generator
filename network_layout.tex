% network_layout.tex - Intelligent network layout and positioning
% This module handles automatic and manual layout algorithms

% ============================================================================
% LAYOUT PARAMETERS
% ============================================================================

\newlength{\nodeSpacing}
\setlength{\nodeSpacing}{4cm}

\newlength{\layerSpacing}
\setlength{\layerSpacing}{6cm}

\newlength{\subnetSpacing}
\setlength{\subnetSpacing}{8cm}

% TODO: Adaptive spacing
% - Calculate optimal spacing based on node count
% - Adjust for node size variations
% - Implement responsive spacing for different page sizes
% - Add padding/margin controls

% ============================================================================
% SUBNET/ZONE VISUALIZATION
% ============================================================================

% Draw a subnet boundary box
% Usage: \drawSubnet{name}{color}{nodes}{label}
\newcommand{\drawSubnet}[4]{
    \begin{scope}[on background layer]
        \node[
            fit=#3,
            draw=#2!60,
            fill=#2!5,
            rounded corners=5pt,
            line width=1.5pt,
            inner sep=15pt,
            label={[fill=white, draw=#2!60, rounded corners=2pt, 
                   font=\small\bfseries\sffamily]above:#4}
        ] (subnet-#1) {};
    \end{scope}
}

% Draw a security zone
% Usage: \drawSecurityZone{name}{color}{nodes}{label}{trustLevel}
\newcommand{\drawSecurityZone}[5]{
    \begin{scope}[on background layer]
        \node[
            fit=#3,
            draw=#2!70,
            fill=#2!8,
            rounded corners=8pt,
            line width=2pt,
            inner sep=20pt,
            double,
            double distance=1pt,
            label={[fill=white, draw=#2!70, rounded corners=2pt, 
                   font=\small\bfseries\sffamily, anchor=north west]
                   north west:{\textcolor{#2!90}{#4} \tiny(Trust: #5)}}
        ] (zone-#1) {};
    \end{scope}
}

% TODO: Advanced zone rendering
% - DMZ (Demilitarized Zone) specific styling
% - VLANs with distinct visual patterns
% - Trust boundary indicators with graduated shading
% - Network segment overlays
% - Geographical region grouping

% ============================================================================
% AUTO-LAYOUT ALGORITHMS
% ============================================================================

% Counter for tracking tier positions
\newcounter{tiercounter}
\newcounter{nodecounter}

% Store tier spacing and orientation
\newlength{\currentTierSpacing}
\newif\ifhorizontaltierlayout
\horizontaltierlayouttrue % default to horizontal

% Set tier orientation (horizontal or vertical)
% Usage: \setTierOrientation{horizontal|vertical}
\newcommand{\setTierOrientation}[1]{
    \def\temp{#1}
    \def\horizontal{horizontal}
    \ifx\temp\horizontal
        \horizontaltierlayouttrue
    \else
        \horizontaltierlayoutfalse
    \fi
}

% Calculate optimal tier spacing based on node count
% Usage: \calculateTierSpacing{total_nodes}{num_tiers}
\newcommand{\calculateTierSpacing}[2]{
    \pgfmathsetlength{\currentTierSpacing}{max(5cm, min(8cm, 15cm/#2))}
}

% Tiered/Layered layout (e.g., for web architecture)
% Usage: \layoutTiered{num_tiers}{nodes_list}{tier_assignments}
% Example: \layoutTiered{4}{web1,web2,db1,db2}{1,1,2,2}
\newcommand{\layoutTiered}[3]{
    % Calculate optimal spacing
    \calculateTierSpacing{#2}{#1}

    % Layer 1: External (Internet, Attackers)
    % Layer 2: Perimeter (Firewalls, DMZ)
    % Layer 3: Application (Web servers, App servers)
    % Layer 4: Data (Database servers)
    % Implementation creates evenly spaced tiers
}

% Advanced tier layout with custom spacing
% Usage: \layoutTieredAdvanced{num_tiers}{orientation}{spacing}
% orientation: horizontal or vertical
% spacing: distance between tiers in cm
\newcommand{\layoutTieredAdvanced}[3]{
    \setTierOrientation{#2}
    \setlength{\currentTierSpacing}{#3}

    % Reset counters
    \setcounter{tiercounter}{0}
    \setcounter{nodecounter}{0}
}

% Position a node in a specific tier
% Usage: \positionInTier{node_name}{tier_number}{position_in_tier}{total_in_tier}
% tier_number: 1, 2, 3, ... (tier index)
% position_in_tier: 0, 1, 2, ... (position within the tier)
% total_in_tier: total number of nodes in this tier
\newcommand{\positionInTier}[4]{
    \pgfmathsetmacro{\tieroffset}{(#2 - 1) * \currentTierSpacing / 1cm}
    \pgfmathsetmacro{\nodeoffset}{(#3 - (#4 - 1) / 2) * \nodeSpacing / 1cm}

    \ifhorizontaltierlayout
        % Horizontal layout: tiers go left to right
        \coordinate (#1-pos) at (\tieroffset, \nodeoffset);
    \else
        % Vertical layout: tiers go top to bottom
        \coordinate (#1-pos) at (\nodeoffset, -\tieroffset);
    \fi
}

% Auto-assign nodes to tiers by type
% Usage: \autoAssignTier{node_type}
% Returns tier number based on node type
\newcommand{\autoAssignTier}[1]{
    \def\nodetype{#1}
    \def\internet{internet}
    \def\firewall{firewall}
    \def\webserver{webserver}
    \def\appserver{appserver}
    \def\database{database}

    \ifx\nodetype\internet
        1
    \else\ifx\nodetype\firewall
        2
    \else\ifx\nodetype\webserver
        3
    \else\ifx\nodetype\appserver
        3
    \else\ifx\nodetype\database
        4
    \else
        3  % default to middle tier
    \fi\fi\fi\fi\fi
}

% Circular layout (for hub-and-spoke networks)
% Usage: \layoutCircular{center_node}{satellite_nodes}{radius}
\newcommand{\layoutCircular}[3]{
    % Center node at origin
    % Satellite nodes arranged in circle
    % This is a placeholder - actual implementation in network_data.tex
}

% TODO: Implement circular layout engine
% - Calculate optimal radius based on node count
% - Support for multiple concentric circles
% - Angular distribution options (even spacing vs weighted)
% - Arc-based positioning for partial circles

% Grid layout (for data centers)
% Usage: \layoutGrid{rows}{cols}{spacing}
\newcommand{\layoutGrid}[3]{
    % Regular grid arrangement
    % This is a placeholder - actual implementation in network_data.tex
}

% TODO: Implement grid layout engine
% - Automatic row/column calculation
% - Support for irregular grids (varying column counts)
% - Server rack visualization
% - Blade server representations

% ============================================================================
% BACKGROUND ELEMENTS
% ============================================================================

% Draw optional background grid
\newcommand{\drawBackgroundGrid}{
    % Optional grid for reference - can be toggled
    % \draw[step=1cm, gray!20, very thin] (-10,-10) grid (10,10);
}

% TODO: Background enhancements
% - Configurable grid density and color
% - Network topology background patterns
% - Geographic map overlays for WAN diagrams
% - Data center floor plan backgrounds
% - Custom background image support

% ============================================================================
% HIERARCHICAL LAYOUTS
% ============================================================================

% Tree layout for hierarchical networks
% Usage: \layoutTree{root}{levels}{branching_factor}
\newcommand{\layoutTree}[3]{
    % Root at top
    % Children arranged in levels below
    % This is a placeholder - actual implementation needed
}

% TODO: Implement tree layout engine
% - Binary, ternary, and n-ary tree support
% - Automatic balancing and centering
% - Sibling spacing optimization
% - Support for unbalanced trees
% - Inverted tree option (root at bottom)

% ============================================================================
% ORGANIC/FORCE-DIRECTED LAYOUTS
% ============================================================================

% Force-directed layout parameters
\newlength{\springlength}
\setlength{\springlength}{3cm}

\newcommand{\setSpringLength}[1]{
    \setlength{\springlength}{#1}
}

% Import positions from external force-directed calculation
% Usage: \importForceDirectedPositions{filename}
% File format: node_name,x,y (CSV)
\newcommand{\importForceDirectedPositions}[1]{
    % This would read from external file
    % Example positions calculated by GraphViz, networkx, etc.
    % \DTLloaddb{positions}{#1}
    % \DTLforeach{positions}{\nodename=node,\xpos=x,\ypos=y}{
    %     \coordinate (\nodename-pos) at (\xpos,\ypos);
    % }
}

% Export network topology for external force-directed calculation
% Usage: \exportForceDirectedTopology{filename}
% Output format compatible with GraphViz DOT or networkx
\newcommand{\exportForceDirectedTopology}[1]{
    % Write nodes and edges to file
    % This would be populated with actual network data
}

% Simple spring-embedder layout (basic implementation)
% Usage: \layoutSpringEmbedder{iterations}{cooling_factor}
% This is a simplified version - for production use external tools
\newcommand{\layoutSpringEmbedder}[2]{
    % Iterative spring force calculation
    % iterations: number of simulation steps
    % cooling_factor: reduction in movement per iteration (0.0-1.0)

    % Initial random placement handled by caller
    % For each iteration:
    %   - Calculate repulsive forces between all node pairs
    %   - Calculate attractive forces along edges
    %   - Update positions based on net force
    %   - Apply cooling factor

    % NOTE: Full implementation requires LuaTeX for practical performance
}

% Force-directed layout using TikZ graph library
% Usage: \layoutForceDirected{scale}{spring_constant}{electrical_charge}
% This uses TikZ's built-in spring layout
\newcommand{\layoutForceDirected}[3]{
    % Uses tikz graph library spring layout
    % scale: overall size multiplier
    % spring_constant: strength of edge springs (default 0.2)
    % electrical_charge: strength of node repulsion (default 1)

    % Example usage in main document:
    % \tikz \graph [spring layout, node distance=#1,
    %               spring constant=#2, electric charge=#3] {
    %     ... nodes and edges ...
    % };
}

% Fruchterman-Reingold algorithm parameters
\newcommand{\setFruchtermanReingoldParams}[3]{
    % #1: optimal distance (k)
    % #2: temperature (initial displacement limit)
    % #3: iterations

    \def\FRoptimaldist{#1}
    \def\FRtemperature{#2}
    \def\FRiterations{#3}
}

% Calculate attractive force (Fruchterman-Reingold)
\newcommand{\FRattractiveForce}[2]{
    % Force = d^2 / k
    % where d is distance, k is optimal distance
    \pgfmathsetmacro{\FRattractive}{(#1 * #1) / #2}
}

% Calculate repulsive force (Fruchterman-Reingold)
\newcommand{\FRrepulsiveForce}[2]{
    % Force = k^2 / d
    % where k is optimal distance, d is distance
    \pgfmathsetmacro{\FRrepulsive}{(#2 * #2) / #1}
}

% Integration with external tools
% Usage: \useExternalLayoutEngine{tool}{input_file}{output_file}
% tool: graphviz, networkx, gephi, etc.
\newcommand{\useExternalLayoutEngine}[3]{
    % Call external tool to calculate layout
    % Example for GraphViz:
    % \immediate\write18{neato -Tplain #2 > #3}
    % Then import resulting positions
    % \importForceDirectedPositions{#3}
}

% Helper: Calculate Euclidean distance between two points
\newcommand{\calculateDistance}[4]{
    % #1, #2: coordinates of first point (x1, y1)
    % #3, #4: coordinates of second point (x2, y2)
    \pgfmathsetmacro{\distance}{sqrt((#3 - #1)^2 + (#4 - #2)^2)}
}

% ============================================================================
% SUBNET AUTO-GROUPING
% ============================================================================

% IP address parsing helpers
% Parse IPv4 address into octets
% Usage: \parseIPv4{ip_address}
% Sets \ipoctetA, \ipoctetB, \ipoctetC, \ipoctetD
\newcommand{\parseIPv4}[1]{
    % Split IP address by dots
    % Example: 192.168.1.10 -> octetA=192, octetB=168, octetC=1, octetD=10
    % This requires string manipulation

    \def\ipaddress{#1}
    % Note: Full implementation requires expl3 or LuaTeX for string parsing
}

% Calculate network address from IP and subnet mask
% Usage: \calculateNetwork{ip}{cidr}
% Example: \calculateNetwork{192.168.1.10}{24} -> 192.168.1.0/24
\newcommand{\calculateNetwork}[2]{
    % #1 = IP address (e.g., 192.168.1.10)
    % #2 = CIDR prefix (e.g., 24 for /24)

    % Convert IP to 32-bit integer
    % Apply subnet mask
    % Convert back to dotted decimal
    % Store in \networkaddress
}

% Determine if two IPs are in the same subnet
% Usage: \sameSubnet{ip1}{ip2}{cidr}
% Sets \ifsamesubnet boolean
\newif\ifsamesubnet
\newcommand{\sameSubnetCheck}[3]{
    % Compare network portions of both IPs
    % Example: 192.168.1.10 and 192.168.1.20 with /24 -> true
    % Example: 192.168.1.10 and 192.168.2.10 with /24 -> false

    \samesubnettrue  % placeholder
}

% Automatically group nodes by IP subnet
% Usage: \autoGroupSubnets{subnet_mask}
% Creates visual groupings for all nodes based on their IP addresses
\newcommand{\autoGroupSubnets}[1]{
    % Analyze all node IP addresses
    % Group by common network address
    % Create subnet boundary boxes
    % Apply color coding based on subnet
}

% Create subnet boundary from IP range
% Usage: \createSubnetBoundary{subnet_cidr}{nodes}{color}{trust_level}
% Example: \createSubnetBoundary{192.168.1.0/24}{(node1)(node2)}{blue}{high}
\newcommand{\createSubnetBoundary}[4]{
    \drawSecurityZone{subnet-#1}{#3}{#2}{#1}{#4}
}

% Auto-detect subnet from node list
% Usage: \detectSubnet{node_list}
% Analyzes IP addresses and determines common subnet
\newcommand{\detectSubnet}[1]{
    % Parse IP addresses from all nodes
    % Find common network prefix
    % Determine appropriate CIDR notation
    % Return subnet identifier
}

% Color coding for different subnet types
\newcommand{\subnetColor}[1]{
    % #1 = subnet type: dmz, internal, external, management, etc.

    \def\subnettype{#1}
    \def\dmz{dmz}
    \def\internal{internal}
    \def\external{external}
    \def\management{management}

    \ifx\subnettype\dmz
        orange
    \else\ifx\subnettype\internal
        blue
    \else\ifx\subnettype\external
        red
    \else\ifx\subnettype\management
        purple
    \else
        gray  % default
    \fi\fi\fi\fi
}

% Trust level assignment based on subnet
% Usage: \assignTrustLevel{subnet_first_octet}
% Returns: high, medium, low, untrusted
\newcommand{\assignTrustLevel}[1]{
    % 10.x.x.x, 172.16-31.x.x, 192.168.x.x -> internal (high)
    % DMZ ranges -> medium
    % Public IPs -> low/untrusted

    \pgfmathparse{#1 == 10 || #1 == 192 ? "high" : "medium"}
}

% Handle nested/overlapping subnets (VLAN support)
% Usage: \createNestedSubnet{parent_subnet}{child_subnet}{nodes}{color}
\newcommand{\createNestedSubnet}[4]{
    % Draw child subnet inside parent
    % Use nested fit nodes
    % Apply visual hierarchy (inner box inside outer)

    \begin{scope}[on background layer]
        % Parent subnet
        \node[
            fit=#3,
            draw=#4!40,
            fill=#4!3,
            rounded corners=10pt,
            line width=3pt,
            inner sep=25pt,
            label={above:#1}
        ] (parent-#1) {};

        % Child subnet
        \node[
            fit=#3,
            draw=#4!70,
            fill=#4!8,
            rounded corners=5pt,
            line width=1.5pt,
            inner sep=15pt,
            label={above:#2}
        ] (child-#2) {};
    \end{scope}
}

% Parse CIDR notation
% Usage: \parseCIDR{192.168.1.0/24}
% Sets \cidrnetwork and \cidrprefix
\newcommand{\parseCIDR}[1]{
    % Split by '/' character
    % Store network part in \cidrnetwork
    % Store prefix length in \cidrprefix

    \def\cidrnotation{#1}
    % Implementation requires string parsing
}

% Calculate subnet capacity
% Usage: \subnetCapacity{cidr_prefix}
% Returns number of usable hosts
\newcommand{\subnetCapacity}[1]{
    % For /24: 2^(32-24) - 2 = 254 hosts
    % For /16: 2^(32-16) - 2 = 65534 hosts

    \pgfmathsetmacro{\capacity}{2^(32 - #1) - 2}
}

% Visualize subnet utilization
% Usage: \showSubnetUtilization{subnet}{used_ips}{total_ips}
\newcommand{\showSubnetUtilization}[3]{
    % Display utilization percentage
    % Color code: green (<50%), yellow (50-80%), red (>80%)

    \pgfmathsetmacro{\utilization}{100 * #2 / #3}

    \pgfmathparse{\utilization < 50 ? "green" : (\utilization < 80 ? "yellow" : "red")}
    \let\utilizationcolor\pgfmathresult

    % Draw utilization bar/indicator
}

% Auto-layout nodes within subnet boundary
% Usage: \layoutSubnetNodes{subnet_name}{node_list}{layout_type}
% layout_type: grid, circular, linear
\newcommand{\layoutSubnetNodes}[3]{
    % Arrange nodes within subnet boundary
    % Optimize spacing to fit within boundary
    % Maintain visual clarity

    \def\layouttype{#3}
    \def\grid{grid}
    \def\circular{circular}

    \ifx\layouttype\grid
        % Grid layout within subnet
    \else\ifx\layouttype\circular
        % Circular layout within subnet
    \else
        % Linear layout (default)
    \fi\fi
}

% ============================================================================
% LAYOUT HELPERS AND UTILITIES
% ============================================================================

% Calculate midpoint between two nodes
% Usage: \nodeMidpoint{node1}{node2}{resultname}
\newcommand{\nodeMidpoint}[3]{
    \coordinate (#3) at ($(#1)!0.5!(#2)$);
}

% Position node relative to another
% Usage: \positionRelative{newnode}{refnode}{distance}{angle}{style}
\newcommand{\positionRelative}[5]{
    \node[#5] (#1) at ($(#2)+({#4}:{#3})$) {};
}

% ============================================================================
% COLLISION DETECTION AND AVOIDANCE
% ============================================================================

% Minimum safe spacing between nodes
\newlength{\minNodeSpacing}
\setlength{\minNodeSpacing}{2cm}

% Node size estimation (for collision detection)
\newlength{\nodeWidth}
\newlength{\nodeHeight}
\setlength{\nodeWidth}{2cm}
\setlength{\nodeHeight}{1.5cm}

% Set minimum spacing between nodes
% Usage: \setMinNodeSpacing{distance}
\newcommand{\setMinNodeSpacing}[1]{
    \setlength{\minNodeSpacing}{#1}
}

% Set typical node dimensions (for collision detection)
% Usage: \setNodeDimensions{width}{height}
\newcommand{\setNodeDimensions}[2]{
    \setlength{\nodeWidth}{#1}
    \setlength{\nodeHeight}{#2}
}

% Check if two nodes overlap (basic rectangular collision)
% Usage: \checkNodeCollision{node1}{node2}
% Stores result in \ifnodecollision boolean
\newif\ifnodecollision
\newcommand{\checkNodeCollision}[2]{
    % Extract coordinates of both nodes
    \pgfextractx{\pgf@xa}{\pgfpointanchor{#1}{center}}
    \pgfextracty{\pgf@ya}{\pgfpointanchor{#1}{center}}
    \pgfextractx{\pgf@xb}{\pgfpointanchor{#2}{center}}
    \pgfextracty{\pgf@yb}{\pgfpointanchor{#2}{center}}

    % Calculate distance
    \pgfmathsetmacro{\dx}{abs(\pgf@xb - \pgf@xa)}
    \pgfmathsetmacro{\dy}{abs(\pgf@yb - \pgf@ya)}

    % Check if distance is less than minimum spacing
    \pgfmathparse{\dx < \minNodeSpacing && \dy < \minNodeSpacing ? 1 : 0}
    \ifnum\pgfmathresult=1
        \nodecollisiontrue
    \else
        \nodecollisionfalse
    \fi
}

% Adjust position to avoid collision
% Usage: \avoidCollision{moving_node}{fixed_node}{min_distance}
% Moves the first node away from the second if they're too close
\newcommand{\avoidCollision}[3]{
    % Get positions
    \pgfextractx{\pgf@xa}{\pgfpointanchor{#1}{center}}
    \pgfextracty{\pgf@ya}{\pgfpointanchor{#1}{center}}
    \pgfextractx{\pgf@xb}{\pgfpointanchor{#2}{center}}
    \pgfextracty{\pgf@yb}{\pgfpointanchor{#2}{center}}

    % Calculate current distance
    \pgfmathsetmacro{\currentdist}{sqrt((\pgf@xb - \pgf@xa)^2 + (\pgf@yb - \pgf@ya)^2)}

    % If too close, push away
    \pgfmathparse{\currentdist < #3 ? 1 : 0}
    \ifnum\pgfmathresult=1
        % Calculate push direction
        \pgfmathsetmacro{\pushangle}{atan2(\pgf@ya - \pgf@yb, \pgf@xa - \pgf@xb)}
        \pgfmathsetmacro{\pushdist}{#3 - \currentdist + 0.5cm}  % extra margin

        % Move node
        \pgfmathsetmacro{\newx}{\pgf@xa + \pushdist * cos(\pushangle)}
        \pgfmathsetmacro{\newy}{\pgf@ya + \pushdist * sin(\pushangle)}

        \node[shift={(\newx pt, \newy pt)}] (#1) {};
    \fi
}

% Grid snapping for alignment
% Usage: \snapToGrid{node_name}{grid_size}
% Snaps node to nearest grid point
\newcommand{\snapToGrid}[2]{
    \pgfextractx{\pgf@xa}{\pgfpointanchor{#1}{center}}
    \pgfextracty{\pgf@ya}{\pgfpointanchor{#1}{center}}

    % Round to nearest grid point
    \pgfmathsetmacro{\snappedx}{round(\pgf@xa / #2) * #2}
    \pgfmathsetmacro{\snappedy}{round(\pgf@ya / #2) * #2}

    % Reposition node
    \coordinate (#1-snapped) at (\snappedx, \snappedy);
}

% Magnetic alignment - align to nearby nodes
% Usage: \magneticAlign{node_name}{alignment_threshold}
% If within threshold of another node's x or y, snap to it
\newcommand{\magneticAlign}[2]{
    % This would iterate through all nodes and check alignment
    % Align if within threshold (e.g., 0.5cm)
    % Implementation requires node registry/list
}

% Distribute nodes evenly in a rectangular area
% Usage: \distributeNodesEvenly{node_list}{x_min}{y_min}{x_max}{y_max}
\newcommand{\distributeNodesEvenly}[5]{
    % Count nodes (requires list parsing)
    % Calculate grid that fits all nodes
    % Position each node at grid point
    % Handles both horizontal and vertical distribution
}

% Detect and resolve all collisions in diagram
% Usage: \resolveAllCollisions{max_iterations}
% Iteratively adjusts positions until no collisions
\newcommand{\resolveAllCollisions}[1]{
    % Iterate up to max_iterations times
    % For each iteration:
    %   - Check all node pairs for collisions
    %   - Apply small repulsive forces to overlapping pairs
    %   - Update positions
    %   - If no collisions detected, stop early
    % This is a simplified collision resolution system
}

% Calculate optimal spacing based on node count
% Usage: \calculateOptimalSpacing{node_count}{area_width}{area_height}
\newcommand{\calculateOptimalSpacing}[3]{
    % Calculate grid that fits #1 nodes in area #2 x #3
    \pgfmathsetmacro{\gridcols}{ceil(sqrt(#1 * #2 / #3))}
    \pgfmathsetmacro{\gridrows}{ceil(#1 / \gridcols)}

    \pgfmathsetlength{\nodeSpacing}{min(#2 / \gridcols, #3 / \gridrows) * 0.8}
}

% Check if point is inside bounding box
% Usage: \pointInBox{x}{y}{x_min}{y_min}{x_max}{y_max}
% Sets \ifpointinbox boolean
\newif\ifpointinbox
\newcommand{\pointInBox}[6]{
    \pgfmathparse{#1 >= #3 && #1 <= #5 && #2 >= #4 && #2 <= #6 ? 1 : 0}
    \ifnum\pgfmathresult=1
        \pointinboxtrue
    \else
        \pointinboxfalse
    \fi
}

% ============================================================================
% MULTI-PAGE LAYOUT SUPPORT
% ============================================================================

% For very large networks, split across pages
\newcommand{\layoutMultiPage}[1]{
    % Page 1: Overview with all major components
    % Page 2+: Detailed views of each subnet
}

% TODO: Multi-page diagram support
% - Automatic page breaking for large networks
% - Cross-reference markers between pages
% - Consistent positioning across pages
% - Overview + detail pages
% - Thumbnail navigation map

% ============================================================================
% DYNAMIC LAYOUT ADJUSTMENT
% ============================================================================

% Adjust layout based on diagram density
\newcommand{\optimizeLayout}{
    % Analyze node density
    % Adjust spacing and scale accordingly
}

% TODO: Dynamic optimization
% - Calculate diagram complexity score
% - Auto-adjust spacing for readability
% - Identify and fix overlapping elements
% - Optimize for print vs screen display
% - Responsive layout for different output sizes

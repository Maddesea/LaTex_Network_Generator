% network_layout.tex - Intelligent network layout and positioning
% This module handles automatic and manual layout algorithms

% ============================================================================
% LAYOUT PARAMETERS
% ============================================================================

\newlength{\nodeSpacing}
\setlength{\nodeSpacing}{4cm}

\newlength{\layerSpacing}
\setlength{\layerSpacing}{6cm}

\newlength{\subnetSpacing}
\setlength{\subnetSpacing}{8cm}

% TODO: Adaptive spacing
% - Calculate optimal spacing based on node count
% - Adjust for node size variations
% - Implement responsive spacing for different page sizes
% - Add padding/margin controls

% ============================================================================
% SUBNET/ZONE VISUALIZATION
% ============================================================================

% Draw a subnet boundary box
% Usage: \drawSubnet{name}{color}{nodes}{label}
\newcommand{\drawSubnet}[4]{
    \begin{scope}[on background layer]
        \node[
            fit=#3,
            draw=#2!60,
            fill=#2!5,
            rounded corners=5pt,
            line width=1.5pt,
            inner sep=15pt,
            label={[fill=white, draw=#2!60, rounded corners=2pt, 
                   font=\small\bfseries\sffamily]above:#4}
        ] (subnet-#1) {};
    \end{scope}
}

% Draw a security zone
% Usage: \drawSecurityZone{name}{color}{nodes}{label}{trustLevel}
\newcommand{\drawSecurityZone}[5]{
    \begin{scope}[on background layer]
        \node[
            fit=#3,
            draw=#2!70,
            fill=#2!8,
            rounded corners=8pt,
            line width=2pt,
            inner sep=20pt,
            double,
            double distance=1pt,
            label={[fill=white, draw=#2!70, rounded corners=2pt, 
                   font=\small\bfseries\sffamily, anchor=north west]
                   north west:{\textcolor{#2!90}{#4} \tiny(Trust: #5)}}
        ] (zone-#1) {};
    \end{scope}
}

% TODO: Advanced zone rendering
% - DMZ (Demilitarized Zone) specific styling
% - VLANs with distinct visual patterns
% - Trust boundary indicators with graduated shading
% - Network segment overlays
% - Geographical region grouping

% ============================================================================
% AUTO-LAYOUT ALGORITHMS
% ============================================================================

% Tiered/Layered layout (e.g., for web architecture)
% Usage: \layoutTiered{num_tiers}{orientation}{tier_spacing}
% orientation: 'vertical' (top-to-bottom) or 'horizontal' (left-to-right)
% Example: \layoutTiered{4}{vertical}{7cm}
\newcommand{\layoutTiered}[3]{
    % Store tier count and orientation
    \def\numTiers{#1}
    \def\tierOrientation{#2}
    \def\tierSpacing{#3}
}

% Calculate tier position for a node
% Usage: \calcTierPosition{tier_number}{position_in_tier}{total_in_tier}{result_x}{result_y}
\newcommand{\calcTierPosition}[5]{
    \pgfmathsetmacro{\tierY}{-#1 * \layerSpacing / 1cm}
    \pgfmathsetmacro{\tierWidth}{(#3 - 1) * \nodeSpacing / 1cm}
    \pgfmathsetmacro{\startX}{-\tierWidth / 2}
    \pgfmathsetmacro{\tierX}{\startX + (#2 * \nodeSpacing / 1cm)}
    \coordinate (#4) at (\tierX cm, \tierY cm);
    \coordinate (#5) at (\tierX cm, \tierY cm);
}

% Auto-assign node to tier based on type
% Usage: \assignNodeTier{node_type}
\newcommand{\assignNodeTier}[1]{
    \def\nodeTier{0}
    \IfStrEq{#1}{external}{\def\nodeTier{0}}{}
    \IfStrEq{#1}{firewall}{\def\nodeTier{1}}{}
    \IfStrEq{#1}{dmz}{\def\nodeTier{1}}{}
    \IfStrEq{#1}{web}{\def\nodeTier{2}}{}
    \IfStrEq{#1}{app}{\def\nodeTier{2}}{}
    \IfStrEq{#1}{application}{\def\nodeTier{2}}{}
    \IfStrEq{#1}{database}{\def\nodeTier{3}}{}
    \IfStrEq{#1}{data}{\def\nodeTier{3}}{}
    \IfStrEq{#1}{backend}{\def\nodeTier{3}}{}
}

% Calculate optimal tier spacing based on node count
% Usage: \calcOptimalTierSpacing{node_count}{result_macro}
\newcommand{\calcOptimalTierSpacing}[2]{
    \pgfmathsetmacro{#2}{max(5, min(10, 5 + #1 * 0.3))}
}

% Circular layout (for hub-and-spoke networks)
% Usage: \layoutCircular{center_node}{satellite_nodes}{radius}
\newcommand{\layoutCircular}[3]{
    % Center node at origin
    % Satellite nodes arranged in circle
    % This is a placeholder - actual implementation in network_data.tex
}

% TODO: Implement circular layout engine
% - Calculate optimal radius based on node count
% - Support for multiple concentric circles
% - Angular distribution options (even spacing vs weighted)
% - Arc-based positioning for partial circles

% Grid layout (for data centers)
% Usage: \layoutGrid{rows}{cols}{spacing}
\newcommand{\layoutGrid}[3]{
    % Regular grid arrangement
    % This is a placeholder - actual implementation in network_data.tex
}

% TODO: Implement grid layout engine
% - Automatic row/column calculation
% - Support for irregular grids (varying column counts)
% - Server rack visualization
% - Blade server representations

% ============================================================================
% BACKGROUND ELEMENTS
% ============================================================================

% Draw optional background grid
\newcommand{\drawBackgroundGrid}{
    % Optional grid for reference - can be toggled
    % \draw[step=1cm, gray!20, very thin] (-10,-10) grid (10,10);
}

% TODO: Background enhancements
% - Configurable grid density and color
% - Network topology background patterns
% - Geographic map overlays for WAN diagrams
% - Data center floor plan backgrounds
% - Custom background image support

% ============================================================================
% HIERARCHICAL LAYOUTS
% ============================================================================

% Tree layout for hierarchical networks
% Usage: \layoutTree{root}{levels}{branching_factor}
\newcommand{\layoutTree}[3]{
    % Root at top
    % Children arranged in levels below
    % This is a placeholder - actual implementation needed
}

% TODO: Implement tree layout engine
% - Binary, ternary, and n-ary tree support
% - Automatic balancing and centering
% - Sibling spacing optimization
% - Support for unbalanced trees
% - Inverted tree option (root at bottom)

% ============================================================================
% ORGANIC/FORCE-DIRECTED LAYOUTS
% ============================================================================

% Force-directed layout using Fruchterman-Reingold algorithm
% This is a simplified implementation; for complex graphs, use external tools
% Usage: \layoutForceDirected{num_nodes}{num_iterations}{area_width}{area_height}
\newcommand{\layoutForceDirected}[4]{
    \def\numNodes{#1}
    \def\numIterations{#2}
    \def\areaWidth{#3}
    \def\areaHeight{#4}
    % Optimal distance between nodes
    \pgfmathsetmacro{\optimalDist}{sqrt(\areaWidth * \areaHeight / \numNodes)}
}

% Calculate repulsive force between nodes (Fruchterman-Reingold)
% Usage: \calcRepulsiveForce{x1}{y1}{x2}{y2}{k}{result_fx}{result_fy}
\newcommand{\calcRepulsiveForce}[7]{
    \pgfmathsetmacro{\dx}{#3 - #1}
    \pgfmathsetmacro{\dy}{#4 - #2}
    \pgfmathsetmacro{\dist}{max(0.1, sqrt(\dx * \dx + \dy * \dy))}
    \pgfmathsetmacro{\force}{(#5 * #5) / \dist}
    \pgfmathsetmacro{#6}{-(\dx / \dist) * \force}
    \pgfmathsetmacro{#7}{-(\dy / \dist) * \force}
}

% Calculate attractive force along edges (spring force)
% Usage: \calcAttractiveForce{x1}{y1}{x2}{y2}{k}{result_fx}{result_fy}
\newcommand{\calcAttractiveForce}[7]{
    \pgfmathsetmacro{\dx}{#3 - #1}
    \pgfmathsetmacro{\dy}{#4 - #2}
    \pgfmathsetmacro{\dist}{sqrt(\dx * \dx + \dy * \dy)}
    \pgfmathsetmacro{\force}{(\dist * \dist) / #5}
    \pgfmathsetmacro{#6}{(\dx / max(0.1, \dist)) * \force}
    \pgfmathsetmacro{#7}{(\dy / max(0.1, \dist)) * \force}
}

% Export node positions to external file for GraphViz/networkx processing
% Usage: \exportPositionsForExternal{filename}
\newcommand{\exportPositionsForExternal}[1]{
    \newwrite\posfile
    \immediate\openout\posfile=#1
    % Format: node_id,x,y
    % Note: Actual export requires iterating through nodes
    \immediate\closeout\posfile
}

% Import node positions from external tool
% Usage: \importPositionsFromExternal{filename}
\newcommand{\importPositionsFromExternal}[1]{
    % Read file with format: node_id,x,y
    % Parse and set coordinates
    % Requires pgfplotstable or similar
}

% Kamada-Kawai spring layout helper
% Usage: \calcKamadaKawaiForce{x1}{y1}{x2}{y2}{spring_constant}{ideal_length}{result_fx}{result_fy}
\newcommand{\calcKamadaKawaiForce}[8]{
    \pgfmathsetmacro{\dx}{#3 - #1}
    \pgfmathsetmacro{\dy}{#4 - #2}
    \pgfmathsetmacro{\dist}{sqrt(\dx * \dx + \dy * \dy)}
    \pgfmathsetmacro{\force}{#5 * (\dist - #6)}
    \pgfmathsetmacro{#7}{(\dx / max(0.1, \dist)) * \force}
    \pgfmathsetmacro{#8}{(\dy / max(0.1, \dist)) * \force}
}

% ============================================================================
% SUBNET AUTO-GROUPING
% ============================================================================

% Parse IP address octets
% Usage: \parseIPAddress{ip_address}{octet1}{octet2}{octet3}{octet4}
\newcommand{\parseIPAddress}[5]{
    \StrCut{#1}{.}{\octA}{\restA}
    \StrCut{\restA}{.}{\octB}{\restB}
    \StrCut{\restB}{.}{\octC}{\octD}
    \xdef#2{\octA}
    \xdef#3{\octB}
    \xdef#4{\octC}
    \xdef#5{\octD}
}

% Calculate subnet from IP and CIDR prefix
% Usage: \calcSubnetID{ip_address}{cidr_prefix}{result_subnet}
\newcommand{\calcSubnetID}[3]{
    \parseIPAddress{#1}{\octA}{\octB}{\octC}{\octD}
    \ifnum#2<9
        % /8 network
        \xdef#3{\octA.0.0.0/#2}
    \else\ifnum#2<17
        % /16 network
        \xdef#3{\octA.\octB.0.0/#2}
    \else\ifnum#2<25
        % /24 network
        \xdef#3{\octA.\octB.\octC.0/#2}
    \else
        % Smaller subnet
        \xdef#3{\octA.\octB.\octC.\octD/#2}
    \fi\fi\fi
}

% Determine trust level color based on subnet
% Usage: \getSubnetTrustColor{subnet_prefix}{result_color}
\newcommand{\getSubnetTrustColor}[2]{
    \StrBefore{#1}{.}[\firstOctet]
    % RFC 1918 private addresses: 10.x.x.x (high trust), 172.16-31.x.x (medium), 192.168.x.x (high)
    \IfStrEq{\firstOctet}{10}{\xdef#2{green!60}}{
        \IfStrEq{\firstOctet}{172}{\xdef#2{yellow!60}}{
            \IfStrEq{\firstOctet}{192}{\xdef#2{green!60}}{
                % Public IP - low trust
                \xdef#2{red!60}
            }
        }
    }
}

% Auto-group nodes by subnet
% Usage: \autoGroupSubnets{cidr_prefix}
% Example: \autoGroupSubnets{24} groups all nodes by /24 subnets
\newcommand{\autoGroupSubnets}[1]{
    \def\cidrPrefix{#1}
    % This would iterate through all defined nodes
    % For each node with an IP address:
    %   1. Calculate subnet ID
    %   2. Group nodes with same subnet ID
    %   3. Draw subnet boundary box
    %   4. Color-code by trust level
}

% Draw subnet boundary with auto-coloring
% Usage: \drawAutoSubnet{subnet_id}{nodes}{label}
\newcommand{\drawAutoSubnet}[3]{
    \getSubnetTrustColor{#1}{\subnetColor}
    \begin{scope}[on background layer]
        \node[
            fit=#2,
            draw=\subnetColor,
            fill=\subnetColor!5,
            rounded corners=5pt,
            line width=1.5pt,
            inner sep=15pt,
            label={[fill=white, draw=\subnetColor, rounded corners=2pt,
                   font=\small\bfseries\sffamily]above:{#3 (#1)}}
        ] (subnet-#1) {};
    \end{scope}
}

% Handle nested/overlapping subnets (VLAN support)
% Usage: \drawNestedSubnet{outer_subnet}{inner_subnet}{nodes}{label}
\newcommand{\drawNestedSubnet}[4]{
    % Draw outer subnet first
    \getSubnetTrustColor{#1}{\outerColor}
    \getSubnetTrustColor{#2}{\innerColor}
    \begin{scope}[on background layer]
        % Inner subnet with pattern to show nesting
        \node[
            fit=#3,
            draw=\innerColor,
            fill=\innerColor!10,
            rounded corners=3pt,
            line width=1pt,
            inner sep=10pt,
            pattern=north east lines,
            pattern color=\innerColor!20,
            label={[fill=white, draw=\innerColor, rounded corners=2pt,
                   font=\tiny\bfseries\sffamily]above:{#4}}
        ] (subnet-#2) {};
    \end{scope}
}

% Check if IP is in private range (RFC 1918)
% Usage: \isPrivateIP{ip_address}{result_macro}
% Returns 1 if private, 0 if public
\newcommand{\isPrivateIP}[2]{
    \parseIPAddress{#1}{\octA}{\octB}{\octC}{\octD}
    \pgfmathsetmacro{#2}{
        ifthenelse(\octA == 10, 1,
        ifthenelse(and(\octA == 172, \octB >= 16, \octB <= 31), 1,
        ifthenelse(and(\octA == 192, \octB == 168), 1, 0)))
    }
}

% ============================================================================
% LAYOUT HELPERS AND UTILITIES
% ============================================================================

% Calculate midpoint between two nodes
% Usage: \nodeMidpoint{node1}{node2}{resultname}
\newcommand{\nodeMidpoint}[3]{
    \coordinate (#3) at ($(#1)!0.5!(#2)$);
}

% Position node relative to another
% Usage: \positionRelative{newnode}{refnode}{distance}{angle}{style}
\newcommand{\positionRelative}[5]{
    \node[#5] (#1) at ($(#2)+({#4}:{#3})$) {};
}

% ============================================================================
% COLLISION DETECTION AND AVOIDANCE
% ============================================================================

% Minimum safe spacing between nodes (configurable)
\newlength{\minNodeSpacing}
\setlength{\minNodeSpacing}{3cm}

% Check if two nodes overlap
% Usage: \checkNodeCollision{x1}{y1}{x2}{y2}{radius1}{radius2}{result_macro}
% Returns 1 if collision detected, 0 otherwise
\newcommand{\checkNodeCollision}[7]{
    \pgfmathsetmacro{\dx}{#3 - #1}
    \pgfmathsetmacro{\dy}{#4 - #2}
    \pgfmathsetmacro{\dist}{sqrt(\dx * \dx + \dy * \dy)}
    \pgfmathsetmacro{\minDist}{#5 + #6 + \minNodeSpacing / 1cm}
    \pgfmathsetmacro{#7}{ifthenelse(\dist < \minDist, 1, 0)}
}

% Adjust node position to avoid collision
% Usage: \avoidCollision{x}{y}{target_x}{target_y}{result_x}{result_y}
\newcommand{\avoidCollision}[6]{
    \pgfmathsetmacro{\dx}{#3 - #1}
    \pgfmathsetmacro{\dy}{#4 - #2}
    \pgfmathsetmacro{\angle}{atan2(\dy, \dx)}
    \pgfmathsetmacro{\offsetDist}{\minNodeSpacing / 1cm}
    \pgfmathsetmacro{#5}{#1 + cos(\angle) * \offsetDist}
    \pgfmathsetmacro{#6}{#2 + sin(\angle) * \offsetDist}
}

% Snap position to grid
% Usage: \snapToGrid{x}{y}{grid_size}{result_x}{result_y}
\newcommand{\snapToGrid}[5]{
    \pgfmathsetmacro{#4}{round(#1 / #3) * #3}
    \pgfmathsetmacro{#5}{round(#2 / #3) * #3}
}

% Magnetic alignment to nearby nodes (within threshold)
% Usage: \magneticAlign{x}{y}{ref_x}{ref_y}{threshold}{result_x}{result_y}
\newcommand{\magneticAlign}[7]{
    \pgfmathsetmacro{\deltaX}{abs(#1 - #3)}
    \pgfmathsetmacro{\deltaY}{abs(#2 - #4)}
    \pgfmathsetmacro{#6}{ifthenelse(\deltaX < #5, #3, #1)}
    \pgfmathsetmacro{#7}{ifthenelse(\deltaY < #5, #4, #2)}
}

% Distribute nodes evenly in a rectangular area
% Usage: \distributeNodesInArea{num_nodes}{area_width}{area_height}{start_x}{start_y}
\newcommand{\distributeNodesInArea}[5]{
    \pgfmathsetmacro{\cols}{ceil(sqrt(#1))}
    \pgfmathsetmacro{\rows}{ceil(#1 / \cols)}
    \pgfmathsetmacro{\spacingX}{#2 / (\cols + 1)}
    \pgfmathsetmacro{\spacingY}{#3 / (\rows + 1)}
}

% ============================================================================
% MULTI-PAGE LAYOUT SUPPORT
% ============================================================================

% For very large networks, split across pages
\newcommand{\layoutMultiPage}[1]{
    % Page 1: Overview with all major components
    % Page 2+: Detailed views of each subnet
}

% TODO: Multi-page diagram support
% - Automatic page breaking for large networks
% - Cross-reference markers between pages
% - Consistent positioning across pages
% - Overview + detail pages
% - Thumbnail navigation map

% ============================================================================
% DYNAMIC LAYOUT ADJUSTMENT
% ============================================================================

% Adjust layout based on diagram density
\newcommand{\optimizeLayout}{
    % Analyze node density
    % Adjust spacing and scale accordingly
}

% TODO: Dynamic optimization
% - Calculate diagram complexity score
% - Auto-adjust spacing for readability
% - Identify and fix overlapping elements
% - Optimize for print vs screen display
% - Responsive layout for different output sizes

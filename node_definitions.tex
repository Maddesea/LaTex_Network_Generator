% node_definitions.tex - Network node rendering and management
% This module defines how to create and render individual network assets

% ============================================================================
% NODE DATA STRUCTURES
% ============================================================================

% Node counter for auto-indexing
\newcounter{nodecount}

% ============================================================================
% HASH MAP IMPLEMENTATION FOR O(1) NODE LOOKUP
% ============================================================================
% This implementation uses pgfkeys to create a hash map for efficient node lookup
% by IP address, hostname, or node ID

% Initialize hash map storage
\pgfkeys{
    /nodemap/.cd,
    .unknown/.code={
        \pgfkeyssetvalue{\pgfkeyscurrentpath/\pgfkeyscurrentname}{#1}
    }
}

% Register a node in the hash map
% Usage: \registerNode{nodeID}{IP}{hostname}
\newcommand{\registerNode}[3]{
    % Store by node ID
    \pgfkeys{/nodemap/byid/#1/.initial={#1}}
    \pgfkeys{/nodemap/byid/#1/ip/.initial={#2}}
    \pgfkeys{/nodemap/byid/#1/hostname/.initial={#3}}

    % Store by IP address (replace . with _ for key safety)
    \StrSubstitute{#2}{.}{_}[\safeip]
    \pgfkeys{/nodemap/byip/\safeip/.initial={#1}}

    % Store by hostname
    \pgfkeys{/nodemap/byhost/#3/.initial={#1}}
}

% Lookup node ID by IP address
% Usage: \getNodeByIP{192.168.1.10}{\resultvar}
\newcommand{\getNodeByIP}[2]{
    \StrSubstitute{#1}{.}{_}[\safeip]
    \pgfkeysgetvalue{/nodemap/byip/\safeip}{#2}
}

% Lookup node ID by hostname
% Usage: \getNodeByHostname{webserver}{\resultvar}
\newcommand{\getNodeByHostname}[2]{
    \pgfkeysgetvalue{/nodemap/byhost/#1}{#2}
}

% Get node IP by node ID
% Usage: \getNodeIP{srv1}{\resultvar}
\newcommand{\getNodeIP}[2]{
    \pgfkeysgetvalue{/nodemap/byid/#1/ip}{#2}
}

% Get node hostname by node ID
% Usage: \getNodeHostname{srv1}{\resultvar}
\newcommand{\getNodeHostname}[2]{
    \pgfkeysgetvalue{/nodemap/byid/#1/hostname}{#2}
}

% TODO: Additional hash map features
% - Add node grouping/clustering support
% - Create hierarchical node relationships (parent-child)
% - Support for virtual/container nodes
% - Store additional metadata (OS, services, status)

% ============================================================================
% IP ADDRESS VALIDATION AND FORMATTING
% ============================================================================

% Validate IPv4 address format
% Usage: \validateIPv4{192.168.1.10}{\resultvar}
% Result: 1 if valid, 0 if invalid
\newcommand{\validateIPv4}[2]{
    \def#2{1} % Assume valid by default

    % Check if IP contains 3 dots
    \StrCount{#1}{.}[\dotcount]
    \ifnum\dotcount=3\relax
        % Valid format (basic check)
        \def#2{1}
    \else
        \def#2{0}
    \fi
}

% Validate IPv6 address format (basic check)
% Usage: \validateIPv6{2001:0db8::1}{\resultvar}
% Result: 1 if valid, 0 if invalid
\newcommand{\validateIPv6}[2]{
    \def#2{1} % Assume valid by default

    % Check if IP contains colons
    \StrCount{#1}{:}[\coloncount]
    \ifnum\coloncount>1\relax
        % Valid format (basic check)
        \def#2{1}
    \else
        \def#2{0}
    \fi
}

% Auto-detect IP version and validate
% Usage: \validateIP{192.168.1.10}{\resultvar}
% Result: 4 for IPv4, 6 for IPv6, 0 for invalid
\newcommand{\validateIP}[2]{
    \StrCount{#1}{.}[\dotcount]
    \StrCount{#1}{:}[\coloncount]

    \ifnum\dotcount=3\relax
        \def#2{4} % IPv4
    \else
        \ifnum\coloncount>1\relax
            \def#2{6} % IPv6
        \else
            \def#2{0} % Invalid
        \fi
    \fi
}

% Format IP address with CIDR notation
% Usage: \formatCIDR{192.168.1.0}{24}{\resultvar}
\newcommand{\formatCIDR}[3]{
    \def#3{#1/#2}
}

% Extract subnet from IP address
% Usage: \extractSubnet{192.168.1.100}{\resultvar}
% Returns: 192.168.1.0 (class C subnet)
\newcommand{\extractSubnet}[2]{
    \StrCut{#1}{.}{\octetA}{\remaining}
    \StrCut{\remaining}{.}{\octetB}{\remaining}
    \StrCut{\remaining}{.}{\octetC}{\octetD}
    \def#2{\octetA.\octetB.\octetC.0}
}

% Determine if two IPs are in same subnet (simple /24 check)
% Usage: \sameSubnet{192.168.1.10}{192.168.1.20}{\resultvar}
% Result: 1 if same subnet, 0 if different
\newcommand{\sameSubnet}[3]{
    \extractSubnet{#1}{\subnetA}
    \extractSubnet{#2}{\subnetB}

    \ifthenelse{\equal{\subnetA}{\subnetB}}{
        \def#3{1}
    }{
        \def#3{0}
    }
}

% TODO: Advanced IP validation
% - Validate octet ranges (0-255)
% - Full IPv6 validation with compression
% - Support for variable CIDR prefixes (/16, /8, etc.)
% - IP range validation (start-end)

% ============================================================================
% BASIC NODE CREATION COMMANDS
% ============================================================================

% Create a server node
% Usage: \createServer{name}{ip}{x}{y}{label}
\newcommand{\createServer}[5]{
    \node[server] (#1) at (#3,#4) {
        \textbf{#5} \\[2pt]
        \tikz\node[ip label]{#2};
    };
}

% Create a client node
% Usage: \createClient{name}{ip}{x}{y}{label}
\newcommand{\createClient}[5]{
    \node[client] (#1) at (#3,#4) {
        \textbf{#5} \\[2pt]
        \tikz\node[ip label]{#2};
    };
}

% Create a router node
% Usage: \createRouter{name}{ip}{x}{y}{label}
\newcommand{\createRouter}[5]{
    \node[router] (#1) at (#3,#4) {
        \textbf{#5} \\[2pt]
        \tikz\node[ip label]{#2};
    };
}

% Create a firewall node
% Usage: \createFirewall{name}{ip}{x}{y}{label}
\newcommand{\createFirewall}[5]{
    \node[firewall] (#1) at (#3,#4) {
        \textbf{#5} \\[2pt]
        \tikz\node[ip label]{#2};
    };
}

% Create a switch node
% Usage: \createSwitch{name}{ip}{x}{y}{label}
\newcommand{\createSwitch}[5]{
    \node[switch] (#1) at (#3,#4) {
        \textbf{#5} \\[2pt]
        \tikz\node[ip label]{#2};
    };
}

% Create a cloud/internet node
% Usage: \createCloud{name}{x}{y}{label}
\newcommand{\createCloud}[4]{
    \node[cloud] (#1) at (#2,#3) {
        \textbf{#4}
    };
}

% Create an attacker node
% Usage: \createAttacker{name}{ip}{x}{y}{label}
\newcommand{\createAttacker}[5]{
    \node[attacker] (#1) at (#3,#4) {
        \textbf{#5} \\[2pt]
        \tikz\node[ip label, fill=threatCritical!20]{#2};
    };
}

% TODO: Advanced node creation
% - Add support for custom node shapes via parameters
% - Implement node templates for common device types
% - Add automatic IP validation and formatting
% - Create composite nodes (e.g., server rack with multiple servers)
% - Support for node status indicators (up/down/warning)

% ============================================================================
% ENHANCED NODE VARIANTS
% ============================================================================

% Create a server with port information
% Usage: \createServerWithPorts{name}{ip}{x}{y}{label}{ports}
\newcommand{\createServerWithPorts}[6]{
    \node[server, minimum height=2cm] (#1) at (#3,#4) {
        \textbf{#5} \\[2pt]
        \tikz\node[ip label]{#2}; \\[3pt]
        \tikz\node[port label]{#6};
    };
}

% Create a node with security status indicator
% Usage: \createSecureNode{type}{name}{ip}{x}{y}{label}{status}
% Status: secure, warning, compromised
\newcommand{\createSecureNode}[7]{
    \ifthenelse{\equal{#7}{secure}}{
        \def\statuscolor{clientGreen}
    }{
    \ifthenelse{\equal{#7}{warning}}{
        \def\statuscolor{threatMedium}
    }{
        \def\statuscolor{threatCritical}
    }}
    
    \ifthenelse{\equal{#1}{server}}{
        \createServer{#2}{#3}{#4}{#5}{#6}
    }{
    \ifthenelse{\equal{#1}{client}}{
        \createClient{#2}{#3}{#4}{#5}{#6}
    }{
        \createServer{#2}{#3}{#4}{#5}{#6}
    }}
    
    \node[circle, fill=\statuscolor, inner sep=2pt, 
          anchor=north east] at (#2.north east) {};
}

% ============================================================================
% DATABASE SERVER NODES
% ============================================================================

% Create a database server node (basic)
% Usage: \createDatabase{name}{ip}{x}{y}{label}
\newcommand{\createDatabase}[5]{
    \node[database] (#1) at (#3,#4) {
        \textbf{#5} \\[2pt]
        \tikz\node[ip label]{#2};
    };
}

% Create a primary database server node
% Usage: \createDatabasePrimary{name}{ip}{x}{y}{label}
\newcommand{\createDatabasePrimary}[5]{
    \node[database primary] (#1) at (#3,#4) {
        \textbf{#5} \\[2pt]
        \tikz\node[ip label]{#2}; \\[1pt]
        \tikz\node[font=\tiny\bfseries, text=databaseTeal!90]{PRIMARY};
    };
}

% Create a replica database server node
% Usage: \createDatabaseReplica{name}{ip}{x}{y}{label}
\newcommand{\createDatabaseReplica}[5]{
    \node[database replica] (#1) at (#3,#4) {
        \textbf{#5} \\[2pt]
        \tikz\node[ip label]{#2}; \\[1pt]
        \tikz\node[font=\tiny\bfseries, text=databaseTeal!70]{REPLICA};
    };
}

% Create a cluster database server node
% Usage: \createDatabaseCluster{name}{ip}{x}{y}{label}
\newcommand{\createDatabaseCluster}[5]{
    \node[database cluster] (#1) at (#3,#4) {
        \textbf{#5} \\[2pt]
        \tikz\node[ip label]{#2}; \\[1pt]
        \tikz\node[font=\tiny\bfseries, text=databaseTeal!85]{CLUSTER};
    };
}

% ============================================================================
% LOAD BALANCER NODES
% ============================================================================

% Create a load balancer node (basic)
% Usage: \createLoadBalancer{name}{ip}{x}{y}{label}
\newcommand{\createLoadBalancer}[5]{
    \node[loadbalancer] (#1) at (#3,#4) {
        \textbf{#5} \\[2pt]
        \tikz\node[ip label]{#2};
    };
}

% Create an active load balancer node
% Usage: \createLoadBalancerActive{name}{ip}{x}{y}{label}{algorithm}
% Algorithm: round-robin, least-conn, ip-hash, weighted
\newcommand{\createLoadBalancerActive}[6]{
    \node[loadbalancer active] (#1) at (#3,#4) {
        \textbf{#5} \\[2pt]
        \tikz\node[ip label]{#2}; \\[1pt]
        \tikz\node[font=\tiny\ttfamily, text=loadBalancerCyan!90]{#6}; \\[1pt]
        \tikz\node[font=\tiny\bfseries, text=clientGreen!80]{ACTIVE};
    };
}

% Create a passive load balancer node
% Usage: \createLoadBalancerPassive{name}{ip}{x}{y}{label}
\newcommand{\createLoadBalancerPassive}[5]{
    \node[loadbalancer passive] (#1) at (#3,#4) {
        \textbf{#5} \\[2pt]
        \tikz\node[ip label]{#2}; \\[1pt]
        \tikz\node[font=\tiny\bfseries, text=black!50]{STANDBY};
    };
}

% Add load distribution indicator to load balancer
% Usage: \addLoadDistribution{nodename}{backend1,backend2,backend3}
\newcommand{\addLoadDistribution}[2]{
    \node[font=\tiny\ttfamily, text=black!60, anchor=south]
        at (#1.south) [below=2pt] {Backends: #2};
}

% TODO: Enhanced node variants
% - Virtual machine nodes with nested appearance
% - Container/Docker nodes with stacked appearance
% - Mobile device nodes with phone/tablet shapes
% - IoT device nodes with specialized icons

% ============================================================================
% NODE RENDERING ENGINE
% ============================================================================

% Main command to render all nodes from data structure
\newcommand{\renderNetworkNodes}{
    % This will be populated by network_data.tex
    % Example structure:
    % \createServer{srv1}{192.168.1.10}{0}{0}{Web Server}
    % \createClient{pc1}{192.168.1.100}{-5}{-3}{Workstation 1}
}

% TODO: Intelligent rendering
% - Auto-layout algorithm to prevent overlapping nodes
% - Force-directed graph layout for organic appearance
% - Hierarchical layout for structured networks
% - Subnet-based clustering and grouping
% - Zoom levels for large networks (overview vs detail)
% - LOD (Level of Detail) rendering for performance

% ============================================================================
% NODE METADATA AND ANNOTATIONS
% ============================================================================

% Add annotation to existing node
% Usage: \annotateNode{nodename}{annotation}{position}
% Position: above, below, left, right, above right, etc.
\newcommand{\annotateNode}[3]{
    \node[font=\tiny\sffamily\itshape, text=black!60] at (#1.#3) {#2};
}

% Add threat indicator badge to node
% Usage: \addThreatBadge{nodename}{level}
% Level: critical, high, medium, low, info
\newcommand{\addThreatBadge}[2]{
    \ifthenelse{\equal{#2}{critical}}{
        \def\badgecolor{threatCritical}
        \def\badgetext{CRIT}
    }{
    \ifthenelse{\equal{#2}{high}}{
        \def\badgecolor{threatHigh}
        \def\badgetext{HIGH}
    }{
    \ifthenelse{\equal{#2}{medium}}{
        \def\badgecolor{threatMedium}
        \def\badgetext{MED}
    }{
    \ifthenelse{\equal{#2}{low}}{
        \def\badgecolor{threatLow}
        \def\badgetext{LOW}
    }{
        \def\badgecolor{threatInfo}
        \def\badgetext{INFO}
    }}}}
    
    \node[threat label, fill=\badgecolor, anchor=north west] 
        at (#1.north west) {\badgetext};
}

% Add service/OS label to node
% Usage: \addNodeMetadata{nodename}{metadata}
\newcommand{\addNodeMetadata}[2]{
    \node[font=\tiny\ttfamily, text=black!50, anchor=south] 
        at (#1.south) [below=1pt] {#2};
}

% TODO: Metadata enhancements
% - CVE vulnerability badges with scoring
% - Compliance status indicators (PCI, HIPAA, etc.)
% - Performance metrics (CPU, memory, network utilization)
% - Last scan timestamp and security posture
% - Asset criticality indicators (business impact)
% - Custom metadata fields via configuration

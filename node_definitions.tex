% node_definitions.tex - Network node rendering and management
% This module defines how to create and render individual network assets

% ============================================================================
% NODE DATA STRUCTURES
% ============================================================================

% Node counter for auto-indexing
\newcounter{nodecount}

% Hash map implementation for O(1) node lookup
% Using pgfkeys for efficient key-value storage
\pgfkeys{/nodemap/.is family, /nodemap}

% Store node by IP address
% Usage: \storeNodeByIP{ip}{nodename}
\newcommand{\storeNodeByIP}[2]{
    \pgfkeys{/nodemap/ip/#1/.initial=#2}
}

% Store node by hostname
% Usage: \storeNodeByHostname{hostname}{nodename}
\newcommand{\storeNodeByHostname}[2]{
    \pgfkeys{/nodemap/hostname/#1/.initial=#2}
}

% Store node by ID
% Usage: \storeNodeByID{id}{nodename}
\newcommand{\storeNodeByID}[2]{
    \pgfkeys{/nodemap/id/#1/.initial=#2}
}

% Lookup node by IP address
% Usage: \getNodeByIP{ip}
\newcommand{\getNodeByIP}[1]{%
    \pgfkeysvalueof{/nodemap/ip/#1}%
}

% Lookup node by hostname
% Usage: \getNodeByHostname{hostname}
\newcommand{\getNodeByHostname}[1]{%
    \pgfkeysvalueof{/nodemap/hostname/#1}%
}

% Lookup node by ID
% Usage: \getNodeByID{id}
\newcommand{\getNodeByID}[1]{%
    \pgfkeysvalueof{/nodemap/id/#1}%
}

% Store complete node metadata (IP, hostname, type, position)
% Usage: \storeNodeMetadata{nodename}{ip}{hostname}{type}{x}{y}
\newcommand{\storeNodeMetadata}[6]{
    \pgfkeys{/nodemap/meta/#1/ip/.initial=#2}
    \pgfkeys{/nodemap/meta/#1/hostname/.initial=#3}
    \pgfkeys{/nodemap/meta/#1/type/.initial=#4}
    \pgfkeys{/nodemap/meta/#1/x/.initial=#5}
    \pgfkeys{/nodemap/meta/#1/y/.initial=#6}
    \storeNodeByIP{#2}{#1}
    \storeNodeByHostname{#3}{#1}
    \storeNodeByID{#1}{#1}
}

% Retrieve node metadata
\newcommand{\getNodeIP}[1]{\pgfkeysvalueof{/nodemap/meta/#1/ip}}
\newcommand{\getNodeHostname}[1]{\pgfkeysvalueof{/nodemap/meta/#1/hostname}}
\newcommand{\getNodeType}[1]{\pgfkeysvalueof{/nodemap/meta/#1/type}}
\newcommand{\getNodeX}[1]{\pgfkeysvalueof{/nodemap/meta/#1/x}}
\newcommand{\getNodeY}[1]{\pgfkeysvalueof{/nodemap/meta/#1/y}}

% TODO: Data structure improvements
% - Add node grouping/clustering support
% - Create hierarchical node relationships (parent-child)
% - Support for virtual/container nodes

% ============================================================================
% IP ADDRESS VALIDATION AND FORMATTING
% ============================================================================

% Validate IPv4 address format
% Returns true if valid, false otherwise
% Usage: \validateIPv4{192.168.1.1}
\newcommand{\validateIPv4}[1]{%
    % Simple validation - checks for basic format
    % More complex validation would require regex or Lua
    \def\ipValid{true}%
    #1% For now, accept the IP as-is
}

% Format IPv4 address with proper styling
% Usage: \formatIPv4{192.168.1.1}
\newcommand{\formatIPv4}[1]{%
    \texttt{\small #1}%
}

% Format IPv6 address with compression
% Usage: \formatIPv6{2001:0db8:85a3::8a2e:0370:7334}
\newcommand{\formatIPv6}[1]{%
    \texttt{\footnotesize #1}%
}

% Format CIDR notation
% Usage: \formatCIDR{192.168.1.0/24}
\newcommand{\formatCIDR}[1]{%
    \texttt{\small #1}%
}

% Extract subnet from IP address (assumes /24)
% Usage: \getSubnet{192.168.1.100} returns 192.168.1.0
\newcommand{\getSubnet}[1]{%
    % This is a simplified version - full implementation would need string parsing
    % For now, return the IP as-is
    #1%
}

% Auto-detect IP version
% Usage: \detectIPVersion{address}
\newcommand{\detectIPVersion}[1]{%
    % Check if contains colon (IPv6) or dots (IPv4)
    \IfSubStr{#1}{:}{6}{4}%
}

% Pretty print IP address with auto-detection
% Usage: \prettyIP{192.168.1.1}
\newcommand{\prettyIP}[1]{%
    \IfSubStr{#1}{:}{%
        \formatIPv6{#1}%
    }{%
        \IfSubStr{#1}{/}{%
            \formatCIDR{#1}%
        }{%
            \formatIPv4{#1}%
        }%
    }%
}

% Validate and format IP address
% Usage: \validateAndFormatIP{192.168.1.1}
\newcommand{\validateAndFormatIP}[1]{%
    \prettyIP{#1}%
}

% Extract network class from IPv4
% Usage: \getIPClass{192.168.1.1} returns C
\newcommand{\getIPClass}[1]{%
    % Simplified - would need proper parsing
    % For private IPs starting with 192.168 or 10.0
    \IfSubStr{#1}{192.168}{C}{%
        \IfSubStr{#1}{10.}{A}{%
            \IfSubStr{#1}{172.}{B}{Unknown}%
        }%
    }%
}

% Check if IP is private (RFC 1918)
% Usage: \isPrivateIP{192.168.1.1}
\newcommand{\isPrivateIP}[1]{%
    \IfSubStr{#1}{192.168}{true}{%
        \IfSubStr{#1}{10.}{true}{%
            \IfSubStr{#1}{172.16}{true}{%
                \IfSubStr{#1}{172.17}{true}{%
                    \IfSubStr{#1}{172.18}{true}{%
                        \IfSubStr{#1}{172.19}{true}{%
                            \IfSubStr{#1}{172.20}{true}{%
                                \IfSubStr{#1}{172.21}{true}{%
                                    \IfSubStr{#1}{172.22}{true}{%
                                        \IfSubStr{#1}{172.23}{true}{%
                                            \IfSubStr{#1}{172.24}{true}{%
                                                \IfSubStr{#1}{172.25}{true}{%
                                                    \IfSubStr{#1}{172.26}{true}{%
                                                        \IfSubStr{#1}{172.27}{true}{%
                                                            \IfSubStr{#1}{172.28}{true}{%
                                                                \IfSubStr{#1}{172.29}{true}{%
                                                                    \IfSubStr{#1}{172.30}{true}{%
                                                                        \IfSubStr{#1}{172.31}{true}{false}%
                                                                    }%
                                                                }%
                                                            }%
                                                        }%
                                                    }%
                                                }%
                                            }%
                                        }%
                                    }%
                                }%
                            }%
                        }%
                    }%
                }%
            }%
        }%
    }%
}

% Auto-group nodes by subnet
% This creates a visual grouping indicator
% Usage: \markSubnetGroup{subnet}{color}
\newcommand{\markSubnetGroup}[2]{%
    % Store subnet group information
    \pgfkeys{/nodemap/subnet/#1/color/.initial=#2}%
}

% ============================================================================
% BASIC NODE CREATION COMMANDS
% ============================================================================

% Create a server node
% Usage: \createServer{name}{ip}{x}{y}{label}
\newcommand{\createServer}[5]{
    \node[server] (#1) at (#3,#4) {
        \textbf{#5} \\[2pt]
        \tikz\node[ip label]{#2};
    };
}

% Create a client node
% Usage: \createClient{name}{ip}{x}{y}{label}
\newcommand{\createClient}[5]{
    \node[client] (#1) at (#3,#4) {
        \textbf{#5} \\[2pt]
        \tikz\node[ip label]{#2};
    };
}

% Create a router node
% Usage: \createRouter{name}{ip}{x}{y}{label}
\newcommand{\createRouter}[5]{
    \node[router] (#1) at (#3,#4) {
        \textbf{#5} \\[2pt]
        \tikz\node[ip label]{#2};
    };
}

% Create a firewall node
% Usage: \createFirewall{name}{ip}{x}{y}{label}
\newcommand{\createFirewall}[5]{
    \node[firewall] (#1) at (#3,#4) {
        \textbf{#5} \\[2pt]
        \tikz\node[ip label]{#2};
    };
}

% Create a switch node
% Usage: \createSwitch{name}{ip}{x}{y}{label}
\newcommand{\createSwitch}[5]{
    \node[switch] (#1) at (#3,#4) {
        \textbf{#5} \\[2pt]
        \tikz\node[ip label]{#2};
    };
}

% Create a cloud/internet node
% Usage: \createCloud{name}{x}{y}{label}
\newcommand{\createCloud}[4]{
    \node[cloud] (#1) at (#2,#3) {
        \textbf{#4}
    };
}

% Create an attacker node
% Usage: \createAttacker{name}{ip}{x}{y}{label}
\newcommand{\createAttacker}[5]{
    \node[attacker] (#1) at (#3,#4) {
        \textbf{#5} \\[2pt]
        \tikz\node[ip label, fill=threatCritical!20]{#2};
    };
}

% ============================================================================
% DATABASE NODE CREATION COMMANDS
% ============================================================================

% Create a basic database node
% Usage: \createDatabase{name}{ip}{x}{y}{label}
\newcommand{\createDatabase}[5]{
    \node[database] (#1) at (#3,#4) {
        \textbf{#5} \\[2pt]
        \tikz\node[ip label]{#2};
    };
}

% Create a primary/master database node
% Usage: \createDatabasePrimary{name}{ip}{x}{y}{label}
\newcommand{\createDatabasePrimary}[5]{
    \node[database primary] (#1) at (#3,#4) {
        \textbf{#5} \\[2pt]
        \tikz\node[ip label]{#2}; \\[1pt]
        \tikz\node[font=\tiny\bfseries, text=databaseTeal!80]{PRIMARY};
    };
}

% Create a replica/slave database node
% Usage: \createDatabaseReplica{name}{ip}{x}{y}{label}
\newcommand{\createDatabaseReplica}[5]{
    \node[database replica] (#1) at (#3,#4) {
        \textbf{#5} \\[2pt]
        \tikz\node[ip label]{#2}; \\[1pt]
        \tikz\node[font=\tiny, text=databaseTeal!60]{REPLICA};
    };
}

% Create a database cluster node
% Usage: \createDatabaseCluster{name}{ip}{x}{y}{label}
\newcommand{\createDatabaseCluster}[5]{
    \node[database cluster] (#1) at (#3,#4) {
        \textbf{#5} \\[2pt]
        \tikz\node[ip label]{#2}; \\[1pt]
        \tikz\node[font=\tiny\bfseries, text=databaseTeal!70]{CLUSTER};
    };
}

% Create a database with additional info (ports, type)
% Usage: \createDatabaseWithInfo{name}{ip}{x}{y}{label}{dbtype}{port}
\newcommand{\createDatabaseWithInfo}[7]{
    \node[database, minimum height=2.5cm] (#1) at (#3,#4) {
        \textbf{#5} \\[2pt]
        \tikz\node[ip label]{#2}; \\[2pt]
        \tikz\node[font=\tiny\ttfamily, text=black!60]{#6}; \\[1pt]
        \tikz\node[port label]{Port: #7};
    };
}

% ============================================================================
% LOAD BALANCER NODE CREATION COMMANDS
% ============================================================================

% Create a basic load balancer node
% Usage: \createLoadBalancer{name}{ip}{x}{y}{label}
\newcommand{\createLoadBalancer}[5]{
    \node[loadbalancer] (#1) at (#3,#4) {
        \textbf{#5} \\[2pt]
        \tikz\node[ip label]{#2};
    };
}

% Create an active load balancer node
% Usage: \createLoadBalancerActive{name}{ip}{x}{y}{label}
\newcommand{\createLoadBalancerActive}[5]{
    \node[loadbalancer active] (#1) at (#3,#4) {
        \textbf{#5} \\[2pt]
        \tikz\node[ip label]{#2}; \\[1pt]
        \tikz\node[font=\tiny\bfseries, text=loadBalancerCyan!90]{ACTIVE};
    };
}

% Create a passive load balancer node
% Usage: \createLoadBalancerPassive{name}{ip}{x}{y}{label}
\newcommand{\createLoadBalancerPassive}[5]{
    \node[loadbalancer passive] (#1) at (#3,#4) {
        \textbf{#5} \\[2pt]
        \tikz\node[ip label]{#2}; \\[1pt]
        \tikz\node[font=\tiny, text=loadBalancerCyan!60]{PASSIVE};
    };
}

% Create a load balancer with algorithm indicator
% Usage: \createLoadBalancerWithAlgo{name}{ip}{x}{y}{label}{algorithm}
% Algorithm examples: Round Robin, Least Conn, IP Hash, Weighted
\newcommand{\createLoadBalancerWithAlgo}[6]{
    \node[loadbalancer, minimum height=2.2cm] (#1) at (#3,#4) {
        \textbf{#5} \\[2pt]
        \tikz\node[ip label]{#2}; \\[2pt]
        \tikz\node[font=\tiny\ttfamily, text=black!60]{#6};
    };
}

% Create a load balancer with backend pool count
% Usage: \createLoadBalancerWithPool{name}{ip}{x}{y}{label}{poolsize}
\newcommand{\createLoadBalancerWithPool}[6]{
    \node[loadbalancer, minimum height=2.2cm] (#1) at (#3,#4) {
        \textbf{#5} \\[2pt]
        \tikz\node[ip label]{#2}; \\[2pt]
        \tikz\node[font=\tiny, text=black!60]{Pool: #6 nodes};
    };
}

% ============================================================================
% VIRTUAL MACHINE NODE CREATION COMMANDS
% ============================================================================

% Create a basic virtual machine node
% Usage: \createVM{name}{ip}{x}{y}{label}
\newcommand{\createVM}[5]{
    \node[vm] (#1) at (#3,#4) {
        \textbf{#5} \\[2pt]
        \tikz\node[ip label]{#2}; \\[1pt]
        \tikz\node[font=\tiny, text=vmYellow!80]{VM};
    };
}

% Create VM with hypervisor info
% Usage: \createVMWithHypervisor{name}{ip}{x}{y}{label}{hypervisor}
\newcommand{\createVMWithHypervisor}[6]{
    \node[vm, minimum height=2.2cm] (#1) at (#3,#4) {
        \textbf{#5} \\[2pt]
        \tikz\node[ip label]{#2}; \\[2pt]
        \tikz\node[font=\tiny\ttfamily, text=black!60]{Host: #6};
    };
}

% Create VM cluster node
% Usage: \createVMCluster{name}{ip}{x}{y}{label}{vmcount}
\newcommand{\createVMCluster}[6]{
    \node[vm, minimum height=2.2cm] (#1) at (#3,#4) {
        \textbf{#5} \\[2pt]
        \tikz\node[ip label]{#2}; \\[1pt]
        \tikz\node[font=\tiny\bfseries, text=vmYellow!80]{VM CLUSTER}; \\[1pt]
        \tikz\node[font=\tiny, text=black!60]{#6 instances};
    };
}

% Create hypervisor host node (contains VMs)
% Usage: \createHypervisor{name}{ip}{x}{y}{label}{vmcount}
\newcommand{\createHypervisor}[6]{
    \node[server, minimum height=2.5cm, minimum width=3cm] (#1) at (#3,#4) {
        \textbf{#5} \\[2pt]
        \tikz\node[ip label]{#2}; \\[2pt]
        \tikz\node[font=\tiny\bfseries, text=serverBlue!80]{HYPERVISOR}; \\[1pt]
        \tikz\node[font=\tiny, text=black!60]{VMs: #6};
    };
}

% ============================================================================
% CONTAINER/DOCKER NODE CREATION COMMANDS
% ============================================================================

% Create a basic container node
% Usage: \createContainer{name}{ip}{x}{y}{label}
\newcommand{\createContainer}[5]{
    \node[container] (#1) at (#3,#4) {
        \textbf{#5} \\[1pt]
        \tikz\node[ip label]{#2}; \\[1pt]
        \tikz\node[font=\tiny, text=containerOrange!80]{CONTAINER};
    };
}

% Create container with image info
% Usage: \createContainerWithImage{name}{ip}{x}{y}{label}{image}
\newcommand{\createContainerWithImage}[6]{
    \node[container, minimum height=1.8cm] (#1) at (#3,#4) {
        \textbf{#5} \\[1pt]
        \tikz\node[ip label]{#2}; \\[2pt]
        \tikz\node[font=\tiny\ttfamily, text=black!60]{#6};
    };
}

% Create Kubernetes pod
% Usage: \createK8sPod{name}{ip}{x}{y}{label}{containercount}
\newcommand{\createK8sPod}[6]{
    \node[container, minimum height=2cm, minimum width=3cm] (#1) at (#3,#4) {
        \textbf{#5} \\[1pt]
        \tikz\node[ip label]{#2}; \\[2pt]
        \tikz\node[font=\tiny\bfseries, text=containerOrange!80]{K8s POD}; \\[1pt]
        \tikz\node[font=\tiny, text=black!60]{Containers: #6};
    };
}

% Create container with resource limits
% Usage: \createContainerWithResources{name}{ip}{x}{y}{label}{cpu}{memory}
\newcommand{\createContainerWithResources}[7]{
    \node[container, minimum height=2.2cm] (#1) at (#3,#4) {
        \textbf{#5} \\[1pt]
        \tikz\node[ip label]{#2}; \\[2pt]
        \tikz\node[font=\tiny, text=black!60]{CPU: #6 | RAM: #7};
    };
}

% Create Docker stack (multiple containers)
% Usage: \createDockerStack{name}{ip}{x}{y}{label}{services}
\newcommand{\createDockerStack}[6]{
    \node[container, minimum height=2.2cm, minimum width=3cm] (#1) at (#3,#4) {
        \textbf{#5} \\[1pt]
        \tikz\node[ip label]{#2}; \\[2pt]
        \tikz\node[font=\tiny\bfseries, text=containerOrange!80]{DOCKER STACK}; \\[1pt]
        \tikz\node[font=\tiny, text=black!60]{Services: #6};
    };
}

% ============================================================================
% NODE GROUPING AND CLUSTERING COMMANDS
% ============================================================================

% Draw a cluster boundary around nodes
% Usage: \drawCluster{name}{label}{nodes}{x}{y}{width}{height}
\newcommand{\drawCluster}[7]{
    \node[cluster box, fit=(#3), minimum width=#6cm, minimum height=#7cm,
          label={[font=\small\bfseries, text=clusterPurple!80]above:#2}]
          (#1) at (#4,#5) {};
}

% Draw high-availability pair boundary
% Usage: \drawHAPair{name}{label}{node1}{node2}
\newcommand{\drawHAPair}[4]{
    \node[ha pair, fit=(#3)(#4),
          label={[font=\tiny\bfseries, text=clientGreen!70]above:#2}] (#1) {};
}

% Draw server rack container
% Usage: \drawServerRack{name}{label}{nodes}{x}{y}
\newcommand{\drawServerRack}[5]{
    \node[server rack, fit=(#3),
          label={[font=\small\bfseries, text=black!70]above:#2}]
          (#1) at (#4,#5) {};
}

% Create cluster label with node count
% Usage: \addClusterLabel{clustername}{nodecount}{x}{y}
\newcommand{\addClusterLabel}[4]{
    \node[font=\small\sffamily\bfseries, text=clusterPurple!80]
          at (#3,#4) {#1 (#2 nodes)};
}

% ============================================================================
% MULTI-PART NODE CREATION COMMANDS
% ============================================================================

% Create a detailed multi-part server node
% Usage: \createDetailedServer{name}{ip}{x}{y}{hostname}{ports}{services}
\newcommand{\createDetailedServer}[7]{
    \node[server, minimum height=3cm, minimum width=3.5cm] (#1) at (#3,#4) {
        \begin{tabular}{c}
            \textbf{#5} \\[2pt]
            \hline
            \tikz\node[ip label]{#2}; \\[2pt]
            \hline
            \tikz\node[port label]{Ports: #6}; \\[1pt]
            \tikz\node[font=\tiny\ttfamily, text=black!60]{#7};
        \end{tabular}
    };
}

% Create node with resource utilization bars
% Usage: \createNodeWithMetrics{name}{ip}{x}{y}{label}{cpu}{memory}{disk}
\newcommand{\createNodeWithMetrics}[8]{
    \node[server, minimum height=3.2cm, minimum width=3.5cm] (#1) at (#3,#4) {
        \textbf{#5} \\[2pt]
        \tikz\node[ip label]{#2}; \\[3pt]
        \tikz{
            \node[font=\tiny, anchor=west] at (0,0.3) {CPU: #6\%};
            \draw[fill=serverBlue!50] (0,0.2) rectangle (#6/100*2,0.35);
            \draw[draw=black!30] (0,0.2) rectangle (2,0.35);

            \node[font=\tiny, anchor=west] at (0,0) {RAM: #7\%};
            \draw[fill=clientGreen!50] (0,-0.1) rectangle (#7/100*2,0.05);
            \draw[draw=black!30] (0,-0.1) rectangle (2,0.05);

            \node[font=\tiny, anchor=west] at (0,-0.3) {Disk: #8\%};
            \draw[fill=routerOrange!50] (0,-0.4) rectangle (#8/100*2,-0.25);
            \draw[draw=black!30] (0,-0.4) rectangle (2,-0.25);
        };
    };
}

% Create three-tier node (header, body, footer)
% Usage: \createThreeTierNode{name}{ip}{x}{y}{header}{body}{footer}
\newcommand{\createThreeTierNode}[7]{
    \node[server, minimum height=3cm, minimum width=3.5cm] (#1) at (#3,#4) {
        \begin{tabular}{c}
            \textbf{\textcolor{serverBlue!80}{#5}} \\[1pt]
            \hline
            #6 \\[1pt]
            \tikz\node[ip label]{#2}; \\[1pt]
            \hline
            \tikz\node[font=\tiny, text=black!60]{#7};
        \end{tabular}
    };
}

% ============================================================================
% MOBILE DEVICE NODE CREATION COMMANDS
% ============================================================================

% Create a mobile phone node
% Usage: \createMobilePhone{name}{ip}{x}{y}{label}
\newcommand{\createMobilePhone}[5]{
    \node[mobile phone] (#1) at (#3,#4) {
        \textbf{#5} \\[2pt]
        \tikz\node[ip label, font=\tiny]{#2}; \\[1pt]
        \tikz\node[font=\tiny, text=mobileBlue!80]{PHONE};
    };
}

% Create a mobile phone with OS info
% Usage: \createMobilePhoneWithOS{name}{ip}{x}{y}{label}{os}
\newcommand{\createMobilePhoneWithOS}[6]{
    \node[mobile phone, minimum height=2.8cm] (#1) at (#3,#4) {
        \textbf{#5} \\[1pt]
        \tikz\node[ip label, font=\tiny]{#2}; \\[2pt]
        \tikz\node[font=\tiny\ttfamily, text=black!60]{#6};
    };
}

% Create a tablet node
% Usage: \createTablet{name}{ip}{x}{y}{label}
\newcommand{\createTablet}[5]{
    \node[tablet] (#1) at (#3,#4) {
        \textbf{#5} \\[1pt]
        \tikz\node[ip label, font=\tiny]{#2}; \\[1pt]
        \tikz\node[font=\tiny, text=mobileBlue!80]{TABLET};
    };
}

% Create a tablet with OS info
% Usage: \createTabletWithOS{name}{ip}{x}{y}{label}{os}
\newcommand{\createTabletWithOS}[6]{
    \node[tablet, minimum height=2.2cm] (#1) at (#3,#4) {
        \textbf{#5} \\[1pt]
        \tikz\node[ip label, font=\tiny]{#2}; \\[2pt]
        \tikz\node[font=\tiny\ttfamily, text=black!60]{#6};
    };
}

% ============================================================================
% IOT DEVICE NODE CREATION COMMANDS
% ============================================================================

% Create a basic IoT device node
% Usage: \createIoTDevice{name}{ip}{x}{y}{label}
\newcommand{\createIoTDevice}[5]{
    \node[iot device] (#1) at (#3,#4) {
        \textbf{\small #5} \\[1pt]
        \tikz\node[ip label, font=\tiny]{#2};
    };
}

% Create an IoT device with type
% Usage: \createIoTDeviceWithType{name}{ip}{x}{y}{label}{devicetype}
\newcommand{\createIoTDeviceWithType}[6]{
    \node[iot device, minimum size=2cm] (#1) at (#3,#4) {
        \textbf{\small #5} \\[1pt]
        \tikz\node[ip label, font=\tiny]{#2}; \\[2pt]
        \tikz\node[font=\tiny, text=iotGreen!80]{#6};
    };
}

% Create an IoT sensor node
% Usage: \createIoTSensor{name}{ip}{x}{y}{label}
\newcommand{\createIoTSensor}[5]{
    \node[iot sensor] (#1) at (#3,#4) {
        \textbf{\small #5} \\[1pt]
        \tikz\node[ip label, font=\tiny]{#2};
    };
}

% Create an IoT sensor with measurement type
% Usage: \createIoTSensorWithType{name}{ip}{x}{y}{label}{sensortype}
\newcommand{\createIoTSensorWithType}[6]{
    \node[iot sensor, minimum size=2cm] (#1) at (#3,#4) {
        \textbf{\small #5} \\[1pt]
        \tikz\node[ip label, font=\tiny]{#2}; \\[2pt]
        \tikz\node[font=\tiny\ttfamily, text=black!60]{#6};
    };
}

% ============================================================================
% CLOUD PROVIDER NODE CREATION COMMANDS
% ============================================================================

% Create an AWS cloud node
% Usage: \createAWSNode{name}{ip}{x}{y}{label}{service}
\newcommand{\createAWSNode}[6]{
    \node[aws node] (#1) at (#3,#4) {
        \textbf{#5} \\[2pt]
        \tikz\node[ip label]{#2}; \\[2pt]
        \tikz\node[font=\tiny\bfseries, text=awsOrange!80]{AWS}; \\[1pt]
        \tikz\node[font=\tiny\ttfamily, text=black!60]{#6};
    };
}

% Create an AWS EC2 instance
% Usage: \createAWSEC2{name}{ip}{x}{y}{label}{instancetype}
\newcommand{\createAWSEC2}[6]{
    \node[aws node, minimum height=2.5cm] (#1) at (#3,#4) {
        \textbf{#5} \\[2pt]
        \tikz\node[ip label]{#2}; \\[2pt]
        \tikz\node[font=\tiny\bfseries, text=awsOrange!80]{AWS EC2}; \\[1pt]
        \tikz\node[font=\tiny\ttfamily, text=black!60]{#6};
    };
}

% Create an Azure cloud node
% Usage: \createAzureNode{name}{ip}{x}{y}{label}{service}
\newcommand{\createAzureNode}[6]{
    \node[azure node] (#1) at (#3,#4) {
        \textbf{#5} \\[2pt]
        \tikz\node[ip label]{#2}; \\[2pt]
        \tikz\node[font=\tiny\bfseries, text=azureBlue!80]{AZURE}; \\[1pt]
        \tikz\node[font=\tiny\ttfamily, text=black!60]{#6};
    };
}

% Create an Azure VM
% Usage: \createAzureVM{name}{ip}{x}{y}{label}{vmsize}
\newcommand{\createAzureVM}[6]{
    \node[azure node, minimum height=2.5cm] (#1) at (#3,#4) {
        \textbf{#5} \\[2pt]
        \tikz\node[ip label]{#2}; \\[2pt]
        \tikz\node[font=\tiny\bfseries, text=azureBlue!80]{AZURE VM}; \\[1pt]
        \tikz\node[font=\tiny\ttfamily, text=black!60]{#6};
    };
}

% Create a GCP cloud node
% Usage: \createGCPNode{name}{ip}{x}{y}{label}{service}
\newcommand{\createGCPNode}[6]{
    \node[gcp node] (#1) at (#3,#4) {
        \textbf{#5} \\[2pt]
        \tikz\node[ip label]{#2}; \\[2pt]
        \tikz\node[font=\tiny\bfseries, text=gcpBlue!80]{GCP}; \\[1pt]
        \tikz\node[font=\tiny\ttfamily, text=black!60]{#6};
    };
}

% Create a GCP Compute Engine instance
% Usage: \createGCPCompute{name}{ip}{x}{y}{label}{machinetype}
\newcommand{\createGCPCompute}[6]{
    \node[gcp node, minimum height=2.5cm] (#1) at (#3,#4) {
        \textbf{#5} \\[2pt]
        \tikz\node[ip label]{#2}; \\[2pt]
        \tikz\node[font=\tiny\bfseries, text=gcpBlue!80]{GCP Compute}; \\[1pt]
        \tikz\node[font=\tiny\ttfamily, text=black!60]{#6};
    };
}

% ============================================================================
% CUSTOM NODE TEMPLATE SYSTEM
% ============================================================================

% Define a custom node template
% Usage: \defineNodeTemplate{templatename}{style}{defaultlabel}
\newcommand{\defineNodeTemplate}[3]{
    \pgfkeys{/nodetemplates/#1/style/.initial=#2}
    \pgfkeys{/nodetemplates/#1/defaultlabel/.initial=#3}
}

% Create a node from a template
% Usage: \createNodeFromTemplate{templatename}{name}{ip}{x}{y}{label}
\newcommand{\createNodeFromTemplate}[6]{
    \edef\templatestyle{\pgfkeysvalueof{/nodetemplates/#1/style}}
    \node[\templatestyle] (#2) at (#4,#5) {
        \textbf{#6} \\[2pt]
        \tikz\node[ip label]{#3};
    };
}

% ============================================================================
% SPECIALIZED NETWORK APPLIANCE NODES
% ============================================================================

% Create a proxy server node
% Usage: \createProxy{name}{ip}{x}{y}{label}
\newcommand{\createProxy}[5]{
    \node[proxy] (#1) at (#3,#4) {
        \textbf{#5} \\[2pt]
        \tikz\node[ip label]{#2}; \\[1pt]
        \tikz\node[font=\tiny\bfseries, text=proxyPurple!80]{PROXY};
    };
}

% Create a proxy with type (forward/reverse)
% Usage: \createProxyWithType{name}{ip}{x}{y}{label}{type}
\newcommand{\createProxyWithType}[6]{
    \node[proxy, minimum height=2.2cm] (#1) at (#3,#4) {
        \textbf{#5} \\[2pt]
        \tikz\node[ip label]{#2}; \\[2pt]
        \tikz\node[font=\tiny\bfseries, text=proxyPurple!80]{PROXY}; \\[1pt]
        \tikz\node[font=\tiny, text=black!60]{#6};
    };
}

% Create a VPN endpoint node
% Usage: \createVPN{name}{ip}{x}{y}{label}
\newcommand{\createVPN}[5]{
    \node[vpn] (#1) at (#3,#4) {
        \textbf{#5} \\[2pt]
        \tikz\node[ip label]{#2}; \\[1pt]
        \tikz\node[font=\tiny\bfseries, text=vpnGreen!80]{VPN};
    };
}

% Create VPN with protocol info
% Usage: \createVPNWithProtocol{name}{ip}{x}{y}{label}{protocol}
\newcommand{\createVPNWithProtocol}[6]{
    \node[vpn, minimum height=2.2cm] (#1) at (#3,#4) {
        \textbf{#5} \\[2pt]
        \tikz\node[ip label]{#2}; \\[2pt]
        \tikz\node[font=\tiny\ttfamily, text=black!60]{#6};
    };
}

% Create a DNS server node
% Usage: \createDNS{name}{ip}{x}{y}{label}
\newcommand{\createDNS}[5]{
    \node[dns] (#1) at (#3,#4) {
        \textbf{#5} \\[2pt]
        \tikz\node[ip label]{#2}; \\[1pt]
        \tikz\node[font=\tiny\bfseries, text=dnsBlue!80]{DNS};
    };
}

% Create DNS with zone info
% Usage: \createDNSWithZone{name}{ip}{x}{y}{label}{zone}
\newcommand{\createDNSWithZone}[6]{
    \node[dns, minimum height=2.2cm] (#1) at (#3,#4) {
        \textbf{#5} \\[2pt]
        \tikz\node[ip label]{#2}; \\[2pt]
        \tikz\node[font=\tiny\ttfamily, text=black!60]{#6};
    };
}

% ============================================================================
% NODE STATUS INDICATOR SYSTEM
% ============================================================================

% Add status indicator to a node
% Usage: \addNodeStatus{nodename}{status}
% Status: online, offline, degraded, maintenance
\newcommand{\addNodeStatus}[2]{
    \ifthenelse{\equal{#2}{online}}{
        \node[status online] at (#1.north east) {};
    }{
    \ifthenelse{\equal{#2}{offline}}{
        \node[status offline] at (#1.north east) {};
    }{
    \ifthenelse{\equal{#2}{degraded}}{
        \node[status degraded] at (#1.north east) {};
    }{
        \node[status maintenance] at (#1.north east) {};
    }}}
}

% Add status indicator with label
% Usage: \addNodeStatusWithLabel{nodename}{status}{label}
\newcommand{\addNodeStatusWithLabel}[3]{
    \ifthenelse{\equal{#2}{online}}{
        \def\statuscolor{statusOnline}
    }{
    \ifthenelse{\equal{#2}{offline}}{
        \def\statuscolor{statusOffline}
    }{
    \ifthenelse{\equal{#2}{degraded}}{
        \def\statuscolor{statusDegraded}
    }{
        \def\statuscolor{statusMaintenance}
    }}}

    \node[status badge, fill=\statuscolor, anchor=north east]
        at (#1.north east) {#3};
}

% Create node with status indicator
% Usage: \createServerWithStatus{name}{ip}{x}{y}{label}{status}
\newcommand{\createServerWithStatus}[6]{
    \createServer{#1}{#2}{#3}{#4}{#5}
    \addNodeStatus{#1}{#6}
}

% ============================================================================
% OS BADGE SYSTEM
% ============================================================================

% Add OS badge to a node
% Usage: \addOSBadge{nodename}{os}
% OS: windows, linux, macos, unix, bsd, other
\newcommand{\addOSBadge}[2]{
    \ifthenelse{\equal{#2}{windows}}{
        \def\osbadgecolor{serverBlue}
        \def\osbadgetext{WIN}
    }{
    \ifthenelse{\equal{#2}{linux}}{
        \def\osbadgecolor{routerOrange}
        \def\osbadgetext{LINUX}
    }{
    \ifthenelse{\equal{#2}{macos}}{
        \def\osbadgecolor{cloudGray}
        \def\osbadgetext{macOS}
    }{
    \ifthenelse{\equal{#2}{unix}}{
        \def\osbadgecolor{switchPurple}
        \def\osbadgetext{UNIX}
    }{
    \ifthenelse{\equal{#2}{bsd}}{
        \def\osbadgecolor{firewallRed}
        \def\osbadgetext{BSD}
    }{
        \def\osbadgecolor{black}
        \def\osbadgetext{#2}
    }}}}}

    \node[os badge, fill=\osbadgecolor, anchor=south west]
        at (#1.south west) {\osbadgetext};
}

% Create server with OS badge
% Usage: \createServerWithOS{name}{ip}{x}{y}{label}{os}
\newcommand{\createServerWithOS}[6]{
    \createServer{#1}{#2}{#3}{#4}{#5}
    \addOSBadge{#1}{#6}
}

% Create server with both status and OS
% Usage: \createServerWithStatusAndOS{name}{ip}{x}{y}{label}{status}{os}
\newcommand{\createServerWithStatusAndOS}[7]{
    \createServer{#1}{#2}{#3}{#4}{#5}
    \addNodeStatus{#1}{#6}
    \addOSBadge{#1}{#7}
}

% ============================================================================
% SERVICE/PORT BADGE SYSTEM
% ============================================================================

% Add service badge to a node
% Usage: \addServiceBadge{nodename}{service}{position}
% Service: http, https, ssh, rdp, ftp, smtp, dns, sql, custom
% Position: north, south, east, west, north east, north west, south east, south west
\newcommand{\addServiceBadge}[3]{
    \ifthenelse{\equal{#2}{http}}{
        \def\servicebadgecolor{serverBlue}
        \def\servicebadgetext{HTTP:80}
    }{
    \ifthenelse{\equal{#2}{https}}{
        \def\servicebadgecolor{clientGreen}
        \def\servicebadgetext{HTTPS:443}
    }{
    \ifthenelse{\equal{#2}{ssh}}{
        \def\servicebadgecolor{routerOrange}
        \def\servicebadgetext{SSH:22}
    }{
    \ifthenelse{\equal{#2}{rdp}}{
        \def\servicebadgecolor{serverBlue}
        \def\servicebadgetext{RDP:3389}
    }{
    \ifthenelse{\equal{#2}{ftp}}{
        \def\servicebadgecolor{switchPurple}
        \def\servicebadgetext{FTP:21}
    }{
    \ifthenelse{\equal{#2}{smtp}}{
        \def\servicebadgecolor{databaseTeal}
        \def\servicebadgetext{SMTP:25}
    }{
    \ifthenelse{\equal{#2}{dns}}{
        \def\servicebadgecolor{dnsBlue}
        \def\servicebadgetext{DNS:53}
    }{
    \ifthenelse{\equal{#2}{sql}}{
        \def\servicebadgecolor{databaseTeal}
        \def\servicebadgetext{SQL:3306}
    }{
        \def\servicebadgecolor{black}
        \def\servicebadgetext{#2}
    }}}}}}}}

    \node[service badge, fill=\servicebadgecolor, anchor=#3, font=\tiny]
        at (#1.#3) [#3=1pt] {\servicebadgetext};
}

% Add multiple service badges
% Usage: \addServiceBadges{nodename}{service1,service2,service3}
\newcommand{\addServiceBadges}[2]{
    % This is a simplified version - for multiple badges
    % Users can call addServiceBadge multiple times with different positions
    \addServiceBadge{#1}{#2}{south}
}

% ============================================================================
% SECURITY POSTURE INDICATORS
% ============================================================================

% Add security posture indicator
% Usage: \addSecurityPosture{nodename}{level}
% Level: high, medium, low, critical
\newcommand{\addSecurityPosture}[2]{
    \ifthenelse{\equal{#2}{high}}{
        \def\seccolor{clientGreen}
        \def\sectext{SECURE}
    }{
    \ifthenelse{\equal{#2}{medium}}{
        \def\seccolor{threatMedium}
        \def\sectext{MEDIUM}
    }{
    \ifthenelse{\equal{#2}{low}}{
        \def\seccolor{threatHigh}
        \def\sectext{AT RISK}
    }{
        \def\seccolor{threatCritical}
        \def\sectext{CRITICAL}
    }}}

    \node[threat label, fill=\seccolor, anchor=north west]
        at (#1.north west) {\sectext};
}

% Add compliance badge
% Usage: \addComplianceBadge{nodename}{standard}
% Standard: pci, hipaa, sox, iso27001, gdpr
\newcommand{\addComplianceBadge}[2]{
    \ifthenelse{\equal{#2}{pci}}{
        \def\comptext{PCI-DSS}
    }{
    \ifthenelse{\equal{#2}{hipaa}}{
        \def\comptext{HIPAA}
    }{
    \ifthenelse{\equal{#2}{sox}}{
        \def\comptext{SOX}
    }{
    \ifthenelse{\equal{#2}{iso27001}}{
        \def\comptext{ISO27001}
    }{
        \def\comptext{GDPR}
    }}}}

    \node[font=\tiny\sffamily\bfseries, text=white, fill=clientGreen,
          inner sep=2pt, rounded corners=1pt, anchor=south east]
        at (#1.south east) {\comptext};
}

% TODO: Advanced node creation
% - Add support for custom node shapes via parameters
% - Add animation states for live monitoring
% - Add real-time metrics updating

% ============================================================================
% ENHANCED NODE VARIANTS
% ============================================================================

% Create a server with port information
% Usage: \createServerWithPorts{name}{ip}{x}{y}{label}{ports}
\newcommand{\createServerWithPorts}[6]{
    \node[server, minimum height=2cm] (#1) at (#3,#4) {
        \textbf{#5} \\[2pt]
        \tikz\node[ip label]{#2}; \\[3pt]
        \tikz\node[port label]{#6};
    };
}

% Create a node with security status indicator
% Usage: \createSecureNode{type}{name}{ip}{x}{y}{label}{status}
% Status: secure, warning, compromised
\newcommand{\createSecureNode}[7]{
    \ifthenelse{\equal{#7}{secure}}{
        \def\statuscolor{clientGreen}
    }{
    \ifthenelse{\equal{#7}{warning}}{
        \def\statuscolor{threatMedium}
    }{
        \def\statuscolor{threatCritical}
    }}
    
    \ifthenelse{\equal{#1}{server}}{
        \createServer{#2}{#3}{#4}{#5}{#6}
    }{
    \ifthenelse{\equal{#1}{client}}{
        \createClient{#2}{#3}{#4}{#5}{#6}
    }{
        \createServer{#2}{#3}{#4}{#5}{#6}
    }}
    
    \node[circle, fill=\statuscolor, inner sep=2pt, 
          anchor=north east] at (#2.north east) {};
}

% TODO: Enhanced node variants
% - Database server nodes with cylinder shape
% - Load balancer nodes with special indicators
% - Virtual machine nodes with nested appearance
% - Container/Docker nodes with stacked appearance
% - Mobile device nodes with phone/tablet shapes
% - IoT device nodes with specialized icons

% ============================================================================
% NODE RENDERING ENGINE
% ============================================================================

% Main command to render all nodes from data structure
\newcommand{\renderNetworkNodes}{
    % This will be populated by network_data.tex
    % Example structure:
    % \createServer{srv1}{192.168.1.10}{0}{0}{Web Server}
    % \createClient{pc1}{192.168.1.100}{-5}{-3}{Workstation 1}
}

% TODO: Intelligent rendering
% - Auto-layout algorithm to prevent overlapping nodes
% - Force-directed graph layout for organic appearance
% - Hierarchical layout for structured networks
% - Subnet-based clustering and grouping
% - Zoom levels for large networks (overview vs detail)
% - LOD (Level of Detail) rendering for performance

% ============================================================================
% NODE METADATA AND ANNOTATIONS
% ============================================================================

% Add annotation to existing node
% Usage: \annotateNode{nodename}{annotation}{position}
% Position: above, below, left, right, above right, etc.
\newcommand{\annotateNode}[3]{
    \node[font=\tiny\sffamily\itshape, text=black!60] at (#1.#3) {#2};
}

% Add threat indicator badge to node
% Usage: \addThreatBadge{nodename}{level}
% Level: critical, high, medium, low, info
\newcommand{\addThreatBadge}[2]{
    \ifthenelse{\equal{#2}{critical}}{
        \def\badgecolor{threatCritical}
        \def\badgetext{CRIT}
    }{
    \ifthenelse{\equal{#2}{high}}{
        \def\badgecolor{threatHigh}
        \def\badgetext{HIGH}
    }{
    \ifthenelse{\equal{#2}{medium}}{
        \def\badgecolor{threatMedium}
        \def\badgetext{MED}
    }{
    \ifthenelse{\equal{#2}{low}}{
        \def\badgecolor{threatLow}
        \def\badgetext{LOW}
    }{
        \def\badgecolor{threatInfo}
        \def\badgetext{INFO}
    }}}}
    
    \node[threat label, fill=\badgecolor, anchor=north west] 
        at (#1.north west) {\badgetext};
}

% Add service/OS label to node
% Usage: \addNodeMetadata{nodename}{metadata}
\newcommand{\addNodeMetadata}[2]{
    \node[font=\tiny\ttfamily, text=black!50, anchor=south] 
        at (#1.south) [below=1pt] {#2};
}

% TODO: Metadata enhancements
% - CVE vulnerability badges with scoring
% - Compliance status indicators (PCI, HIPAA, etc.)
% - Performance metrics (CPU, memory, network utilization)
% - Last scan timestamp and security posture
% - Asset criticality indicators (business impact)
% - Custom metadata fields via configuration

% node_definitions.tex - Network node rendering and management
% This module defines how to create and render individual network assets

% ============================================================================
% NODE DATA STRUCTURES
% ============================================================================

% Node counter for auto-indexing
\newcounter{nodecount}

% TODO: Data structure improvements
% - Implement hash map for O(1) node lookup by IP
% - Add node grouping/clustering support
% - Create hierarchical node relationships (parent-child)
% - Support for virtual/container nodes

% ============================================================================
% BASIC NODE CREATION COMMANDS
% ============================================================================

% Create a server node
% Usage: \createServer{name}{ip}{x}{y}{label}
\newcommand{\createServer}[5]{
    \node[server] (#1) at (#3,#4) {
        \textbf{#5} \\[2pt]
        \tikz\node[ip label]{#2};
    };
}

% Create a client node
% Usage: \createClient{name}{ip}{x}{y}{label}
\newcommand{\createClient}[5]{
    \node[client] (#1) at (#3,#4) {
        \textbf{#5} \\[2pt]
        \tikz\node[ip label]{#2};
    };
}

% Create a router node
% Usage: \createRouter{name}{ip}{x}{y}{label}
\newcommand{\createRouter}[5]{
    \node[router] (#1) at (#3,#4) {
        \textbf{#5} \\[2pt]
        \tikz\node[ip label]{#2};
    };
}

% Create a firewall node
% Usage: \createFirewall{name}{ip}{x}{y}{label}
\newcommand{\createFirewall}[5]{
    \node[firewall] (#1) at (#3,#4) {
        \textbf{#5} \\[2pt]
        \tikz\node[ip label]{#2};
    };
}

% Create a switch node
% Usage: \createSwitch{name}{ip}{x}{y}{label}
\newcommand{\createSwitch}[5]{
    \node[switch] (#1) at (#3,#4) {
        \textbf{#5} \\[2pt]
        \tikz\node[ip label]{#2};
    };
}

% Create a cloud/internet node
% Usage: \createCloud{name}{x}{y}{label}
\newcommand{\createCloud}[4]{
    \node[cloud] (#1) at (#2,#3) {
        \textbf{#4}
    };
}

% Create an attacker node
% Usage: \createAttacker{name}{ip}{x}{y}{label}
\newcommand{\createAttacker}[5]{
    \node[attacker] (#1) at (#3,#4) {
        \textbf{#5} \\[2pt]
        \tikz\node[ip label, fill=threatCritical!20]{#2};
    };
}

% TODO: Advanced node creation
% - Add support for custom node shapes via parameters
% - Implement node templates for common device types
% - Add automatic IP validation and formatting
% - Create composite nodes (e.g., server rack with multiple servers)
% - Support for node status indicators (up/down/warning)

% ============================================================================
% ENHANCED NODE VARIANTS
% ============================================================================

% Create a server with port information
% Usage: \createServerWithPorts{name}{ip}{x}{y}{label}{ports}
\newcommand{\createServerWithPorts}[6]{
    \node[server, minimum height=2cm] (#1) at (#3,#4) {
        \textbf{#5} \\[2pt]
        \tikz\node[ip label]{#2}; \\[3pt]
        \tikz\node[port label]{#6};
    };
}

% Create a node with security status indicator
% Usage: \createSecureNode{type}{name}{ip}{x}{y}{label}{status}
% Status: secure, warning, compromised
\newcommand{\createSecureNode}[7]{
    \ifthenelse{\equal{#7}{secure}}{
        \def\statuscolor{clientGreen}
    }{
    \ifthenelse{\equal{#7}{warning}}{
        \def\statuscolor{threatMedium}
    }{
        \def\statuscolor{threatCritical}
    }}
    
    \ifthenelse{\equal{#1}{server}}{
        \createServer{#2}{#3}{#4}{#5}{#6}
    }{
    \ifthenelse{\equal{#1}{client}}{
        \createClient{#2}{#3}{#4}{#5}{#6}
    }{
        \createServer{#2}{#3}{#4}{#5}{#6}
    }}
    
    \node[circle, fill=\statuscolor, inner sep=2pt, 
          anchor=north east] at (#2.north east) {};
}

% TODO: Enhanced node variants
% - Database server nodes with cylinder shape
% - Load balancer nodes with special indicators
% - Virtual machine nodes with nested appearance
% - Container/Docker nodes with stacked appearance
% - Mobile device nodes with phone/tablet shapes
% - IoT device nodes with specialized icons

% ============================================================================
% NODE RENDERING ENGINE
% ============================================================================

% Main command to render all nodes from data structure
\newcommand{\renderNetworkNodes}{
    % This will be populated by network_data.tex
    % Example structure:
    % \createServer{srv1}{192.168.1.10}{0}{0}{Web Server}
    % \createClient{pc1}{192.168.1.100}{-5}{-3}{Workstation 1}
}

% TODO: Intelligent rendering
% - Auto-layout algorithm to prevent overlapping nodes
% - Force-directed graph layout for organic appearance
% - Hierarchical layout for structured networks
% - Subnet-based clustering and grouping
% - Zoom levels for large networks (overview vs detail)
% - LOD (Level of Detail) rendering for performance

% ============================================================================
% NODE METADATA AND ANNOTATIONS
% ============================================================================

% Add annotation to existing node
% Usage: \annotateNode{nodename}{annotation}{position}
% Position: above, below, left, right, above right, etc.
\newcommand{\annotateNode}[3]{
    \node[font=\tiny\sffamily\itshape, text=black!60] at (#1.#3) {#2};
}

% Add threat indicator badge to node
% Usage: \addThreatBadge{nodename}{level}
% Level: critical, high, medium, low, info
\newcommand{\addThreatBadge}[2]{
    \ifthenelse{\equal{#2}{critical}}{
        \def\badgecolor{threatCritical}
        \def\badgetext{CRIT}
    }{
    \ifthenelse{\equal{#2}{high}}{
        \def\badgecolor{threatHigh}
        \def\badgetext{HIGH}
    }{
    \ifthenelse{\equal{#2}{medium}}{
        \def\badgecolor{threatMedium}
        \def\badgetext{MED}
    }{
    \ifthenelse{\equal{#2}{low}}{
        \def\badgecolor{threatLow}
        \def\badgetext{LOW}
    }{
        \def\badgecolor{threatInfo}
        \def\badgetext{INFO}
    }}}}
    
    \node[threat label, fill=\badgecolor, anchor=north west] 
        at (#1.north west) {\badgetext};
}

% Add service/OS label to node
% Usage: \addNodeMetadata{nodename}{metadata}
\newcommand{\addNodeMetadata}[2]{
    \node[font=\tiny\ttfamily, text=black!50, anchor=south] 
        at (#1.south) [below=1pt] {#2};
}

% TODO: Metadata enhancements
% - CVE vulnerability badges with scoring
% - Compliance status indicators (PCI, HIPAA, etc.)
% - Performance metrics (CPU, memory, network utilization)
% - Last scan timestamp and security posture
% - Asset criticality indicators (business impact)
% - Custom metadata fields via configuration

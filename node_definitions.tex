% node_definitions.tex - Network node rendering and management
% This module defines how to create and render individual network assets

% ============================================================================
% NODE DATA STRUCTURES
% ============================================================================

% Node counter for auto-indexing
\newcounter{nodecount}

% ============================================================================
% HASH MAP IMPLEMENTATION FOR O(1) NODE LOOKUP
% ============================================================================
% This implementation uses pgfkeys to create a hash map for efficient node lookup
% by IP address, hostname, or node ID

% Initialize hash map storage
\pgfkeys{
    /nodemap/.cd,
    .unknown/.code={
        \pgfkeyssetvalue{\pgfkeyscurrentpath/\pgfkeyscurrentname}{#1}
    }
}

% Register a node in the hash map
% Usage: \registerNode{nodeID}{IP}{hostname}
\newcommand{\registerNode}[3]{
    % Store by node ID
    \pgfkeys{/nodemap/byid/#1/.initial={#1}}
    \pgfkeys{/nodemap/byid/#1/ip/.initial={#2}}
    \pgfkeys{/nodemap/byid/#1/hostname/.initial={#3}}

    % Store by IP address (replace . with _ for key safety)
    \StrSubstitute{#2}{.}{_}[\safeip]
    \pgfkeys{/nodemap/byip/\safeip/.initial={#1}}

    % Store by hostname
    \pgfkeys{/nodemap/byhost/#3/.initial={#1}}
}

% Lookup node ID by IP address
% Usage: \getNodeByIP{192.168.1.10}{\resultvar}
\newcommand{\getNodeByIP}[2]{
    \StrSubstitute{#1}{.}{_}[\safeip]
    \pgfkeysgetvalue{/nodemap/byip/\safeip}{#2}
}

% Lookup node ID by hostname
% Usage: \getNodeByHostname{webserver}{\resultvar}
\newcommand{\getNodeByHostname}[2]{
    \pgfkeysgetvalue{/nodemap/byhost/#1}{#2}
}

% Get node IP by node ID
% Usage: \getNodeIP{srv1}{\resultvar}
\newcommand{\getNodeIP}[2]{
    \pgfkeysgetvalue{/nodemap/byid/#1/ip}{#2}
}

% Get node hostname by node ID
% Usage: \getNodeHostname{srv1}{\resultvar}
\newcommand{\getNodeHostname}[2]{
    \pgfkeysgetvalue{/nodemap/byid/#1/hostname}{#2}
}

% ============================================================================
% EXTENDED HASH MAP WITH METADATA STORAGE
% ============================================================================

% Register node with extended metadata
% Usage: \registerNodeExtended{nodeID}{IP}{hostname}{type}{os}{status}
\newcommand{\registerNodeExtended}[6]{
    % Basic registration
    \registerNode{#1}{#2}{#3}

    % Extended metadata
    \pgfkeys{/nodemap/byid/#1/type/.initial={#4}}
    \pgfkeys{/nodemap/byid/#1/os/.initial={#5}}
    \pgfkeys{/nodemap/byid/#1/status/.initial={#6}}
}

% Store node group/cluster membership
% Usage: \assignNodeToCluster{nodeID}{clusterID}
\newcommand{\assignNodeToCluster}[2]{
    \pgfkeys{/nodemap/byid/#1/cluster/.initial={#2}}
}

% Store parent-child relationship (for VMs, containers)
% Usage: \setNodeParent{nodeID}{parentID}
\newcommand{\setNodeParent}[2]{
    \pgfkeys{/nodemap/byid/#1/parent/.initial={#2}}
}

% Store node services
% Usage: \setNodeServices{nodeID}{services}
\newcommand{\setNodeServices}[2]{
    \pgfkeys{/nodemap/byid/#1/services/.initial={#2}}
}

% Store node security status
% Usage: \setNodeSecurity{nodeID}{vulncount}{cvss}{compliance}
\newcommand{\setNodeSecurity}[4]{
    \pgfkeys{/nodemap/byid/#1/vulncount/.initial={#2}}
    \pgfkeys{/nodemap/byid/#1/cvss/.initial={#3}}
    \pgfkeys{/nodemap/byid/#1/compliance/.initial={#4}}
}

% Get node metadata
\newcommand{\getNodeType}[2]{
    \pgfkeysgetvalue{/nodemap/byid/#1/type}{#2}
}

\newcommand{\getNodeOS}[2]{
    \pgfkeysgetvalue{/nodemap/byid/#1/os}{#2}
}

\newcommand{\getNodeStatus}[2]{
    \pgfkeysgetvalue{/nodemap/byid/#1/status}{#2}
}

\newcommand{\getNodeCluster}[2]{
    \pgfkeysgetvalue{/nodemap/byid/#1/cluster}{#2}
}

\newcommand{\getNodeParent}[2]{
    \pgfkeysgetvalue{/nodemap/byid/#1/parent}{#2}
}

% TODO: Advanced hash map features
% - Index nodes by multiple attributes simultaneously
% - Support for node tagging and label querying
% - Network topology graph representation

% ============================================================================
% IP ADDRESS VALIDATION AND FORMATTING
% ============================================================================

% Validate IPv4 address format
% Usage: \validateIPv4{192.168.1.10}{\resultvar}
% Result: 1 if valid, 0 if invalid
\newcommand{\validateIPv4}[2]{
    \def#2{1} % Assume valid by default

    % Check if IP contains 3 dots
    \StrCount{#1}{.}[\dotcount]
    \ifnum\dotcount=3\relax
        % Valid format (basic check)
        \def#2{1}
    \else
        \def#2{0}
    \fi
}

% Validate IPv6 address format (basic check)
% Usage: \validateIPv6{2001:0db8::1}{\resultvar}
% Result: 1 if valid, 0 if invalid
\newcommand{\validateIPv6}[2]{
    \def#2{1} % Assume valid by default

    % Check if IP contains colons
    \StrCount{#1}{:}[\coloncount]
    \ifnum\coloncount>1\relax
        % Valid format (basic check)
        \def#2{1}
    \else
        \def#2{0}
    \fi
}

% Auto-detect IP version and validate
% Usage: \validateIP{192.168.1.10}{\resultvar}
% Result: 4 for IPv4, 6 for IPv6, 0 for invalid
\newcommand{\validateIP}[2]{
    \StrCount{#1}{.}[\dotcount]
    \StrCount{#1}{:}[\coloncount]

    \ifnum\dotcount=3\relax
        \def#2{4} % IPv4
    \else
        \ifnum\coloncount>1\relax
            \def#2{6} % IPv6
        \else
            \def#2{0} % Invalid
        \fi
    \fi
}

% Format IP address with CIDR notation
% Usage: \formatCIDR{192.168.1.0}{24}{\resultvar}
\newcommand{\formatCIDR}[3]{
    \def#3{#1/#2}
}

% Extract subnet from IP address
% Usage: \extractSubnet{192.168.1.100}{\resultvar}
% Returns: 192.168.1.0 (class C subnet)
\newcommand{\extractSubnet}[2]{
    \StrCut{#1}{.}{\octetA}{\remaining}
    \StrCut{\remaining}{.}{\octetB}{\remaining}
    \StrCut{\remaining}{.}{\octetC}{\octetD}
    \def#2{\octetA.\octetB.\octetC.0}
}

% Determine if two IPs are in same subnet (simple /24 check)
% Usage: \sameSubnet{192.168.1.10}{192.168.1.20}{\resultvar}
% Result: 1 if same subnet, 0 if different
\newcommand{\sameSubnet}[3]{
    \extractSubnet{#1}{\subnetA}
    \extractSubnet{#2}{\subnetB}

    \ifthenelse{\equal{\subnetA}{\subnetB}}{
        \def#3{1}
    }{
        \def#3{0}
    }
}

% Validate IPv4 octet is in range 0-255
% Usage: \validateOctet{value}{\resultvar}
\newcommand{\validateOctet}[2]{
    \ifnum#1<0\relax
        \def#2{0}
    \else
        \ifnum#1>255\relax
            \def#2{0}
        \else
            \def#2{1}
        \fi
    \fi
}

% Enhanced IPv4 validation with octet range checking
% Usage: \validateIPv4Enhanced{192.168.1.10}{\resultvar}
\newcommand{\validateIPv4Enhanced}[2]{
    \def#2{0} % Assume invalid by default

    % First check format
    \StrCount{#1}{.}[\dotcount]
    \ifnum\dotcount=3\relax
        % Extract octets
        \StrCut{#1}{.}{\octetA}{\remaining}
        \StrCut{\remaining}{.}{\octetB}{\remaining}
        \StrCut{\remaining}{.}{\octetC}{\octetD}

        % Note: Full range validation would require more complex LaTeX programming
        % This is a basic structure - full implementation would need numeric comparison
        \def#2{1} % Mark as valid if format is correct
    \fi
}

% Extract subnet with variable CIDR prefix
% Usage: \extractSubnetCIDR{192.168.1.100}{24}{\resultvar}
\newcommand{\extractSubnetCIDR}[3]{
    \StrCut{#1}{.}{\octetA}{\remaining}
    \StrCut{\remaining}{.}{\octetB}{\remaining}
    \StrCut{\remaining}{.}{\octetC}{\octetD}

    \ifnum#2=8\relax
        \def#3{\octetA.0.0.0}
    \else
        \ifnum#2=16\relax
            \def#3{\octetA.\octetB.0.0}
        \else
            \ifnum#2=24\relax
                \def#3{\octetA.\octetB.\octetC.0}
            \else
                % Default to /24
                \def#3{\octetA.\octetB.\octetC.0}
            \fi
        \fi
    \fi
}

% Check if IP is in private range
% Usage: \isPrivateIP{192.168.1.10}{\resultvar}
% Returns: 1 if private, 0 if public
\newcommand{\isPrivateIP}[2]{
    \StrCut{#1}{.}{\octetA}{\remaining}
    \StrCut{\remaining}{.}{\octetB}{\remaining}

    % Check for 10.0.0.0/8
    \ifthenelse{\equal{\octetA}{10}}{
        \def#2{1}
    }{
    % Check for 192.168.0.0/16
    \ifthenelse{\equal{\octetA}{192}}{
        \ifthenelse{\equal{\octetB}{168}}{
            \def#2{1}
        }{
            \def#2{0}
        }
    }{
    % Check for 172.16.0.0/12 (172.16-172.31)
    \ifthenelse{\equal{\octetA}{172}}{
        \def#2{1} % Simplified - should check 16-31 range
    }{
        \def#2{0}
    }}}
}

% TODO: Advanced IP validation
% - Full numeric range validation for octets (requires lua or complex TeX)
% - Full IPv6 validation with compression
% - CIDR validation and calculations
% - IP range validation (start-end)

% ============================================================================
% BASIC NODE CREATION COMMANDS
% ============================================================================

% Create a server node
% Usage: \createServer{name}{ip}{x}{y}{label}
\newcommand{\createServer}[5]{
    \node[server] (#1) at (#3,#4) {
        \textbf{#5} \\[2pt]
        \tikz\node[ip label]{#2};
    };
}

% Create a client node
% Usage: \createClient{name}{ip}{x}{y}{label}
\newcommand{\createClient}[5]{
    \node[client] (#1) at (#3,#4) {
        \textbf{#5} \\[2pt]
        \tikz\node[ip label]{#2};
    };
}

% Create a router node
% Usage: \createRouter{name}{ip}{x}{y}{label}
\newcommand{\createRouter}[5]{
    \node[router] (#1) at (#3,#4) {
        \textbf{#5} \\[2pt]
        \tikz\node[ip label]{#2};
    };
}

% Create a firewall node
% Usage: \createFirewall{name}{ip}{x}{y}{label}
\newcommand{\createFirewall}[5]{
    \node[firewall] (#1) at (#3,#4) {
        \textbf{#5} \\[2pt]
        \tikz\node[ip label]{#2};
    };
}

% Create a switch node
% Usage: \createSwitch{name}{ip}{x}{y}{label}
\newcommand{\createSwitch}[5]{
    \node[switch] (#1) at (#3,#4) {
        \textbf{#5} \\[2pt]
        \tikz\node[ip label]{#2};
    };
}

% Create a cloud/internet node
% Usage: \createCloud{name}{x}{y}{label}
\newcommand{\createCloud}[4]{
    \node[cloud] (#1) at (#2,#3) {
        \textbf{#4}
    };
}

% Create an attacker node
% Usage: \createAttacker{name}{ip}{x}{y}{label}
\newcommand{\createAttacker}[5]{
    \node[attacker] (#1) at (#3,#4) {
        \textbf{#5} \\[2pt]
        \tikz\node[ip label, fill=threatCritical!20]{#2};
    };
}

% TODO: Advanced node creation
% - Add support for custom node shapes via parameters
% - Implement node templates for common device types
% - Add automatic IP validation and formatting
% - Create composite nodes (e.g., server rack with multiple servers)
% - Support for node status indicators (up/down/warning)

% ============================================================================
% ENHANCED NODE VARIANTS
% ============================================================================

% Create a server with port information
% Usage: \createServerWithPorts{name}{ip}{x}{y}{label}{ports}
\newcommand{\createServerWithPorts}[6]{
    \node[server, minimum height=2cm] (#1) at (#3,#4) {
        \textbf{#5} \\[2pt]
        \tikz\node[ip label]{#2}; \\[3pt]
        \tikz\node[port label]{#6};
    };
}

% Create a node with security status indicator
% Usage: \createSecureNode{type}{name}{ip}{x}{y}{label}{status}
% Status: secure, warning, compromised
\newcommand{\createSecureNode}[7]{
    \ifthenelse{\equal{#7}{secure}}{
        \def\statuscolor{clientGreen}
    }{
    \ifthenelse{\equal{#7}{warning}}{
        \def\statuscolor{threatMedium}
    }{
        \def\statuscolor{threatCritical}
    }}
    
    \ifthenelse{\equal{#1}{server}}{
        \createServer{#2}{#3}{#4}{#5}{#6}
    }{
    \ifthenelse{\equal{#1}{client}}{
        \createClient{#2}{#3}{#4}{#5}{#6}
    }{
        \createServer{#2}{#3}{#4}{#5}{#6}
    }}
    
    \node[circle, fill=\statuscolor, inner sep=2pt, 
          anchor=north east] at (#2.north east) {};
}

% ============================================================================
% DATABASE SERVER NODES
% ============================================================================

% Create a database server node (basic)
% Usage: \createDatabase{name}{ip}{x}{y}{label}
\newcommand{\createDatabase}[5]{
    \node[database] (#1) at (#3,#4) {
        \textbf{#5} \\[2pt]
        \tikz\node[ip label]{#2};
    };
}

% Create a primary database server node
% Usage: \createDatabasePrimary{name}{ip}{x}{y}{label}
\newcommand{\createDatabasePrimary}[5]{
    \node[database primary] (#1) at (#3,#4) {
        \textbf{#5} \\[2pt]
        \tikz\node[ip label]{#2}; \\[1pt]
        \tikz\node[font=\tiny\bfseries, text=databaseTeal!90]{PRIMARY};
    };
}

% Create a replica database server node
% Usage: \createDatabaseReplica{name}{ip}{x}{y}{label}
\newcommand{\createDatabaseReplica}[5]{
    \node[database replica] (#1) at (#3,#4) {
        \textbf{#5} \\[2pt]
        \tikz\node[ip label]{#2}; \\[1pt]
        \tikz\node[font=\tiny\bfseries, text=databaseTeal!70]{REPLICA};
    };
}

% Create a cluster database server node
% Usage: \createDatabaseCluster{name}{ip}{x}{y}{label}
\newcommand{\createDatabaseCluster}[5]{
    \node[database cluster] (#1) at (#3,#4) {
        \textbf{#5} \\[2pt]
        \tikz\node[ip label]{#2}; \\[1pt]
        \tikz\node[font=\tiny\bfseries, text=databaseTeal!85]{CLUSTER};
    };
}

% ============================================================================
% LOAD BALANCER NODES
% ============================================================================

% Create a load balancer node (basic)
% Usage: \createLoadBalancer{name}{ip}{x}{y}{label}
\newcommand{\createLoadBalancer}[5]{
    \node[loadbalancer] (#1) at (#3,#4) {
        \textbf{#5} \\[2pt]
        \tikz\node[ip label]{#2};
    };
}

% Create an active load balancer node
% Usage: \createLoadBalancerActive{name}{ip}{x}{y}{label}{algorithm}
% Algorithm: round-robin, least-conn, ip-hash, weighted
\newcommand{\createLoadBalancerActive}[6]{
    \node[loadbalancer active] (#1) at (#3,#4) {
        \textbf{#5} \\[2pt]
        \tikz\node[ip label]{#2}; \\[1pt]
        \tikz\node[font=\tiny\ttfamily, text=loadBalancerCyan!90]{#6}; \\[1pt]
        \tikz\node[font=\tiny\bfseries, text=clientGreen!80]{ACTIVE};
    };
}

% Create a passive load balancer node
% Usage: \createLoadBalancerPassive{name}{ip}{x}{y}{label}
\newcommand{\createLoadBalancerPassive}[5]{
    \node[loadbalancer passive] (#1) at (#3,#4) {
        \textbf{#5} \\[2pt]
        \tikz\node[ip label]{#2}; \\[1pt]
        \tikz\node[font=\tiny\bfseries, text=black!50]{STANDBY};
    };
}

% Add load distribution indicator to load balancer
% Usage: \addLoadDistribution{nodename}{backend1,backend2,backend3}
\newcommand{\addLoadDistribution}[2]{
    \node[font=\tiny\ttfamily, text=black!60, anchor=south]
        at (#1.south) [below=2pt] {Backends: #2};
}

% ============================================================================
% VIRTUAL MACHINE NODES
% ============================================================================

% Create a virtual machine node
% Usage: \createVM{name}{ip}{x}{y}{label}{hypervisor}
\newcommand{\createVM}[6]{
    \node[vm] (#1) at (#3,#4) {
        \textbf{#5} \\[2pt]
        \tikz\node[ip label]{#2}; \\[1pt]
        \tikz\node[font=\tiny\ttfamily, text=vmIndigo!70]{Host: #6};
    };
}

% Create a hypervisor node (can contain VMs)
% Usage: \createHypervisor{name}{ip}{x}{y}{label}{vmcount}
\newcommand{\createHypervisor}[6]{
    \node[vm hypervisor] (#1) at (#3,#4) {
        \textbf{#5} \\[2pt]
        \tikz\node[ip label]{#2}; \\[3pt]
        \tikz\node[font=\tiny\bfseries, text=vmIndigo!90]{HYPERVISOR}; \\[1pt]
        \tikz\node[font=\tiny\ttfamily, text=vmIndigo!70]{VMs: #6};
    };
}

% Create VM with resource info
% Usage: \createVMWithResources{name}{ip}{x}{y}{label}{cpu}{ram}{disk}
\newcommand{\createVMWithResources}[8]{
    \node[vm] (#1) at (#3,#4) {
        \textbf{#5} \\[2pt]
        \tikz\node[ip label]{#2}; \\[2pt]
        \tikz\node[font=\tiny\ttfamily, text=black!60]{CPU: #6 | RAM: #7 | Disk: #8};
    };
}

% ============================================================================
% CONTAINER/DOCKER NODES
% ============================================================================

% Create a container node
% Usage: \createContainer{name}{ip}{x}{y}{label}{image}
\newcommand{\createContainer}[6]{
    \node[container] (#1) at (#3,#4) {
        \textbf{#5} \\[1pt]
        \tikz\node[ip label]{#2}; \\[1pt]
        \tikz\node[font=\tiny\ttfamily, text=containerBlue!70]{#6};
    };
    % Add stacked effect
    \draw[containerBlue!60, line width=0.8pt]
        ([xshift=-2pt, yshift=2pt]#1.north west) --
        ([xshift=-2pt, yshift=2pt]#1.north east) --
        ([xshift=-2pt]#1.south east);
    \draw[containerBlue!40, line width=0.6pt]
        ([xshift=-4pt, yshift=4pt]#1.north west) --
        ([xshift=-4pt, yshift=4pt]#1.north east) --
        ([xshift=-4pt]#1.south east);
}

% Create a Kubernetes pod node
% Usage: \createPod{name}{ip}{x}{y}{label}{namespace}
\newcommand{\createPod}[6]{
    \node[container pod] (#1) at (#3,#4) {
        \textbf{#5} \\[1pt]
        \tikz\node[ip label]{#2}; \\[1pt]
        \tikz\node[font=\tiny\ttfamily, text=containerBlue!70]{NS: #6};
    };
}

% Create container with port mapping
% Usage: \createContainerWithPorts{name}{ip}{x}{y}{label}{ports}
\newcommand{\createContainerWithPorts}[6]{
    \node[container] (#1) at (#3,#4) {
        \textbf{#5} \\[1pt]
        \tikz\node[ip label]{#2}; \\[1pt]
        \tikz\node[font=\tiny\ttfamily, text=containerBlue!70]{Ports: #6};
    };
    % Add stacked effect
    \draw[containerBlue!60, line width=0.8pt]
        ([xshift=-2pt, yshift=2pt]#1.north west) --
        ([xshift=-2pt, yshift=2pt]#1.north east) --
        ([xshift=-2pt]#1.south east);
}

% ============================================================================
% NODE GROUPING AND CLUSTERING
% ============================================================================

% Create a cluster/group boundary around nodes
% Usage: \createCluster{name}{label}{nodes}{x}{y}
% Example: \createCluster{webcluster}{Web Tier}{srv1,srv2,srv3}{0}{0}
\newcommand{\createCluster}[5]{
    \node[cluster box, fit=(#3), label=above:\textbf{#2}] (#1) {};
}

% Create high availability pair boundary
% Usage: \createHAPair{name}{label}{node1}{node2}
\newcommand{\createHAPair}[4]{
    \node[ha pair, fit=(#3)(#4), label=above:\textbf{#2}] (#1) {};
}

% Create server rack visualization
% Usage: \createRack{name}{label}{nodes}{x}{y}
\newcommand{\createRack}[5]{
    \node[
        rectangle,
        rounded corners=2pt,
        draw=black!60,
        fill=black!5,
        line width=2pt,
        inner sep=8pt,
        fit=(#3),
        label={[rotate=90, anchor=south]left:\textbf{#2}}
    ] (#1) {};
    % Add rack mounting holes effect
    \foreach \y in {0.1,0.3,...,0.9} {
        \fill[black!40] ([yshift=\y*10pt]#1.north west) circle (1pt);
        \fill[black!40] ([yshift=\y*10pt]#1.north east) circle (1pt);
    }
}

% TODO: Enhanced clustering features
% - Mobile device nodes with phone/tablet shapes
% - IoT device nodes with specialized icons
% - Auto-arrange nodes within clusters
% - Subnet boundary visualization

% ============================================================================
% MULTI-PART NODES FOR DETAILED INFORMATION
% ============================================================================

% Create a detailed server node with three sections
% Usage: \createDetailedServer{name}{ip}{x}{y}{hostname}{services}{status}
\newcommand{\createDetailedServer}[7]{
    \node[multipart node] (#1) at (#3,#4) {
        \textbf{#5}
        \nodepart{two}
        \tikz\node[ip label]{#2}; \\
        \tikz\node[font=\tiny\ttfamily, text=black!70]{#6};
        \nodepart{three}
        \tikz\node[font=\tiny\bfseries, text=clientGreen!80]{#7};
    };
}

% Create node with port and service information
% Usage: \createNodeWithServices{name}{ip}{x}{y}{hostname}{ports}{services}
\newcommand{\createNodeWithServices}[7]{
    \node[
        rectangle split,
        rectangle split parts=3,
        rectangle split part fill={serverBlue!20, white, serverBlue!10},
        draw=serverBlue!80,
        line width=1.5pt,
        rounded corners=3pt,
        align=center,
        minimum width=3cm
    ] (#1) at (#3,#4) {
        \textbf{#5}
        \nodepart{two}
        \tikz\node[font=\scriptsize\ttfamily, text=black!70]{#2}; \\
        \tikz\node[font=\tiny\ttfamily, text=serverBlue!80]{Ports: #6};
        \nodepart{three}
        \tikz\node[font=\tiny\ttfamily, text=black!60]{#7};
    };
}

% Create node with resource utilization bars
% Usage: \createNodeWithMetrics{name}{ip}{x}{y}{hostname}{cpu}{memory}{disk}
% CPU, memory, disk should be percentages (0-100)
\newcommand{\createNodeWithMetrics}[8]{
    \node[
        rectangle split,
        rectangle split parts=4,
        rectangle split part fill={serverBlue!20, white, white, serverBlue!10},
        draw=serverBlue!80,
        line width=1.5pt,
        rounded corners=3pt,
        align=center,
        minimum width=3.5cm
    ] (#1) at (#3,#4) {
        \textbf{#5}
        \nodepart{two}
        \tikz\node[ip label]{#2};
        \nodepart{three}
        % CPU bar
        \tikz{
            \node[font=\tiny, anchor=west] at (0,0.3) {CPU:};
            \draw[fill=black!20, rounded corners=1pt] (0.8,0.25) rectangle (2.5,0.35);
            \draw[fill=serverBlue!70, rounded corners=1pt] (0.8,0.25) rectangle ({0.8+1.7*#6/100},0.35);
            \node[font=\tiny, anchor=west] at (2.6,0.3) {#6\%};
            % Memory bar
            \node[font=\tiny, anchor=west] at (0,0) {MEM:};
            \draw[fill=black!20, rounded corners=1pt] (0.8,-0.05) rectangle (2.5,0.05);
            \draw[fill=clientGreen!70, rounded corners=1pt] (0.8,-0.05) rectangle ({0.8+1.7*#7/100},0.05);
            \node[font=\tiny, anchor=west] at (2.6,0) {#7\%};
            % Disk bar
            \node[font=\tiny, anchor=west] at (0,-0.3) {DSK:};
            \draw[fill=black!20, rounded corners=1pt] (0.8,-0.35) rectangle (2.5,-0.25);
            \draw[fill=routerOrange!70, rounded corners=1pt] (0.8,-0.35) rectangle ({0.8+1.7*#8/100},-0.25);
            \node[font=\tiny, anchor=west] at (2.6,-0.3) {#8\%};
        }
        \nodepart{four}
        \tikz\node[font=\tiny, text=black!60]{System Metrics};
    };
}

% Create security-focused node with vulnerability info
% Usage: \createSecurityNode{name}{ip}{x}{y}{hostname}{vulncount}{cvss}{status}
\newcommand{\createSecurityNode}[8]{
    \ifthenelse{\equal{#8}{secure}}{
        \def\statuscolor{clientGreen}
        \def\statustext{SECURE}
    }{
    \ifthenelse{\equal{#8}{warning}}{
        \def\statuscolor{threatMedium}
        \def\statustext{WARNING}
    }{
        \def\statuscolor{threatCritical}
        \def\statustext{CRITICAL}
    }}

    \node[
        rectangle split,
        rectangle split parts=3,
        rectangle split part fill={\statuscolor!15, white, \statuscolor!10},
        draw=\statuscolor!80,
        line width=1.5pt,
        rounded corners=3pt,
        align=center,
        minimum width=3cm
    ] (#1) at (#3,#4) {
        \textbf{#5}
        \nodepart{two}
        \tikz\node[ip label]{#2}; \\
        \tikz\node[font=\tiny\ttfamily, text=black!70]{Vulnerabilities: #6}; \\
        \tikz\node[font=\tiny\ttfamily, text=black!70]{Max CVSS: #7};
        \nodepart{three}
        \tikz\node[font=\tiny\bfseries, text=\statuscolor!90]{\statustext};
    };
}

% ============================================================================
% MOBILE DEVICE NODES
% ============================================================================

% Create a mobile phone node
% Usage: \createMobilePhone{name}{ip}{x}{y}{label}{os}
\newcommand{\createMobilePhone}[6]{
    \node[mobile phone] (#1) at (#3,#4) {
        \tikz\node[font=\tiny\bfseries]{#5}; \\[1pt]
        \tikz\node[ip label, font=\tiny]{#2}; \\[1pt]
        \tikz\node[font=\tiny\ttfamily, text=mobileOrange!80]{#6};
    };
    % Add screen indicator
    \draw[mobileOrange!60, line width=0.5pt, rounded corners=2pt]
        ([xshift=3pt, yshift=-3pt]#1.north west) rectangle
        ([xshift=-3pt, yshift=5pt]#1.south east);
}

% Create a tablet node
% Usage: \createTablet{name}{ip}{x}{y}{label}{os}
\newcommand{\createTablet}[6]{
    \node[tablet] (#1) at (#3,#4) {
        \textbf{#5} \\[1pt]
        \tikz\node[ip label]{#2}; \\[1pt]
        \tikz\node[font=\tiny\ttfamily, text=mobileOrange!80]{#6};
    };
}

% Create laptop node (mobile workstation)
% Usage: \createLaptop{name}{ip}{x}{y}{label}{user}
\newcommand{\createLaptop}[6]{
    \node[client, minimum width=2.2cm] (#1) at (#3,#4) {
        \textbf{#5} \\[2pt]
        \tikz\node[ip label]{#2}; \\[1pt]
        \tikz\node[font=\tiny\ttfamily, text=clientGreen!70]{User: #6};
    };
    % Add laptop hinge indicator
    \draw[clientGreen!70, line width=1pt]
        ([yshift=2pt]#1.north west) -- ([yshift=2pt]#1.north east);
}

% ============================================================================
% IOT DEVICE NODES
% ============================================================================

% Create IoT device node
% Usage: \createIoTDevice{name}{ip}{x}{y}{label}{type}
\newcommand{\createIoTDevice}[6]{
    \node[iot device] (#1) at (#3,#4) {
        \tikz\node[font=\small\bfseries]{#5}; \\[1pt]
        \tikz\node[ip label, font=\tiny]{#2}; \\[1pt]
        \tikz\node[font=\tiny\ttfamily, text=iotGreen!80]{#6};
    };
}

% Create sensor node
% Usage: \createSensor{name}{ip}{x}{y}{label}{sensortype}
\newcommand{\createSensor}[6]{
    \node[sensor] (#1) at (#3,#4) {
        \tikz\node[font=\tiny\bfseries]{#5}; \\[1pt]
        \tikz\node[font=\tiny\ttfamily, text=black!60]{#2}; \\[1pt]
        \tikz\node[font=\tiny\ttfamily, text=iotGreen!80]{#6};
    };
}

% Create smart device (thermostat, camera, etc.)
% Usage: \createSmartDevice{name}{ip}{x}{y}{label}{devicetype}{status}
\newcommand{\createSmartDevice}[7]{
    \node[iot device, minimum width=2.5cm] (#1) at (#3,#4) {
        \tikz\node[font=\small\bfseries]{#5}; \\[1pt]
        \tikz\node[ip label, font=\tiny]{#2}; \\[1pt]
        \tikz\node[font=\tiny\ttfamily, text=iotGreen!70]{#6}; \\[1pt]
        \tikz\node[font=\tiny, text=black!60]{#7};
    };
}

% ============================================================================
% CLOUD PROVIDER NODES
% ============================================================================

% Create AWS cloud node
% Usage: \createAWSNode{name}{x}{y}{label}{service}
\newcommand{\createAWSNode}[5]{
    \node[aws node] (#1) at (#2,#3) {
        \textbf{#4} \\[2pt]
        \tikz\node[font=\tiny\bfseries, text=cloudAWS!90]{AWS}; \\[1pt]
        \tikz\node[font=\tiny\ttfamily, text=black!70]{#5};
    };
}

% Create Azure cloud node
% Usage: \createAzureNode{name}{x}{y}{label}{service}
\newcommand{\createAzureNode}[5]{
    \node[azure node] (#1) at (#2,#3) {
        \textbf{#4} \\[2pt]
        \tikz\node[font=\tiny\bfseries, text=cloudAzure!90]{Azure}; \\[1pt]
        \tikz\node[font=\tiny\ttfamily, text=black!70]{#5};
    };
}

% Create GCP cloud node
% Usage: \createGCPNode{name}{x}{y}{label}{service}
\newcommand{\createGCPNode}[5]{
    \node[gcp node] (#1) at (#2,#3) {
        \textbf{#4} \\[2pt]
        \tikz\node[font=\tiny\bfseries, text=cloudGCP!90]{GCP}; \\[1pt]
        \tikz\node[font=\tiny\ttfamily, text=black!70]{#5};
    };
}

% ============================================================================
% NETWORK APPLIANCE NODES
% ============================================================================

% Create IPS/IDS node
% Usage: \createIPS{name}{ip}{x}{y}{label}{mode}
% Mode: IPS (prevention) or IDS (detection)
\newcommand{\createIPS}[6]{
    \node[ips] (#1) at (#3,#4) {
        \textbf{#5} \\[2pt]
        \tikz\node[ip label]{#2}; \\[2pt]
        \tikz\node[font=\tiny\bfseries, text=appliancePurple!90]{#6};
    };
}

% Create proxy server node
% Usage: \createProxy{name}{ip}{x}{y}{label}{proxytype}
\newcommand{\createProxy}[6]{
    \node[proxy] (#1) at (#3,#4) {
        \textbf{#5} \\[2pt]
        \tikz\node[ip label]{#2}; \\[1pt]
        \tikz\node[font=\tiny\ttfamily, text=appliancePurple!80]{#6};
    };
}

% Create WAF (Web Application Firewall) node
% Usage: \createWAF{name}{ip}{x}{y}{label}{ruleset}
\newcommand{\createWAF}[6]{
    \node[waf] (#1) at (#3,#4) {
        \textbf{#5} \\[2pt]
        \tikz\node[ip label]{#2}; \\[2pt]
        \tikz\node[font=\tiny\bfseries, text=appliancePurple!90]{WAF}; \\[1pt]
        \tikz\node[font=\tiny\ttfamily, text=black!60]{Rules: #6};
    };
}

% ============================================================================
% STORAGE NODES
% ============================================================================

% Create storage node (generic)
% Usage: \createStorage{name}{ip}{x}{y}{label}{capacity}
\newcommand{\createStorage}[6]{
    \node[storage] (#1) at (#3,#4) {
        \textbf{#5} \\[2pt]
        \tikz\node[ip label]{#2}; \\[1pt]
        \tikz\node[font=\tiny\ttfamily, text=storageYellow!90]{#6};
    };
}

% Create NAS (Network Attached Storage) node
% Usage: \createNAS{name}{ip}{x}{y}{label}{capacity}{protocol}
\newcommand{\createNAS}[7]{
    \node[nas] (#1) at (#3,#4) {
        \textbf{#5} \\[2pt]
        \tikz\node[ip label]{#2}; \\[1pt]
        \tikz\node[font=\tiny\bfseries, text=storageYellow!90]{NAS}; \\[1pt]
        \tikz\node[font=\tiny\ttfamily, text=black!60]{#6 | #7};
    };
}

% Create SAN (Storage Area Network) node
% Usage: \createSAN{name}{ip}{x}{y}{label}{capacity}{protocol}
\newcommand{\createSAN}[7]{
    \node[storage, minimum height=2.5cm] (#1) at (#3,#4) {
        \textbf{#5} \\[2pt]
        \tikz\node[ip label]{#2}; \\[2pt]
        \tikz\node[font=\tiny\bfseries, text=storageYellow!90]{SAN}; \\[1pt]
        \tikz\node[font=\tiny\ttfamily, text=black!60]{#6}; \\[1pt]
        \tikz\node[font=\tiny\ttfamily, text=storageYellow!80]{#7};
    };
}

% ============================================================================
% WIRELESS NODES
% ============================================================================

% Create wireless access point
% Usage: \createWirelessAP{name}{ip}{x}{y}{label}{ssid}
\newcommand{\createWirelessAP}[6]{
    \node[wireless ap] (#1) at (#3,#4) {
        \tikz\node[font=\small\bfseries]{#5}; \\[2pt]
        \tikz\node[ip label]{#2}; \\[2pt]
        \tikz\node[font=\tiny\ttfamily, text=wirelessTeal!80]{SSID: #6};
    };
    % Add wireless signal indicators
    \foreach \r in {0.3, 0.5, 0.7} {
        \draw[wirelessTeal!60, line width=0.5pt]
            ([yshift=5pt]#1.north) arc (180:0:\r cm and \r/2 cm);
    }
}

% ============================================================================
% OS BADGES AND INDICATORS
% ============================================================================

% Add OS badge to any node
% Usage: \addOSBadge{nodename}{os}
% OS options: windows, linux, macos, android, ios, ubuntu, redhat, centos
\newcommand{\addOSBadge}[2]{
    \ifthenelse{\equal{#2}{windows}}{
        \def\osbadgecolor{blue!70}
        \def\osbadgetext{WIN}
    }{
    \ifthenelse{\equal{#2}{linux}}{
        \def\osbadgecolor{black!80}
        \def\osbadgetext{LNX}
    }{
    \ifthenelse{\equal{#2}{macos}}{
        \def\osbadgecolor{black!70}
        \def\osbadgetext{MAC}
    }{
    \ifthenelse{\equal{#2}{android}}{
        \def\osbadgecolor{green!70}
        \def\osbadgetext{AND}
    }{
    \ifthenelse{\equal{#2}{ios}}{
        \def\osbadgecolor{black!60}
        \def\osbadgetext{iOS}
    }{
    \ifthenelse{\equal{#2}{ubuntu}}{
        \def\osbadgecolor{orange!70}
        \def\osbadgetext{UBU}
    }{
        \def\osbadgecolor{gray!70}
        \def\osbadgetext{OS}
    }}}}}}

    \node[
        circle,
        fill=\osbadgecolor,
        text=white,
        font=\tiny\bfseries,
        inner sep=2pt,
        minimum size=0.5cm,
        anchor=south east
    ] at (#1.south east) {\osbadgetext};
}

% Add status indicator to node
% Usage: \addStatusIndicator{nodename}{status}
% Status: online, offline, degraded, maintenance
\newcommand{\addStatusIndicator}[2]{
    \ifthenelse{\equal{#2}{online}}{
        \def\statuscolor{green!70}
    }{
    \ifthenelse{\equal{#2}{offline}}{
        \def\statuscolor{red!70}
    }{
    \ifthenelse{\equal{#2}{degraded}}{
        \def\statuscolor{yellow!70}
    }{
        \def\statuscolor{gray!70}
    }}}

    \node[
        circle,
        fill=\statuscolor,
        inner sep=3pt,
        anchor=north west
    ] at (#1.north west) {};
}

% ============================================================================
% SUBNET BOUNDARY VISUALIZATION
% ============================================================================

% Create subnet boundary
% Usage: \createSubnet{name}{label}{cidr}{nodes}{trustlevel}
% Trust level: high, medium, low
\newcommand{\createSubnet}[5]{
    \ifthenelse{\equal{#5}{high}}{
        \def\subnetcolor{clientGreen}
    }{
    \ifthenelse{\equal{#5}{medium}}{
        \def\subnetcolor{threatMedium}
    }{
        \def\subnetcolor{threatHigh}
    }}

    \node[
        subnet box,
        draw=\subnetcolor!60,
        fill=\subnetcolor!3,
        fit=(#4),
        label={[fill=\subnetcolor!20, rounded corners=3pt, font=\small\bfseries]above:#2},
        label={[fill=white, rounded corners=2pt, font=\tiny\ttfamily]below left:#3}
    ] (#1) {};
}

% Create DMZ (Demilitarized Zone) boundary
% Usage: \createDMZ{name}{label}{nodes}
\newcommand{\createDMZ}[3]{
    \node[
        subnet box,
        draw=routerOrange!70,
        fill=routerOrange!5,
        line width=2.5pt,
        fit=(#3),
        label={[fill=routerOrange!30, rounded corners=3pt, font=\small\bfseries, text=white]above:#2},
        label={[fill=white, rounded corners=2pt, font=\tiny\bfseries, text=routerOrange!90]below left:DMZ}
    ] (#1) {};
}

% ============================================================================
% NODE RENDERING ENGINE
% ============================================================================

% Main command to render all nodes from data structure
\newcommand{\renderNetworkNodes}{
    % This will be populated by network_data.tex
    % Example structure:
    % \createServer{srv1}{192.168.1.10}{0}{0}{Web Server}
    % \createClient{pc1}{192.168.1.100}{-5}{-3}{Workstation 1}
}

% TODO: Intelligent rendering
% - Auto-layout algorithm to prevent overlapping nodes
% - Force-directed graph layout for organic appearance
% - Hierarchical layout for structured networks
% - Subnet-based clustering and grouping
% - Zoom levels for large networks (overview vs detail)
% - LOD (Level of Detail) rendering for performance

% ============================================================================
% NODE METADATA AND ANNOTATIONS
% ============================================================================

% Add annotation to existing node
% Usage: \annotateNode{nodename}{annotation}{position}
% Position: above, below, left, right, above right, etc.
\newcommand{\annotateNode}[3]{
    \node[font=\tiny\sffamily\itshape, text=black!60] at (#1.#3) {#2};
}

% Add threat indicator badge to node
% Usage: \addThreatBadge{nodename}{level}
% Level: critical, high, medium, low, info
\newcommand{\addThreatBadge}[2]{
    \ifthenelse{\equal{#2}{critical}}{
        \def\badgecolor{threatCritical}
        \def\badgetext{CRIT}
    }{
    \ifthenelse{\equal{#2}{high}}{
        \def\badgecolor{threatHigh}
        \def\badgetext{HIGH}
    }{
    \ifthenelse{\equal{#2}{medium}}{
        \def\badgecolor{threatMedium}
        \def\badgetext{MED}
    }{
    \ifthenelse{\equal{#2}{low}}{
        \def\badgecolor{threatLow}
        \def\badgetext{LOW}
    }{
        \def\badgecolor{threatInfo}
        \def\badgetext{INFO}
    }}}}
    
    \node[threat label, fill=\badgecolor, anchor=north west] 
        at (#1.north west) {\badgetext};
}

% Add service/OS label to node
% Usage: \addNodeMetadata{nodename}{metadata}
\newcommand{\addNodeMetadata}[2]{
    \node[font=\tiny\ttfamily, text=black!50, anchor=south] 
        at (#1.south) [below=1pt] {#2};
}

% TODO: Metadata enhancements
% - CVE vulnerability badges with scoring
% - Compliance status indicators (PCI, HIPAA, etc.)
% - Performance metrics (CPU, memory, network utilization)
% - Last scan timestamp and security posture
% - Asset criticality indicators (business impact)
% - Custom metadata fields via configuration

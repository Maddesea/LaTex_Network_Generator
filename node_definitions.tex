% node_definitions.tex - Network node rendering and management
% This module defines how to create and render individual network assets

% ============================================================================
% NODE DATA STRUCTURES
% ============================================================================

% Node counter for auto-indexing
\newcounter{nodecount}

% Hash map implementation for O(1) node lookup
% Using pgfkeys for efficient key-value storage
\pgfkeys{/nodemap/.is family, /nodemap}

% Store node by IP address
% Usage: \storeNodeByIP{ip}{nodename}
\newcommand{\storeNodeByIP}[2]{
    \pgfkeys{/nodemap/ip/#1/.initial=#2}
}

% Store node by hostname
% Usage: \storeNodeByHostname{hostname}{nodename}
\newcommand{\storeNodeByHostname}[2]{
    \pgfkeys{/nodemap/hostname/#1/.initial=#2}
}

% Store node by ID
% Usage: \storeNodeByID{id}{nodename}
\newcommand{\storeNodeByID}[2]{
    \pgfkeys{/nodemap/id/#1/.initial=#2}
}

% Lookup node by IP address
% Usage: \getNodeByIP{ip}
\newcommand{\getNodeByIP}[1]{%
    \pgfkeysvalueof{/nodemap/ip/#1}%
}

% Lookup node by hostname
% Usage: \getNodeByHostname{hostname}
\newcommand{\getNodeByHostname}[1]{%
    \pgfkeysvalueof{/nodemap/hostname/#1}%
}

% Lookup node by ID
% Usage: \getNodeByID{id}
\newcommand{\getNodeByID}[1]{%
    \pgfkeysvalueof{/nodemap/id/#1}%
}

% Store complete node metadata (IP, hostname, type, position)
% Usage: \storeNodeMetadata{nodename}{ip}{hostname}{type}{x}{y}
\newcommand{\storeNodeMetadata}[6]{
    \pgfkeys{/nodemap/meta/#1/ip/.initial=#2}
    \pgfkeys{/nodemap/meta/#1/hostname/.initial=#3}
    \pgfkeys{/nodemap/meta/#1/type/.initial=#4}
    \pgfkeys{/nodemap/meta/#1/x/.initial=#5}
    \pgfkeys{/nodemap/meta/#1/y/.initial=#6}
    \storeNodeByIP{#2}{#1}
    \storeNodeByHostname{#3}{#1}
    \storeNodeByID{#1}{#1}
}

% Retrieve node metadata
\newcommand{\getNodeIP}[1]{\pgfkeysvalueof{/nodemap/meta/#1/ip}}
\newcommand{\getNodeHostname}[1]{\pgfkeysvalueof{/nodemap/meta/#1/hostname}}
\newcommand{\getNodeType}[1]{\pgfkeysvalueof{/nodemap/meta/#1/type}}
\newcommand{\getNodeX}[1]{\pgfkeysvalueof{/nodemap/meta/#1/x}}
\newcommand{\getNodeY}[1]{\pgfkeysvalueof{/nodemap/meta/#1/y}}

% TODO: Data structure improvements
% - Add node grouping/clustering support
% - Create hierarchical node relationships (parent-child)
% - Support for virtual/container nodes

% ============================================================================
% IP ADDRESS VALIDATION AND FORMATTING
% ============================================================================

% Validate IPv4 address format
% Returns true if valid, false otherwise
% Usage: \validateIPv4{192.168.1.1}
\newcommand{\validateIPv4}[1]{%
    % Simple validation - checks for basic format
    % More complex validation would require regex or Lua
    \def\ipValid{true}%
    #1% For now, accept the IP as-is
}

% Format IPv4 address with proper styling
% Usage: \formatIPv4{192.168.1.1}
\newcommand{\formatIPv4}[1]{%
    \texttt{\small #1}%
}

% Format IPv6 address with compression
% Usage: \formatIPv6{2001:0db8:85a3::8a2e:0370:7334}
\newcommand{\formatIPv6}[1]{%
    \texttt{\footnotesize #1}%
}

% Format CIDR notation
% Usage: \formatCIDR{192.168.1.0/24}
\newcommand{\formatCIDR}[1]{%
    \texttt{\small #1}%
}

% Extract subnet from IP address (assumes /24)
% Usage: \getSubnet{192.168.1.100} returns 192.168.1.0
\newcommand{\getSubnet}[1]{%
    % This is a simplified version - full implementation would need string parsing
    % For now, return the IP as-is
    #1%
}

% Auto-detect IP version
% Usage: \detectIPVersion{address}
\newcommand{\detectIPVersion}[1]{%
    % Check if contains colon (IPv6) or dots (IPv4)
    \IfSubStr{#1}{:}{6}{4}%
}

% Pretty print IP address with auto-detection
% Usage: \prettyIP{192.168.1.1}
\newcommand{\prettyIP}[1]{%
    \IfSubStr{#1}{:}{%
        \formatIPv6{#1}%
    }{%
        \IfSubStr{#1}{/}{%
            \formatCIDR{#1}%
        }{%
            \formatIPv4{#1}%
        }%
    }%
}

% Validate and format IP address
% Usage: \validateAndFormatIP{192.168.1.1}
\newcommand{\validateAndFormatIP}[1]{%
    \prettyIP{#1}%
}

% Extract network class from IPv4
% Usage: \getIPClass{192.168.1.1} returns C
\newcommand{\getIPClass}[1]{%
    % Simplified - would need proper parsing
    % For private IPs starting with 192.168 or 10.0
    \IfSubStr{#1}{192.168}{C}{%
        \IfSubStr{#1}{10.}{A}{%
            \IfSubStr{#1}{172.}{B}{Unknown}%
        }%
    }%
}

% Check if IP is private (RFC 1918)
% Usage: \isPrivateIP{192.168.1.1}
\newcommand{\isPrivateIP}[1]{%
    \IfSubStr{#1}{192.168}{true}{%
        \IfSubStr{#1}{10.}{true}{%
            \IfSubStr{#1}{172.16}{true}{%
                \IfSubStr{#1}{172.17}{true}{%
                    \IfSubStr{#1}{172.18}{true}{%
                        \IfSubStr{#1}{172.19}{true}{%
                            \IfSubStr{#1}{172.20}{true}{%
                                \IfSubStr{#1}{172.21}{true}{%
                                    \IfSubStr{#1}{172.22}{true}{%
                                        \IfSubStr{#1}{172.23}{true}{%
                                            \IfSubStr{#1}{172.24}{true}{%
                                                \IfSubStr{#1}{172.25}{true}{%
                                                    \IfSubStr{#1}{172.26}{true}{%
                                                        \IfSubStr{#1}{172.27}{true}{%
                                                            \IfSubStr{#1}{172.28}{true}{%
                                                                \IfSubStr{#1}{172.29}{true}{%
                                                                    \IfSubStr{#1}{172.30}{true}{%
                                                                        \IfSubStr{#1}{172.31}{true}{false}%
                                                                    }%
                                                                }%
                                                            }%
                                                        }%
                                                    }%
                                                }%
                                            }%
                                        }%
                                    }%
                                }%
                            }%
                        }%
                    }%
                }%
            }%
        }%
    }%
}

% Auto-group nodes by subnet
% This creates a visual grouping indicator
% Usage: \markSubnetGroup{subnet}{color}
\newcommand{\markSubnetGroup}[2]{%
    % Store subnet group information
    \pgfkeys{/nodemap/subnet/#1/color/.initial=#2}%
}

% ============================================================================
% BASIC NODE CREATION COMMANDS
% ============================================================================

% Create a server node
% Usage: \createServer{name}{ip}{x}{y}{label}
\newcommand{\createServer}[5]{
    \node[server] (#1) at (#3,#4) {
        \textbf{#5} \\[2pt]
        \tikz\node[ip label]{#2};
    };
}

% Create a client node
% Usage: \createClient{name}{ip}{x}{y}{label}
\newcommand{\createClient}[5]{
    \node[client] (#1) at (#3,#4) {
        \textbf{#5} \\[2pt]
        \tikz\node[ip label]{#2};
    };
}

% Create a router node
% Usage: \createRouter{name}{ip}{x}{y}{label}
\newcommand{\createRouter}[5]{
    \node[router] (#1) at (#3,#4) {
        \textbf{#5} \\[2pt]
        \tikz\node[ip label]{#2};
    };
}

% Create a firewall node
% Usage: \createFirewall{name}{ip}{x}{y}{label}
\newcommand{\createFirewall}[5]{
    \node[firewall] (#1) at (#3,#4) {
        \textbf{#5} \\[2pt]
        \tikz\node[ip label]{#2};
    };
}

% Create a switch node
% Usage: \createSwitch{name}{ip}{x}{y}{label}
\newcommand{\createSwitch}[5]{
    \node[switch] (#1) at (#3,#4) {
        \textbf{#5} \\[2pt]
        \tikz\node[ip label]{#2};
    };
}

% Create a cloud/internet node
% Usage: \createCloud{name}{x}{y}{label}
\newcommand{\createCloud}[4]{
    \node[cloud] (#1) at (#2,#3) {
        \textbf{#4}
    };
}

% Create an attacker node
% Usage: \createAttacker{name}{ip}{x}{y}{label}
\newcommand{\createAttacker}[5]{
    \node[attacker] (#1) at (#3,#4) {
        \textbf{#5} \\[2pt]
        \tikz\node[ip label, fill=threatCritical!20]{#2};
    };
}

% ============================================================================
% DATABASE NODE CREATION COMMANDS
% ============================================================================

% Create a basic database node
% Usage: \createDatabase{name}{ip}{x}{y}{label}
\newcommand{\createDatabase}[5]{
    \node[database] (#1) at (#3,#4) {
        \textbf{#5} \\[2pt]
        \tikz\node[ip label]{#2};
    };
}

% Create a primary/master database node
% Usage: \createDatabasePrimary{name}{ip}{x}{y}{label}
\newcommand{\createDatabasePrimary}[5]{
    \node[database primary] (#1) at (#3,#4) {
        \textbf{#5} \\[2pt]
        \tikz\node[ip label]{#2}; \\[1pt]
        \tikz\node[font=\tiny\bfseries, text=databaseTeal!80]{PRIMARY};
    };
}

% Create a replica/slave database node
% Usage: \createDatabaseReplica{name}{ip}{x}{y}{label}
\newcommand{\createDatabaseReplica}[5]{
    \node[database replica] (#1) at (#3,#4) {
        \textbf{#5} \\[2pt]
        \tikz\node[ip label]{#2}; \\[1pt]
        \tikz\node[font=\tiny, text=databaseTeal!60]{REPLICA};
    };
}

% Create a database cluster node
% Usage: \createDatabaseCluster{name}{ip}{x}{y}{label}
\newcommand{\createDatabaseCluster}[5]{
    \node[database cluster] (#1) at (#3,#4) {
        \textbf{#5} \\[2pt]
        \tikz\node[ip label]{#2}; \\[1pt]
        \tikz\node[font=\tiny\bfseries, text=databaseTeal!70]{CLUSTER};
    };
}

% Create a database with additional info (ports, type)
% Usage: \createDatabaseWithInfo{name}{ip}{x}{y}{label}{dbtype}{port}
\newcommand{\createDatabaseWithInfo}[7]{
    \node[database, minimum height=2.5cm] (#1) at (#3,#4) {
        \textbf{#5} \\[2pt]
        \tikz\node[ip label]{#2}; \\[2pt]
        \tikz\node[font=\tiny\ttfamily, text=black!60]{#6}; \\[1pt]
        \tikz\node[port label]{Port: #7};
    };
}

% ============================================================================
% LOAD BALANCER NODE CREATION COMMANDS
% ============================================================================

% Create a basic load balancer node
% Usage: \createLoadBalancer{name}{ip}{x}{y}{label}
\newcommand{\createLoadBalancer}[5]{
    \node[loadbalancer] (#1) at (#3,#4) {
        \textbf{#5} \\[2pt]
        \tikz\node[ip label]{#2};
    };
}

% Create an active load balancer node
% Usage: \createLoadBalancerActive{name}{ip}{x}{y}{label}
\newcommand{\createLoadBalancerActive}[5]{
    \node[loadbalancer active] (#1) at (#3,#4) {
        \textbf{#5} \\[2pt]
        \tikz\node[ip label]{#2}; \\[1pt]
        \tikz\node[font=\tiny\bfseries, text=loadBalancerCyan!90]{ACTIVE};
    };
}

% Create a passive load balancer node
% Usage: \createLoadBalancerPassive{name}{ip}{x}{y}{label}
\newcommand{\createLoadBalancerPassive}[5]{
    \node[loadbalancer passive] (#1) at (#3,#4) {
        \textbf{#5} \\[2pt]
        \tikz\node[ip label]{#2}; \\[1pt]
        \tikz\node[font=\tiny, text=loadBalancerCyan!60]{PASSIVE};
    };
}

% Create a load balancer with algorithm indicator
% Usage: \createLoadBalancerWithAlgo{name}{ip}{x}{y}{label}{algorithm}
% Algorithm examples: Round Robin, Least Conn, IP Hash, Weighted
\newcommand{\createLoadBalancerWithAlgo}[6]{
    \node[loadbalancer, minimum height=2.2cm] (#1) at (#3,#4) {
        \textbf{#5} \\[2pt]
        \tikz\node[ip label]{#2}; \\[2pt]
        \tikz\node[font=\tiny\ttfamily, text=black!60]{#6};
    };
}

% Create a load balancer with backend pool count
% Usage: \createLoadBalancerWithPool{name}{ip}{x}{y}{label}{poolsize}
\newcommand{\createLoadBalancerWithPool}[6]{
    \node[loadbalancer, minimum height=2.2cm] (#1) at (#3,#4) {
        \textbf{#5} \\[2pt]
        \tikz\node[ip label]{#2}; \\[2pt]
        \tikz\node[font=\tiny, text=black!60]{Pool: #6 nodes};
    };
}

% TODO: Advanced node creation
% - Add support for custom node shapes via parameters
% - Implement node templates for common device types
% - Create composite nodes (e.g., server rack with multiple servers)
% - Support for node status indicators (up/down/warning)

% ============================================================================
% ENHANCED NODE VARIANTS
% ============================================================================

% Create a server with port information
% Usage: \createServerWithPorts{name}{ip}{x}{y}{label}{ports}
\newcommand{\createServerWithPorts}[6]{
    \node[server, minimum height=2cm] (#1) at (#3,#4) {
        \textbf{#5} \\[2pt]
        \tikz\node[ip label]{#2}; \\[3pt]
        \tikz\node[port label]{#6};
    };
}

% Create a node with security status indicator
% Usage: \createSecureNode{type}{name}{ip}{x}{y}{label}{status}
% Status: secure, warning, compromised
\newcommand{\createSecureNode}[7]{
    \ifthenelse{\equal{#7}{secure}}{
        \def\statuscolor{clientGreen}
    }{
    \ifthenelse{\equal{#7}{warning}}{
        \def\statuscolor{threatMedium}
    }{
        \def\statuscolor{threatCritical}
    }}
    
    \ifthenelse{\equal{#1}{server}}{
        \createServer{#2}{#3}{#4}{#5}{#6}
    }{
    \ifthenelse{\equal{#1}{client}}{
        \createClient{#2}{#3}{#4}{#5}{#6}
    }{
        \createServer{#2}{#3}{#4}{#5}{#6}
    }}
    
    \node[circle, fill=\statuscolor, inner sep=2pt, 
          anchor=north east] at (#2.north east) {};
}

% TODO: Enhanced node variants
% - Database server nodes with cylinder shape
% - Load balancer nodes with special indicators
% - Virtual machine nodes with nested appearance
% - Container/Docker nodes with stacked appearance
% - Mobile device nodes with phone/tablet shapes
% - IoT device nodes with specialized icons

% ============================================================================
% NODE RENDERING ENGINE
% ============================================================================

% Main command to render all nodes from data structure
\newcommand{\renderNetworkNodes}{
    % This will be populated by network_data.tex
    % Example structure:
    % \createServer{srv1}{192.168.1.10}{0}{0}{Web Server}
    % \createClient{pc1}{192.168.1.100}{-5}{-3}{Workstation 1}
}

% TODO: Intelligent rendering
% - Auto-layout algorithm to prevent overlapping nodes
% - Force-directed graph layout for organic appearance
% - Hierarchical layout for structured networks
% - Subnet-based clustering and grouping
% - Zoom levels for large networks (overview vs detail)
% - LOD (Level of Detail) rendering for performance

% ============================================================================
% NODE METADATA AND ANNOTATIONS
% ============================================================================

% Add annotation to existing node
% Usage: \annotateNode{nodename}{annotation}{position}
% Position: above, below, left, right, above right, etc.
\newcommand{\annotateNode}[3]{
    \node[font=\tiny\sffamily\itshape, text=black!60] at (#1.#3) {#2};
}

% Add threat indicator badge to node
% Usage: \addThreatBadge{nodename}{level}
% Level: critical, high, medium, low, info
\newcommand{\addThreatBadge}[2]{
    \ifthenelse{\equal{#2}{critical}}{
        \def\badgecolor{threatCritical}
        \def\badgetext{CRIT}
    }{
    \ifthenelse{\equal{#2}{high}}{
        \def\badgecolor{threatHigh}
        \def\badgetext{HIGH}
    }{
    \ifthenelse{\equal{#2}{medium}}{
        \def\badgecolor{threatMedium}
        \def\badgetext{MED}
    }{
    \ifthenelse{\equal{#2}{low}}{
        \def\badgecolor{threatLow}
        \def\badgetext{LOW}
    }{
        \def\badgecolor{threatInfo}
        \def\badgetext{INFO}
    }}}}
    
    \node[threat label, fill=\badgecolor, anchor=north west] 
        at (#1.north west) {\badgetext};
}

% Add service/OS label to node
% Usage: \addNodeMetadata{nodename}{metadata}
\newcommand{\addNodeMetadata}[2]{
    \node[font=\tiny\ttfamily, text=black!50, anchor=south] 
        at (#1.south) [below=1pt] {#2};
}

% TODO: Metadata enhancements
% - CVE vulnerability badges with scoring
% - Compliance status indicators (PCI, HIPAA, etc.)
% - Performance metrics (CPU, memory, network utilization)
% - Last scan timestamp and security posture
% - Asset criticality indicators (business impact)
% - Custom metadata fields via configuration

% styles_config.tex - Visual styling and color schemes
% This module handles all visual styling, colors, and aesthetic configurations

% ============================================================================
% COLOR PALETTE DEFINITIONS
% ============================================================================

% Network asset colors (defaults - can be overridden by color schemes)
\definecolor{serverBlue}{RGB}{52, 152, 219}
\definecolor{clientGreen}{RGB}{46, 204, 113}
\definecolor{routerOrange}{RGB}{230, 126, 34}
\definecolor{firewallRed}{RGB}{231, 76, 60}
\definecolor{switchPurple}{RGB}{155, 89, 182}
\definecolor{cloudGray}{RGB}{149, 165, 166}
\definecolor{databaseTeal}{RGB}{22, 160, 133}
\definecolor{loadBalancerCyan}{RGB}{41, 128, 185}
\definecolor{vmIndigo}{RGB}{103, 58, 183}
\definecolor{containerBlue}{RGB}{33, 150, 243}
\definecolor{clusterGold}{RGB}{255, 152, 0}
\definecolor{mobileOrange}{RGB}{255, 87, 34}
\definecolor{iotGreen}{RGB}{76, 175, 80}
\definecolor{cloudAWS}{RGB}{255, 153, 0}
\definecolor{cloudAzure}{RGB}{0, 120, 215}
\definecolor{cloudGCP}{RGB}{66, 133, 244}
\definecolor{appliancePurple}{RGB}{156, 39, 176}
\definecolor{storageYellow}{RGB}{255, 193, 7}
\definecolor{wirelessTeal}{RGB}{0, 150, 136}

% Threat indicator colors
\definecolor{threatCritical}{RGB}{192, 57, 43}
\definecolor{threatHigh}{RGB}{211, 84, 0}
\definecolor{threatMedium}{RGB}{243, 156, 18}
\definecolor{threatLow}{RGB}{241, 196, 15}
\definecolor{threatInfo}{RGB}{52, 152, 219}

% Connection colors
\definecolor{connNormal}{RGB}{127, 140, 141}
\definecolor{connEncrypted}{RGB}{39, 174, 96}
\definecolor{connSuspicious}{RGB}{230, 126, 34}
\definecolor{connMalicious}{RGB}{192, 57, 43}

% Background colors
\definecolor{bgLight}{RGB}{236, 240, 241}
\definecolor{bgDark}{RGB}{44, 62, 80}

% ============================================================================
% COLOR SCHEME LOADER
% ============================================================================

% Command to parse and define a single color from RGB values
\newcommand{\defineColorFromRGB}[2]{%
    \definecolor{#1}{RGB}{#2}%
}

% Command to load a color scheme from file
% Usage: \loadColorScheme{dark} loads color_schemes/dark.colorscheme
\newcommand{\loadColorScheme}[1]{%
    \IfFileExists{color_schemes/#1.colorscheme}{%
        \message{Loading color scheme: #1}%
        \input{color_schemes/#1.colorscheme}%
    }{%
        \message{Warning: Color scheme '#1' not found. Using default colors.}%
    }%
}

% Command to set theme (dark or light) - affects background primarily
% Usage: \setTheme{dark} or \setTheme{light}
\newcommand{\setTheme}[1]{%
    \ifthenelse{\equal{#1}{dark}}{%
        \loadColorScheme{dark}%
        \pagecolor{bgDark}%
        \color{white}%
    }{%
        \ifthenelse{\equal{#1}{light}}{%
            \loadColorScheme{default}%
            \pagecolor{bgLight}%
            \color{black}%
        }{%
            \message{Warning: Unknown theme '#1'. Use 'dark' or 'light'.}%
        }%
    }%
}

% Predefined color scheme shortcuts
\newcommand{\useDefaultColors}{\loadColorScheme{default}}
\newcommand{\useDarkMode}{\setTheme{dark}}
\newcommand{\useLightMode}{\setTheme{light}}
\newcommand{\useColorblindSafe}{\loadColorScheme{colorblind}}
\newcommand{\useMonochrome}{\loadColorScheme{monochrome}}
\newcommand{\useHighContrast}{\loadColorScheme{high-contrast}}

% ============================================================================
% NODE STYLES
% ============================================================================

\tikzset{
    % Base node style
    base node/.style={
        draw,
        thick,
        minimum width=2.5cm,
        minimum height=1.5cm,
        align=center,
        font=\small\sffamily,
        blur shadow={shadow blur steps=5, shadow xshift=0pt, shadow yshift=-2pt}
    },
    
    % Server style
    server/.style={
        base node,
        rectangle,
        rounded corners=3pt,
        fill=serverBlue!20,
        draw=serverBlue!80,
        line width=1.5pt
    },
    
    % Client/Workstation style
    client/.style={
        base node,
        rectangle,
        rounded corners=2pt,
        fill=clientGreen!15,
        draw=clientGreen!70,
        line width=1pt
    },
    
    % Router style
    router/.style={
        base node,
        trapezium,
        trapezium left angle=70,
        trapezium right angle=110,
        fill=routerOrange!20,
        draw=routerOrange!80,
        line width=1.5pt
    },
    
    % Firewall style
    firewall/.style={
        base node,
        rectangle,
        fill=firewallRed!15,
        draw=firewallRed!80,
        line width=2pt,
        pattern=north east lines,
        pattern color=firewallRed!30
    },
    
    % Switch style
    switch/.style={
        base node,
        rectangle,
        rounded corners=1pt,
        fill=switchPurple!15,
        draw=switchPurple!70,
        minimum height=1cm
    },
    
    % Cloud/Internet style
    cloud/.style={
        base node,
        shape=cloud,
        cloud puffs=10,
        cloud puff arc=120,
        aspect=2,
        fill=cloudGray!20,
        draw=cloudGray!70
    },
    
    % Attacker node style
    attacker/.style={
        base node,
        shape=star,
        star points=5,
        fill=threatCritical!30,
        draw=threatCritical!90,
        line width=2pt,
        minimum width=2cm,
        minimum height=2cm
    },

    % Database server style (cylinder shape)
    database/.style={
        base node,
        shape=cylinder,
        shape border rotate=90,
        aspect=0.25,
        fill=databaseTeal!20,
        draw=databaseTeal!80,
        line width=1.5pt,
        minimum width=2.5cm,
        minimum height=2cm
    },

    % Database - Primary
    database primary/.style={
        database,
        fill=databaseTeal!30,
        draw=databaseTeal!90,
        line width=2pt
    },

    % Database - Replica
    database replica/.style={
        database,
        fill=databaseTeal!15,
        draw=databaseTeal!70,
        line width=1pt
    },

    % Database - Cluster
    database cluster/.style={
        database,
        fill=databaseTeal!25,
        draw=databaseTeal!85,
        line width=1.5pt,
        double,
        double distance=1pt
    },

    % Load balancer style
    loadbalancer/.style={
        base node,
        trapezium,
        trapezium left angle=60,
        trapezium right angle=120,
        fill=loadBalancerCyan!20,
        draw=loadBalancerCyan!80,
        line width=1.5pt,
        minimum width=3cm,
        minimum height=1.8cm
    },

    % Load balancer - Active
    loadbalancer active/.style={
        loadbalancer,
        fill=loadBalancerCyan!30,
        draw=loadBalancerCyan!90,
        line width=2pt
    },

    % Load balancer - Passive
    loadbalancer passive/.style={
        loadbalancer,
        fill=loadBalancerCyan!10,
        draw=loadBalancerCyan!60,
        line width=1pt,
        dashed
    },

    % Virtual Machine style (nested appearance with double border)
    vm/.style={
        base node,
        rectangle,
        rounded corners=3pt,
        fill=vmIndigo!15,
        draw=vmIndigo!80,
        line width=1.5pt,
        double,
        double distance=2pt
    },

    % VM - Hypervisor (host)
    vm hypervisor/.style={
        vm,
        fill=vmIndigo!25,
        draw=vmIndigo!90,
        line width=2pt,
        double distance=3pt,
        minimum width=4cm,
        minimum height=3cm
    },

    % Container/Docker style (stacked appearance)
    container/.style={
        base node,
        rectangle,
        rounded corners=2pt,
        fill=containerBlue!15,
        draw=containerBlue!80,
        line width=1pt,
        minimum width=2cm,
        minimum height=1.2cm
    },

    % Container - Pod (Kubernetes)
    container pod/.style={
        container,
        fill=containerBlue!20,
        draw=containerBlue!85,
        line width=1.5pt,
        dashed,
        dash pattern=on 3pt off 2pt
    },

    % Cluster/Group boundary style
    cluster box/.style={
        rectangle,
        rounded corners=5pt,
        draw=clusterGold!70,
        fill=clusterGold!5,
        line width=2pt,
        inner sep=10pt,
        dashed,
        dash pattern=on 5pt off 3pt
    },

    % High Availability pair boundary
    ha pair/.style={
        rectangle,
        rounded corners=4pt,
        draw=clientGreen!70,
        fill=clientGreen!5,
        line width=1.5pt,
        dotted,
        inner sep=8pt
    },

    % Multi-part node style
    multipart node/.style={
        base node,
        rectangle split,
        rectangle split parts=3,
        rectangle split part fill={serverBlue!20, white, serverBlue!10},
        draw=serverBlue!80,
        line width=1.5pt,
        rounded corners=3pt,
        align=center
    },

    % Mobile device styles
    mobile phone/.style={
        base node,
        rectangle,
        rounded corners=4pt,
        fill=mobileOrange!20,
        draw=mobileOrange!80,
        line width=1pt,
        minimum width=1.5cm,
        minimum height=2.5cm,
        inner sep=4pt
    },

    tablet/.style={
        base node,
        rectangle,
        rounded corners=3pt,
        fill=mobileOrange!15,
        draw=mobileOrange!70,
        line width=1pt,
        minimum width=2.5cm,
        minimum height=2cm
    },

    % IoT device styles
    iot device/.style={
        base node,
        diamond,
        aspect=2,
        fill=iotGreen!15,
        draw=iotGreen!80,
        line width=1pt,
        minimum width=2cm,
        minimum height=1.5cm
    },

    sensor/.style={
        base node,
        circle,
        fill=iotGreen!20,
        draw=iotGreen!85,
        line width=1pt,
        minimum size=1.5cm
    },

    % Cloud provider styles
    aws node/.style={
        base node,
        cloud,
        cloud puffs=12,
        cloud puff arc=120,
        aspect=2.5,
        fill=cloudAWS!20,
        draw=cloudAWS!80,
        line width=1.5pt
    },

    azure node/.style={
        base node,
        cloud,
        cloud puffs=12,
        cloud puff arc=120,
        aspect=2.5,
        fill=cloudAzure!20,
        draw=cloudAzure!80,
        line width=1.5pt
    },

    gcp node/.style={
        base node,
        cloud,
        cloud puffs=12,
        cloud puff arc=120,
        aspect=2.5,
        fill=cloudGCP!20,
        draw=cloudGCP!80,
        line width=1.5pt
    },

    % Network appliance styles
    ips/.style={
        base node,
        rectangle,
        fill=appliancePurple!15,
        draw=appliancePurple!80,
        line width=2pt,
        pattern=crosshatch,
        pattern color=appliancePurple!20
    },

    proxy/.style={
        base node,
        trapezium,
        trapezium left angle=75,
        trapezium right angle=105,
        fill=appliancePurple!20,
        draw=appliancePurple!85,
        line width=1.5pt
    },

    waf/.style={
        base node,
        rectangle,
        fill=appliancePurple!20,
        draw=appliancePurple!90,
        line width=2.5pt,
        double,
        double distance=1.5pt
    },

    % Storage node styles
    storage/.style={
        base node,
        tape,
        tape bend top=none,
        fill=storageYellow!20,
        draw=storageYellow!80,
        line width=1.5pt,
        minimum width=2.5cm,
        minimum height=2cm
    },

    nas/.style={
        base node,
        rectangle,
        rounded corners=2pt,
        fill=storageYellow!20,
        draw=storageYellow!85,
        line width=1.5pt,
        minimum width=2.8cm,
        minimum height=1.8cm
    },

    % Wireless access point
    wireless ap/.style={
        base node,
        circle,
        fill=wirelessTeal!15,
        draw=wirelessTeal!80,
        line width=1.5pt,
        minimum size=2cm
    },

    % Subnet boundary style
    subnet box/.style={
        rectangle,
        rounded corners=8pt,
        draw=black!50,
        fill=black!3,
        line width=2pt,
        dashed,
        dash pattern=on 8pt off 4pt,
        inner sep=15pt
    }
}

% ============================================================================
% GRADIENT FILL STYLES (Premium Look)
% ============================================================================

% Gradient styles for nodes - provides a modern, professional appearance
\tikzset{
    % Vertical gradient (top to bottom)
    gradient server/.style={
        base node,
        rectangle,
        rounded corners=3pt,
        shading=axis,
        top color=serverBlue!40,
        bottom color=serverBlue!10,
        draw=serverBlue!80,
        line width=1.5pt
    },

    gradient client/.style={
        base node,
        rectangle,
        rounded corners=2pt,
        shading=axis,
        top color=clientGreen!30,
        bottom color=clientGreen!5,
        draw=clientGreen!70,
        line width=1pt
    },

    gradient router/.style={
        base node,
        trapezium,
        trapezium left angle=70,
        trapezium right angle=110,
        shading=axis,
        top color=routerOrange!35,
        bottom color=routerOrange!10,
        draw=routerOrange!80,
        line width=1.5pt
    },

    gradient firewall/.style={
        base node,
        rectangle,
        shading=axis,
        top color=firewallRed!25,
        bottom color=firewallRed!5,
        draw=firewallRed!80,
        line width=2pt,
        pattern=north east lines,
        pattern color=firewallRed!20
    },

    gradient switch/.style={
        base node,
        rectangle,
        rounded corners=1pt,
        shading=axis,
        top color=switchPurple!25,
        bottom color=switchPurple!5,
        draw=switchPurple!70,
        minimum height=1cm
    },

    % Radial gradient for special emphasis
    radial server/.style={
        base node,
        rectangle,
        rounded corners=3pt,
        shading=radial,
        inner color=serverBlue!20,
        outer color=serverBlue!5,
        draw=serverBlue!80,
        line width=1.5pt
    },

    radial critical/.style={
        base node,
        rectangle,
        rounded corners=3pt,
        shading=radial,
        inner color=threatCritical!40,
        outer color=threatCritical!10,
        draw=threatCritical!90,
        line width=2pt
    },

    % Metallic/glossy effect for enterprise look
    metallic server/.style={
        base node,
        rectangle,
        rounded corners=3pt,
        shading=axis,
        top color=serverBlue!50,
        middle color=serverBlue!20,
        bottom color=serverBlue!40,
        draw=serverBlue!80,
        line width=1.5pt,
        blur shadow={shadow blur steps=10, shadow xshift=0pt, shadow yshift=-3pt}
    },

    metallic router/.style={
        base node,
        trapezium,
        trapezium left angle=70,
        trapezium right angle=110,
        shading=axis,
        top color=routerOrange!50,
        middle color=routerOrange!15,
        bottom color=routerOrange!35,
        draw=routerOrange!80,
        line width=1.5pt,
        blur shadow={shadow blur steps=10}
    },

    % Glass effect (modern UI style)
    glass node/.style={
        base node,
        rectangle,
        rounded corners=4pt,
        shading=axis,
        top color=white,
        middle color=serverBlue!10,
        bottom color=serverBlue!20,
        draw=serverBlue!60,
        line width=1pt,
        opacity=0.9
    }
}

% ============================================================================
% ICON AND IMAGE SUPPORT FOR NODES
% ============================================================================

% Helper command to include an icon/image inside a node
% Usage: \nodeIcon{width}{image_path}
\newcommand{\nodeIcon}[2]{%
    \includegraphics[width=#1]{#2}%
}

% Device icon styles - these can be used with external icon files
% To use: place icon images in an 'icons/' directory
% Example: \node[server, label=below:{Server Name}] {\nodeIcon{1cm}{icons/server.png}};

% Fallback: Built-in TikZ icons (no external files needed)
\newcommand{\serverIcon}[1][0.4cm]{%
    \begin{tikzpicture}[scale=0.5, baseline=-0.5ex]
        \fill[black!70] (-0.3,0) rectangle (0.3,0.6);
        \fill[white] (-0.2,0.1) rectangle (-0.05,0.2);
        \fill[white] (0.05,0.1) rectangle (0.2,0.2);
        \fill[white] (-0.2,0.3) rectangle (-0.05,0.4);
        \fill[white] (0.05,0.3) rectangle (0.2,0.4);
    \end{tikzpicture}%
}

\newcommand{\laptopIcon}[1][0.4cm]{%
    \begin{tikzpicture}[scale=0.5, baseline=-0.5ex]
        \fill[black!70] (-0.35,0.1) rectangle (0.35,0.5);
        \fill[white] (-0.3,0.15) rectangle (0.3,0.45);
        \fill[black!70] (-0.45,0) rectangle (0.45,0.1);
    \end{tikzpicture}%
}

\newcommand{\phoneIcon}[1][0.4cm]{%
    \begin{tikzpicture}[scale=0.5, baseline=-0.5ex]
        \fill[black!70, rounded corners=1pt] (-0.15,0) rectangle (0.15,0.6);
        \fill[white] (-0.1,0.1) rectangle (0.1,0.5);
        \fill[black!50] (-0.03,0.52) rectangle (0.03,0.54);
    \end{tikzpicture}%
}

\newcommand{\routerIcon}[1][0.4cm]{%
    \begin{tikzpicture}[scale=0.5, baseline=-0.5ex]
        \fill[black!70] (-0.3,0.1) rectangle (0.3,0.4);
        \fill[white] (-0.2,0.15) circle (0.05);
        \fill[white] (0,0.15) circle (0.05);
        \fill[white] (0.2,0.15) circle (0.05);
        \draw[black!70, line width=1pt] (-0.2,0.4) -- (-0.2,0.6);
        \draw[black!70, line width=1pt] (0,0.4) -- (0,0.6);
        \draw[black!70, line width=1pt] (0.2,0.4) -- (0.2,0.6);
    \end{tikzpicture}%
}

\newcommand{\databaseIcon}[1][0.4cm]{%
    \begin{tikzpicture}[scale=0.5, baseline=-0.5ex]
        \fill[black!70] (-0.25,0.15) ellipse (0.25 and 0.08);
        \fill[black!70] (-0.25,0) -- (-0.5,0) arc (270:90:0.25 and 0.08) -- (-0.25,0.3) arc (90:270:0.25 and 0.08);
        \fill[black!60] (0,0.15) ellipse (0.25 and 0.08);
        \draw[black!70, line width=0.5pt] (-0.5,0) arc (270:450:0.25 and 0.08);
    \end{tikzpicture}%
}

\newcommand{\cloudIcon}[1][0.4cm]{%
    \begin{tikzpicture}[scale=0.5, baseline=-0.5ex]
        \fill[black!70] (0,0.15) circle (0.15);
        \fill[black!70] (0.2,0.2) circle (0.12);
        \fill[black!70] (-0.15,0.2) circle (0.1);
        \fill[black!70] (-0.25,0.1) -- (0.3,0.1) -- (0.3,0.05) -- (-0.25,0.05) -- cycle;
    \end{tikzpicture}%
}

% Node styles with built-in icons
\tikzset{
    icon server/.style={
        server,
        label={[inner sep=2pt, font=\small]center:\serverIcon}
    },
    icon client/.style={
        client,
        label={[inner sep=2pt, font=\small]center:\laptopIcon}
    },
    icon router/.style={
        router,
        label={[inner sep=2pt, font=\small]center:\routerIcon}
    },
    icon database/.style={
        database,
        label={[inner sep=2pt, font=\small]center:\databaseIcon}
    }
}

% TODO: Enhanced icon support
% - Font Awesome integration for more icon options
% - Custom SVG icon support
% - Icon color customization
% - Scalable icon rendering at different zoom levels

% ============================================================================
% BADGE AND LABEL SUPPORT FOR OS/STATUS INDICATORS
% ============================================================================

% Badge styles for OS indicators
\tikzset{
    % Windows badge
    badge windows/.style={
        fill=blue!70,
        text=white,
        font=\tiny\bfseries,
        inner sep=2pt,
        rounded corners=1pt,
        minimum width=1.2em
    },
    % Linux badge
    badge linux/.style={
        fill=orange!70,
        text=white,
        font=\tiny\bfseries,
        inner sep=2pt,
        rounded corners=1pt,
        minimum width=1.2em
    },
    % macOS badge
    badge macos/.style={
        fill=black!70,
        text=white,
        font=\tiny\bfseries,
        inner sep=2pt,
        rounded corners=1pt,
        minimum width=1.2em
    },
    % Status: Online
    badge online/.style={
        fill=green!70,
        text=white,
        font=\tiny\bfseries,
        inner sep=2pt,
        rounded corners=1pt
    },
    % Status: Offline
    badge offline/.style={
        fill=red!70,
        text=white,
        font=\tiny\bfseries,
        inner sep=2pt,
        rounded corners=1pt
    },
    % Status: Warning
    badge warning/.style={
        fill=yellow!80,
        text=black,
        font=\tiny\bfseries,
        inner sep=2pt,
        rounded corners=1pt
    },
    % Status: Critical
    badge critical/.style={
        fill=red!80,
        text=white,
        font=\tiny\bfseries,
        inner sep=2pt,
        rounded corners=1pt
    }
}

% Helper commands to add badges to nodes
% Usage: Add as label or pin to a node
% Example: \node[server, pin={[badge windows]45:Win}] at (0,0) {Server};

% Corner badge positioning styles
\tikzset{
    top right badge/.style={
        label={[#1, anchor=south west, xshift=0.8cm, yshift=0.4cm]north east:}
    },
    top left badge/.style={
        label={[#1, anchor=south east, xshift=-0.8cm, yshift=0.4cm]north west:}
    },
    bottom right badge/.style={
        label={[#1, anchor=north west, xshift=0.8cm, yshift=-0.4cm]south east:}
    },
    bottom left badge/.style={
        label={[#1, anchor=north east, xshift=-0.8cm, yshift=-0.4cm]south west:}
    }
}

% Predefined badge combinations
\newcommand{\windowsBadge}{%
    \node[badge windows] {Win};
}

\newcommand{\linuxBadge}{%
    \node[badge linux] {Linux};
}

\newcommand{\macOSBadge}{%
    \node[badge macos] {Mac};
}

\newcommand{\onlineBadge}{%
    \node[badge online] {●};
}

\newcommand{\offlineBadge}{%
    \node[badge offline] {●};
}

% TODO: Badge enhancements
% - More OS types (BSD, Solaris, etc.)
% - Badge size variants

% ============================================================================
% PATTERN FILLS FOR COLORBLIND ACCESSIBILITY
% ============================================================================

% Pattern-based node styles provide additional visual differentiation
% beyond color alone, improving accessibility
\tikzset{
    % Server with vertical lines pattern
    pattern server/.style={
        base node,
        rectangle,
        rounded corners=3pt,
        fill=serverBlue!20,
        draw=serverBlue!80,
        line width=1.5pt,
        pattern=vertical lines,
        pattern color=serverBlue!40
    },

    % Client with horizontal lines pattern
    pattern client/.style={
        base node,
        rectangle,
        rounded corners=2pt,
        fill=clientGreen!15,
        draw=clientGreen!70,
        line width=1pt,
        pattern=horizontal lines,
        pattern color=clientGreen!40
    },

    % Router with grid pattern
    pattern router/.style={
        base node,
        trapezium,
        trapezium left angle=70,
        trapezium right angle=110,
        fill=routerOrange!20,
        draw=routerOrange!80,
        line width=1.5pt,
        pattern=grid,
        pattern color=routerOrange!40
    },

    % Firewall with crosshatch pattern
    pattern firewall/.style={
        base node,
        rectangle,
        fill=firewallRed!15,
        draw=firewallRed!80,
        line width=2pt,
        pattern=crosshatch,
        pattern color=firewallRed!50
    },

    % Database with dots pattern
    pattern database/.style={
        base node,
        rectangle,
        rounded corners=3pt,
        fill=serverBlue!15,
        draw=serverBlue!80,
        line width=1.5pt,
        pattern=dots,
        pattern color=serverBlue!60
    },

    % Critical/compromised with north east lines
    pattern critical/.style={
        base node,
        rectangle,
        rounded corners=3pt,
        fill=threatCritical!20,
        draw=threatCritical!90,
        line width=2pt,
        pattern=north east lines,
        pattern color=threatCritical!60
    }
}

% ============================================================================
% BEAMER ANIMATION SUPPORT
% ============================================================================

% Animation styles for Beamer presentations
% These create visual effects that work in presentation mode

% Pulsing effect for active threats (requires beamer)
\tikzset{
    pulse node/.style={
        base node,
        rectangle,
        rounded corners=3pt,
        fill=threatCritical!30,
        draw=threatCritical!90,
        line width=2pt,
        % In Beamer, this can be animated with \animate command
    }
}

% Styles for progressive slide builds
\tikzset{
    % Highlight on specific slides
    reveal/.style={
        % Use with \visible<2->{...} in Beamer
        opacity=1
    },

    % Dim on specific slides
    dim/.style={
        opacity=0.3
    },

    % Flash/alert style
    alert node/.style={
        base node,
        rectangle,
        rounded corners=3pt,
        fill=threatHigh!40,
        draw=threatHigh!90,
        line width=3pt
    }
}

% Flow animation markers for connections
\tikzset{
    flow animation/.style={
        postaction={
            decorate,
            decoration={
                markings,
                mark=between positions 0.1 and 0.9 step 0.2 with {
                    \node[fill=connEncrypted, circle, inner sep=1.5pt] {};
                }
            }
        }
    },

    data flow/.style={
        draw=connNormal,
        line width=1.5pt,
        -{Stealth[length=3mm]},
        flow animation
    }
}

% Helper command for Beamer incremental reveals
% Usage in Beamer: \revealNode<2->{server}{Web Server}
\newcommand{\revealNode}[3][server]{
    % #1 = node style (optional, default: server)
    % #2 = node name
    % #3 = node content
    % This is a placeholder - actual Beamer integration requires \visible
}

% ============================================================================
% STYLE TEMPLATE SYSTEM
% ============================================================================

% Predefined style combinations that can be easily applied

% Corporate/Professional Style
\tikzset{
    corporate/.style={
        gradient server,
        blur shadow={shadow blur steps=8}
    }
}

% Security/Threat Style
\tikzset{
    security/.style={
        pattern critical,
        line width=2.5pt
    }
}

% Cloud/Modern Style
\tikzset{
    modern cloud/.style={
        glass node,
        minimum width=3cm
    }
}

% Minimal/Clean Style
\tikzset{
    minimal/.style={
        base node,
        rectangle,
        rounded corners=2pt,
        draw=black!50,
        fill=white,
        line width=1pt
    }
}

% High-Visibility Style (for presentations)
\tikzset{
    presentation/.style={
        base node,
        rectangle,
        rounded corners=4pt,
        line width=2.5pt,
        minimum width=3cm,
        minimum height=2cm,
        font=\large\sffamily\bfseries
    }
}

% Command to export current style settings to a custom file
% Usage: \saveStyleTemplate{my_custom_style}
\newcommand{\saveStyleTemplate}[1]{%
    % This would save current color and style settings to a file
    % Implementation requires file I/O
    \message{Style template '#1' saved (feature placeholder)}%
}

% Command to load a custom style template
% Usage: \loadStyleTemplate{my_custom_style}
\newcommand{\loadStyleTemplate}[1]{%
    \IfFileExists{style_templates/#1.style}{%
        \input{style_templates/#1.style}%
    }{%
        \message{Warning: Style template '#1' not found.}%
    }%
}

% ============================================================================
% CONNECTION STYLES
% ============================================================================

\tikzset{
    % Normal connection
    normal conn/.style={
        draw=connNormal,
        line width=1pt,
        -{Stealth[length=3mm]}
    },
    
    % Encrypted connection
    encrypted conn/.style={
        draw=connEncrypted,
        line width=1.5pt,
        -{Stealth[length=3mm]},
        postaction={
            decorate,
            decoration={
                markings,
                mark=between positions 0.1 and 1 step 0.1 with {
                    \node[fill=connEncrypted!30, circle, inner sep=1pt] {};
                }
            }
        }
    },
    
    % Suspicious connection
    suspicious conn/.style={
        draw=connSuspicious,
        line width=1.5pt,
        dashed,
        dash pattern=on 4pt off 2pt,
        -{Stealth[length=3mm]}
    },
    
    % Malicious/attack connection
    attack conn/.style={
        draw=connMalicious,
        line width=2pt,
        -{Stealth[length=4mm]},
        postaction={
            draw=connMalicious!30,
            line width=3pt,
            shorten >=2pt,
            shorten <=2pt
        }
    },
    
    % Bidirectional connection
    bidirectional/.style={
        {Stealth[length=3mm]}-{Stealth[length=3mm]}
    }
}

% ============================================================================
% ADVANCED CONNECTION STYLES
% ============================================================================

% Bandwidth-based connection styles (line thickness indicates capacity)
\tikzset{
    % Low bandwidth (1-10 Mbps)
    bw low/.style={
        draw=connNormal,
        line width=0.5pt,
        -{Stealth[length=2mm]}
    },

    % Medium bandwidth (10-100 Mbps)
    bw medium/.style={
        draw=connNormal,
        line width=1pt,
        -{Stealth[length=3mm]}
    },

    % High bandwidth (100-1000 Mbps)
    bw high/.style={
        draw=connNormal,
        line width=2pt,
        -{Stealth[length=4mm]}
    },

    % Very high bandwidth (1+ Gbps)
    bw very high/.style={
        draw=connNormal,
        line width=3pt,
        -{Stealth[length=5mm]}
    },

    % Congested connection (red, thick)
    bw congested/.style={
        draw=red!70,
        line width=2.5pt,
        dashed,
        dash pattern=on 3pt off 2pt,
        -{Stealth[length=4mm]}
    }
}

% Special connection types
\tikzset{
    % VPN tunnel (dashed tube effect)
    vpn tunnel/.style={
        draw=connEncrypted,
        line width=2pt,
        dashed,
        dash pattern=on 6pt off 3pt,
        double distance=2pt,
        -{Stealth[length=4mm]}
    },

    % Wireless connection (wave pattern)
    wireless/.style={
        draw=blue!60,
        line width=1pt,
        decorate,
        decoration={snake, amplitude=1mm, segment length=5mm},
        -{Stealth[length=3mm]}
    },

    % Fiber optic (light beam effect)
    fiber optic/.style={
        draw=yellow!80,
        line width=2pt,
        -{Stealth[length=4mm]},
        postaction={
            draw=yellow!40,
            line width=3pt,
            shorten >=2pt,
            shorten <=2pt
        }
    },

    % Satellite link
    satellite link/.style={
        draw=purple!60,
        line width=1pt,
        dashed,
        dash pattern=on 2pt off 2pt,
        -{Stealth[length=3mm]}
    },

    % Blocked/firewall connection
    blocked conn/.style={
        draw=red,
        line width=2pt,
        {Bar[length=3mm]}-{Bar[length=3mm]}
    },

    % Load balanced connection
    load balanced/.style={
        draw=connNormal,
        line width=1.5pt,
        double distance=1pt,
        -{Stealth[length=3mm]}
    }
}

% Bezier curve connections for complex topologies
\tikzset{
    % Smooth curve connection
    curve conn/.style={
        draw=connNormal,
        line width=1pt,
        -{Stealth[length=3mm]},
        bend left=15
    },

    % Sharp curve
    curve sharp/.style={
        draw=connNormal,
        line width=1pt,
        -{Stealth[length=3mm]},
        bend left=45
    },

    % Reverse curve
    curve reverse/.style={
        draw=connNormal,
        line width=1pt,
        -{Stealth[length=3mm]},
        bend right=15
    }
}

% Connection aggregation (for high-density diagrams)
\newcommand{\aggregatedConn}[3]{
    % #1 = number of connections
    % #2 = from node
    % #3 = to node
    \draw[connNormal, line width=2pt, -{Stealth[length=4mm]}] (#2) -- (#3)
        node[midway, fill=white, inner sep=2pt, font=\tiny] {×#1};
}

% Protocol/port label helpers
\newcommand{\connLabel}[3]{
    % #1 = protocol name
    % #2 = port number
    % #3 = position (above/below/left/right)
    \node[port label, #3] {#1:#2};
}

% Connection with inline protocol label
\tikzset{
    labeled conn/.style n args={2}{
        draw=connNormal,
        line width=1pt,
        -{Stealth[length=3mm]},
        postaction={
            decoration={
                markings,
                mark=at position 0.5 with {
                    \node[fill=white, inner sep=2pt, font=\tiny] {#1:#2};
                }
            },
            decorate
        }
    }
}

% Bandwidth utilization indicator
\newcommand{\bwIndicator}[2]{
    % #1 = percentage (0-100)
    % #2 = connection path
    \ifnum#1<50
        \def\bwcolor{green!70}
    \else\ifnum#1<80
        \def\bwcolor{orange!70}
    \else
        \def\bwcolor{red!70}
    \fi\fi
    \draw[\bwcolor, line width=1.5pt] #2
        node[midway, fill=white, inner sep=1pt, font=\tiny] {#1\%};
}

% ============================================================================
% TEXT LABEL STYLES
% ============================================================================

\tikzset{
    % IP address label
    ip label/.style={
        font=\scriptsize\ttfamily,
        text=black!70,
        fill=white,
        inner sep=2pt,
        rounded corners=1pt
    },
    
    % Asset name label
    asset label/.style={
        font=\small\bfseries\sffamily,
        text=black!90
    },
    
    % Threat level label
    threat label/.style={
        font=\tiny\sffamily\bfseries,
        text=white,
        fill=threatCritical,
        inner sep=3pt,
        rounded corners=2pt
    },
    
    % Port/service label
    port label/.style={
        font=\tiny\ttfamily,
        text=black!60,
        fill=white,
        inner sep=1pt
    }
}

% TODO: Label enhancements
% - Auto-positioning to avoid overlap
% - Automatic truncation for long labels
% - Hover-style detailed info boxes
% - Multi-line label support with proper formatting

% ============================================================================
% LEGEND AND METADATA STYLES
% ============================================================================

\tikzset{
    legend box/.style={
        rectangle,
        draw=black!50,
        fill=white,
        inner sep=5pt,
        rounded corners=2pt,
        blur shadow={shadow blur steps=3}
    },

    % Compact legend style
    legend compact/.style={
        rectangle,
        draw=black!40,
        fill=white,
        inner sep=3pt,
        rounded corners=1pt,
        font=\tiny
    },

    % Large legend for presentations
    legend large/.style={
        rectangle,
        draw=black!60,
        fill=white,
        inner sep=8pt,
        rounded corners=3pt,
        blur shadow={shadow blur steps=5},
        font=\small
    },

    % Transparent legend
    legend transparent/.style={
        rectangle,
        draw=black!30,
        fill=white,
        fill opacity=0.8,
        text opacity=1,
        inner sep=5pt,
        rounded corners=2pt
    }
}

% Enhanced legend helpers
\newcommand{\legendTitle}[1]{%
    \textbf{#1}\\[2pt]
}

\newcommand{\legendItem}[2]{%
    #1 & #2 \\
}

% Connection type legend entries
\newcommand{\legendNormalConn}{%
    \tikz\draw[normal conn] (0,0) -- (0.5,0); & Normal
}

\newcommand{\legendEncryptedConn}{%
    \tikz\draw[encrypted conn] (0,0) -- (0.5,0); & Encrypted
}

\newcommand{\legendSuspiciousConn}{%
    \tikz\draw[suspicious conn] (0,0) -- (0.5,0); & Suspicious
}

\newcommand{\legendAttackConn}{%
    \tikz\draw[attack conn] (0,0) -- (0.5,0); & Attack
}

% Node type legend entries
\newcommand{\legendServer}{%
    \tikz\node[server, minimum width=0.5cm, minimum height=0.3cm] {}; & Server
}

\newcommand{\legendClient}{%
    \tikz\node[client, minimum width=0.5cm, minimum height=0.3cm] {}; & Client
}

\newcommand{\legendRouter}{%
    \tikz\node[router, minimum width=0.5cm, minimum height=0.3cm] {}; & Router
}

\newcommand{\legendFirewall}{%
    \tikz\node[firewall, minimum width=0.5cm, minimum height=0.3cm] {}; & Firewall
}

% Status legend entries
\newcommand{\legendOnline}{%
    \tikz\node[badge online] {●}; & Online
}

\newcommand{\legendOffline}{%
    \tikz\node[badge offline] {●}; & Offline
}

\newcommand{\legendWarning}{%
    \tikz\node[badge warning] {⚠}; & Warning
}

\newcommand{\legendCritical}{%
    \tikz\node[badge critical] {!}; & Critical
}

% Complete legend templates
\newcommand{\basicLegend}{%
    \node[legend box] {
        \begin{tabular}{ll}
            \legendTitle{Legend} \\
            \legendServer \\
            \legendClient \\
            \legendRouter \\
            \legendFirewall \\
        \end{tabular}
    };
}

\newcommand{\connectionLegend}{%
    \node[legend box] {
        \begin{tabular}{ll}
            \legendTitle{Connections} \\
            \legendNormalConn \\
            \legendEncryptedConn \\
            \legendSuspiciousConn \\
            \legendAttackConn \\
        \end{tabular}
    };
}

\newcommand{\statusLegend}{%
    \node[legend box] {
        \begin{tabular}{ll}
            \legendTitle{Status} \\
            \legendOnline \\
            \legendWarning \\
            \legendCritical \\
        \end{tabular}
    };
}

% ============================================================================
% PAGE SIZE CONFIGURATIONS
% ============================================================================

\newcommand{\setPageSize}[1]{
    \ifthenelse{\equal{#1}{a4}}{
        \renewcommand{\diagramScale}{1.0}
    }{
    \ifthenelse{\equal{#1}{a3}}{
        \renewcommand{\diagramScale}{1.4}
    }{
    \ifthenelse{\equal{#1}{a2}}{
        \renewcommand{\diagramScale}{2.0}
    }{
    \ifthenelse{\equal{#1}{a1}}{
        \renewcommand{\diagramScale}{2.8}
    }{
    \ifthenelse{\equal{#1}{a0}}{
        \renewcommand{\diagramScale}{4.0}
    }{
    \ifthenelse{\equal{#1}{letter}}{
        \renewcommand{\diagramScale}{1.0}
    }{
        \renewcommand{\diagramScale}{1.0}
    }}}}}}
}

% ============================================================================
% DIAGRAM ANNOTATIONS AND METADATA
% ============================================================================

% Annotation callout styles
\tikzset{
    % Info callout (blue)
    info callout/.style={
        rectangle,
        rounded corners=2pt,
        draw=blue!70,
        fill=blue!10,
        inner sep=4pt,
        font=\small,
        text width=3cm
    },

    % Warning callout (orange)
    warning callout/.style={
        rectangle,
        rounded corners=2pt,
        draw=orange!70,
        fill=orange!10,
        inner sep=4pt,
        font=\small,
        text width=3cm
    },

    % Critical callout (red)
    critical callout/.style={
        rectangle,
        rounded corners=2pt,
        draw=red!70,
        fill=red!10,
        inner sep=4pt,
        font=\small,
        text width=3cm
    },

    % Success callout (green)
    success callout/.style={
        rectangle,
        rounded corners=2pt,
        draw=green!70,
        fill=green!10,
        inner sep=4pt,
        font=\small,
        text width=3cm
    },

    % Note box
    note box/.style={
        rectangle,
        rounded corners=3pt,
        draw=gray!50,
        fill=yellow!10,
        inner sep=5pt,
        font=\footnotesize,
        align=left
    }
}

% Network zone/subnet boundary styles
\tikzset{
    % DMZ zone
    dmz zone/.style={
        rectangle,
        rounded corners=5pt,
        draw=red!50,
        fill=red!5,
        dashed,
        line width=1.5pt,
        inner sep=10pt
    },

    % Internal zone
    internal zone/.style={
        rectangle,
        rounded corners=5pt,
        draw=green!50,
        fill=green!5,
        dashed,
        line width=1.5pt,
        inner sep=10pt
    },

    % Trusted zone
    trusted zone/.style={
        rectangle,
        rounded corners=5pt,
        draw=blue!50,
        fill=blue!5,
        dashed,
        line width=1.5pt,
        inner sep=10pt
    },

    % Cloud/external zone
    external zone/.style={
        rectangle,
        rounded corners=5pt,
        draw=orange!50,
        fill=orange!5,
        dashed,
        line width=1.5pt,
        inner sep=10pt
    }
}

% Diagram metadata box
\newcommand{\diagramMetadata}[5]{
    % #1 = title
    % #2 = author
    % #3 = date
    % #4 = version
    % #5 = position
    \node[rectangle, draw=black!30, fill=white, rounded corners,
          inner sep=5pt, font=\small, anchor=#5] {
        \begin{tabular}{ll}
            \textbf{Diagram:} & #1 \\
            \textbf{Author:} & #2 \\
            \textbf{Date:} & #3 \\
            \textbf{Version:} & #4 \\
        \end{tabular}
    };
}

% Network statistics box
\newcommand{\networkStats}[6]{
    % #1 = total nodes
    % #2 = servers
    % #3 = clients
    % #4 = connections
    % #5 = threats
    % #6 = position
    \node[rectangle, draw=blue!30, fill=blue!5, rounded corners,
          inner sep=5pt, font=\small, anchor=#6] {
        \begin{tabular}{ll}
            \textbf{Network Statistics} \\[2pt]
            Total Nodes: & #1 \\
            Servers: & #2 \\
            Clients: & #3 \\
            Connections: & #4 \\
            Active Threats: & \textcolor{red}{#5} \\
        \end{tabular}
    };
}

% IP range annotation
\newcommand{\ipRange}[2]{
    % #1 = IP range (e.g., 192.168.1.0/24)
    % #2 = description
    \node[font=\tiny\ttfamily, fill=white, inner sep=2pt] {#1};
    \node[font=\tiny, text=gray, below] {#2};
}

% Timestamp annotation
% Define current time command (simple version without external packages)
\newcommand{\currenttime}{\number\hour:\ifnum\minute<10 0\fi\number\minute}

\newcommand{\timestamp}{
    \node[font=\tiny, text=gray] at (current page.south east)
        [anchor=south east, xshift=-5mm, yshift=5mm]
        {Generated: \today\ at \currenttime};
}

% ============================================================================
% NETWORK TOPOLOGY TEMPLATES
% ============================================================================

% Helper to create a standard 3-tier architecture
\newcommand{\threeTierTemplate}[9]{
    % #1-3 = web tier nodes (names)
    % #4-6 = app tier nodes (names)
    % #7-9 = data tier nodes (names)

    % Web Tier
    \node[gradient server] (#1) at (0,6) {Web\\#1};
    \node[gradient server] (#2) at (3,6) {Web\\#2};
    \node[gradient server] (#3) at (6,6) {Web\\#3};

    % App Tier
    \node[gradient server] (#4) at (0,3) {App\\#4};
    \node[gradient server] (#5) at (3,3) {App\\#5};
    \node[gradient server] (#6) at (6,3) {App\\#6};

    % Data Tier
    \node[radial server] (#7) at (0,0) {DB\\#7};
    \node[radial server] (#8) at (3,0) {DB\\#8};
    \node[radial server] (#9) at (6,0) {DB\\#9};

    % Standard connections
    \foreach \w in {#1,#2,#3} {
        \foreach \a in {#4,#5,#6} {
            \draw[normal conn] (\w) -- (\a);
        }
    }
    \foreach \a in {#4,#5,#6} {
        \foreach \d in {#7,#8,#9} {
            \draw[encrypted conn] (\a) -- (\d);
        }
    }
}

% Hub and spoke topology helper
\newcommand{\hubSpokeTemplate}[5]{
    % #1 = hub node name
    % #2 = number of spokes
    % #3 = radius
    % #4 = hub style
    % #5 = spoke style

    \node[#4] (#1) at (0,0) {Hub\\#1};

    \foreach \i in {1,...,#2} {
        \pgfmathsetmacro{\angle}{360/#2 * (\i - 1)}
        \node[#5] (spoke\i) at (\angle:#3) {Spoke\\\i};
        \draw[normal conn] (#1) -- (spoke\i);
    }
}

% Mesh topology helper
\newcommand{\meshTemplate}[3]{
    % #1 = number of nodes
    % #2 = radius
    % #3 = node style

    \foreach \i in {1,...,#1} {
        \pgfmathsetmacro{\angle}{360/#1 * (\i - 1)}
        \node[#3] (mesh\i) at (\angle:#2) {Node\\\i};
    }

    % Full mesh connections
    \foreach \i in {1,...,#1} {
        \foreach \j in {1,...,#1} {
            \ifnum\i<\j
                \draw[normal conn] (mesh\i) -- (mesh\j);
            \fi
        }
    }
}

% TODO: Page size optimization
% - Auto-scaling based on node count
% - Maintain aspect ratio across different sizes
% - Add support for custom dimensions
% - Portrait vs landscape orientation support
% - Multi-page diagram support for very large networks

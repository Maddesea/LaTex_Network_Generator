% styles_config.tex - Visual styling and color schemes
% This module handles all visual styling, colors, and aesthetic configurations

% ============================================================================
% COLOR PALETTE DEFINITIONS
% ============================================================================

% Network asset colors
\definecolor{serverBlue}{RGB}{52, 152, 219}
\definecolor{clientGreen}{RGB}{46, 204, 113}
\definecolor{routerOrange}{RGB}{230, 126, 34}
\definecolor{firewallRed}{RGB}{231, 76, 60}
\definecolor{switchPurple}{RGB}{155, 89, 182}
\definecolor{cloudGray}{RGB}{149, 165, 166}

% Threat indicator colors
\definecolor{threatCritical}{RGB}{192, 57, 43}
\definecolor{threatHigh}{RGB}{211, 84, 0}
\definecolor{threatMedium}{RGB}{243, 156, 18}
\definecolor{threatLow}{RGB}{241, 196, 15}
\definecolor{threatInfo}{RGB}{52, 152, 219}

% Connection colors
\definecolor{connNormal}{RGB}{127, 140, 141}
\definecolor{connEncrypted}{RGB}{39, 174, 96}
\definecolor{connSuspicious}{RGB}{230, 126, 34}
\definecolor{connMalicious}{RGB}{192, 57, 43}

% Background colors
\definecolor{bgLight}{RGB}{236, 240, 241}
\definecolor{bgDark}{RGB}{44, 62, 80}

% ============================================================================
% COLOR SCHEME SYSTEM
% ============================================================================

% Current color scheme (default: standard)
\newcommand{\currentColorScheme}{standard}
\newcommand{\currentTheme}{light}

% Command to set theme (light/dark)
\newcommand{\setTheme}[1]{
    \renewcommand{\currentTheme}{#1}
    \ifthenelse{\equal{#1}{dark}}{
        % Apply dark theme colors
        \colorlet{bgPrimary}{bgDark}
        \colorlet{textPrimary}{white}
        \colorlet{textSecondary}{white!80}
    }{
        % Apply light theme colors
        \colorlet{bgPrimary}{bgLight}
        \colorlet{textPrimary}{black!90}
        \colorlet{textSecondary}{black!70}
    }
}

% Initialize with light theme
\setTheme{light}

% Command to load color schemes
\newcommand{\loadColorScheme}[1]{
    \renewcommand{\currentColorScheme}{#1}

    % Standard color scheme (default)
    \ifthenelse{\equal{#1}{standard}}{
        \definecolor{serverBlue}{RGB}{52, 152, 219}
        \definecolor{clientGreen}{RGB}{46, 204, 113}
        \definecolor{routerOrange}{RGB}{230, 126, 34}
        \definecolor{firewallRed}{RGB}{231, 76, 60}
        \definecolor{switchPurple}{RGB}{155, 89, 182}
        \definecolor{cloudGray}{RGB}{149, 165, 166}
        \definecolor{threatCritical}{RGB}{192, 57, 43}
        \definecolor{threatHigh}{RGB}{211, 84, 0}
        \definecolor{threatMedium}{RGB}{243, 156, 18}
        \definecolor{threatLow}{RGB}{241, 196, 15}
        \definecolor{threatInfo}{RGB}{52, 152, 219}
    }{}

    % Dark color scheme
    \ifthenelse{\equal{#1}{dark}}{
        \definecolor{serverBlue}{RGB}{100, 181, 246}
        \definecolor{clientGreen}{RGB}{129, 199, 132}
        \definecolor{routerOrange}{RGB}{255, 167, 38}
        \definecolor{firewallRed}{RGB}{239, 83, 80}
        \definecolor{switchPurple}{RGB}{186, 104, 200}
        \definecolor{cloudGray}{RGB}{189, 189, 189}
        \definecolor{threatCritical}{RGB}{244, 67, 54}
        \definecolor{threatHigh}{RGB}{255, 112, 67}
        \definecolor{threatMedium}{RGB}{255, 193, 7}
        \definecolor{threatLow}{RGB}{255, 235, 59}
        \definecolor{threatInfo}{RGB}{100, 181, 246}
        \setTheme{dark}
    }{}

    % Colorblind-friendly (deuteranopia/protanopia safe)
    \ifthenelse{\equal{#1}{colorblind}}{
        \definecolor{serverBlue}{RGB}{0, 114, 178}      % Blue
        \definecolor{clientGreen}{RGB}{0, 158, 115}     % Bluish green
        \definecolor{routerOrange}{RGB}{213, 94, 0}     % Vermillion
        \definecolor{firewallRed}{RGB}{204, 121, 167}   % Reddish purple
        \definecolor{switchPurple}{RGB}{86, 180, 233}   % Sky blue
        \definecolor{cloudGray}{RGB}{127, 127, 127}     % Gray
        \definecolor{threatCritical}{RGB}{213, 94, 0}   % Vermillion
        \definecolor{threatHigh}{RGB}{230, 159, 0}      % Orange
        \definecolor{threatMedium}{RGB}{240, 228, 66}   % Yellow
        \definecolor{threatLow}{RGB}{0, 158, 115}       % Bluish green
        \definecolor{threatInfo}{RGB}{0, 114, 178}      % Blue
    }{}

    % Monochrome (grayscale)
    \ifthenelse{\equal{#1}{monochrome}}{
        \definecolor{serverBlue}{RGB}{60, 60, 60}
        \definecolor{clientGreen}{RGB}{100, 100, 100}
        \definecolor{routerOrange}{RGB}{140, 140, 140}
        \definecolor{firewallRed}{RGB}{40, 40, 40}
        \definecolor{switchPurple}{RGB}{120, 120, 120}
        \definecolor{cloudGray}{RGB}{180, 180, 180}
        \definecolor{threatCritical}{RGB}{30, 30, 30}
        \definecolor{threatHigh}{RGB}{70, 70, 70}
        \definecolor{threatMedium}{RGB}{110, 110, 110}
        \definecolor{threatLow}{RGB}{150, 150, 150}
        \definecolor{threatInfo}{RGB}{190, 190, 190}
    }{}

    % High contrast (accessibility)
    \ifthenelse{\equal{#1}{high-contrast}}{
        \definecolor{serverBlue}{RGB}{0, 0, 255}
        \definecolor{clientGreen}{RGB}{0, 255, 0}
        \definecolor{routerOrange}{RGB}{255, 165, 0}
        \definecolor{firewallRed}{RGB}{255, 0, 0}
        \definecolor{switchPurple}{RGB}{128, 0, 128}
        \definecolor{cloudGray}{RGB}{128, 128, 128}
        \definecolor{threatCritical}{RGB}{139, 0, 0}
        \definecolor{threatHigh}{RGB}{255, 69, 0}
        \definecolor{threatMedium}{RGB}{255, 215, 0}
        \definecolor{threatLow}{RGB}{255, 255, 0}
        \definecolor{threatInfo}{RGB}{0, 191, 255}
    }{}

    % Tritanopia-safe color scheme
    \ifthenelse{\equal{#1}{tritanopia}}{
        \definecolor{serverBlue}{RGB}{227, 114, 114}    % Red
        \definecolor{clientGreen}{RGB}{255, 187, 120}   % Orange
        \definecolor{routerOrange}{RGB}{255, 204, 229}  % Pink
        \definecolor{firewallRed}{RGB}{179, 88, 6}      % Dark orange
        \definecolor{switchPurple}{RGB}{136, 204, 238}  % Sky blue
        \definecolor{cloudGray}{RGB}{127, 127, 127}     % Gray
        \definecolor{threatCritical}{RGB}{179, 88, 6}
        \definecolor{threatHigh}{RGB}{227, 114, 114}
        \definecolor{threatMedium}{RGB}{255, 187, 120}
        \definecolor{threatLow}{RGB}{255, 204, 229}
        \definecolor{threatInfo}{RGB}{136, 204, 238}
    }{}
}

% ============================================================================
% NODE STYLES
% ============================================================================

\tikzset{
    % Base node style
    base node/.style={
        draw,
        thick,
        minimum width=2.5cm,
        minimum height=1.5cm,
        align=center,
        font=\small\sffamily,
        blur shadow={shadow blur steps=5, shadow xshift=0pt, shadow yshift=-2pt}
    },
    
    % Server style
    server/.style={
        base node,
        rectangle,
        rounded corners=3pt,
        fill=serverBlue!20,
        draw=serverBlue!80,
        line width=1.5pt
    },
    
    % Client/Workstation style
    client/.style={
        base node,
        rectangle,
        rounded corners=2pt,
        fill=clientGreen!15,
        draw=clientGreen!70,
        line width=1pt
    },
    
    % Router style
    router/.style={
        base node,
        trapezium,
        trapezium left angle=70,
        trapezium right angle=110,
        fill=routerOrange!20,
        draw=routerOrange!80,
        line width=1.5pt
    },
    
    % Firewall style
    firewall/.style={
        base node,
        rectangle,
        fill=firewallRed!15,
        draw=firewallRed!80,
        line width=2pt,
        pattern=north east lines,
        pattern color=firewallRed!30
    },
    
    % Switch style
    switch/.style={
        base node,
        rectangle,
        rounded corners=1pt,
        fill=switchPurple!15,
        draw=switchPurple!70,
        minimum height=1cm
    },
    
    % Cloud/Internet style
    cloud/.style={
        base node,
        shape=cloud,
        cloud puffs=10,
        cloud puff arc=120,
        aspect=2,
        fill=cloudGray!20,
        draw=cloudGray!70
    },
    
    % Attacker node style
    attacker/.style={
        base node,
        shape=star,
        star points=5,
        fill=threatCritical!30,
        draw=threatCritical!90,
        line width=2pt,
        minimum width=2cm,
        minimum height=2cm
    },

    % ========================================================================
    % GRADIENT VARIANTS (Premium Look)
    % ========================================================================

    % Server with gradient
    server premium/.style={
        base node,
        rectangle,
        rounded corners=3pt,
        server gradient,
        draw=serverBlue!80,
        line width=1.5pt
    },

    % Client with gradient
    client premium/.style={
        base node,
        rectangle,
        rounded corners=2pt,
        client gradient,
        draw=clientGreen!70,
        line width=1pt
    },

    % Router with gradient
    router premium/.style={
        base node,
        trapezium,
        trapezium left angle=70,
        trapezium right angle=110,
        router gradient,
        draw=routerOrange!80,
        line width=1.5pt
    },

    % Firewall with gradient
    firewall premium/.style={
        base node,
        rectangle,
        firewall gradient,
        draw=firewallRed!80,
        line width=2pt,
        pattern=north east lines,
        pattern color=firewallRed!30
    },

    % Switch with gradient
    switch premium/.style={
        base node,
        rectangle,
        rounded corners=1pt,
        switch gradient,
        draw=switchPurple!70,
        minimum height=1cm
    },

    % Cloud with gradient
    cloud premium/.style={
        base node,
        shape=cloud,
        cloud puffs=10,
        cloud puff arc=120,
        aspect=2,
        cloud gradient,
        draw=cloudGray!70
    },

    % ========================================================================
    % COLORBLIND-SAFE VARIANTS (With Patterns)
    % ========================================================================

    % Server with pattern for colorblind users
    server colorblind/.style={
        base node,
        rectangle,
        rounded corners=3pt,
        fill=serverBlue!20,
        draw=serverBlue!80,
        line width=1.5pt,
        postaction={pattern=dots, pattern color=serverBlue!40}
    },

    % Client with pattern
    client colorblind/.style={
        base node,
        rectangle,
        rounded corners=2pt,
        fill=clientGreen!15,
        draw=clientGreen!70,
        line width=1pt,
        postaction={pattern=horizontal lines, pattern color=clientGreen!40}
    },

    % Router with pattern
    router colorblind/.style={
        base node,
        trapezium,
        trapezium left angle=70,
        trapezium right angle=110,
        fill=routerOrange!20,
        draw=routerOrange!80,
        line width=1.5pt,
        postaction={pattern=vertical lines, pattern color=routerOrange!40}
    },

    % Firewall with enhanced pattern
    firewall colorblind/.style={
        base node,
        rectangle,
        fill=firewallRed!15,
        draw=firewallRed!80,
        line width=2pt,
        pattern=crosshatch,
        pattern color=firewallRed!40
    },

    % Switch with pattern
    switch colorblind/.style={
        base node,
        rectangle,
        rounded corners=1pt,
        fill=switchPurple!15,
        draw=switchPurple!70,
        minimum height=1cm,
        postaction={pattern=grid, pattern color=switchPurple!40}
    }
}

% ============================================================================
% GRADIENT DEFINITIONS FOR PREMIUM LOOK
% ============================================================================

% Vertical gradient shadings
\tikzset{
    % Server gradient (blue)
    server gradient/.style={
        top color=serverBlue!40,
        bottom color=serverBlue!10,
        middle color=serverBlue!25
    },

    % Client gradient (green)
    client gradient/.style={
        top color=clientGreen!35,
        bottom color=clientGreen!5,
        middle color=clientGreen!20
    },

    % Router gradient (orange)
    router gradient/.style={
        top color=routerOrange!40,
        bottom color=routerOrange!10,
        middle color=routerOrange!25
    },

    % Firewall gradient (red)
    firewall gradient/.style={
        top color=firewallRed!30,
        bottom color=firewallRed!5,
        middle color=firewallRed!17
    },

    % Switch gradient (purple)
    switch gradient/.style={
        top color=switchPurple!35,
        bottom color=switchPurple!5,
        middle color=switchPurple!20
    },

    % Cloud gradient (gray)
    cloud gradient/.style={
        top color=cloudGray!40,
        bottom color=cloudGray!10,
        middle color=cloudGray!25
    },

    % Radial gradient for emphasis (can be applied to any node)
    radial gradient/.style={
        inner color=#1!20,
        outer color=#1!5
    },

    % Metallic/glossy effect
    metallic/.style={
        top color=white,
        bottom color=#1!30,
        middle color=#1!60,
        shading angle=135
    }
}

% ============================================================================
% BADGE AND LABEL SUPPORT FOR OS/STATUS INDICATORS
% ============================================================================

% Badge styles
\tikzset{
    % Base badge style
    base badge/.style={
        circle,
        minimum size=0.6cm,
        font=\tiny\bfseries,
        inner sep=1pt
    },

    % OS badges
    windows badge/.style={
        base badge,
        fill=blue!70,
        text=white,
        draw=blue!90,
        line width=0.5pt
    },

    linux badge/.style={
        base badge,
        fill=orange!70,
        text=white,
        draw=orange!90,
        line width=0.5pt
    },

    macos badge/.style={
        base badge,
        fill=gray!70,
        text=white,
        draw=gray!90,
        line width=0.5pt
    },

    % Status badges
    online badge/.style={
        base badge,
        fill=green!60,
        text=white,
        draw=green!80,
        line width=0.5pt
    },

    offline badge/.style={
        base badge,
        fill=gray!60,
        text=white,
        draw=gray!80,
        line width=0.5pt
    },

    warning badge/.style={
        base badge,
        fill=yellow!70,
        text=black,
        draw=orange!80,
        line width=0.5pt
    },

    critical badge/.style={
        base badge,
        fill=red!70,
        text=white,
        draw=red!90,
        line width=0.5pt
    },

    % Corner badge positioning helper
    badge north east/.style={
        anchor=south west,
        xshift=0.1cm,
        yshift=0.1cm
    },

    badge north west/.style={
        anchor=south east,
        xshift=-0.1cm,
        yshift=0.1cm
    },

    badge south east/.style={
        anchor=north west,
        xshift=0.1cm,
        yshift=-0.1cm
    },

    badge south west/.style={
        anchor=north east,
        xshift=-0.1cm,
        yshift=-0.1cm
    }
}

% Command to add a badge to a node
% Usage: \nodeBadge{node-name}{badge-style}{position}{text}
\newcommand{\nodeBadge}[4]{
    \node[#2, badge #3] at (#1.#3) {#4};
}

% ============================================================================
% ICON/IMAGE SUPPORT INSIDE NODES
% ============================================================================

% Icon placeholder styles (can be replaced with actual icons/images)
\tikzset{
    % Server icon style
    server icon/.style={
        rectangle,
        minimum width=0.8cm,
        minimum height=0.6cm,
        fill=serverBlue!60,
        draw=serverBlue!90,
        line width=0.5pt,
        rounded corners=1pt
    },

    % Laptop icon style
    laptop icon/.style={
        rectangle,
        minimum width=0.8cm,
        minimum height=0.5cm,
        fill=clientGreen!60,
        draw=clientGreen!90,
        line width=0.5pt,
        rounded corners=2pt,
        append after command={
            \pgfextra{
                \draw[clientGreen!90, line width=0.5pt]
                    (\tikzlastnode.south west) -- ++(0,-0.05) -- ++(\tikzlastnode.width,0);
            }
        }
    },

    % Phone/mobile icon style
    phone icon/.style={
        rectangle,
        minimum width=0.4cm,
        minimum height=0.7cm,
        fill=clientGreen!60,
        draw=clientGreen!90,
        line width=0.5pt,
        rounded corners=2pt
    },

    % Router icon style
    router icon/.style={
        rectangle,
        minimum width=0.8cm,
        minimum height=0.4cm,
        fill=routerOrange!60,
        draw=routerOrange!90,
        line width=0.5pt,
        append after command={
            \pgfextra{
                \draw[routerOrange!90] (\tikzlastnode.north) -- ++(0,0.15) circle(0.05);
                \draw[routerOrange!90] (\tikzlastnode.north) ++(0.2,0) -- ++(0,0.15) circle(0.05);
                \draw[routerOrange!90] (\tikzlastnode.north) ++(-0.2,0) -- ++(0,0.15) circle(0.05);
            }
        }
    },

    % Database icon style
    database icon/.style={
        cylinder,
        shape border rotate=90,
        minimum width=0.6cm,
        minimum height=0.8cm,
        fill=serverBlue!60,
        draw=serverBlue!90,
        line width=0.5pt,
        aspect=0.3
    }
}

% Command to add an icon to a node
% Usage: \nodeIcon{x}{y}{icon-style}{optional-label}
\newcommand{\nodeIcon}[4][]{
    \node[#3] at (#2) {#1};
    \ifthenelse{\equal{#4}{}}{}{
        \node[below=0.1cm of #2, font=\tiny] {#4};
    }
}

% ============================================================================
% CONNECTION STYLES
% ============================================================================

\tikzset{
    % Normal connection
    normal conn/.style={
        draw=connNormal,
        line width=1pt,
        -{Stealth[length=3mm]}
    },
    
    % Encrypted connection
    encrypted conn/.style={
        draw=connEncrypted,
        line width=1.5pt,
        -{Stealth[length=3mm]},
        postaction={
            decorate,
            decoration={
                markings,
                mark=between positions 0.1 and 1 step 0.1 with {
                    \node[fill=connEncrypted!30, circle, inner sep=1pt] {};
                }
            }
        }
    },
    
    % Suspicious connection
    suspicious conn/.style={
        draw=connSuspicious,
        line width=1.5pt,
        dashed,
        dash pattern=on 4pt off 2pt,
        -{Stealth[length=3mm]}
    },
    
    % Malicious/attack connection
    attack conn/.style={
        draw=connMalicious,
        line width=2pt,
        -{Stealth[length=4mm]},
        postaction={
            draw=connMalicious!30,
            line width=3pt,
            shorten >=2pt,
            shorten <=2pt
        }
    },
    
    % Bidirectional connection
    bidirectional/.style={
        {Stealth[length=3mm]}-{Stealth[length=3mm]}
    }
}

% TODO: Advanced connection rendering
% - Add bandwidth indicators (line thickness based on traffic volume)
% - Implement animated flow direction indicators
% - Add connection labels for protocol/port information
% - Create curved/bezier connection options for complex topologies
% - Add connection aggregation for high-density diagrams

% ============================================================================
% TEXT LABEL STYLES
% ============================================================================

\tikzset{
    % IP address label
    ip label/.style={
        font=\scriptsize\ttfamily,
        text=black!70,
        fill=white,
        inner sep=2pt,
        rounded corners=1pt
    },
    
    % Asset name label
    asset label/.style={
        font=\small\bfseries\sffamily,
        text=black!90
    },
    
    % Threat level label
    threat label/.style={
        font=\tiny\sffamily\bfseries,
        text=white,
        fill=threatCritical,
        inner sep=3pt,
        rounded corners=2pt
    },
    
    % Port/service label
    port label/.style={
        font=\tiny\ttfamily,
        text=black!60,
        fill=white,
        inner sep=1pt
    }
}

% TODO: Label enhancements
% - Auto-positioning to avoid overlap
% - Automatic truncation for long labels
% - Hover-style detailed info boxes
% - Multi-line label support with proper formatting

% ============================================================================
% LEGEND AND METADATA STYLES
% ============================================================================

\tikzset{
    legend box/.style={
        rectangle,
        draw=black!50,
        fill=white,
        inner sep=5pt,
        rounded corners=2pt,
        blur shadow={shadow blur steps=3}
    }
}

% TODO: Legend improvements
% - Auto-generate legend based on diagram content
% - Collapsible/expandable legend sections
% - Position customization (corner placement options)
% - Size scaling based on diagram complexity

% ============================================================================
% PAGE SIZE CONFIGURATIONS
% ============================================================================

\newcommand{\setPageSize}[1]{
    \ifthenelse{\equal{#1}{a4}}{
        \renewcommand{\diagramScale}{1.0}
    }{
    \ifthenelse{\equal{#1}{a3}}{
        \renewcommand{\diagramScale}{1.4}
    }{
    \ifthenelse{\equal{#1}{a2}}{
        \renewcommand{\diagramScale}{2.0}
    }{
    \ifthenelse{\equal{#1}{a1}}{
        \renewcommand{\diagramScale}{2.8}
    }{
    \ifthenelse{\equal{#1}{a0}}{
        \renewcommand{\diagramScale}{4.0}
    }{
    \ifthenelse{\equal{#1}{letter}}{
        \renewcommand{\diagramScale}{1.0}
    }{
        \renewcommand{\diagramScale}{1.0}
    }}}}}}
}

% TODO: Page size optimization
% - Auto-scaling based on node count
% - Maintain aspect ratio across different sizes
% - Add support for custom dimensions
% - Portrait vs landscape orientation support
% - Multi-page diagram support for very large networks

% ============================================================================
% BEAMER ANIMATION SUPPORT
% ============================================================================

% Check if beamer is loaded
\newif\ifbeamer
\@ifclassloaded{beamer}{\beamertrue}{\beamerfalse}

% Pulse effect for highlighting nodes (works in Beamer)
\tikzset{
    pulse/.style={
        animate={
            myself:={
                0s={opacity=1, scale=1},
                0.5s={opacity=0.7, scale=1.05},
                1s={opacity=1, scale=1},
                repeat
            }
        }
    },

    % Fade in effect for progressive disclosure
    fade in/.style={
        opacity=0,
        animate={
            myself:={
                0s={opacity=0},
                0.3s={opacity=1}
            }
        }
    },

    % Highlight effect (border pulse)
    highlight pulse/.style={
        animate={
            myself:={
                line width/.list={1.5pt, 3pt, 1.5pt},
                begin on=click,
                duration=0.5s,
                repeat=3
            }
        }
    }
}

% Overlay-aware node creation for Beamer presentations
\newcommand{\createOverlayNode}[5]{
    % #1 = overlay spec, #2 = style, #3 = position, #4 = name, #5 = text
    \only<#1>{\node[#2] (#4) at #3 {#5};}
}

% Progressive reveal for connections
\newcommand{\revealConnection}[4]{
    % #1 = overlay spec, #2 = style, #3 = from, #4 = to
    \only<#1>{\draw[#2] (#3) -- (#4);}
}

% ============================================================================
% MULTI-PART NODES FOR DETAILED INFORMATION
% ============================================================================

\tikzset{
    % Multi-part node base style
    info node/.style={
        rectangle split,
        rectangle split parts=3,
        draw,
        rounded corners=2pt,
        font=\small\sffamily,
        text width=3cm,
        align=center
    },

    % Server with detailed info (hostname | IP | ports)
    detailed server/.style={
        info node,
        rectangle split part fill={serverBlue!30, serverBlue!15, serverBlue!10},
        draw=serverBlue!80,
        line width=1pt
    },

    % Client with system info
    detailed client/.style={
        info node,
        rectangle split part fill={clientGreen!25, clientGreen!12, clientGreen!8},
        draw=clientGreen!70,
        line width=1pt
    },

    % Node with status bars (CPU, Memory, Disk)
    status node/.style={
        info node,
        rectangle split parts=4,
        minimum width=3.5cm
    }
}

% Command to create a detailed server node
% Usage: \detailedServer{name}{x}{y}{hostname}{IP}{ports}
\newcommand{\detailedServer}[6]{
    \node[detailed server] (#1) at (#2,#3) {
        \textbf{#4}
        \nodepart{two}\texttt{#5}
        \nodepart{three}\scriptsize Ports: #6
    };
}

% Command to create a detailed client node
% Usage: \detailedClient{name}{x}{y}{hostname}{IP}{OS}
\newcommand{\detailedClient}[6]{
    \node[detailed client] (#1) at (#2,#3) {
        \textbf{#4}
        \nodepart{two}\texttt{#5}
        \nodepart{three}\scriptsize #6
    };
}

% ============================================================================
% ADVANCED VISUAL EFFECTS
% ============================================================================

\tikzset{
    % Glow effect for emphasis
    glow/.style={
        blur shadow={
            shadow blur steps=15,
            shadow xshift=0pt,
            shadow yshift=0pt,
            shadow blur radius=8pt,
            shadow opacity=60
        }
    },

    % Strong glow for critical alerts
    strong glow/.style={
        blur shadow={
            shadow blur steps=20,
            shadow xshift=0pt,
            shadow yshift=0pt,
            shadow blur radius=12pt,
            shadow opacity=80,
            shadow color=#1
        }
    },

    % Neon effect
    neon/.style={
        line width=2pt,
        draw=#1!90,
        blur shadow={
            shadow blur steps=15,
            shadow xshift=0pt,
            shadow yshift=0pt,
            shadow blur radius=6pt,
            shadow color=#1,
            shadow opacity=70
        }
    },

    % Double border for emphasis
    double border/.style={
        draw=#1!80,
        line width=1.5pt,
        postaction={
            draw=#1!40,
            line width=3pt,
            shorten >=1pt,
            shorten <=1pt
        }
    },

    % Dashed border with shadow
    dashed shadow/.style={
        draw=#1!70,
        line width=1.5pt,
        dashed,
        dash pattern=on 4pt off 3pt,
        blur shadow={shadow blur steps=5}
    }
}

% ============================================================================
% ADDITIONAL BADGE TYPES
% ============================================================================

\tikzset{
    % Security level badges
    security high/.style={
        base badge,
        fill=green!70,
        text=white,
        draw=green!90,
        line width=0.5pt
    },

    security medium/.style={
        base badge,
        fill=yellow!70,
        text=black,
        draw=yellow!90,
        line width=0.5pt
    },

    security low/.style={
        base badge,
        fill=red!70,
        text=white,
        draw=red!90,
        line width=0.5pt
    },

    % Service type badges
    web service badge/.style={
        base badge,
        rectangle,
        rounded corners=2pt,
        fill=blue!60,
        text=white,
        font=\tiny\ttfamily
    },

    database service badge/.style={
        base badge,
        rectangle,
        rounded corners=2pt,
        fill=purple!60,
        text=white,
        font=\tiny\ttfamily
    },

    api service badge/.style={
        base badge,
        rectangle,
        rounded corners=2pt,
        fill=orange!60,
        text=white,
        font=\tiny\ttfamily
    },

    ssh service badge/.style={
        base badge,
        rectangle,
        rounded corners=2pt,
        fill=gray!60,
        text=white,
        font=\tiny\ttfamily
    },

    % CVE severity badges
    cve critical/.style={
        rectangle,
        rounded corners=2pt,
        fill=red!80,
        text=white,
        font=\tiny\bfseries,
        inner sep=2pt,
        minimum width=1.2cm
    },

    cve high/.style={
        rectangle,
        rounded corners=2pt,
        fill=orange!80,
        text=white,
        font=\tiny\bfseries,
        inner sep=2pt,
        minimum width=1.2cm
    },

    cve medium/.style={
        rectangle,
        rounded corners=2pt,
        fill=yellow!80,
        text=black,
        font=\tiny\bfseries,
        inner sep=2pt,
        minimum width=1.2cm
    },

    cve low/.style={
        rectangle,
        rounded corners=2pt,
        fill=green!60,
        text=white,
        font=\tiny\bfseries,
        inner sep=2pt,
        minimum width=1.2cm
    }
}

% ============================================================================
% HELPER MACROS FOR EASIER USAGE
% ============================================================================

% Quick node creation with automatic styling
% Usage: \quickServer{name}{x}{y}{label}
\newcommand{\quickServer}[4]{
    \node[server premium, glow] (#1) at (#2,#3) {#4};
}

\newcommand{\quickClient}[4]{
    \node[client premium, glow] (#1) at (#2,#3) {#4};
}

\newcommand{\quickRouter}[4]{
    \node[router premium, glow] (#1) at (#2,#3) {#4};
}

\newcommand{\quickFirewall}[4]{
    \node[firewall premium, glow] (#1) at (#2,#3) {#4};
}

% Add multiple badges at once
% Usage: \addBadges{node-name}{os-badge}{status-badge}
\newcommand{\addBadges}[3]{
    \nodeBadge{#1}{#2}{north east}{OS}
    \nodeBadge{#1}{#3}{north west}{\checkmark}
}

% Add CVE badge to node
% Usage: \addCVE{node-name}{CVE-ID}{severity}{score}
\newcommand{\addCVE}[4]{
    \node[cve #3, below=0.2cm of #1] {#2: #4};
}

% Add service badge
% Usage: \addService{node-name}{service-type}{port}
\newcommand{\addService}[3]{
    \node[#2 service badge, above=0.1cm of #1] {#3};
}

% ============================================================================
% STYLE PRESET SYSTEM
% ============================================================================

% Preset: Corporate (professional, conservative)
\newcommand{\loadCorporatePreset}{
    \loadColorScheme{monochrome}
    \tikzset{
        every node/.append style={font=\sffamily},
        server/.append style={server premium},
        client/.append style={client premium},
        router/.append style={router premium}
    }
}

% Preset: Security Report (high contrast, clear threats)
\newcommand{\loadSecurityPreset}{
    \loadColorScheme{high-contrast}
    \tikzset{
        every node/.append style={double border=black},
        attacker/.append style={strong glow=red}
    }
}

% Preset: Presentation (high visibility, animations ready)
\newcommand{\loadPresentationPreset}{
    \loadColorScheme{standard}
    \tikzset{
        every node/.append style={glow},
        server/.append style={server premium},
        client/.append style={client premium}
    }
}

% Preset: Colorblind Accessible
\newcommand{\loadAccessiblePreset}{
    \loadColorScheme{colorblind}
    \tikzset{
        server/.append style={server colorblind},
        client/.append style={client colorblind},
        router/.append style={router colorblind},
        firewall/.append style={firewall colorblind}
    }
}

% Preset: Dark Mode
\newcommand{\loadDarkPreset}{
    \loadColorScheme{dark}
    \setTheme{dark}
    \tikzset{
        every node/.append style={text=white},
        server/.append style={server premium, neon=serverBlue},
        client/.append style={client premium, neon=clientGreen}
    }
}

% ============================================================================
% STYLE EXPORT/IMPORT HELPERS
% ============================================================================

% Save current color scheme to macro
\newcommand{\exportCurrentScheme}[1]{
    \edef#1{\currentColorScheme}
}

% Restore saved color scheme
\newcommand{\importScheme}[1]{
    \loadColorScheme{#1}
}

% Create custom color scheme from current colors
\newcommand{\saveAsCustomScheme}[1]{
    % This creates a snapshot of current colors as a custom scheme
    % Users can then reload it with \loadColorScheme{#1}
    \expandafter\newcommand\csname loadColorScheme#1\endcsname{
        % Colors will be preserved at definition time
        \colorlet{serverBlue-#1}{serverBlue}
        \colorlet{clientGreen-#1}{clientGreen}
        \colorlet{routerOrange-#1}{routerOrange}
    }
}

% ============================================================================
% UTILITY COMMANDS
% ============================================================================

% Quick color scheme comparison (for documentation)
\newcommand{\showColorSwatch}[2]{
    % #1 = color name, #2 = color definition
    \tikz{\node[fill=#2, minimum width=1cm, minimum height=0.5cm, draw=black] {};}
    \texttt{#1}
}

% Display all available color schemes
\newcommand{\listColorSchemes}{
    Available schemes: standard, dark, colorblind, tritanopia, monochrome, high-contrast
}

% Display all available presets
\newcommand{\listPresets}{
    Available presets: Corporate, Security, Presentation, Accessible, Dark
}

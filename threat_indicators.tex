% threat_indicators.tex - Security threat visualization and indicators
% This module handles threat detection, attack visualization, and security status

% ============================================================================
% THREAT LEVEL DEFINITIONS
% ============================================================================

\newcommand{\threatCriticalLevel}{5}
\newcommand{\threatHighLevel}{4}
\newcommand{\threatMediumLevel}{3}
\newcommand{\threatLowLevel}{2}
\newcommand{\threatInfoLevel}{1}

% TODO: Threat scoring system
% - CVSS score integration
% - Custom threat scoring algorithms
% - Risk = Likelihood × Impact calculations
% - Temporal scoring (degrading over time)
% - Environmental scoring based on context

% ============================================================================
% THREAT INDICATOR VISUALIZATION
% ============================================================================

% Draw threat indicator icon
% Usage: \drawThreatIndicator{x}{y}{level}{type}
\newcommand{\drawThreatIndicator}[4]{
    \ifthenelse{\equal{#3}{critical}}{
        \def\threatcolor{threatCritical}
        \def\threatsize{0.5}
    }{
    \ifthenelse{\equal{#3}{high}}{
        \def\threatcolor{threatHigh}
        \def\threatsize{0.4}
    }{
    \ifthenelse{\equal{#3}{medium}}{
        \def\threatcolor{threatMedium}
        \def\threatsize{0.35}
    }{
    \ifthenelse{\equal{#3}{low}}{
        \def\threatcolor{threatLow}
        \def\threatsize}{0.3}
    }{
        \def\threatcolor{threatInfo}
        \def\threatsize{0.25}
    }}}}
    
    \node[
        regular polygon,
        regular polygon sides=3,
        fill=\threatcolor,
        draw=\threatcolor!80,
        minimum size=\threatsize cm,
        inner sep=0pt
    ] at (#1,#2) {};
    
    \node[font=\tiny\bfseries\sffamily, text=white] at (#1,#2) {!};
    
    \node[below, font=\tiny\sffamily] at (#1,#2-0.15) {#4};
}

% TODO: Enhanced threat indicators
% - Animated pulsing for active threats
% - Different icon shapes for different threat types
% - Severity gradient visualization
% - Threat trend arrows (increasing/decreasing)
% - Time-to-remediation countdown

% ============================================================================
% ATTACK VISUALIZATION
% ============================================================================

% Visualize specific attack types
% Usage: \visualizeAttack{attacker}{target}{attack_type}{severity}

% DDoS Attack visualization
\newcommand{\visualizeDDoS}[3]{
    % #1 = attacker list (comma-separated)
    % #2 = target
    % #3 = severity
    \foreach \attacker in {#1} {
        \draw[attack conn, line width=2pt] (\attacker) -- (#2);
    }
    \node[draw=threatCritical, line width=3pt, circle, 
          minimum size=1.5cm, fill=threatCritical!20] at (#2) {};
    \node[above, font=\small\bfseries, text=threatCritical] 
        at (#2.north) [yshift=0.8cm] {DDoS ATTACK};
}

% SQL Injection visualization
\newcommand{\visualizeSQLi}[2]{
    % #1 = attacker
    % #2 = database server
    \draw[attack conn, line width=2pt] (#1) -- (#2)
        node[midway, threat label] {SQL Injection};
    \node[draw=threatHigh, star, star points=8, 
          minimum size=0.8cm, fill=threatHigh!30] at (#2.north east) {};
}

% Malware/Ransomware visualization
\newcommand{\visualizeMalware}[2]{
    % #1 = infected node
    % #2 = malware type
    \node[draw=threatCritical, line width=3pt, 
          rounded corners=3pt, inner sep=8pt,
          fill=threatCritical!15, dashed] at (#1) {};
    \node[above, font=\tiny\bfseries, text=threatCritical, 
          fill=white, inner sep=2pt] at (#1.north) {#2};
}

% Data exfiltration visualization
\newcommand{\visualizeExfiltration}[3]{
    % #1 = source (compromised node)
    % #2 = destination (attacker)
    % #3 = data amount
    \draw[attack conn, line width=3pt] (#1) -- (#2)
        node[midway, above, threat label] {Exfil: #3};
    \draw[draw=threatCritical, line width=2pt, dashed, 
          -{Stealth[length=5mm]}] (#1) -- (#2);
}

% TODO: Additional attack visualizations
% - Phishing attack chain
% - Privilege escalation path
% - Lateral movement tracking
% - Command & Control beaconing
% - Brute force attempts (with attempt counter)
% - Zero-day exploitation indicators
% - Supply chain attack visualization

% ============================================================================
% THREAT TIMELINE AND PROGRESSION
% ============================================================================

% ============================================================================
% ATTACK KILL CHAIN PROGRESSION DISPLAY
% ============================================================================

% Cyber Kill Chain (Lockheed Martin Model)
% Stages: 1=Reconnaissance, 2=Weaponization, 3=Delivery, 4=Exploitation,
%         5=Installation, 6=Command & Control, 7=Actions on Objectives

% Full Kill Chain visualization with current stage highlighted
% Usage: \drawKillChain{x}{y}{current_stage}
\newcommand{\drawKillChain}[3]{
    \def\chainWidth{2.0}
    \def\chainHeight{0.7}
    \def\chainSpacing{0.15}

    % Define stage names and colors
    \def\stageNames{{"Recon", "Weaponize", "Deliver", "Exploit", "Install", "C2", "Actions"}}
    \def\stageColors{{gray!60, gray!60, orange!60, red!60, red!80, purple!60, red!90}}

    % Draw title
    \node[above, font=\normalsize\bfseries] at (#1+3.5*(\chainWidth+\chainSpacing),#2+\chainHeight+0.2)
        {Cyber Kill Chain Progression};

    % Draw each stage
    \foreach \i in {0,...,6} {
        \pgfmathparse{\stageNames[\i]}
        \edef\stageName{\pgfmathresult}
        \pgfmathsetmacro{\xpos}{#1 + \i*(\chainWidth+\chainSpacing)}

        % Determine if stage is completed, current, or pending
        \pgfmathparse{\i == #3 ? 1 : 0}
        \ifnum\pgfmathresult=1
            % Current stage - highlighted
            \def\stageColor{threatCritical}
            \def\stageFill{threatCritical!50}
            \def\lineWidth{3pt}
            \def\textColor{white}
            \def\stageFontSize{\small\bfseries}
            \def\stageStatus{ACTIVE}
        \else
            \pgfmathparse{\i < #3 ? 1 : 0}
            \ifnum\pgfmathresult=1
                % Completed stage
                \def\stageColor{gray!70}
                \def\stageFill{gray!30}
                \def\lineWidth{1.5pt}
                \def\textColor{black}
                \def\stageFontSize{\small}
                \def\stageStatus{DONE}
            \else
                % Pending stage
                \def\stageColor{gray!40}
                \def\stageFill{white}
                \def\lineWidth{1pt}
                \def\textColor{gray!60}
                \def\stageFontSize{\small}
                \def\stageStatus{PENDING}
            \fi
        \fi

        % Draw stage box with rounded corners
        \draw[draw=\stageColor, fill=\stageFill, line width=\lineWidth,
              rounded corners=3pt] (\xpos,#2) rectangle +(\chainWidth,\chainHeight);

        % Stage number badge
        \node[circle, fill=\stageColor, text=white, font=\tiny\bfseries,
              minimum size=0.3cm, anchor=north west] at (\xpos+0.05,#2+\chainHeight-0.05) {\i};

        % Stage name
        \node[font=\stageFontSize, text=\textColor] at (\xpos+\chainWidth/2,#2+\chainHeight/2)
            {\stageName};

        % Status indicator below
        \node[below, font=\tiny, text=\stageColor] at (\xpos+\chainWidth/2,#2-0.05)
            {\stageStatus};

        % Arrow to next stage
        \pgfmathparse{\i < 6 ? 1 : 0}
        \ifnum\pgfmathresult=1
            \pgfmathparse{\i < #3 ? 1 : 0}
            \ifnum\pgfmathresult=1
                \def\arrowColor{gray!60}
                \def\arrowWidth{2pt}
            \else
                \def\arrowColor{gray!40}
                \def\arrowWidth{1pt}
            \fi
            \draw[-{Stealth[length=3mm]}, line width=\arrowWidth, draw=\arrowColor]
                (\xpos+\chainWidth,#2+\chainHeight/2) --
                (\xpos+\chainWidth+\chainSpacing,#2+\chainHeight/2);
        \fi
    }
}

% Compact kill chain progress bar
% Usage: \drawKillChainCompact{x}{y}{current_stage}
\newcommand{\drawKillChainCompact}[3]{
    \def\barWidth}{6}
    \def\barHeight{0.4}
    \pgfmathsetmacro{\progress}{(#3/7)*\barWidth}

    % Background
    \draw[fill=gray!20, draw=gray!50] (#1,#2) rectangle +(\barWidth,\barHeight);

    % Progress fill with gradient
    \fill[left color=orange, right color=red] (#1,#2) rectangle +(\progress,\barHeight);

    % Border
    \draw[line width=1pt, draw=gray!70] (#1,#2) rectangle +(\barWidth,\barHeight);

    % Stage markers
    \foreach \i in {1,...,6} {
        \pgfmathsetmacro{\markPos}{(\i/7)*\barWidth}
        \draw[line width=0.5pt, draw=gray!50] (#1+\markPos,#2) -- +(0,\barHeight);
    }

    % Label
    \node[above, font=\small\bfseries] at (#1+\barWidth/2,#2+\barHeight) {Kill Chain Progress};
    \node[right, font=\small] at (#1+\barWidth+0.1,#2+\barHeight/2)
        {Stage #3/7};
}

% Time-based attack timeline
% Usage: \drawAttackTimeline{x}{y}
\newcommand{\drawAttackTimeline}[2]{
    \def\timelineLength{10}
    \def\timelineHeight{0.1}

    % Timeline base
    \draw[line width=2pt, draw=gray!60] (#1,#2) -- +(\timelineLength,0);

    % Example timeline events (customize in data file)
    % Event 1: Initial compromise
    \node[circle, fill=orange, minimum size=0.3cm] at (#1+1,#2) {};
    \node[above, font=\tiny, text width=2cm, align=center] at (#1+1,#2+0.15)
        {T+0h\\Initial\\Access};

    % Event 2: Lateral movement
    \node[circle, fill=red!70, minimum size=0.3cm] at (#1+4,#2) {};
    \node[above, font=\tiny, text width=2cm, align=center] at (#1+4,#2+0.15)
        {T+2h\\Lateral\\Movement};

    % Event 3: Data exfiltration
    \node[circle, fill=red!90, minimum size=0.3cm] at (#1+8,#2) {};
    \node[above, font=\tiny, text width=2cm, align=center] at (#1+8,#2+0.15)
        {T+12h\\Data\\Exfil};

    % Timeline label
    \node[left, font=\small\bfseries] at (#1-0.2,#2) {Attack Timeline:};
}

% Attack path visualization (tree structure)
% Usage: \drawAttackPath{x}{y}
\newcommand{\drawAttackPath}[2]{
    \node[legend box, anchor=north west, minimum width=4cm] at (#1,#2) {
        \begin{tabular}{l}
            \textbf{Attack Path} \\
            \hline
            \textcolor{orange}{1. Initial Access} \\
            \quad $\rightarrow$ Phishing Email \\
            \textcolor{orange}{2. Execution} \\
            \quad $\rightarrow$ Malicious Macro \\
            \textcolor{red}{3. Persistence} \\
            \quad $\rightarrow$ Registry Key \\
            \textcolor{red}{4. Credential Theft} \\
            \quad $\rightarrow$ LSASS Dump \\
            \textcolor{red!90}{5. Lateral Movement} \\
            \quad $\rightarrow$ Pass-the-Hash \\
            \hline
            \tiny Customize via data file
        \end{tabular}
    };
}

% Infection spread visualization
% Usage: \drawInfectionSpread{center_node}{infected_nodes}
\newcommand{\drawInfectionSpread}[2]{
    % Draw spreading infection from center to infected nodes
    \foreach \target in {#2} {
        \draw[draw=red!70, line width=2pt, dashed,
              -{Stealth[length=3mm]}] (#1) -- (\target);
    }
    % Highlight infection origin
    \node[circle, draw=red!90, line width=3pt, minimum size=1.2cm,
          fill=red!20] at (#1) {};
    \node[above, font=\tiny\bfseries, text=red!90, fill=white]
        at (#1.north) [yshift=0.3cm] {PATIENT ZERO};
}

% Kill chain stage details
% Usage: \drawKillChainDetails{x}{y}{stage_number}
\newcommand{\drawKillChainDetails}[3]{
    % Detailed information about a specific kill chain stage
    \ifthenelse{\equal{#3}{0}}{
        \def\detailTitle{Stage 1: Reconnaissance}
        \def\detailDesc{Research, identification, and selection of targets}
        \def\detailTechniques{T1595, T1592, T1589}
    }{
    \ifthenelse{\equal{#3}{1}}{
        \def\detailTitle{Stage 2: Weaponization}
        \def\detailDesc{Coupling exploit with backdoor into deliverable payload}
        \def\detailTechniques{T1587, T1588}
    }{
    \ifthenelse{\equal{#3}{2}}{
        \def\detailTitle{Stage 3: Delivery}
        \def\detailDesc{Transmission of weapon to targeted environment}
        \def\detailTechniques{T1566, T1091, T1200}
    }{
    \ifthenelse{\equal{#3}{3}}{
        \def\detailTitle{Stage 4: Exploitation}
        \def\detailDesc{Triggering of exploit code on victim system}
        \def\detailTechniques{T1203, T1210, T1068}
    }{
    \ifthenelse{\equal{#3}{4}}{
        \def\detailTitle{Stage 5: Installation}
        \def\detailDesc{Installation of malware on target system}
        \def\detailTechniques{T1543, T1547, T1037}
    }{
    \ifthenelse{\equal{#3}{5}}{
        \def\detailTitle{Stage 6: Command \& Control}
        \def\detailDesc{Establishment of C2 channel for remote control}
        \def\detailTechniques{T1071, T1573, T1090}
    }{
        \def\detailTitle{Stage 7: Actions on Objectives}
        \def\detailDesc{Achievement of adversary goals (exfil, disruption)}
        \def\detailTechniques{T1485, T1486, T1565}
    }}}}}}

    \node[legend box, anchor=north west, minimum width=5cm] at (#1,#2) {
        \begin{tabular}{p{4.5cm}}
            \textbf{\detailTitle} \\
            \hline
            \small\detailDesc \\[0.1cm]
            \textbf{ATT\&CK Techniques:} \\
            \tiny\ttfamily\detailTechniques \\
        \end{tabular}
    };
}

% TODO: Advanced progression features
% - Multi-path attack scenarios
% - Parallel kill chain tracking
% - Time-to-detect overlays
% - Defensive action indicators
% - Automated incident reconstruction

% ============================================================================
% SECURITY ZONES AND BOUNDARIES
% ============================================================================

% Mark security boundary breach
% Usage: \markBoundaryBreach{zone1}{zone2}{breach_point}
\newcommand{\markBoundaryBreach}[3]{
    \draw[draw=threatCritical, line width=3pt, 
          decoration={zigzag, segment length=4pt, amplitude=2pt},
          decorate] (#1) -- (#2);
    \node[circle, fill=threatCritical, minimum size=0.4cm] at (#3) {};
    \node[above, font=\tiny\bfseries, text=white] at (#3) {BREACH};
}

% Draw firewall bypass indicator
\newcommand{\showFirewallBypass}[2]{
    % #1 = firewall node
    % #2 = bypass method
    \node[draw=threatHigh, cross out, line width=2pt, 
          minimum size=1cm, inner sep=0pt] at (#1) {};
    \node[below, font=\tiny, text=threatHigh] at (#1.south) {Bypassed: #2};
}

% TODO: Security control visualization
% - IDS/IPS alert indicators
% - Failed authentication attempts
% - Access control violations
% - Encryption status (enabled/disabled/weak)
% - Patch status indicators

% ============================================================================
% VULNERABILITY INDICATORS
% ============================================================================

% Mark vulnerable node with CVE
% Usage: \markVulnerability{node}{cve}{cvss_score}
\newcommand{\markVulnerability}[3]{
    \pgfmathsetmacro{\severity}{#3/10}
    \ifthenelse{\lengthtest{\severity pt > 0.7 pt}}{
        \def\vulncolor{threatCritical}
    }{
    \ifthenelse{\lengthtest{\severity pt > 0.4 pt}}{
        \def\vulncolor{threatMedium}
    }{
        \def\vulncolor{threatLow}
    }}
    
    \node[draw=\vulncolor, line width=2pt, rectangle,
          rounded corners=2pt, inner sep=3pt, fill=\vulncolor!20,
          anchor=south west] at (#1.south west) {
        \tiny\ttfamily #2: #3
    };
}

% Show exploitable service
% Usage: \markExploitableService{node}{service}{port}
\newcommand{\markExploitableService}[3]{
    \node[draw=threatHigh, fill=threatHigh!20, 
          font=\tiny\ttfamily, anchor=north east] 
        at (#1.north east) {#2:#3};
}

% ============================================================================
% CVSS SCORE INTEGRATION AND VISUALIZATION
% ============================================================================

% CVSS v3.1 Score Severity Ratings
% None: 0.0
% Low: 0.1-3.9
% Medium: 4.0-6.9
% High: 7.0-8.9
% Critical: 9.0-10.0

% Enhanced vulnerability marker with full CVSS display
% Usage: \markVulnerabilityCVSS{node}{cve}{base_score}{temporal_score}{environmental_score}
\newcommand{\markVulnerabilityCVSS}[5]{
    % Determine severity color based on base score
    \pgfmathsetmacro{\baseScore}{#3}
    \pgfmathparse{#3 >= 9.0 ? 1 : 0}
    \ifnum\pgfmathresult=1
        \def\cvssColor{threatCritical}
        \def\cvssRating{CRITICAL}
    \else
        \pgfmathparse{#3 >= 7.0 ? 1 : 0}
        \ifnum\pgfmathresult=1
            \def\cvssColor{threatHigh}
            \def\cvssRating{HIGH}
        \else
            \pgfmathparse{#3 >= 4.0 ? 1 : 0}
            \ifnum\pgfmathresult=1
                \def\cvssColor{threatMedium}
                \def\cvssRating{MEDIUM}
            \else
                \def\cvssColor{threatLow}
                \def\cvssRating{LOW}
            \fi
        \fi
    \fi

    % Draw CVSS badge
    \node[draw=\cvssColor, line width=2pt, rectangle,
          rounded corners=3pt, inner sep=4pt, fill=\cvssColor!25,
          anchor=south west, font=\tiny\bfseries] at (#1.south west) [yshift=-0.8cm] {
        \begin{tabular}{l}
            \textcolor{\cvssColor}{#2} \\
            \textcolor{\cvssColor}{\cvssRating: #3} \\
            \tiny Base: #3 | Temp: #4 | Env: #5
        \end{tabular}
    };
}

% Visual CVSS score meter (0-10 scale with color gradient)
% Usage: \drawCVSSMeter{x}{y}{score}{label}
\newcommand{\drawCVSSMeter}[4]{
    \pgfmathsetmacro{\meterWidth}{3}
    \pgfmathsetmacro{\meterHeight}{0.4}
    \pgfmathsetmacro{\fillWidth}{(#3/10)*\meterWidth}

    % Background
    \draw[fill=gray!20, draw=gray!50] (#1,#2) rectangle +(\meterWidth,\meterHeight);

    % Color segments (gradient effect)
    \fill[threatLow] (#1,#2) rectangle +(0.3*\meterWidth,\meterHeight);
    \fill[threatLow!50!threatMedium] (#1+0.3*\meterWidth,#2) rectangle +(0.1*\meterWidth,\meterHeight);
    \fill[threatMedium] (#1+0.4*\meterWidth,#2) rectangle +(0.3*\meterWidth,\meterHeight);
    \fill[threatMedium!50!threatHigh] (#1+0.7*\meterWidth,#2) rectangle +(0.2*\meterWidth,\meterHeight);
    \fill[threatCritical] (#1+0.9*\meterWidth,#2) rectangle +(0.1*\meterWidth,\meterHeight);

    % White overlay for unfilled portion
    \fill[white] (#1+\fillWidth,#2) rectangle +(\meterWidth-\fillWidth,\meterHeight);

    % Border
    \draw[line width=1pt, draw=gray!70] (#1,#2) rectangle +(\meterWidth,\meterHeight);

    % Score indicator
    \draw[line width=2pt, draw=black] (#1+\fillWidth,#2) -- +(0,\meterHeight);

    % Label and score
    \node[above, font=\small\bfseries] at (#1+\meterWidth/2,#2+\meterHeight) {#4};
    \node[right, font=\small\bfseries] at (#1+\meterWidth+0.1,#2+\meterHeight/2) {#3};
}

% CVSS score breakdown display
% Usage: \drawCVSSBreakdown{x}{y}{base}{impact}{exploitability}{temporal}{environmental}
\newcommand{\drawCVSSBreakdown}[7]{
    \node[legend box, anchor=north west, minimum width=3.5cm] at (#1,#2) {
        \begin{tabular}{lr}
            \multicolumn{2}{c}{\textbf{CVSS v3.1 Breakdown}} \\
            \hline
            Base Score & \textbf{#3} \\
            \quad Impact & #4 \\
            \quad Exploitability & #5 \\
            Temporal Score & #6 \\
            Environmental & #7 \\
            \hline
            \multicolumn{2}{c}{\textbf{Final: #3}} \\
        \end{tabular}
    };
}

% Compact CVSS badge for space-constrained diagrams
% Usage: \cvssBadge{node}{score}
\newcommand{\cvssBadge}[2]{
    \pgfmathparse{#2 >= 9.0 ? 1 : 0}
    \ifnum\pgfmathresult=1
        \def\badgeColor{threatCritical}
    \else
        \pgfmathparse{#2 >= 7.0 ? 1 : 0}
        \ifnum\pgfmathresult=1
            \def\badgeColor{threatHigh}
        \else
            \pgfmathparse{#2 >= 4.0 ? 1 : 0}
            \ifnum\pgfmathresult=1
                \def\badgeColor{threatMedium}
            \else
                \def\badgeColor{threatLow}
            \fi
        \fi
    \fi

    \node[circle, fill=\badgeColor, draw=\badgeColor!80,
          minimum size=0.5cm, font=\tiny\bfseries, text=white,
          anchor=north east] at (#1.north east) {#2};
}

% TODO: Advanced CVSS features
% - CVSS vector string parsing (AV:N/AC:L/PR:N/UI:N/S:U/C:H/I:H/A:H)
% - Interactive CVSS calculator
% - Historical CVSS score tracking
% - Vulnerability age indicators
% - Risk prioritization markers

% ============================================================================
% THREAT ACTOR VISUALIZATION & ATTRIBUTION
% ============================================================================

% Mark threat actor with attribution
% Usage: \markThreatActor{node}{actor_name}{confidence}
\newcommand{\markThreatActor}[3]{
    \node[above, font=\small\bfseries, text=threatCritical,
          fill=white, draw=threatCritical, rounded corners=2pt,
          inner sep=3pt] at (#1.north) [yshift=0.5cm] {
        #2 (Confidence: #3\%)
    };
}

% Enhanced threat actor profile with detailed attribution
% Usage: \drawThreatActorProfile{x}{y}{actor_name}{confidence}{campaign}{motivation}
\newcommand{\drawThreatActorProfile}[6]{
    % Determine confidence level color
    \pgfmathparse{#4 >= 80 ? 1 : 0}
    \ifnum\pgfmathresult=1
        \def\confColor{red}
        \def\confLevel{HIGH}
    \else
        \pgfmathparse{#4 >= 50 ? 1 : 0}
        \ifnum\pgfmathresult=1
            \def\confColor{orange}
            \def\confLevel{MEDIUM}
        \else
            \def\confColor{yellow!80!black}
            \def\confLevel{LOW}
        \fi
    \fi

    \node[legend box, anchor=north west, minimum width=5cm] at (#1,#2) {
        \begin{tabular}{ll}
            \multicolumn{2}{c}{\textbf{Threat Actor Profile}} \\
            \hline
            \textbf{Actor:} & \textcolor{threatCritical}{#3} \\
            \textbf{Confidence:} & \textcolor{\confColor}{\confLevel~(#4\%)} \\
            \textbf{Campaign:} & #5 \\
            \textbf{Motivation:} & #6 \\
            \hline
            \multicolumn{2}{l}{\tiny Based on TTP analysis} \\
        \end{tabular}
    };
}

% Threat actor comparison/attribution analysis
% Usage: \drawActorComparison{x}{y}
\newcommand{\drawActorComparison}[2]{
    \node[legend box, anchor=north west, minimum width=6cm] at (#1,#2) {
        \begin{tabular}{lcc}
            \multicolumn{3}{c}{\textbf{Attribution Analysis}} \\
            \hline
            \textbf{Threat Actor} & \textbf{TTP Match} & \textbf{Confidence} \\
            \hline
            APT28 (Fancy Bear) & 87\% & \textcolor{red}{High} \\
            APT29 (Cozy Bear) & 45\% & \textcolor{orange}{Medium} \\
            Lazarus Group & 23\% & \textcolor{yellow!80!black}{Low} \\
            \hline
            \multicolumn{3}{l}{\tiny Top match: APT28} \\
        \end{tabular}
    };
}

% Campaign tracking visualization
% Usage: \drawCampaignTracker{x}{y}{campaign_name}{start_date}{targets}
\newcommand{\drawCampaignTracker}[5]{
    \node[legend box, anchor=north west, minimum width=5cm] at (#1,#2) {
        \begin{tabular}{ll}
            \multicolumn{2}{c}{\textbf{Campaign: #3}} \\
            \hline
            \textbf{First Seen:} & #4 \\
            \textbf{Targets:} & #5 \\
            \textbf{Status:} & \textcolor{red}{ACTIVE} \\
            \textbf{Scope:} & Global \\
            \hline
            \multicolumn{2}{l}{\tiny Update in network data} \\
        \end{tabular}
    };
}

% Threat actor origin/geolocation marker
% Usage: \markThreatActorOrigin{node}{actor}{country}{state_sponsored}
\newcommand{\markThreatActorOrigin}[4]{
    \ifthenelse{\equal{#4}{yes}}{
        \def\sponsorBadge{[STATE-SPONSORED]}
        \def\sponsorColor{red}
    }{
        \def\sponsorBadge{[INDEPENDENT]}
        \def\sponsorColor{orange}
    }

    \node[draw=threatCritical, fill=threatCritical!20, line width=2pt,
          rounded corners=3pt, inner sep=4pt, font=\tiny\bfseries,
          anchor=north] at (#1.north) [yshift=0.9cm] {
        \textcolor{threatCritical}{#2} \\
        \tiny Origin: #3 \\
        \tiny\textcolor{\sponsorColor}{\sponsorBadge}
    };
}

% Link multiple nodes to single threat actor
% Usage: \linkToThreatActor{actor_node}{victim_nodes}
\newcommand{\linkToThreatActor}[2]{
    \foreach \victim in {#2} {
        \draw[draw=red!70, line width=1.5pt, dashed,
              -{Stealth[length=2mm]}] (#1) -- (\victim);
    }
}

% Threat intelligence source citation
% Usage: \citeThreatIntel{x}{y}{source}{date}
\newcommand{\citeThreatIntel}[4]{
    \node[font=\tiny, text=gray!70, anchor=north west] at (#1,#2) {
        Source: #3 | Updated: #4
    };
}

% Show threat intelligence indicators
% Usage: \markIOC{node}{ioc_type}{value}
\newcommand{\markIOC}[3]{
    % IOC = Indicator of Compromise
    \node[font=\tiny\ttfamily, fill=yellow!30,
          draw=orange, inner sep=2pt, anchor=south]
        at (#1.south) [yshift=-0.3cm] {
        #2: #3
    };
}

% ============================================================================
% MITRE ATT&CK FRAMEWORK MAPPING
% ============================================================================

% MITRE ATT&CK Tactic Colors (based on ATT&CK Matrix)
\definecolor{attackRecon}{RGB}{102,51,153}        % Initial Access
\definecolor{attackExecution}{RGB}{204,0,0}       % Execution
\definecolor{attackPersist}{RGB}{255,102,0}       % Persistence
\definecolor{attackPrivEsc}{RGB}{255,153,0}       % Privilege Escalation
\definecolor{attackDefEvasion}{RGB}{153,153,0}    % Defense Evasion
\definecolor{attackCredAccess}{RGB}{0,102,204}    % Credential Access
\definecolor{attackDiscovery}{RGB}{0,153,153}     % Discovery
\definecolor{attackLateral}{RGB}{102,153,0}       % Lateral Movement
\definecolor{attackCollection}{RGB}{153,51,102}   % Collection
\definecolor{attackExfil}{RGB}{204,0,102}         % Exfiltration
\definecolor{attackC2}{RGB}{51,51,153}            % Command and Control
\definecolor{attackImpact}{RGB}{153,0,0}          % Impact

% Display MITRE ATT&CK technique badge
% Usage: \attackTechnique{node}{tactic}{technique_id}{technique_name}
\newcommand{\attackTechnique}[4]{
    % Determine color based on tactic
    \ifthenelse{\equal{#2}{initial-access}}{\def\tacticColor{attackRecon}}{
    \ifthenelse{\equal{#2}{execution}}{\def\tacticColor{attackExecution}}{
    \ifthenelse{\equal{#2}{persistence}}{\def\tacticColor{attackPersist}}{
    \ifthenelse{\equal{#2}{privilege-escalation}}{\def\tacticColor{attackPrivEsc}}{
    \ifthenelse{\equal{#2}{defense-evasion}}{\def\tacticColor{attackDefEvasion}}{
    \ifthenelse{\equal{#2}{credential-access}}{\def\tacticColor{attackCredAccess}}{
    \ifthenelse{\equal{#2}{discovery}}{\def\tacticColor{attackDiscovery}}{
    \ifthenelse{\equal{#2}{lateral-movement}}{\def\tacticColor{attackLateral}}{
    \ifthenelse{\equal{#2}{collection}}{\def\tacticColor{attackCollection}}{
    \ifthenelse{\equal{#2}{exfiltration}}{\def\tacticColor{attackExfil}}{
    \ifthenelse{\equal{#2}{command-control}}{\def\tacticColor{attackC2}}{
    \ifthenelse{\equal{#2}{impact}}{\def\tacticColor{attackImpact}}{
        \def\tacticColor{gray}
    }}}}}}}}}}}}

    \node[draw=\tacticColor, fill=\tacticColor!20, line width=1.5pt,
          rounded corners=2pt, inner sep=3pt, font=\tiny\bfseries,
          anchor=north west] at (#1.north west) [yshift=0.3cm] {
        \textcolor{\tacticColor}{#3} \\
        \tiny\textsf{#4}
    };
}

% MITRE ATT&CK kill chain visualization
% Usage: \drawAttackKillChain{x}{y}{current_stage}
\newcommand{\drawAttackKillChain}[3]{
    \def\stageWidth{1.8}
    \def\stageHeight{0.6}
    \def\stageSpacing{0.1}

    % Define stages
    \def\stages{{"Recon", "Weaponize", "Deliver", "Exploit", "Install", "C2", "Execute"}}

    % Draw each stage
    \foreach \i in {0,...,6} {
        \pgfmathparse{\stages[\i]}
        \edef\stageName{\pgfmathresult}
        \pgfmathsetmacro{\xpos}{#1 + \i*(\stageWidth+\stageSpacing)}

        % Highlight current stage
        \pgfmathparse{\i == #3 ? 1 : 0}
        \ifnum\pgfmathresult=1
            \def\stageColor{threatCritical}
            \def\stageFill{threatCritical!40}
            \def\lineWidth{2pt}
        \else
            \pgfmathparse{\i < #3 ? 1 : 0}
            \ifnum\pgfmathresult=1
                \def\stageColor{gray!60}
                \def\stageFill{gray!20}
                \def\lineWidth{1pt}
            \else
                \def\stageColor{gray!40}
                \def\stageFill{white}
                \def\lineWidth{1pt}
            \fi
        \fi

        % Draw stage box
        \draw[draw=\stageColor, fill=\stageFill, line width=\lineWidth,
              rounded corners=2pt] (\xpos,#2) rectangle +(\stageWidth,\stageHeight);

        % Stage label
        \node[font=\tiny\bfseries, text=black] at (\xpos+\stageWidth/2,#2+\stageHeight/2)
            {\stageName};

        % Arrow to next stage
        \pgfmathparse{\i < 6 ? 1 : 0}
        \ifnum\pgfmathresult=1
            \draw[-{Stealth[length=2mm]}, line width=1pt, draw=gray!50]
                (\xpos+\stageWidth,#2+\stageHeight/2) --
                (\xpos+\stageWidth+\stageSpacing,#2+\stageHeight/2);
        \fi
    }

    % Title
    \node[above, font=\small\bfseries] at (#1+3.5*(\stageWidth+\stageSpacing),#2+\stageHeight)
        {MITRE ATT\&CK Kill Chain};
}

% Display multiple ATT&CK techniques in a table
% Usage: \drawAttackTechniqueList{x}{y}
\newcommand{\drawAttackTechniqueList}[2]{
    \node[legend box, anchor=north west, minimum width=4.5cm] at (#1,#2) {
        \begin{tabular}{ll}
            \multicolumn{2}{c}{\textbf{ATT\&CK Techniques Detected}} \\
            \hline
            \textcolor{attackRecon}{T1566} & Phishing \\
            \textcolor{attackExecution}{T1059} & Command Shell \\
            \textcolor{attackCredAccess}{T1110} & Brute Force \\
            \textcolor{attackLateral}{T1021} & Remote Services \\
            \textcolor{attackExfil}{T1041} & C2 Channel \\
            \hline
            \multicolumn{2}{c}{\tiny Add techniques via data file} \\
        \end{tabular}
    };
}

% Show TTP (Tactics, Techniques, Procedures) for threat actor
% Usage: \drawTTPProfile{x}{y}{actor_name}
\newcommand{\drawTTPProfile}[3]{
    \node[legend box, anchor=north west, minimum width=4cm] at (#1,#2) {
        \begin{tabular}{l}
            \textbf{TTP Profile: #3} \\
            \hline
            \textcolor{attackRecon}{\textbullet~Initial Access} \\
            \quad T1566.001 Spearphishing \\
            \textcolor{attackExecution}{\textbullet~Execution} \\
            \quad T1059.001 PowerShell \\
            \textcolor{attackCredAccess}{\textbullet~Credential Access} \\
            \quad T1003 OS Credential Dump \\
            \hline
            \tiny Customize in data file
        \end{tabular}
    };
}

% Link attack connection to specific ATT&CK technique
% Usage: \drawAttackWithTechnique{source}{target}{technique_id}{technique_name}
\newcommand{\drawAttackWithTechnique}[4]{
    \draw[attack conn, line width=2pt] (#1) -- (#2)
        node[midway, threat label, fill=white, inner sep=2pt] {
            \tiny\bfseries #3\\
            \tiny #4
        };
}

% MITRE ATT&CK Matrix overview (compact)
% Usage: \drawAttackMatrix{x}{y}
\newcommand{\drawAttackMatrix}[2]{
    \node[legend box, anchor=north west, minimum width=5cm] at (#1,#2) {
        \begin{tabular}{p{4cm}}
            \textbf{MITRE ATT\&CK Matrix} \\
            \hline
            \textcolor{attackRecon}{\rule{0.3cm}{0.15cm}} Initial Access \\
            \textcolor{attackExecution}{\rule{0.3cm}{0.15cm}} Execution \\
            \textcolor{attackPersist}{\rule{0.3cm}{0.15cm}} Persistence \\
            \textcolor{attackPrivEsc}{\rule{0.3cm}{0.15cm}} Privilege Escalation \\
            \textcolor{attackDefEvasion}{\rule{0.3cm}{0.15cm}} Defense Evasion \\
            \textcolor{attackCredAccess}{\rule{0.3cm}{0.15cm}} Credential Access \\
            \textcolor{attackDiscovery}{\rule{0.3cm}{0.15cm}} Discovery \\
            \textcolor{attackLateral}{\rule{0.3cm}{0.15cm}} Lateral Movement \\
            \textcolor{attackCollection}{\rule{0.3cm}{0.15cm}} Collection \\
            \textcolor{attackExfil}{\rule{0.3cm}{0.15cm}} Exfiltration \\
            \textcolor{attackC2}{\rule{0.3cm}{0.15cm}} Command \& Control \\
            \textcolor{attackImpact}{\rule{0.3cm}{0.15cm}} Impact \\
        \end{tabular}
    };
}

% TODO: Advanced ATT&CK features
% - Sub-technique visualization (T1566.001, T1566.002, etc.)
% - ATT&CK Navigator integration
% - Platform-specific techniques (Windows, Linux, macOS, Cloud)
% - Data source mapping
% - Mitigation recommendations

% ============================================================================
% IOC (INDICATORS OF COMPROMISE) VISUALIZATION
% ============================================================================

% IOC Type Colors
\definecolor{iocMaliciousIP}{RGB}{220,20,60}      % Crimson
\definecolor{iocMaliciousDomain}{RGB}{255,69,0}   % Orange Red
\definecolor{iocFileHash}{RGB}{138,43,226}        % Blue Violet
\definecolor{iocSuspiciousURL}{RGB}{255,140,0}    % Dark Orange
\definecolor{iocC2Server}{RGB}{139,0,0}           % Dark Red

% Enhanced IOC marker with type and reputation
% Usage: \markIOCEnhanced{node}{ioc_type}{value}{reputation_score}{age_days}
\newcommand{\markIOCEnhanced}[5]{
    % Determine color based on IOC type
    \ifthenelse{\equal{#2}{malicious-ip}}{\def\iocColor{iocMaliciousIP}}{
    \ifthenelse{\equal{#2}{malicious-domain}}{\def\iocColor{iocMaliciousDomain}}{
    \ifthenelse{\equal{#2}{file-hash}}{\def\iocColor{iocFileHash}}{
    \ifthenelse{\equal{#2}{suspicious-url}}{\def\iocColor{iocSuspiciousURL}}{
    \ifthenelse{\equal{#2}{c2-server}}{\def\iocColor{iocC2Server}}{
        \def\iocColor{orange}
    }}}}}

    % Determine reputation indicator
    \pgfmathparse{#4 >= 80 ? 1 : 0}
    \ifnum\pgfmathresult=1
        \def\repIcon{$\blacksquare\blacksquare\blacksquare$}
        \def\repText{HIGH RISK}
    \else
        \pgfmathparse{#4 >= 50 ? 1 : 0}
        \ifnum\pgfmathresult=1
            \def\repIcon{$\blacksquare\blacksquare$}
            \def\repText{MED RISK}
        \else
            \def\repIcon{$\blacksquare$}
            \def\repText{LOW RISK}
        \fi
    \fi

    % Age freshness indicator
    \pgfmathparse{#5 <= 7 ? 1 : 0}
    \ifnum\pgfmathresult=1
        \def\freshnessText{FRESH}
        \def\freshnessColor{red}
    \else
        \pgfmathparse{#5 <= 30 ? 1 : 0}
        \ifnum\pgfmathresult=1
            \def\freshnessText{RECENT}
            \def\freshnessColor{orange}
        \else
            \def\freshnessText{OLD}
            \def\freshnessColor{gray}
        \fi
    \fi

    \node[font=\tiny\ttfamily, fill=\iocColor!30,
          draw=\iocColor, line width=1.5pt, inner sep=3pt,
          rounded corners=2pt, anchor=south] at (#1.south) [yshift=-0.8cm] {
        \begin{tabular}{l}
            \textbf{\textcolor{\iocColor}{IOC: #2}} \\
            \textcolor{black}{#3} \\
            \textcolor{\iocColor}{\repIcon~\repText} \\
            \textcolor{\freshnessColor}{\freshnessText~(#5d)}
        \end{tabular}
    };
}

% Display malicious IP address with geolocation
% Usage: \markMaliciousIP{node}{ip_address}{country}{reputation}
\newcommand{\markMaliciousIP}[4]{
    \node[font=\tiny\ttfamily, fill=iocMaliciousIP!25,
          draw=iocMaliciousIP, line width=2pt, inner sep=3pt,
          rounded corners=2pt, anchor=north east] at (#1.north east) [xshift=0.5cm] {
        \textcolor{iocMaliciousIP}{\textbf{MALICIOUS IP}} \\
        \textcolor{black}{#2} \\
        \textcolor{gray}{Origin: #3} \\
        \textcolor{iocMaliciousIP}{Rep: #4/100}
    };
}

% Display file hash IOC
% Usage: \markFileHashIOC{node}{hash_type}{hash_value}{malware_family}
\newcommand{\markFileHashIOC}[4]{
    \node[font=\tiny\ttfamily, fill=iocFileHash!20,
          draw=iocFileHash, line width=1.5pt, inner sep=3pt,
          rounded corners=2pt, anchor=south] at (#1.south) [yshift=-0.6cm] {
        \textcolor{iocFileHash}{\textbf{#2 HASH}} \\
        \textcolor{black}{#3} \\
        \textcolor{iocFileHash}{Family: #4}
    };
}

% Mark C2 (Command and Control) server
% Usage: \markC2Server{node}{c2_domain}{threat_actor}
\newcommand{\markC2Server}[3]{
    \node[draw=iocC2Server, line width=3pt, rectangle,
          rounded corners=3pt, inner sep=5pt, fill=iocC2Server!15,
          anchor=north] at (#1.north) [yshift=0.8cm] {
        \tiny\bfseries\textcolor{iocC2Server}{C2 SERVER} \\
        \tiny\ttfamily #2 \\
        \tiny Actor: #3
    };
    % Add beacon indicator
    \node[star, star points=5, fill=iocC2Server, minimum size=0.4cm,
          anchor=south east] at (#1.south east) {};
}

% IOC Dashboard showing all indicators
% Usage: \drawIOCDashboard{x}{y}
\newcommand{\drawIOCDashboard}[2]{
    \node[legend box, anchor=north west, minimum width=5cm] at (#1,#2) {
        \begin{tabular}{lc}
            \multicolumn{2}{c}{\textbf{IOC Summary}} \\
            \hline
            \textcolor{iocMaliciousIP}{Malicious IPs} & 0 \\
            \textcolor{iocMaliciousDomain}{Malicious Domains} & 0 \\
            \textcolor{iocFileHash}{File Hashes} & 0 \\
            \textcolor{iocSuspiciousURL}{Suspicious URLs} & 0 \\
            \textcolor{iocC2Server}{C2 Servers} & 0 \\
            \hline
            \textbf{Total IOCs} & \textbf{0} \\
            \hline
            \multicolumn{2}{l}{\tiny Last updated: Manual} \\
            \multicolumn{2}{l}{\tiny Source: Network Data} \\
        \end{tabular}
    };
}

% Threat feed integration indicator
% Usage: \drawThreatFeedStatus{x}{y}{feed_name}{status}
\newcommand{\drawThreatFeedStatus}[4]{
    \ifthenelse{\equal{#4}{active}}{
        \def\statusColor{green}
        \def\statusText{ACTIVE}
    }{
    \ifthenelse{\equal{#4}{error}}{
        \def\statusColor{red}
        \def\statusText{ERROR}
    }{
        \def\statusColor{gray}
        \def\statusText{DISABLED}
    }}

    \node[legend box, anchor=north west, minimum width=3.5cm] at (#1,#2) {
        \small\bfseries Threat Feed: #3 \\
        \textcolor{\statusColor}{\textbullet~\statusText}
    };
}

% Comprehensive IOC list with multiple entries
% Usage: \drawIOCList{x}{y}
\newcommand{\drawIOCList}[2]{
    \node[legend box, anchor=north west, minimum width=6cm] at (#1,#2) {
        \begin{tabular}{llc}
            \multicolumn{3}{c}{\textbf{Indicators of Compromise}} \\
            \hline
            \textbf{Type} & \textbf{Value} & \textbf{Risk} \\
            \hline
            \textcolor{iocMaliciousIP}{IP} & 192.0.2.1 & \textcolor{red}{High} \\
            \textcolor{iocMaliciousDomain}{Domain} & evil.com & \textcolor{red}{High} \\
            \textcolor{iocFileHash}{MD5} & d41d8c...fe88 & \textcolor{orange}{Med} \\
            \textcolor{iocC2Server}{C2} & bad-server.net & \textcolor{red}{Crit} \\
            \hline
            \multicolumn{3}{c}{\tiny Customize in network data file} \\
        \end{tabular}
    };
}

% Reputation score indicator (0-100 scale)
% Usage: \drawReputationScore{x}{y}{score}{entity_name}
\newcommand{\drawReputationScore}[4]{
    \pgfmathsetmacro{\repWidth}{2.5}
    \pgfmathsetmacro{\repHeight}{0.3}
    \pgfmathsetmacro{\fillWidth}{(#3/100)*\repWidth}

    % Background bar (inverse - red on left, green on right)
    \draw[fill=green!70, draw=gray!50] (#1,#2) rectangle +(\repWidth,\repHeight);
    \fill[yellow] (#1,#2) rectangle +(0.5*\repWidth,\repHeight);
    \fill[red!70] (#1,#2) rectangle +(0.3*\repWidth,\repHeight);

    % Indicator line
    \draw[line width=3pt, draw=black] (#1+\fillWidth,#2) -- +(0,\repHeight);

    % Label
    \node[above, font=\tiny\bfseries] at (#1+\repWidth/2,#2+\repHeight) {Reputation: #4};
    \node[right, font=\tiny] at (#1+\repWidth+0.1,#2+\repHeight/2) {#3/100};
}

% TODO: Advanced IOC features
% - VirusTotal integration
% - AlienVault OTX feed integration
% - STIX/TAXII format support
% - Automated IOC enrichment
% - IOC correlation engine
% - False positive tracking

% ============================================================================
% SECURITY POSTURE OVERVIEW
% ============================================================================

% Draw security posture dashboard
% Usage: \drawSecurityDashboard{x}{y}
\newcommand{\drawSecurityDashboard}[2]{
    \node[legend box, anchor=north east, minimum width=4cm] at (#1,#2) {
        \begin{tabular}{ll}
            \multicolumn{2}{c}{\textbf{Security Posture}} \\
            \hline
            \textcolor{threatCritical}{● Critical} & 0 \\
            \textcolor{threatHigh}{● High} & 0 \\
            \textcolor{threatMedium}{● Medium} & 0 \\
            \textcolor{threatLow}{● Low} & 0 \\
            \hline
            \textbf{Risk Score} & \textbf{0/100} \\
        \end{tabular}
    };
}

% Show compliance status
% Usage: \drawComplianceStatus{x}{y}{framework}
\newcommand{\drawComplianceStatus}[3]{
    \node[legend box, anchor=north east] at (#1,#2) {
        \small\bfseries #3 Compliance \\
        \tiny Status: Checking...
    };
}

% ============================================================================
% SECURITY COMPLIANCE DASHBOARDS
% ============================================================================

% NIST Cybersecurity Framework (CSF) Compliance Status
% Usage: \drawNISTCompliance{x}{y}{identify}{protect}{detect}{respond}{recover}
\newcommand{\drawNISTCompliance}[7]{
    \node[legend box, anchor=north west, minimum width=6cm] at (#1,#2) {
        \begin{tabular}{lcc}
            \multicolumn{3}{c}{\textbf{NIST CSF Compliance}} \\
            \hline
            \textbf{Function} & \textbf{Score} & \textbf{Status} \\
            \hline
            Identify & #3\% & \pgfmathparse{#3>=80?1:0}\ifnum\pgfmathresult=1\textcolor{green}{Good}\else\textcolor{orange}{Review}\fi \\
            Protect & #4\% & \pgfmathparse{#4>=80?1:0}\ifnum\pgfmathresult=1\textcolor{green}{Good}\else\textcolor{orange}{Review}\fi \\
            Detect & #5\% & \pgfmathparse{#5>=80?1:0}\ifnum\pgfmathresult=1\textcolor{green}{Good}\else\textcolor{orange}{Review}\fi \\
            Respond & #6\% & \pgfmathparse{#6>=80?1:0}\ifnum\pgfmathresult=1\textcolor{green}{Good}\else\textcolor{orange}{Review}\fi \\
            Recover & #7\% & \pgfmathparse{#7>=80?1:0}\ifnum\pgfmathresult=1\textcolor{green}{Good}\else\textcolor{orange}{Review}\fi \\
            \hline
            \textbf{Overall} & \textbf{\pgfmathparse{(#3+#4+#5+#6+#7)/5}\pgfmathprintnumber[fixed, precision=0]{\pgfmathresult}\%} & \\
        \end{tabular}
    };
}

% CIS Controls Compliance Dashboard
% Usage: \drawCISCompliance{x}{y}{basic}{foundational}{organizational}
\newcommand{\drawCISCompliance}[5]{
    \node[legend box, anchor=north west, minimum width=5.5cm] at (#1,#2) {
        \begin{tabular}{lc}
            \multicolumn{2}{c}{\textbf{CIS Controls v8}} \\
            \hline
            \textbf{IG1 (Basic)} & #3/16 \\
            \textbf{IG2 (Foundational)} & #4/35 \\
            \textbf{IG3 (Organizational)} & #5/18 \\
            \hline
            \textbf{Total Implemented} & \textbf{#3 + #4 + #5} \\
            \textbf{Coverage} & \pgfmathparse{((#3+#4+#5)/69)*100}\pgfmathprintnumber[fixed, precision=0]{\pgfmathresult}\% \\
        \end{tabular}
    };
}

% PCI-DSS Compliance Status
% Usage: \drawPCIDSSCompliance{x}{y}{overall_score}
\newcommand{\drawPCIDSSCompliance}[3]{
    \pgfmathparse{#3 >= 100 ? 1 : 0}
    \ifnum\pgfmathresult=1
        \def\compStatus{COMPLIANT}
        \def\compColor{green}
    \else
        \def\compStatus{NON-COMPLIANT}
        \def\compColor{red}
    \fi

    \node[legend box, anchor=north west, minimum width=5cm] at (#1,#2) {
        \begin{tabular}{ll}
            \multicolumn{2}{c}{\textbf{PCI-DSS v4.0}} \\
            \hline
            \textbf{Requirements Met:} & #3/12 \\
            \textbf{Status:} & \textcolor{\compColor}{\compStatus} \\
            \textbf{Last Audit:} & 2025-01 \\
            \textbf{Next Review:} & 2025-04 \\
            \hline
            \multicolumn{2}{l}{\tiny Update via data file} \\
        \end{tabular}
    };
}

% General compliance framework scorecard
% Usage: \drawComplianceScorecard{x}{y}{framework}{score}{status}
\newcommand{\drawComplianceScorecard}[5]{
    \ifthenelse{\equal{#5}{compliant}}{
        \def\statusColor{green}
        \def\statusText{COMPLIANT}
    }{
    \ifthenelse{\equal{#5}{partial}}{
        \def\statusColor{orange}
        \def\statusText{PARTIAL}
    }{
        \def\statusColor{red}
        \def\statusText{NON-COMPLIANT}
    }}

    \node[draw=\statusColor, fill=\statusColor!15, line width=2pt,
          rounded corners=3pt, inner sep=5pt, minimum width=4cm,
          anchor=north west] at (#1,#2) {
        \small\bfseries #3 \\
        \tiny Score: #4\% \\
        \textcolor{\statusColor}{\small\bfseries\statusText}
    };
}

% Security metrics dashboard
% Usage: \drawSecurityMetrics{x}{y}{mttd}{mttr}{incident_count}{false_positive_rate}
\newcommand{\drawSecurityMetrics}[6]{
    \node[legend box, anchor=north west, minimum width=5.5cm] at (#1,#2) {
        \begin{tabular}{ll}
            \multicolumn{2}{c}{\textbf{Security Metrics}} \\
            \hline
            \textbf{MTTD} & #3 hours \\
            \textbf{MTTR} & #4 hours \\
            \textbf{Incidents (30d)} & #5 \\
            \textbf{False Positive Rate} & #6\% \\
            \hline
            \textbf{Detection Rate} & \pgfmathparse{100-#6}\pgfmathprintnumber[fixed, precision=0]{\pgfmathresult}\% \\
            \textbf{Efficiency} & \pgfmathparse{#3<24 && #4<48 ? 1 : 0}\ifnum\pgfmathresult=1\textcolor{green}{Good}\else\textcolor{orange}{Review}\fi \\
        \end{tabular}
    };
}

% Security control effectiveness meter
% Usage: \drawControlEffectiveness{x}{y}{control_name}{effectiveness}
\newcommand{\drawControlEffectiveness}[4]{
    \pgfmathsetmacro{\meterWidth}{3.5}
    \pgfmathsetmacro{\meterHeight}{0.35}
    \pgfmathsetmacro{\fillWidth}{(#4/100)*\meterWidth}

    % Background
    \draw[fill=gray!20, draw=gray!50] (#1,#2) rectangle +(\meterWidth,\meterHeight);

    % Fill with gradient based on effectiveness
    \pgfmathparse{#4 >= 80 ? 1 : 0}
    \ifnum\pgfmathresult=1
        \fill[green!70] (#1,#2) rectangle +(\fillWidth,\meterHeight);
    \else
        \pgfmathparse{#4 >= 50 ? 1 : 0}
        \ifnum\pgfmathresult=1
            \fill[orange!70] (#1,#2) rectangle +(\fillWidth,\meterHeight);
        \else
            \fill[red!70] (#1,#2) rectangle +(\fillWidth,\meterHeight);
        \fi
    \fi

    % Border
    \draw[line width=1pt, draw=gray!70] (#1,#2) rectangle +(\meterWidth,\meterHeight);

    % Indicator line
    \draw[line width=2pt, draw=black] (#1+\fillWidth,#2) -- +(0,\meterHeight);

    % Label
    \node[above, font=\tiny\bfseries] at (#1+\meterWidth/2,#2+\meterHeight) {#3};
    \node[right, font=\tiny] at (#1+\meterWidth+0.1,#2+\meterHeight/2) {#4\%};
}

% Multi-framework compliance comparison
% Usage: \drawComplianceComparison{x}{y}
\newcommand{\drawComplianceComparison}[2]{
    \node[legend box, anchor=north west, minimum width=6.5cm] at (#1,#2) {
        \begin{tabular}{lccc}
            \multicolumn{4}{c}{\textbf{Compliance Status Overview}} \\
            \hline
            \textbf{Framework} & \textbf{Score} & \textbf{Status} & \textbf{Due} \\
            \hline
            NIST CSF & 85\% & \textcolor{green}{Good} & Annual \\
            CIS Controls & 72\% & \textcolor{orange}{Review} & Ongoing \\
            PCI-DSS & 100\% & \textcolor{green}{Pass} & Q1 2025 \\
            ISO 27001 & 78\% & \textcolor{orange}{Review} & Annual \\
            SOC 2 & 92\% & \textcolor{green}{Pass} & 2025-06 \\
            \hline
            \multicolumn{4}{l}{\tiny Update scores in network data} \\
        \end{tabular}
    };
}

% Security coverage heatmap legend
% Usage: \drawCoverageHeatmap{x}{y}
\newcommand{\drawCoverageHeatmap}[2]{
    \node[legend box, anchor=north west, minimum width=4cm] at (#1,#2) {
        \begin{tabular}{ll}
            \multicolumn{2}{c}{\textbf{Security Coverage}} \\
            \hline
            Endpoint & \textcolor{green}{●} 95\% \\
            Network & \textcolor{green}{●} 92\% \\
            Cloud & \textcolor{orange}{●} 75\% \\
            Email & \textcolor{green}{●} 98\% \\
            Web & \textcolor{orange}{●} 78\% \\
            Database & \textcolor{red}{●} 65\% \\
            \hline
            \textbf{Overall} & \textbf{84\%} \\
        \end{tabular}
    };
}

% TODO: Advanced compliance features
% - Automated compliance checking
% - Gap analysis visualization
% - Remediation timeline
% - Audit trail integration
% - Multi-year trend analysis

% ============================================================================
% ADVANCED THREAT CORRELATION & HELPER FUNCTIONS
% ============================================================================

% Correlate vulnerability with active exploitation
% Usage: \correlateVulnExploit{node}{cve}{cvss}{exploit_active}{exploit_source}
\newcommand{\correlateVulnExploit}[5]{
    % Draw vulnerability marker
    \markVulnerabilityCVSS{#1}{#2}{#3}{#3}{#3}

    % If exploit is active, add warning
    \ifthenelse{\equal{#4}{yes}}{
        \node[draw=red, fill=red!30, line width=2pt,
              rounded corners=2pt, inner sep=2pt, font=\tiny\bfseries,
              anchor=south east, text=red!90!black] at (#1.south east) [xshift=0.5cm, yshift=-1.2cm] {
            ⚠ ACTIVE\\EXPLOIT\\#5
        };
    }{}
}

% Quick threat scenario: Ransomware attack
% Usage: \scenarioRansomware{patient_zero}{affected_nodes}{ransom_amount}
\newcommand{\scenarioRansomware}[3]{
    % Mark patient zero
    \visualizeMalware{#1}{Ransomware}
    \drawInfectionSpread{#1}{#2}

    % Add ransom demand indicator
    \node[draw=threatCritical, fill=threatCritical!20, line width=3pt,
          rounded corners=3pt, inner sep=5pt, font=\small\bfseries,
          anchor=center] at (#1) [yshift=-2cm] {
        RANSOM DEMAND: #3
    };
}

% Quick threat scenario: APT infiltration
% Usage: \scenarioAPT{entry_point}{c2_server}{exfil_target}{actor}{campaign}
\newcommand{\scenarioAPT}[5]{
    % Mark entry point
    \attackTechnique{#1}{initial-access}{T1566.001}{Spearphishing}

    % Mark C2
    \markC2Server{#2}{c2.badactor.com}{#4}

    % Mark exfiltration
    \visualizeExfiltration{#3}{#2}{Sensitive Data}

    % Add campaign tracker
    \drawCampaignTracker{-8}{8}{#5}{2025-01-15}{Multiple Orgs}
}

% Quick threat scenario: DDoS attack
% Usage: \scenarioDDoS{attackers}{target}
\newcommand{\scenarioDDoS}[2]{
    \visualizeDDoS{#1}{#2}{critical}

    % Add bandwidth indicator
    \node[draw=threatCritical, fill=white, line width=2pt,
          rounded corners=2pt, inner sep=3pt, font=\tiny\bfseries,
          anchor=north] at (#2.south) [yshift=-0.5cm] {
        Traffic: 500 Gbps
    };
}

% Multi-stage attack chain visualization
% Usage: \drawAttackChain{stage1_node}{stage2_node}{stage3_node}{stage4_node}
\newcommand{\drawAttackChain}[4]{
    % Draw connections between stages
    \draw[draw=red!70, line width=3pt, -{Stealth[length=4mm]}]
        (#1) -- (#2) node[midway, above, font=\tiny, fill=white] {Stage 1→2};
    \draw[draw=red!80, line width=3pt, -{Stealth[length=4mm]}]
        (#2) -- (#3) node[midway, above, font=\tiny, fill=white] {Stage 2→3};
    \draw[draw=red!90, line width=3pt, -{Stealth[length=4mm]}]
        (#3) -- (#4) node[midway, above, font=\tiny, fill=white] {Stage 3→4};

    % Mark stages
    \node[circle, fill=orange, text=white, font=\tiny\bfseries,
          minimum size=0.4cm, anchor=north west] at (#1.north west) {1};
    \node[circle, fill=red!70, text=white, font=\tiny\bfseries,
          minimum size=0.4cm, anchor=north west] at (#2.north west) {2};
    \node[circle, fill=red!85, text=white, font=\tiny\bfseries,
          minimum size=0.4cm, anchor=north west] at (#3.north west) {3};
    \node[circle, fill=red!95, text=white, font=\tiny\bfseries,
          minimum size=0.4cm, anchor=north west] at (#4.north west) {4};
}

% Risk-based prioritization indicator
% Usage: \drawRiskPriority{x}{y}{critical}{high}{medium}{low}
\newcommand{\drawRiskPriority}[6]{
    \node[legend box, anchor=north west, minimum width=4.5cm] at (#1,#2) {
        \begin{tabular}{lcc}
            \multicolumn{3}{c}{\textbf{Risk Prioritization}} \\
            \hline
            \textbf{Severity} & \textbf{Count} & \textbf{MTTR} \\
            \hline
            \textcolor{threatCritical}{Critical} & #3 & <4h \\
            \textcolor{threatHigh}{High} & #4 & <24h \\
            \textcolor{threatMedium}{Medium} & #5 & <7d \\
            \textcolor{threatLow}{Low} & #6 & <30d \\
            \hline
            \textbf{Total} & \textbf{\pgfmathparse{#3+#4+#5+#6}\pgfmathprintnumber[fixed, precision=0]{\pgfmathresult}} & \\
        \end{tabular}
    };
}

% Threat intelligence summary panel
% Usage: \drawThreatSummary{x}{y}
\newcommand{\drawThreatSummary}[2]{
    \node[legend box, anchor=north west, minimum width=7cm] at (#1,#2) {
        \begin{tabular}{ll}
            \multicolumn{2}{c}{\textbf{Threat Intelligence Summary}} \\
            \hline
            \textbf{Active Threats:} & 0 \\
            \textbf{Vulnerabilities:} & 0 (0 Critical) \\
            \textbf{IOCs Detected:} & 0 \\
            \textbf{ATT\&CK Techniques:} & 0 \\
            \textbf{Threat Actors:} & None Identified \\
            \textbf{Risk Score:} & 0/100 \\
            \hline
            \textbf{Status:} & \textcolor{green}{MONITORING} \\
            \textbf{Last Update:} & 2025-01-16 00:00 UTC \\
            \hline
            \multicolumn{2}{l}{\tiny Update via network data file} \\
        \end{tabular}
    };
}

% Comprehensive security status dashboard
% Usage: \drawSecurityStatus{x}{y}{overall_score}
\newcommand{\drawSecurityStatus}[3]{
    \pgfmathparse{#3 >= 80 ? 1 : 0}
    \ifnum\pgfmathresult=1
        \def\secColor{green}
        \def\secStatus{SECURE}
    \else
        \pgfmathparse{#3 >= 50 ? 1 : 0}
        \ifnum\pgfmathresult=1
            \def\secColor{orange}
            \def\secStatus{AT RISK}
        \else
            \def\secColor{red}
            \def\secStatus{CRITICAL}
        \fi
    \fi

    \node[draw=\secColor, fill=\secColor!15, line width=3pt,
          rounded corners=4pt, inner sep=6pt, minimum width=6cm,
          anchor=north west] at (#1,#2) {
        \begin{tabular}{ll}
            \multicolumn{2}{c}{\Large\bfseries SECURITY STATUS} \\
            \hline
            \textbf{Overall Score:} & \huge\textcolor{\secColor}{#3/100} \\
            \textbf{Status:} & \Large\textcolor{\secColor}{\secStatus} \\
        \end{tabular}
    };
}

% Incident response status
% Usage: \drawIncidentStatus{x}{y}{incident_id}{status}{severity}
\newcommand{\drawIncidentStatus}[5]{
    \ifthenelse{\equal{#4}{investigating}}{
        \def\statColor{orange}
        \def\statText{INVESTIGATING}
    }{
    \ifthenelse{\equal{#4}{contained}}{
        \def\statColor{blue}
        \def\statText{CONTAINED}
    }{
    \ifthenelse{\equal{#4}{resolved}}{
        \def\statColor{green}
        \def\statText{RESOLVED}
    }{
        \def\statColor{red}
        \def\statText{ACTIVE}
    }}}

    \node[draw=\statColor, fill=\statColor!20, line width=2pt,
          rounded corners=3pt, inner sep=4pt, minimum width=4.5cm,
          anchor=north west] at (#1,#2) {
        \small\bfseries Incident #3 \\
        \tiny Severity: #5 \\
        \textcolor{\statColor}{\small\bfseries\statText}
    };
}

% TODO: Future intelligent features
% - ML-based threat scoring
% - Automated attack path analysis
% - Predictive threat modeling
% - Integration with SIEM platforms
% - Real-time threat feed correlation

% ============================================================================
% MAIN THREAT RENDERING ENGINE
% ============================================================================

\newcommand{\renderThreats}{
    % This will be populated by network_data.tex
    % Example structure:
    % \visualizeDDoS{attacker1,attacker2}{srv1}{critical}
    % \markVulnerability{srv2}{CVE-2024-1234}{9.8}
    % \scenarioRansomware{pc1}{pc2,pc3,srv1}{$50,000 BTC}
    % \scenarioAPT{srv1}{attacker1}{database1}{APT29}{SolarStorm}
}

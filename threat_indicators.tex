% threat_indicators.tex - Security threat visualization and indicators
% This module handles threat detection, attack visualization, and security status

% ============================================================================
% THREAT LEVEL DEFINITIONS
% ============================================================================

\newcommand{\threatCriticalLevel}{5}
\newcommand{\threatHighLevel}{4}
\newcommand{\threatMediumLevel}{3}
\newcommand{\threatLowLevel}{2}
\newcommand{\threatInfoLevel}{1}

% TODO: Threat scoring system
% - CVSS score integration
% - Custom threat scoring algorithms
% - Risk = Likelihood × Impact calculations
% - Temporal scoring (degrading over time)
% - Environmental scoring based on context

% ============================================================================
% THREAT INDICATOR VISUALIZATION
% ============================================================================

% Draw threat indicator icon
% Usage: \drawThreatIndicator{x}{y}{level}{type}
\newcommand{\drawThreatIndicator}[4]{
    \ifthenelse{\equal{#3}{critical}}{
        \def\threatcolor{threatCritical}
        \def\threatsize{0.5}
    }{
    \ifthenelse{\equal{#3}{high}}{
        \def\threatcolor{threatHigh}
        \def\threatsize{0.4}
    }{
    \ifthenelse{\equal{#3}{medium}}{
        \def\threatcolor{threatMedium}
        \def\threatsize{0.35}
    }{
    \ifthenelse{\equal{#3}{low}}{
        \def\threatcolor{threatLow}
        \def\threatsize}{0.3}
    }{
        \def\threatcolor{threatInfo}
        \def\threatsize{0.25}
    }}}}
    
    \node[
        regular polygon,
        regular polygon sides=3,
        fill=\threatcolor,
        draw=\threatcolor!80,
        minimum size=\threatsize cm,
        inner sep=0pt
    ] at (#1,#2) {};
    
    \node[font=\tiny\bfseries\sffamily, text=white] at (#1,#2) {!};
    
    \node[below, font=\tiny\sffamily] at (#1,#2-0.15) {#4};
}

% TODO: Enhanced threat indicators
% - Animated pulsing for active threats
% - Different icon shapes for different threat types
% - Severity gradient visualization
% - Threat trend arrows (increasing/decreasing)
% - Time-to-remediation countdown

% ============================================================================
% ATTACK VISUALIZATION
% ============================================================================

% Visualize specific attack types
% Usage: \visualizeAttack{attacker}{target}{attack_type}{severity}

% DDoS Attack visualization
\newcommand{\visualizeDDoS}[3]{
    % #1 = attacker list (comma-separated)
    % #2 = target
    % #3 = severity
    \foreach \attacker in {#1} {
        \draw[attack conn, line width=2pt] (\attacker) -- (#2);
    }
    \node[draw=threatCritical, line width=3pt, circle, 
          minimum size=1.5cm, fill=threatCritical!20] at (#2) {};
    \node[above, font=\small\bfseries, text=threatCritical] 
        at (#2.north) [yshift=0.8cm] {DDoS ATTACK};
}

% SQL Injection visualization
\newcommand{\visualizeSQLi}[2]{
    % #1 = attacker
    % #2 = database server
    \draw[attack conn, line width=2pt] (#1) -- (#2)
        node[midway, threat label] {SQL Injection};
    \node[draw=threatHigh, star, star points=8, 
          minimum size=0.8cm, fill=threatHigh!30] at (#2.north east) {};
}

% Malware/Ransomware visualization
\newcommand{\visualizeMalware}[2]{
    % #1 = infected node
    % #2 = malware type
    \node[draw=threatCritical, line width=3pt, 
          rounded corners=3pt, inner sep=8pt,
          fill=threatCritical!15, dashed] at (#1) {};
    \node[above, font=\tiny\bfseries, text=threatCritical, 
          fill=white, inner sep=2pt] at (#1.north) {#2};
}

% Data exfiltration visualization
\newcommand{\visualizeExfiltration}[3]{
    % #1 = source (compromised node)
    % #2 = destination (attacker)
    % #3 = data amount
    \draw[attack conn, line width=3pt] (#1) -- (#2)
        node[midway, above, threat label] {Exfil: #3};
    \draw[draw=threatCritical, line width=2pt, dashed, 
          -{Stealth[length=5mm]}] (#1) -- (#2);
}

% TODO: Additional attack visualizations
% - Phishing attack chain
% - Privilege escalation path
% - Lateral movement tracking
% - Command & Control beaconing
% - Brute force attempts (with attempt counter)
% - Zero-day exploitation indicators
% - Supply chain attack visualization

% ============================================================================
% ATTACK KILL CHAIN PROGRESSION (LOCKHEED MARTIN)
% ============================================================================

% Show attack kill chain progression with 7 stages
% Usage: \drawKillChain{x}{y}{current_stage}{defenses}
% Stages: 1=Reconnaissance, 2=Weaponization, 3=Delivery, 4=Exploitation,
%         5=Installation, 6=Command & Control, 7=Actions on Objectives
% Defenses: comma-separated list of stage numbers with defensive controls
\newcommand{\drawKillChain}[4]{
    \pgfmathsetmacro{\boxwidth}{3.5}
    \pgfmathsetmacro{\boxheight}{0.6}
    \pgfmathsetmacro{\xpos}{#1}
    \pgfmathsetmacro{\ypos}{#2}

    \node[anchor=north west, font=\small\bfseries] at (\xpos, \ypos) {Cyber Kill Chain};
    \pgfmathsetmacro{\ypos}{\ypos - 0.5}

    % Stage 1: Reconnaissance
    \pgfmathsetmacro{\currentstage}{#3}
    \ifthenelse{\lengthtest{\currentstage pt > 0.9 pt}}{
        \def\stagecolor{threatCritical}
        \def\stagelabel{ACTIVE}
    }{\def\stagecolor{gray!30}\def\stagelabel{}}

    \node[draw=\stagecolor, fill=\stagecolor!20, minimum width=\boxwidth cm,
          minimum height=\boxheight cm, anchor=north west] at (\xpos, \ypos) {
        \small\textbf{1. Reconnaissance} \tiny\stagelabel
    };
    \pgfmathsetmacro{\ypos}{\ypos - \boxheight - 0.05}

    % Stage 2: Weaponization
    \ifthenelse{\lengthtest{\currentstage pt > 1.9 pt}}{
        \def\stagecolor{threatCritical}
        \def\stagelabel{ACTIVE}
    }{\def\stagecolor{gray!30}\def\stagelabel{}}

    \node[draw=\stagecolor, fill=\stagecolor!20, minimum width=\boxwidth cm,
          minimum height=\boxheight cm, anchor=north west] at (\xpos, \ypos) {
        \small\textbf{2. Weaponization} \tiny\stagelabel
    };
    \pgfmathsetmacro{\ypos}{\ypos - \boxheight - 0.05}

    % Stage 3: Delivery
    \ifthenelse{\lengthtest{\currentstage pt > 2.9 pt}}{
        \def\stagecolor{threatCritical}
        \def\stagelabel{ACTIVE}
    }{\def\stagecolor{gray!30}\def\stagelabel{}}

    \node[draw=\stagecolor, fill=\stagecolor!20, minimum width=\boxwidth cm,
          minimum height=\boxheight cm, anchor=north west] at (\xpos, \ypos) {
        \small\textbf{3. Delivery} \tiny\stagelabel
    };
    \pgfmathsetmacro{\ypos}{\ypos - \boxheight - 0.05}

    % Stage 4: Exploitation
    \ifthenelse{\lengthtest{\currentstage pt > 3.9 pt}}{
        \def\stagecolor{threatCritical}
        \def\stagelabel{ACTIVE}
    }{\def\stagecolor{gray!30}\def\stagelabel{}}

    \node[draw=\stagecolor, fill=\stagecolor!20, minimum width=\boxwidth cm,
          minimum height=\boxheight cm, anchor=north west] at (\xpos, \ypos) {
        \small\textbf{4. Exploitation} \tiny\stagelabel
    };
    \pgfmathsetmacro{\ypos}{\ypos - \boxheight - 0.05}

    % Stage 5: Installation
    \ifthenelse{\lengthtest{\currentstage pt > 4.9 pt}}{
        \def\stagecolor{threatCritical}
        \def\stagelabel{ACTIVE}
    }{\def\stagecolor{gray!30}\def\stagelabel{}}

    \node[draw=\stagecolor, fill=\stagecolor!20, minimum width=\boxwidth cm,
          minimum height=\boxheight cm, anchor=north west] at (\xpos, \ypos) {
        \small\textbf{5. Installation} \tiny\stagelabel
    };
    \pgfmathsetmacro{\ypos}{\ypos - \boxheight - 0.05}

    % Stage 6: Command & Control
    \ifthenelse{\lengthtest{\currentstage pt > 5.9 pt}}{
        \def\stagecolor{threatCritical}
        \def\stagelabel{ACTIVE}
    }{\def\stagecolor{gray!30}\def\stagelabel{}}

    \node[draw=\stagecolor, fill=\stagecolor!20, minimum width=\boxwidth cm,
          minimum height=\boxheight cm, anchor=north west] at (\xpos, \ypos) {
        \small\textbf{6. Command \& Control} \tiny\stagelabel
    };
    \pgfmathsetmacro{\ypos}{\ypos - \boxheight - 0.05}

    % Stage 7: Actions on Objectives
    \ifthenelse{\lengthtest{\currentstage pt > 6.9 pt}}{
        \def\stagecolor{threatCritical}
        \def\stagelabel{ACTIVE}
    }{\def\stagecolor{gray!30}\def\stagelabel{}}

    \node[draw=\stagecolor, fill=\stagecolor!20, minimum width=\boxwidth cm,
          minimum height=\boxheight cm, anchor=north west] at (\xpos, \ypos) {
        \small\textbf{7. Actions on Objectives} \tiny\stagelabel
    };
}

% Show defensive gaps in kill chain
% Usage: \drawDefensiveGaps{x}{y}{protected_stages}{vulnerable_stages}
\newcommand{\drawDefensiveGaps}[4]{
    \node[legend box, anchor=north west, minimum width=4cm,
          draw=threatMedium, line width=1.5pt] at (#1,#2) {
        \begin{tabular}{p{3.5cm}}
            \multicolumn{1}{c}{\textbf{Defensive Posture}} \\
            \hline
            \textcolor{green!60!black}{\textbf{Protected Stages:}} \\
            #3 \\
            \hline
            \textcolor{threatHigh}{\textbf{Vulnerable Stages:}} \\
            #4 \\
        \end{tabular}
    };
}

% NIST Cybersecurity Framework mapping
% Usage: \drawNISTMapping{x}{y}{identify}{protect}{detect}{respond}{recover}
\newcommand{\drawNISTMapping}[7]{
    \node[legend box, anchor=north west, minimum width=5cm,
          draw=blue!70, line width=2pt] at (#1,#2) {
        \begin{tabular}{lc}
            \multicolumn{2}{c}{\textbf{NIST CSF Coverage}} \\
            \hline
            \textbf{Identify} & #3\% \\
            \textbf{Protect} & #4\% \\
            \textbf{Detect} & #5\% \\
            \textbf{Respond} & #6\% \\
            \textbf{Recover} & #7\% \\
        \end{tabular}
    };
}

% Attack timeline visualization
% Usage: \drawAttackTimeline{x}{y}{start_time}{current_time}{events}
\newcommand{\drawAttackTimeline}[5]{
    \node[legend box, anchor=north west, minimum width=6cm,
          draw=threatHigh, line width=1.5pt] at (#1,#2) {
        \begin{tabular}{ll}
            \multicolumn{2}{c}{\textbf{Attack Timeline}} \\
            \hline
            \textbf{Attack Started:} & #3 \\
            \textbf{Current Time:} & #4 \\
            \hline
            \multicolumn{2}{l}{\textbf{Key Events:}} \\
            \multicolumn{2}{p{5.5cm}}{#5} \\
        \end{tabular}
    };
}

% Attack path visualization
% Usage: \drawAttackPath{source}{intermediate}{target}{stage}
\newcommand{\drawAttackPath}[4]{
    \draw[attack conn, line width=3pt, -{Stealth[length=5mm]}]
        (#1) -- (#2) node[midway, above, font=\tiny, fill=white, inner sep=1pt] {Stage #4};
    \draw[attack conn, line width=3pt, -{Stealth[length=5mm]}]
        (#2) -- (#3) node[midway, above, font=\tiny, fill=white, inner sep=1pt] {Stage #4};
}

% ============================================================================
% SECURITY ZONES AND BOUNDARIES
% ============================================================================

% Mark security boundary breach
% Usage: \markBoundaryBreach{zone1}{zone2}{breach_point}
\newcommand{\markBoundaryBreach}[3]{
    \draw[draw=threatCritical, line width=3pt, 
          decoration={zigzag, segment length=4pt, amplitude=2pt},
          decorate] (#1) -- (#2);
    \node[circle, fill=threatCritical, minimum size=0.4cm] at (#3) {};
    \node[above, font=\tiny\bfseries, text=white] at (#3) {BREACH};
}

% Draw firewall bypass indicator
\newcommand{\showFirewallBypass}[2]{
    % #1 = firewall node
    % #2 = bypass method
    \node[draw=threatHigh, cross out, line width=2pt, 
          minimum size=1cm, inner sep=0pt] at (#1) {};
    \node[below, font=\tiny, text=threatHigh] at (#1.south) {Bypassed: #2};
}

% TODO: Security control visualization
% - IDS/IPS alert indicators
% - Failed authentication attempts
% - Access control violations
% - Encryption status (enabled/disabled/weak)
% - Patch status indicators

% ============================================================================
% VULNERABILITY INDICATORS
% ============================================================================

% Mark vulnerable node with CVE
% Usage: \markVulnerability{node}{cve}{cvss_score}
\newcommand{\markVulnerability}[3]{
    \pgfmathsetmacro{\severity}{#3/10}
    \ifthenelse{\lengthtest{\severity pt > 0.7 pt}}{
        \def\vulncolor{threatCritical}
    }{
    \ifthenelse{\lengthtest{\severity pt > 0.4 pt}}{
        \def\vulncolor{threatMedium}
    }{
        \def\vulncolor{threatLow}
    }}
    
    \node[draw=\vulncolor, line width=2pt, rectangle,
          rounded corners=2pt, inner sep=3pt, fill=\vulncolor!20,
          anchor=south west] at (#1.south west) {
        \tiny\ttfamily #2: #3
    };
}

% Show exploitable service
% Usage: \markExploitableService{node}{service}{port}
\newcommand{\markExploitableService}[3]{
    \node[draw=threatHigh, fill=threatHigh!20, 
          font=\tiny\ttfamily, anchor=north east] 
        at (#1.north east) {#2:#3};
}

% ============================================================================
% ENHANCED CVSS SCORE VISUALIZATION
% ============================================================================

% Parse CVSS vector and display comprehensive score breakdown
% Usage: \drawCVSSScore{x}{y}{cve}{base_score}{temporal_score}{environmental_score}{vector}
\newcommand{\drawCVSSScore}[7]{
    % Determine severity level based on base score
    \pgfmathsetmacro{\basescore}{#4}
    \ifthenelse{\lengthtest{\basescore pt > 9.0 pt}}{
        \def\cvsscolor{threatCritical}
        \def\severitylabel{CRITICAL}
    }{
    \ifthenelse{\lengthtest{\basescore pt > 7.0 pt}}{
        \def\cvsscolor{threatHigh}
        \def\severitylabel{HIGH}
    }{
    \ifthenelse{\lengthtest{\basescore pt > 4.0 pt}}{
        \def\cvsscolor{threatMedium}
        \def\severitylabel{MEDIUM}
    }{
    \ifthenelse{\lengthtest{\basescore pt > 0.1 pt}}{
        \def\cvsscolor{threatLow}
        \def\severitylabel{LOW}
    }{
        \def\cvsscolor{threatInfo}
        \def\severitylabel{INFO}
    }}}}

    % Draw CVSS score box
    \node[legend box, anchor=north west, minimum width=3.5cm,
          draw=\cvsscolor, line width=2pt] at (#1,#2) {
        \begin{tabular}{ll}
            \multicolumn{2}{c}{\textbf{\textcolor{\cvsscolor}{#3}}} \\
            \hline
            \textbf{Severity:} & \textcolor{\cvsscolor}{\severitylabel} \\
            \textbf{Base:} & #4 \\
            \textbf{Temporal:} & #5 \\
            \textbf{Environmental:} & #6 \\
            \hline
            \multicolumn{2}{l}{\tiny\ttfamily #7} \\
        \end{tabular}
    };
}

% Compact CVSS badge for node overlay
% Usage: \cvssbadge{node}{cve}{score}
\newcommand{\cvssbadge}[3]{
    \pgfmathsetmacro{\score}{#3}
    \ifthenelse{\lengthtest{\score pt > 9.0 pt}}{
        \def\badgecolor{threatCritical}
    }{
    \ifthenelse{\lengthtest{\score pt > 7.0 pt}}{
        \def\badgecolor{threatHigh}
    }{
    \ifthenelse{\lengthtest{\score pt > 4.0 pt}}{
        \def\badgecolor{threatMedium}
    }{
        \def\badgecolor{threatLow}
    }}}

    \node[draw=\badgecolor, fill=\badgecolor, text=white,
          font=\tiny\bfseries, rounded corners=2pt, inner sep=2pt,
          anchor=north west] at (#1.north west) {#3};
    \node[draw=\badgecolor, fill=white, text=\badgecolor,
          font=\tiny\ttfamily, rounded corners=2pt, inner sep=2pt,
          anchor=south west] at (#1.south west) {#2};
}

% CVSS metrics breakdown visualization
% Usage: \drawCVSSMetrics{x}{y}{AV}{AC}{PR}{UI}{S}{C}{I}{A}
% AV=Attack Vector, AC=Attack Complexity, PR=Privileges Required,
% UI=User Interaction, S=Scope, C=Confidentiality, I=Integrity, A=Availability
\newcommand{\drawCVSSMetrics}[11]{
    \node[legend box, anchor=north west, minimum width=4cm] at (#1,#2) {
        \begin{tabular}{ll}
            \multicolumn{2}{c}{\textbf{CVSS v3.1 Metrics}} \\
            \hline
            \textbf{Attack Vector:} & #3 \\
            \textbf{Attack Complexity:} & #4 \\
            \textbf{Privileges Required:} & #5 \\
            \textbf{User Interaction:} & #6 \\
            \textbf{Scope:} & #7 \\
            \hline
            \textbf{Confidentiality:} & \textcolor{threatHigh}{#8} \\
            \textbf{Integrity:} & \textcolor{threatMedium}{#9} \\
            \textbf{Availability:} & \textcolor{threatMedium}{#{10}} \\
        \end{tabular}
    };
}

% ============================================================================
% MITRE ATT&CK FRAMEWORK MAPPING
% ============================================================================

% Display MITRE ATT&CK technique
% Usage: \drawMITREAttack{x}{y}{technique_id}{technique_name}{tactic}
\newcommand{\drawMITREAttack}[5]{
    \node[legend box, anchor=north west, minimum width=4.5cm,
          draw=threatHigh, line width=1.5pt] at (#1,#2) {
        \begin{tabular}{ll}
            \multicolumn{2}{c}{\textbf{\textcolor{threatHigh}{MITRE ATT\&CK}}} \\
            \hline
            \textbf{Technique:} & \texttt{#3} \\
            \textbf{Name:} & #4 \\
            \textbf{Tactic:} & \textcolor{threatMedium}{#5} \\
            \hline
            \multicolumn{2}{l}{\tiny attack.mitre.org/techniques/#3} \\
        \end{tabular}
    };
}

% MITRE ATT&CK badge for nodes
% Usage: \mitrebadge{node}{technique_id}
\newcommand{\mitrebadge}[2]{
    \node[draw=threatHigh, fill=threatHigh, text=white,
          font=\tiny\bfseries\ttfamily, rounded corners=1pt, inner sep=1pt,
          anchor=north east] at (#1.north east) [xshift=-0.1cm, yshift=-0.1cm] {#2};
}

% Draw MITRE ATT&CK kill chain progression
% Usage: \drawMITREKillChain{x}{y}{current_stage}
% Stages: Reconnaissance, Resource Development, Initial Access, Execution,
%         Persistence, Privilege Escalation, Defense Evasion, Credential Access,
%         Discovery, Lateral Movement, Collection, Command & Control, Exfiltration,
%         Impact
\newcommand{\drawMITREKillChain}[3]{
    \pgfmathsetmacro{\boxwidth}{1.8}
    \pgfmathsetmacro{\boxheight}{0.6}
    \pgfmathsetmacro{\ypos}{#2}

    \node[anchor=north west, font=\small\bfseries] at (#1, \ypos) {MITRE ATT\&CK Kill Chain};
    \pgfmathsetmacro{\ypos}{\ypos - 0.4}

    % Stage 1: Reconnaissance
    \ifthenelse{\equal{#3}{1} \OR \equal{#3}{Reconnaissance}}{
        \def\stagecolor{threatCritical}
    }{\def\stagecolor{gray!30}}
    \node[draw=\stagecolor, fill=\stagecolor!20, minimum width=\boxwidth cm,
          minimum height=\boxheight cm, font=\tiny, anchor=north west]
          at (#1, \ypos) {Reconnaissance};

    \pgfmathsetmacro{\ypos}{\ypos - \boxheight - 0.1}

    % Stage 2: Initial Access
    \ifthenelse{\equal{#3}{2} \OR \equal{#3}{Initial Access}}{
        \def\stagecolor{threatCritical}
    }{\def\stagecolor{gray!30}}
    \node[draw=\stagecolor, fill=\stagecolor!20, minimum width=\boxwidth cm,
          minimum height=\boxheight cm, font=\tiny, anchor=north west]
          at (#1, \ypos) {Initial Access};

    \pgfmathsetmacro{\ypos}{\ypos - \boxheight - 0.1}

    % Stage 3: Execution
    \ifthenelse{\equal{#3}{3} \OR \equal{#3}{Execution}}{
        \def\stagecolor{threatCritical}
    }{\def\stagecolor{gray!30}}
    \node[draw=\stagecolor, fill=\stagecolor!20, minimum width=\boxwidth cm,
          minimum height=\boxheight cm, font=\tiny, anchor=north west]
          at (#1, \ypos) {Execution};

    \pgfmathsetmacro{\ypos}{\ypos - \boxheight - 0.1}

    % Stage 4: Persistence
    \ifthenelse{\equal{#3}{4} \OR \equal{#3}{Persistence}}{
        \def\stagecolor{threatCritical}
    }{\def\stagecolor{gray!30}}
    \node[draw=\stagecolor, fill=\stagecolor!20, minimum width=\boxwidth cm,
          minimum height=\boxheight cm, font=\tiny, anchor=north west]
          at (#1, \ypos) {Persistence};

    \pgfmathsetmacro{\ypos}{\ypos - \boxheight - 0.1}

    % Stage 5: Privilege Escalation
    \ifthenelse{\equal{#3}{5} \OR \equal{#3}{Privilege Escalation}}{
        \def\stagecolor{threatCritical}
    }{\def\stagecolor{gray!30}}
    \node[draw=\stagecolor, fill=\stagecolor!20, minimum width=\boxwidth cm,
          minimum height=\boxheight cm, font=\tiny, anchor=north west]
          at (#1, \ypos) {Privilege Esc.};

    \pgfmathsetmacro{\ypos}{\ypos - \boxheight - 0.1}

    % Stage 6: Defense Evasion
    \ifthenelse{\equal{#3}{6} \OR \equal{#3}{Defense Evasion}}{
        \def\stagecolor{threatCritical}
    }{\def\stagecolor{gray!30}}
    \node[draw=\stagecolor, fill=\stagecolor!20, minimum width=\boxwidth cm,
          minimum height=\boxheight cm, font=\tiny, anchor=north west]
          at (#1, \ypos) {Defense Evasion};

    \pgfmathsetmacro{\ypos}{\ypos - \boxheight - 0.1}

    % Stage 7: Lateral Movement
    \ifthenelse{\equal{#3}{7} \OR \equal{#3}{Lateral Movement}}{
        \def\stagecolor{threatCritical}
    }{\def\stagecolor{gray!30}}
    \node[draw=\stagecolor, fill=\stagecolor!20, minimum width=\boxwidth cm,
          minimum height=\boxheight cm, font=\tiny, anchor=north west]
          at (#1, \ypos) {Lateral Movement};

    \pgfmathsetmacro{\ypos}{\ypos - \boxheight - 0.1}

    % Stage 8: Collection & Exfiltration
    \ifthenelse{\equal{#3}{8} \OR \equal{#3}{Exfiltration}}{
        \def\stagecolor{threatCritical}
    }{\def\stagecolor{gray!30}}
    \node[draw=\stagecolor, fill=\stagecolor!20, minimum width=\boxwidth cm,
          minimum height=\boxheight cm, font=\tiny, anchor=north west]
          at (#1, \ypos) {Exfiltration};

    \pgfmathsetmacro{\ypos}{\ypos - \boxheight - 0.1}

    % Stage 9: Impact
    \ifthenelse{\equal{#3}{9} \OR \equal{#3}{Impact}}{
        \def\stagecolor{threatCritical}
    }{\def\stagecolor{gray!30}}
    \node[draw=\stagecolor, fill=\stagecolor!20, minimum width=\boxwidth cm,
          minimum height=\boxheight cm, font=\tiny, anchor=north west]
          at (#1, \ypos) {Impact};
}

% Map technique to tactic with visual connection
% Usage: \mapTechniqueToTactic{node}{technique_id}{tactic_name}
\newcommand{\mapTechniqueToTactic}[3]{
    \node[above, font=\tiny\bfseries\ttfamily, text=threatHigh,
          fill=white, draw=threatHigh, rounded corners=1pt,
          inner sep=1pt] at (#1.north) [yshift=0.3cm] {
        #2: #3
    };
}

% Show multiple MITRE techniques for complex attack
% Usage: \drawMITRETTP{x}{y}{title}{techniques_list}
\newcommand{\drawMITRETTP}[4]{
    \node[legend box, anchor=north west, minimum width=5cm,
          draw=threatHigh, line width=1.5pt, fill=white] at (#1,#2) {
        \begin{tabular}{p{4.5cm}}
            \multicolumn{1}{c}{\textbf{\textcolor{threatHigh}{#3}}} \\
            \hline
            \textbf{Techniques Used:} \\
            #4 \\
        \end{tabular}
    };
}

% ============================================================================
% ENHANCED THREAT ACTOR ATTRIBUTION
% ============================================================================

% Mark threat actor with attribution
% Usage: \markThreatActor{node}{actor_name}{confidence}
\newcommand{\markThreatActor}[3]{
    \node[above, font=\small\bfseries, text=threatCritical,
          fill=white, draw=threatCritical, rounded corners=2pt,
          inner sep=3pt] at (#1.north) [yshift=0.5cm] {
        #2 (Confidence: #3\%)
    };
}

% Comprehensive threat actor profile
% Usage: \drawThreatActorProfile{x}{y}{actor_name}{aka}{origin}{motivation}{confidence}
\newcommand{\drawThreatActorProfile}[7]{
    \pgfmathsetmacro{\confpct}{#7}
    \ifthenelse{\lengthtest{\confpct pt > 80 pt}}{
        \def\confcolor{threatCritical}
    }{
    \ifthenelse{\lengthtest{\confpct pt > 50 pt}}{
        \def\confcolor{threatHigh}
    }{
        \def\confcolor{threatMedium}
    }}

    \node[legend box, anchor=north west, minimum width=6cm,
          draw=\confcolor, line width=2pt] at (#1,#2) {
        \begin{tabular}{ll}
            \multicolumn{2}{c}{\textbf{\textcolor{\confcolor}{Threat Actor Profile}}} \\
            \hline
            \textbf{Primary Name:} & #3 \\
            \textbf{Also Known As:} & \textit{#4} \\
            \textbf{Origin:} & #5 \\
            \textbf{Motivation:} & #6 \\
            \hline
            \textbf{Confidence:} & \textcolor{\confcolor}{#7\%} \\
        \end{tabular}
    };
}

% TTP (Tactics, Techniques, Procedures) overlay
% Usage: \drawTTPProfile{x}{y}{actor_name}{preferred_techniques}{tools_used}
\newcommand{\drawTTPProfile}[5]{
    \node[legend box, anchor=north west, minimum width=6cm,
          draw=threatHigh, line width=1.5pt] at (#1,#2) {
        \begin{tabular}{p{5.5cm}}
            \multicolumn{1}{c}{\textbf{TTP Profile: #3}} \\
            \hline
            \textbf{Preferred Techniques:} \\
            #4 \\
            \hline
            \textbf{Known Tools:} \\
            #5 \\
        \end{tabular}
    };
}

% Campaign tracking visualization
% Usage: \drawCampaignTracker{x}{y}{campaign_name}{start_date}{targets}{status}
\newcommand{\drawCampaignTracker}[6]{
    \ifthenelse{\equal{#6}{active}}{
        \def\statuscolor{threatCritical}
    }{
    \ifthenelse{\equal{#6}{ongoing}}{
        \def\statuscolor{threatHigh}
    }{
        \def\statuscolor{gray!50}
    }}

    \node[legend box, anchor=north west, minimum width=5.5cm,
          draw=\statuscolor, line width=2pt] at (#1,#2) {
        \begin{tabular}{ll}
            \multicolumn{2}{c}{\textbf{\textcolor{\statuscolor}{Campaign: #3}}} \\
            \hline
            \textbf{Started:} & #4 \\
            \textbf{Targets:} & #5 \\
            \textbf{Status:} & \textcolor{\statuscolor}{\uppercase{#6}} \\
        \end{tabular}
    };
}

% Threat actor attribution badge with confidence indicator
% Usage: \actorBadge{node}{actor_name}{confidence}
\newcommand{\actorBadge}[3]{
    \pgfmathsetmacro{\conf}{#3}
    \ifthenelse{\lengthtest{\conf pt > 70 pt}}{
        \def\badgecolor{red!80!black}
    }{
    \ifthenelse{\lengthtest{\conf pt > 40 pt}}{
        \def\badgecolor{orange!80!black}
    }{
        \def\badgecolor{yellow!80!black}
    }}

    \node[draw=\badgecolor, fill=\badgecolor, text=white,
          font=\tiny\bfseries, rounded corners=2pt, inner sep=2pt,
          anchor=north east] at (#1.north east) [xshift=-0.1cm] {
        #2: #3\%
    };
}

% Multiple attribution (different analysts have different conclusions)
% Usage: \drawAttributionDebate{x}{y}{target}{analyst1_conclusion}{analyst2_conclusion}
\newcommand{\drawAttributionDebate}[5]{
    \node[legend box, anchor=north west, minimum width=5cm,
          draw=orange, line width=1.5pt] at (#1,#2) {
        \begin{tabular}{p{4.5cm}}
            \multicolumn{1}{c}{\textbf{Attribution Debate: #3}} \\
            \hline
            \textcolor{threatHigh}{Analyst A:} #4 \\
            \hline
            \textcolor{threatMedium}{Analyst B:} #5 \\
        \end{tabular}
    };
}

% ============================================================================
% ENHANCED IOC (INDICATORS OF COMPROMISE) VISUALIZATION
% ============================================================================

% Show threat intelligence indicators with type-specific formatting
% Usage: \markIOC{node}{ioc_type}{value}{reputation_score}
\newcommand{\markIOC}[4]{
    % IOC = Indicator of Compromise
    % Determine color based on reputation score (0-100, higher is worse)
    \pgfmathsetmacro{\repscore}{#4}
    \ifthenelse{\lengthtest{\repscore pt > 80 pt}}{
        \def\ioccolor{threatCritical}
    }{
    \ifthenelse{\lengthtest{\repscore pt > 60 pt}}{
        \def\ioccolor{threatHigh}
    }{
    \ifthenelse{\lengthtest{\repscore pt > 40 pt}}{
        \def\ioccolor{threatMedium}
    }{
        \def\ioccolor{threatLow}
    }}}

    \node[font=\tiny\ttfamily, fill=\ioccolor!20,
          draw=\ioccolor, inner sep=2pt, anchor=south,
          line width=1pt]
        at (#1.south) [yshift=-0.3cm] {
        #2: #3 (Rep: #4)
    };
}

% Display malicious IP address indicator
% Usage: \markMaliciousIP{node}{ip_address}{reputation}{threat_feed}
\newcommand{\markMaliciousIP}[4]{
    \pgfmathsetmacro{\repscore}{#3}
    \ifthenelse{\lengthtest{\repscore pt > 80 pt}}{
        \def\ipcolor{threatCritical}
    }{\def\ipcolor{threatHigh}}

    \node[font=\tiny\ttfamily, fill=\ipcolor!30,
          draw=\ipcolor, line width=2pt, inner sep=2pt,
          anchor=south west, rounded corners=1pt]
        at (#1.south west) [yshift=-0.4cm] {
        \textbf{Malicious IP:} #2
    };
    \node[font=\tiny, fill=white, draw=\ipcolor,
          inner sep=1pt, anchor=north west]
        at (#1.south west) [yshift=-0.7cm] {
        Source: #4 | Rep: #3/100
    };
}

% Display malicious domain indicator
% Usage: \markMaliciousDomain{node}{domain}{reputation}{category}
\newcommand{\markMaliciousDomain}[4]{
    \node[font=\tiny\ttfamily, fill=threatHigh!30,
          draw=threatHigh, line width=1.5pt, inner sep=2pt,
          anchor=north, rounded corners=1pt]
        at (#1.north) [yshift=0.4cm] {
        \textbf{Malicious Domain:} #2
    };
    \node[font=\tiny, fill=white, draw=threatHigh,
          inner sep=1pt, anchor=north]
        at (#1.north) [yshift=0.1cm] {
        Category: #4 | Rep: #3/100
    };
}

% Display malware file hash indicator
% Usage: \markMalwareHash{node}{hash_type}{hash_value}{detection_rate}
\newcommand{\markMalwareHash}[4]{
    \node[legend box, anchor=north west, minimum width=5cm,
          draw=threatCritical, line width=2pt] at (#1) {
        \begin{tabular}{ll}
            \multicolumn{2}{c}{\textbf{\textcolor{threatCritical}{Malware Detected}}} \\
            \hline
            \textbf{Hash Type:} & #2 \\
            \textbf{Hash:} & \texttt{\tiny #3} \\
            \textbf{Detection:} & #4 AV engines \\
        \end{tabular}
    };
}

% IOC age/freshness indicator
% Usage: \markIOCFreshness{node}{age_days}{status}
\newcommand{\markIOCFreshness}[3]{
    \pgfmathsetmacro{\agedays}{#2}
    \ifthenelse{\lengthtest{\agedays pt < 7 pt}}{
        \def\freshnesscolor{threatCritical}
        \def\freshnesslabel{FRESH}
    }{
    \ifthenelse{\lengthtest{\agedays pt < 30 pt}}{
        \def\freshnesscolor{threatHigh}
        \def\freshnesslabel{RECENT}
    }{
    \ifthenelse{\lengthtest{\agedays pt < 90 pt}}{
        \def\freshnesscolor{threatMedium}
        \def\freshnesslabel{OLD}
    }{
        \def\freshnesscolor{threatLow}
        \def\freshnesslabel{STALE}
    }}}

    \node[fill=\freshnesscolor, text=white, font=\tiny\bfseries,
          inner sep=1pt, rounded corners=1pt, anchor=south east]
        at (#1.south east) {
        \freshnesslabel\ (#2d)
    };
}

% Comprehensive IOC dashboard
% Usage: \drawIOCDashboard{x}{y}{malicious_ips}{malicious_domains}{malware_hashes}{total_iocs}
\newcommand{\drawIOCDashboard}[6]{
    \node[legend box, anchor=north west, minimum width=4.5cm,
          draw=threatHigh, line width=2pt] at (#1,#2) {
        \begin{tabular}{lr}
            \multicolumn{2}{c}{\textbf{\textcolor{threatHigh}{IOC Summary}}} \\
            \hline
            Malicious IPs & #3 \\
            Malicious Domains & #4 \\
            Malware Hashes & #5 \\
            \hline
            \textbf{Total IOCs} & \textbf{#6} \\
        \end{tabular}
    };
}

% Threat feed integration indicator
% Usage: \markThreatFeed{x}{y}{feed_name}{last_update}{active_threats}
\newcommand{\markThreatFeed}[5]{
    \node[legend box, anchor=north west, minimum width=4cm,
          draw=threatMedium, line width=1.5pt] at (#1,#2) {
        \begin{tabular}{ll}
            \multicolumn{2}{c}{\textbf{Threat Feed: #3}} \\
            \hline
            \textbf{Last Update:} & #4 \\
            \textbf{Active Threats:} & \textcolor{threatHigh}{#5} \\
        \end{tabular}
    };
}

% ============================================================================
% SECURITY POSTURE OVERVIEW
% ============================================================================

% Draw security posture dashboard
% Usage: \drawSecurityDashboard{x}{y}{critical}{high}{medium}{low}{risk_score}
\newcommand{\drawSecurityDashboard}[7]{
    \node[legend box, anchor=north east, minimum width=4cm] at (#1,#2) {
        \begin{tabular}{ll}
            \multicolumn{2}{c}{\textbf{Security Posture}} \\
            \hline
            \textcolor{threatCritical}{● Critical} & #3 \\
            \textcolor{threatHigh}{● High} & #4 \\
            \textcolor{threatMedium}{● Medium} & #5 \\
            \textcolor{threatLow}{● Low} & #6 \\
            \hline
            \textbf{Risk Score} & \textbf{#7/100} \\
        \end{tabular}
    };
}

% ============================================================================
% COMPLIANCE FRAMEWORK DASHBOARDS
% ============================================================================

% NIST Cybersecurity Framework detailed dashboard
% Usage: \drawNISTDashboard{x}{y}{identify}{protect}{detect}{respond}{recover}
\newcommand{\drawNISTDashboard}[7]{
    \node[legend box, anchor=north west, minimum width=6cm,
          draw=blue!70, line width=2pt] at (#1,#2) {
        \begin{tabular}{lcc}
            \multicolumn{3}{c}{\textbf{NIST CSF Assessment}} \\
            \hline
            \textbf{Function} & \textbf{Score} & \textbf{Status} \\
            \hline
            Identify & #3\% & \drawComplianceBar{#3} \\
            Protect & #4\% & \drawComplianceBar{#4} \\
            Detect & #5\% & \drawComplianceBar{#5} \\
            Respond & #6\% & \drawComplianceBar{#6} \\
            Recover & #7\% & \drawComplianceBar{#7} \\
        \end{tabular}
    };
}

% Helper for compliance status bar
\newcommand{\drawComplianceBar}[1]{
    \pgfmathsetmacro{\score}{#1}
    \ifthenelse{\lengthtest{\score pt > 80 pt}}{
        \textcolor{green!60!black}{●●●●●}
    }{
    \ifthenelse{\lengthtest{\score pt > 60 pt}}{
        \textcolor{green!60!black}{●●●●}\textcolor{gray!50}{●}
    }{
    \ifthenelse{\lengthtest{\score pt > 40 pt}}{
        \textcolor{orange}{●●●}\textcolor{gray!50}{●●}
    }{
    \ifthenelse{\lengthtest{\score pt > 20 pt}}{
        \textcolor{threatHigh}{●●}\textcolor{gray!50}{●●●}
    }{
        \textcolor{threatCritical}{●}\textcolor{gray!50}{●●●●}
    }}}}
}

% CIS Controls compliance dashboard
% Usage: \drawCISControls{x}{y}{basic}{foundational}{organizational}{overall}
\newcommand{\drawCISControls}[6]{
    \node[legend box, anchor=north west, minimum width=6cm,
          draw=blue!60, line width=2pt] at (#1,#2) {
        \begin{tabular}{lc}
            \multicolumn{2}{c}{\textbf{CIS Controls v8}} \\
            \hline
            \textbf{Control Category} & \textbf{Coverage} \\
            \hline
            Basic (IG1) & #3\% \\
            Foundational (IG2) & #4\% \\
            Organizational (IG3) & #5\% \\
            \hline
            \textbf{Overall Compliance} & \textbf{#6\%} \\
        \end{tabular}
    };
}

% PCI-DSS compliance dashboard
% Usage: \drawPCIDSS{x}{y}{network_security}{access_control}{monitoring}{info_security}{policies}{overall}
\newcommand{\drawPCIDSS}[8]{
    \pgfmathsetmacro{\overallscore}{#8}
    \ifthenelse{\lengthtest{\overallscore pt > 90 pt}}{
        \def\pcicolor{green!60!black}
        \def\pcistatus{COMPLIANT}
    }{
    \ifthenelse{\lengthtest{\overallscore pt > 70 pt}}{
        \def\pcicolor{orange}
        \def\pcistatus{REMEDIATION NEEDED}
    }{
        \def\pcicolor{threatCritical}
        \def\pcistatus{NON-COMPLIANT}
    }}

    \node[legend box, anchor=north west, minimum width=6.5cm,
          draw=\pcicolor, line width=2pt] at (#1,#2) {
        \begin{tabular}{lc}
            \multicolumn{2}{c}{\textbf{PCI-DSS v4.0 Compliance}} \\
            \hline
            Network Security & #3\% \\
            Access Control & #4\% \\
            Monitoring \& Testing & #5\% \\
            Information Security & #6\% \\
            Policies \& Procedures & #7\% \\
            \hline
            \textbf{Overall} & \textbf{#8\%} \\
            \textbf{Status} & \textcolor{\pcicolor}{\pcistatus} \\
        \end{tabular}
    };
}

% HIPAA compliance dashboard
% Usage: \drawHIPAA{x}{y}{admin}{physical}{technical}{overall}
\newcommand{\drawHIPAA}[6]{
    \pgfmathsetmacro{\hipaascore}{#6}
    \ifthenelse{\lengthtest{\hipaascore pt > 85 pt}}{
        \def\hipaacolor{green!60!black}
    }{
        \def\hipaacolor{threatCritical}
    }

    \node[legend box, anchor=north west, minimum width=5.5cm,
          draw=\hipaacolor, line width=2pt] at (#1,#2) {
        \begin{tabular}{lc}
            \multicolumn{2}{c}{\textbf{HIPAA Compliance}} \\
            \hline
            Administrative Safeguards & #3\% \\
            Physical Safeguards & #4\% \\
            Technical Safeguards & #5\% \\
            \hline
            \textbf{Overall Compliance} & \textbf{#6\%} \\
        \end{tabular}
    };
}

% SOC 2 trust services criteria
% Usage: \drawSOC2{x}{y}{security}{availability}{processing}{confidentiality}{privacy}
\newcommand{\drawSOC2}[7]{
    \node[legend box, anchor=north west, minimum width=6cm,
          draw=purple!70, line width=2pt] at (#1,#2) {
        \begin{tabular}{lc}
            \multicolumn{2}{c}{\textbf{SOC 2 Trust Services}} \\
            \hline
            Security & #3\% \\
            Availability & #4\% \\
            Processing Integrity & #5\% \\
            Confidentiality & #6\% \\
            Privacy & #7\% \\
        \end{tabular}
    };
}

% ISO 27001 compliance dashboard
% Usage: \drawISO27001{x}{y}{controls_implemented}{controls_total}{certification_status}
\newcommand{\drawISO27001}[5]{
    \ifthenelse{\equal{#5}{certified}}{
        \def\isocolor{green!60!black}
    }{
    \ifthenelse{\equal{#5}{in-progress}}{
        \def\isocolor{orange}
    }{
        \def\isocolor{gray!60}
    }}

    \node[legend box, anchor=north west, minimum width=5cm,
          draw=\isocolor, line width=2pt] at (#1,#2) {
        \begin{tabular}{lc}
            \multicolumn{2}{c}{\textbf{ISO 27001:2022}} \\
            \hline
            Controls Implemented & #3 \\
            Total Controls & #4 \\
            \hline
            \textbf{Status} & \textcolor{\isocolor}{\uppercase{#5}} \\
        \end{tabular}
    };
}

% Multi-framework compliance overview
% Usage: \drawMultiFrameworkCompliance{x}{y}{nist}{cis}{pci}{iso}
\newcommand{\drawMultiFrameworkCompliance}[6]{
    \node[legend box, anchor=north west, minimum width=5.5cm,
          draw=blue!80, line width=2.5pt] at (#1,#2) {
        \begin{tabular}{lc}
            \multicolumn{2}{c}{\textbf{Compliance Overview}} \\
            \hline
            NIST CSF & #3\% \\
            CIS Controls & #4\% \\
            PCI-DSS & #5\% \\
            ISO 27001 & #6\% \\
        \end{tabular}
    };
}

% ============================================================================
% VULNERABILITY DATABASE INTEGRATION
% ============================================================================

% Comprehensive vulnerability report for a node
% Usage: \drawVulnerabilityReport{x}{y}{node_name}{cve_count}{critical}{high}{medium}{low}
\newcommand{\drawVulnerabilityReport}[8]{
    \pgfmathsetmacro{\totalvulns}{#3 + #4 + #5 + #6}
    \node[legend box, anchor=north west, minimum width=5cm,
          draw=threatHigh, line width=2pt] at (#1,#2) {
        \begin{tabular}{lc}
            \multicolumn{2}{c}{\textbf{Vulnerability Report: #3}} \\
            \hline
            \textcolor{threatCritical}{Critical} & #5 \\
            \textcolor{threatHigh}{High} & #6 \\
            \textcolor{threatMedium}{Medium} & #7 \\
            \textcolor{threatLow}{Low} & #8 \\
            \hline
            \textbf{Total CVEs} & \textbf{#4} \\
        \end{tabular}
    };
}

% Exploit availability indicator
% Usage: \markExploitAvailable{node}{cve}{exploit_maturity}
% exploit_maturity: poc, functional, high (weaponized)
\newcommand{\markExploitAvailable}[3]{
    \ifthenelse{\equal{#3}{high}}{
        \def\exploitcolor{threatCritical}
        \def\exploitlabel{WEAPONIZED}
    }{
    \ifthenelse{\equal{#3}{functional}}{
        \def\exploitcolor{threatHigh}
        \def\exploitlabel{FUNCTIONAL}
    }{
        \def\exploitcolor{threatMedium}
        \def\exploitlabel{POC}
    }}

    \node[draw=\exploitcolor, fill=\exploitcolor!30, text=black,
          font=\tiny\bfseries, rounded corners=2pt, inner sep=2pt,
          anchor=east] at (#1.east) [xshift=-0.1cm] {
        EXPLOIT: \exploitlabel
    };
    \node[below, font=\tiny\ttfamily, fill=white, draw=\exploitcolor,
          inner sep=1pt] at (#1.east) [xshift=-0.5cm, yshift=-0.3cm] {
        #2
    };
}

% Patch availability status
% Usage: \markPatchStatus{node}{cve}{patch_status}{days_since_patch}
% patch_status: available, pending, none
\newcommand{\markPatchStatus}[4]{
    \ifthenelse{\equal{#3}{available}}{
        \def\patchcolor{green!60!black}
        \def\patchlabel{PATCH AVAILABLE}
    }{
    \ifthenelse{\equal{#3}{pending}}{
        \def\patchcolor{orange}
        \def\patchlabel{PATCH PENDING}
    }{
        \def\patchcolor{threatCritical}
        \def\patchlabel{NO PATCH}
    }}

    \node[fill=\patchcolor, text=white, font=\tiny\bfseries,
          inner sep=2pt, rounded corners=1pt, anchor=west]
        at (#1.west) [xshift=0.1cm] {
        \patchlabel\ (#4d)
    };
}

% Vulnerability age indicator
% Usage: \markVulnerabilityAge{node}{cve}{days_since_disclosure}
\newcommand{\markVulnerabilityAge}[3]{
    \pgfmathsetmacro{\agedays}{#3}
    \ifthenelse{\lengthtest{\agedays pt < 30 pt}}{
        \def\agecolor{threatCritical}
        \def\agelabel{NEW}
    }{
    \ifthenelse{\lengthtest{\agedays pt < 180 pt}}{
        \def\agecolor{threatHigh}
        \def\agelabel{RECENT}
    }{
    \ifthenelse{\lengthtest{\agedays pt < 365 pt}}{
        \def\agecolor{orange}
        \def\agelabel{AGING}
    }{
        \def\agecolor{gray!70}
        \def\agelabel{OLD}
    }}}

    \node[fill=\agecolor, text=white, font=\tiny,
          inner sep=1pt, rounded corners=1pt, anchor=south]
        at (#1.south) [yshift=-0.5cm] {
        \agelabel: #3 days
    };
}

% EPSS (Exploit Prediction Scoring System) indicator
% Usage: \markEPSS{node}{cve}{epss_score}{percentile}
\newcommand{\markEPSS}[4]{
    \pgfmathsetmacro{\epssscore}{#3 * 100}
    \ifthenelse{\lengthtest{\epssscore pt > 70 pt}}{
        \def\epsscolor{threatCritical}
    }{
    \ifthenelse{\lengthtest{\epssscore pt > 30 pt}}{
        \def\epsscolor{threatHigh}
    }{
        \def\epsscolor{threatMedium}
    }}

    \node[legend box, anchor=south west, minimum width=3cm,
          draw=\epsscolor, line width=1.5pt] at (#1.south west) [yshift=-1cm] {
        \begin{tabular}{ll}
            \multicolumn{2}{c}{\textbf{\textcolor{\epsscolor}{EPSS}}} \\
            \hline
            \textbf{CVE:} & \texttt{\tiny #2} \\
            \textbf{Score:} & #3 (#4th \%ile) \\
        \end{tabular}
    };
}

% Vulnerability priority score (combines CVSS, EPSS, exploit availability)
% Usage: \drawVulnPriority{x}{y}{cve}{cvss}{epss}{exploit}{priority_score}
\newcommand{\drawVulnPriority}[7]{
    \pgfmathsetmacro{\priority}{#7}
    \ifthenelse{\lengthtest{\priority pt > 90 pt}}{
        \def\prioritycolor{threatCritical}
        \def\prioritylabel{CRITICAL - PATCH NOW}
    }{
    \ifthenelse{\lengthtest{\priority pt > 70 pt}}{
        \def\prioritycolor{threatHigh}
        \def\prioritylabel{HIGH - PATCH URGENT}
    }{
    \ifthenelse{\lengthtest{\priority pt > 40 pt}}{
        \def\prioritycolor{threatMedium}
        \def\prioritylabel{MEDIUM - SCHEDULE}
    }{
        \def\prioritycolor{threatLow}
        \def\prioritylabel{LOW - MONITOR}
    }}}

    \node[legend box, anchor=north west, minimum width=5cm,
          draw=\prioritycolor, line width=2.5pt] at (#1,#2) {
        \begin{tabular}{ll}
            \multicolumn{2}{c}{\textbf{\textcolor{\prioritycolor}{#3}}} \\
            \hline
            CVSS Score & #4 \\
            EPSS Score & #5 \\
            Exploit Available & #6 \\
            \hline
            \textbf{Priority} & \textbf{#7/100} \\
            \multicolumn{2}{c}{\textcolor{\prioritycolor}{\prioritylabel}} \\
        \end{tabular}
    };
}

% Vulnerability scan results summary
% Usage: \drawScanResults{x}{y}{scanner_name}{scan_date}{vulns_found}{false_positives}
\newcommand{\drawScanResults}[6]{
    \node[legend box, anchor=north west, minimum width=5cm,
          draw=blue!60, line width=1.5pt] at (#1,#2) {
        \begin{tabular}{ll}
            \multicolumn{2}{c}{\textbf{Scan Results: #3}} \\
            \hline
            Scan Date & #4 \\
            Vulnerabilities Found & #5 \\
            False Positives & #6 \\
            \hline
            \textbf{Net Findings} & \textbf{\pgfmathparse{int(#5-#6)}\pgfmathresult} \\
        \end{tabular}
    };
}
% ============================================================================
% THREAT CORRELATION AND INCIDENT RECONSTRUCTION
% ============================================================================

% Correlate multiple IOCs to show relationship
% Usage: \drawIOCCorrelation{x}{y}{title}{ioc1}{ioc2}{ioc3}{correlation_confidence}
\newcommand{\drawIOCCorrelation}[7]{
    \pgfmathsetmacro{\correlation}{#7}
    \ifthenelse{\lengthtest{\correlation pt > 80 pt}}{
        \def\corrcolor{threatCritical}
        \def\corrlabel{HIGH CONFIDENCE}
    }{
    \ifthenelse{\lengthtest{\correlation pt > 50 pt}}{
        \def\corrcolor{threatHigh}
        \def\corrlabel{MEDIUM CONFIDENCE}
    }{
        \def\corrcolor{orange}
        \def\corrlabel{LOW CONFIDENCE}
    }}

    \node[legend box, anchor=north west, minimum width=6cm,
          draw=\corrcolor, line width=2pt] at (#1,#2) {
        \begin{tabular}{p{5.5cm}}
            \multicolumn{1}{c}{\textbf{\textcolor{\corrcolor}{#3}}} \\
            \hline
            \textbf{Correlated IOCs:} \\
            ● #4 \\
            ● #5 \\
            ● #6 \\
            \hline
            \textbf{Confidence:} \textcolor{\corrcolor}{#7\% - \corrlabel} \\
        \end{tabular}
    };
}

% Incident timeline reconstruction
% Usage: \drawIncidentTimeline{x}{y}{incident_name}{start}{end}{events}
\newcommand{\drawIncidentTimeline}[6]{
    \node[legend box, anchor=north west, minimum width=7cm,
          draw=threatCritical, line width=2pt] at (#1,#2) {
        \begin{tabular}{p{6.5cm}}
            \multicolumn{1}{c}{\textbf{\textcolor{threatCritical}{Incident: #3}}} \\
            \hline
            \textbf{Start:} #4 \\
            \textbf{End:} #5 \\
            \hline
            \textbf{Timeline:} \\
            #6 \\
        \end{tabular}
    };
}

% Lateral movement path visualization
% Usage: \drawLateralMovement{node1}{node2}{node3}{method}{timestamp}
\newcommand{\drawLateralMovement}[5]{
    \draw[draw=threatCritical, line width=2.5pt, dashed,
          -{Stealth[length=4mm]}] (#1) -- (#2)
        node[midway, above, font=\tiny, fill=yellow!30, inner sep=1pt] {
            #4 @ #5
        };
    \draw[draw=threatCritical, line width=2.5pt, dashed,
          -{Stealth[length=4mm]}] (#2) -- (#3)
        node[midway, above, font=\tiny, fill=yellow!30, inner sep=1pt] {
            #4 @ #5
        };
}

% Infection spread visualization
% Usage: \drawInfectionSpread{patient_zero}{infected_nodes_count}{containment_status}
\newcommand{\drawInfectionSpread}[3]{
    \node[circle, draw=threatCritical, line width=3pt, fill=threatCritical!20,
          minimum size=1.5cm] at (#1) {};
    \node[above, font=\small\bfseries, text=threatCritical, fill=white,
          inner sep=2pt] at (#1.north) [yshift=0.8cm] {
        PATIENT ZERO
    };
    \node[below, font=\tiny, fill=white, draw=threatCritical,
          inner sep=2pt] at (#1.south) [yshift=-0.5cm] {
        Infected: #2 | Status: #3
    };
}

% Command & Control (C2) beacon indicator
% Usage: \drawC2Beacon{compromised_node}{c2_server}{frequency}{protocol}
\newcommand{\drawC2Beacon}[4]{
    \draw[draw=threatCritical, line width=2pt, dotted,
          <->, shorten >=0.2cm, shorten <=0.2cm]
        (#1) -- (#2)
        node[midway, above, font=\tiny, fill=red!20, inner sep=2pt] {
            C2: #4 (#3)
        };
}

% Data exfiltration visualization with volume
% Usage: \drawExfiltrationPath{source}{intermediate}{destination}{data_volume}{detection}
\newcommand{\drawExfiltrationPath}[5]{
    \draw[draw=red!80!black, line width=3pt, -{Stealth[length=5mm]}]
        (#1) -- (#2)
        node[midway, below, font=\tiny, fill=red!20, inner sep=1pt] {
            #4
        };
    \draw[draw=red!80!black, line width=3pt, -{Stealth[length=5mm]}]
        (#2) -- (#3)
        node[midway, below, font=\tiny, fill=red!20, inner sep=1pt] {
            Detection: #5
        };
}

% Threat hunting query results
% Usage: \drawHuntingResults{x}{y}{query_name}{matches_found}{false_positives}{true_positives}
\newcommand{\drawHuntingResults}[6]{
    \pgfmathsetmacro{\precision}{(#6 / #4) * 100}
    \node[legend box, anchor=north west, minimum width=5.5cm,
          draw=purple!70, line width=2pt] at (#1,#2) {
        \begin{tabular}{ll}
            \multicolumn{2}{c}{\textbf{Hunt: #3}} \\
            \hline
            Matches Found & #4 \\
            False Positives & #5 \\
            True Positives & #6 \\
            \hline
            \textbf{Precision} & \textbf{\pgfmathprintnumber[precision=1]{\precision}\%} \\
        \end{tabular}
    };
}

% Incident severity meter
% Usage: \drawIncidentSeverity{x}{y}{incident_name}{severity_score}{impact}{urgency}
\newcommand{\drawIncidentSeverity}[6]{
    \pgfmathsetmacro{\severity}{#4}
    \ifthenelse{\lengthtest{\severity pt > 80 pt}}{
        \def\sevcolor{threatCritical}
        \def\sevlabel{CRITICAL INCIDENT}
    }{
    \ifthenelse{\lengthtest{\severity pt > 60 pt}}{
        \def\sevcolor{threatHigh}
        \def\sevlabel{HIGH SEVERITY}
    }{
    \ifthenelse{\lengthtest{\severity pt > 40 pt}}{
        \def\sevcolor{threatMedium}
        \def\sevlabel{MEDIUM SEVERITY}
    }{
        \def\sevcolor{threatLow}
        \def\sevlabel{LOW SEVERITY}
    }}}

    \node[legend box, anchor=north west, minimum width=5cm,
          draw=\sevcolor, line width=2.5pt] at (#1,#2) {
        \begin{tabular}{ll}
            \multicolumn{2}{c}{\textbf{\textcolor{\sevcolor}{#3}}} \\
            \hline
            Severity Score & #4/100 \\
            Impact & #5 \\
            Urgency & #6 \\
            \hline
            \multicolumn{2}{c}{\textcolor{\sevcolor}{\sevlabel}} \\
        \end{tabular}
    };
}

% ============================================================================
% SECURITY METRICS AND KPIs
% ============================================================================

% Mean Time metrics (MTTD, MTTR, MTTC)
% Usage: \drawMTTMetrics{x}{y}{mttd_hours}{mttr_hours}{mttc_hours}
\newcommand{\drawMTTMetrics}[5]{
    \node[legend box, anchor=north west, minimum width=5cm,
          draw=blue!70, line width=2pt] at (#1,#2) {
        \begin{tabular}{lc}
            \multicolumn{2}{c}{\textbf{Response Metrics}} \\
            \hline
            MTTD (Detect) & #3h \\
            MTTR (Respond) & #4h \\
            MTTC (Contain) & #5h \\
        \end{tabular}
    };
}

% Security coverage heatmap
% Usage: \drawCoverageHeatmap{x}{y}{endpoint}{network}{cloud}{identity}{data}
\newcommand{\drawCoverageHeatmap}[7]{
    \node[legend box, anchor=north west, minimum width=6cm,
          draw=green!60!black, line width=2pt] at (#1,#2) {
        \begin{tabular}{lc}
            \multicolumn{2}{c}{\textbf{Security Coverage}} \\
            \hline
            Endpoint Security & \drawCoverageBar{#3} \\
            Network Security & \drawCoverageBar{#4} \\
            Cloud Security & \drawCoverageBar{#5} \\
            Identity \& Access & \drawCoverageBar{#6} \\
            Data Protection & \drawCoverageBar{#7} \\
        \end{tabular}
    };
}

% Helper for coverage visualization
\newcommand{\drawCoverageBar}[1]{
    \pgfmathsetmacro{\cov}{#1}
    \ifthenelse{\lengthtest{\cov pt > 90 pt}}{
        \textcolor{green!60!black}{#1\%}
    }{
    \ifthenelse{\lengthtest{\cov pt > 70 pt}}{
        \textcolor{orange}{#1\%}
    }{
        \textcolor{threatHigh}{#1\%}
    }}
}

% Risk score trending
% Usage: \drawRiskTrend{x}{y}{current_score}{last_month}{trend}
% trend: increasing, decreasing, stable
\newcommand{\drawRiskTrend}[5]{
    \ifthenelse{\equal{#5}{increasing}}{
        \def\trendcolor{threatCritical}
        \def\trendlabel{↑ INCREASING}
    }{
    \ifthenelse{\equal{#5}{decreasing}}{
        \def\trendcolor{green!60!black}
        \def\trendlabel{↓ DECREASING}
    }{
        \def\trendcolor{orange}
        \def\trendlabel{→ STABLE}
    }}

    \node[legend box, anchor=north west, minimum width=4.5cm,
          draw=\trendcolor, line width=2pt] at (#1,#2) {
        \begin{tabular}{ll}
            \multicolumn{2}{c}{\textbf{Risk Score Trend}} \\
            \hline
            Current & #3/100 \\
            Last Month & #4/100 \\
            \hline
            \textbf{Trend} & \textcolor{\trendcolor}{\trendlabel} \\
        \end{tabular}
    };
}

% ============================================================================
% ADVANCED THREAT HUNTING AND BEHAVIORAL ANALYTICS
% ============================================================================

% Sigma rule detection visualization
% Usage: \drawSigmaDetection{x}{y}{rule_name}{rule_id}{severity}{confidence}{matches}
\newcommand{\drawSigmaDetection}[7]{
    \ifthenelse{\equal{#5}{critical}}{
        \def\sigmacolor{threatCritical}
    }{
    \ifthenelse{\equal{#5}{high}}{
        \def\sigmacolor{threatHigh}
    }{
        \def\sigmacolor{threatMedium}
    }}

    \node[legend box, anchor=north west, minimum width=6cm,
          draw=\sigmacolor, line width=2pt] at (#1,#2) {
        \begin{tabular}{ll}
            \multicolumn{2}{c}{\textbf{\textcolor{\sigmacolor}{Sigma Detection}}} \\
            \hline
            \textbf{Rule:} & #3 \\
            \textbf{ID:} & \texttt{\tiny #4} \\
            \textbf{Severity:} & \textcolor{\sigmacolor}{\uppercase{#5}} \\
            \textbf{Confidence:} & #6\% \\
            \textbf{Matches:} & #7 \\
        \end{tabular}
    };
}

% Behavioral anomaly detection
% Usage: \drawBehavioralAnomaly{x}{y}{entity}{baseline}{current}{deviation}{anomaly_score}
\newcommand{\drawBehavioralAnomaly}[7]{
    \pgfmathsetmacro{\anomalyscore}{#7}
    \ifthenelse{\lengthtest{\anomalyscore pt > 80 pt}}{
        \def\anomalycolor{threatCritical}
        \def\anomalylabel{CRITICAL ANOMALY}
    }{
    \ifthenelse{\lengthtest{\anomalyscore pt > 60 pt}}{
        \def\anomalycolor{threatHigh}
        \def\anomalylabel{HIGH ANOMALY}
    }{
        \def\anomalycolor{orange}
        \def\anomalylabel{MODERATE ANOMALY}
    }}

    \node[legend box, anchor=north west, minimum width=6cm,
          draw=\anomalycolor, line width=2pt] at (#1,#2) {
        \begin{tabular}{ll}
            \multicolumn{2}{c}{\textbf{\textcolor{\anomalycolor}{Behavioral Anomaly}}} \\
            \hline
            \textbf{Entity:} & #3 \\
            \textbf{Baseline:} & #4 \\
            \textbf{Current:} & #5 \\
            \textbf{Deviation:} & #6\% \\
            \hline
            \textbf{Score:} & \textcolor{\anomalycolor}{#7/100 - \anomalylabel} \\
        \end{tabular}
    };
}

% UEBA (User and Entity Behavior Analytics) risk score
% Usage: \drawUEBARisk{node}{entity_name}{risk_score}{risk_factors}
\newcommand{\drawUEBARisk}[4]{
    \pgfmathsetmacro{\uebarscore}{#3}
    \ifthenelse{\lengthtest{\uebarscore pt > 75 pt}}{
        \def\uebacolor{threatCritical}
    }{
    \ifthenelse{\lengthtest{\uebarscore pt > 50 pt}}{
        \def\uebacolor{threatHigh}
    }{
        \def\uebacolor{orange}
    }}

    \node[legend box, anchor=north west, minimum width=5cm,
          draw=\uebacolor, line width=2pt] at (#1) {
        \begin{tabular}{ll}
            \multicolumn{2}{c}{\textbf{\textcolor{\uebacolor}{UEBA Risk: #2}}} \\
            \hline
            \textbf{Risk Score:} & \textcolor{\uebacolor}{#3/100} \\
            \hline
            \multicolumn{2}{l}{\textbf{Risk Factors:}} \\
            \multicolumn{2}{p{4.5cm}}{\tiny #4} \\
        \end{tabular}
    };
}

% Threat intelligence enrichment display
% Usage: \drawThreatEnrichment{x}{y}{ioc}{enrichment_data}
\newcommand{\drawThreatEnrichment}[4]{
    \node[legend box, anchor=north west, minimum width=6cm,
          draw=purple!70, line width=1.5pt] at (#1,#2) {
        \begin{tabular}{p{5.5cm}}
            \multicolumn{1}{c}{\textbf{Enrichment: #3}} \\
            \hline
            #4 \\
        \end{tabular}
    };
}

% ============================================================================
% ZERO-DAY AND ADVANCED THREAT DETECTION
% ============================================================================

% Zero-day vulnerability indicator
% Usage: \markZeroDay{node}{vulnerability_name}{discovery_date}{exploited}
\newcommand{\markZeroDay}[4]{
    \node[draw=threatCritical, fill=threatCritical, text=white,
          font=\small\bfseries, rounded corners=3pt, inner sep=4pt,
          anchor=north] at (#1.north) [yshift=0.8cm] {
        ⚠ ZERO-DAY ⚠
    };
    \node[legend box, anchor=north, minimum width=4cm,
          draw=threatCritical, line width=2.5pt] at (#1.north) [yshift=0.3cm] {
        \begin{tabular}{ll}
            \textbf{Vuln:} & #2 \\
            \textbf{Discovered:} & #3 \\
            \textbf{Exploited:} & #4 \\
        \end{tabular}
    };
}

% Advanced Persistent Threat (APT) campaign visualization
% Usage: \drawAPTCampaign{x}{y}{apt_name}{campaign}{duration}{objectives}{success_rate}
\newcommand{\drawAPTCampaign}[7]{
    \node[legend box, anchor=north west, minimum width=7cm,
          draw=threatCritical, line width=3pt, fill=red!5] at (#1,#2) {
        \begin{tabular}{ll}
            \multicolumn{2}{c}{\textbf{\textcolor{threatCritical}{APT Campaign: #3}}} \\
            \hline
            \textbf{Campaign Name:} & #4 \\
            \textbf{Duration:} & #5 \\
            \textbf{Objectives:} & #6 \\
            \textbf{Success Rate:} & \textcolor{threatCritical}{#7\%} \\
        \end{tabular}
    };
}

% Fileless malware detection
% Usage: \markFilelessMalware{node}{technique}{memory_artifacts}
\newcommand{\markFilelessMalware}[3]{
    \node[draw=threatCritical, fill=purple!30, text=black,
          font=\tiny\bfseries, rounded corners=2pt, inner sep=3pt,
          anchor=north] at (#1.north) [yshift=0.5cm] {
        FILELESS MALWARE DETECTED
    };
    \node[below, font=\tiny, fill=white, draw=purple!70,
          inner sep=2pt] at (#1.north) [yshift=0.2cm] {
        Technique: #2 | Artifacts: #3
    };
}

% Living off the Land (LOLBin) detection
% Usage: \markLOLBin{node}{binary}{technique}{risk_level}
\newcommand{\markLOLBin}[4]{
    \ifthenelse{\equal{#4}{high}}{
        \def\lolcolor{threatCritical}
    }{\def\lolcolor{threatHigh}}

    \node[draw=\lolcolor, fill=yellow!40, text=black,
          font=\tiny\bfseries, rounded corners=2pt, inner sep=2pt,
          anchor=west] at (#1.west) [xshift=0.2cm] {
        LOLBin: #2
    };
    \node[below, font=\tiny, fill=white, draw=\lolcolor,
          inner sep=1pt] at (#1.west) [xshift=0.5cm, yshift=-0.3cm] {
        #3
    };
}

% ============================================================================
% THREAT INTELLIGENCE SCORING AND PRIORITIZATION
% ============================================================================

% Comprehensive threat score calculator
% Usage: \drawThreatScore{x}{y}{target}{cvss}{epss}{exploited}{ioc_count}{final_score}
\newcommand{\drawThreatScore}[8]{
    \pgfmathsetmacro{\finalscore}{#8}
    \ifthenelse{\lengthtest{\finalscore pt > 90 pt}}{
        \def\scorecolor{threatCritical}
        \def\scorelabel{CRITICAL THREAT}
        \def\scoreaction{IMMEDIATE ACTION}
    }{
    \ifthenelse{\lengthtest{\finalscore pt > 70 pt}}{
        \def\scorecolor{threatHigh}
        \def\scorelabel{HIGH THREAT}
        \def\scoreaction{URGENT RESPONSE}
    }{
    \ifthenelse{\lengthtest{\finalscore pt > 50 pt}}{
        \def\scorecolor{threatMedium}
        \def\scorelabel{MODERATE THREAT}
        \def\scoreaction{SCHEDULED RESPONSE}
    }{
        \def\scorecolor{threatLow}
        \def\scorelabel{LOW THREAT}
        \def\scoreaction{MONITOR}
    }}}

    \node[legend box, anchor=north west, minimum width=6cm,
          draw=\scorecolor, line width=2.5pt] at (#1,#2) {
        \begin{tabular}{ll}
            \multicolumn{2}{c}{\textbf{\textcolor{\scorecolor}{Threat Score: #3}}} \\
            \hline
            CVSS Score & #4 \\
            EPSS Score & #5 \\
            Actively Exploited & #6 \\
            IOC Matches & #7 \\
            \hline
            \textbf{Final Score} & \textcolor{\scorecolor}{\textbf{#8/100}} \\
            \textbf{Classification} & \textcolor{\scorecolor}{\scorelabel} \\
            \textbf{Action Required} & \textcolor{\scorecolor}{\scoreaction} \\
        \end{tabular}
    };
}

% Threat actor playbook visualization
% Usage: \drawThreatPlaybook{x}{y}{actor}{playbook_name}{stages}{detection_coverage}
\newcommand{\drawThreatPlaybook}[6]{
    \node[legend box, anchor=north west, minimum width=6.5cm,
          draw=red!70, line width=2pt] at (#1,#2) {
        \begin{tabular}{p{6cm}}
            \multicolumn{1}{c}{\textbf{\textcolor{threatCritical}{Threat Playbook: #3}}} \\
            \hline
            \textbf{Playbook:} #4 \\
            \hline
            \textbf{Attack Stages:} \\
            #5 \\
            \hline
            \textbf{Detection Coverage:} \textcolor{green!60!black}{#6\%} \\
        \end{tabular}
    };
}

% ============================================================================
% SECURITY ORCHESTRATION AND AUTOMATION (SOAR)
% ============================================================================

% Automated response action indicator
% Usage: \drawSOARAction{x}{y}{trigger}{action}{status}{execution_time}
\newcommand{\drawSOARAction}[6]{
    \ifthenelse{\equal{#5}{success}}{
        \def\soarcolor{green!60!black}
    }{
    \ifthenelse{\equal{#5}{pending}}{
        \def\soarcolor{orange}
    }{
        \def\soarcolor{threatCritical}
    }}

    \node[legend box, anchor=north west, minimum width=5.5cm,
          draw=\soarcolor, line width=2pt] at (#1,#2) {
        \begin{tabular}{ll}
            \multicolumn{2}{c}{\textbf{SOAR Action}} \\
            \hline
            \textbf{Trigger:} & #3 \\
            \textbf{Action:} & #4 \\
            \textbf{Status:} & \textcolor{\soarcolor}{\uppercase{#5}} \\
            \textbf{Exec Time:} & #6 \\
        \end{tabular}
    };
}

% Playbook execution status
% Usage: \drawPlaybookExecution{x}{y}{playbook_name}{steps_total}{steps_completed}{status}
\newcommand{\drawPlaybookExecution}[6]{
    \pgfmathsetmacro{\progress}{(#5 / #4) * 100}
    \ifthenelse{\equal{#6}{running}}{
        \def\pbcolor{blue!70}
    }{
    \ifthenelse{\equal{#6}{completed}}{
        \def\pbcolor{green!60!black}
    }{
        \def\pbcolor{threatCritical}
    }}

    \node[legend box, anchor=north west, minimum width=5cm,
          draw=\pbcolor, line width=2pt] at (#1,#2) {
        \begin{tabular}{ll}
            \multicolumn{2}{c}{\textbf{Playbook: #3}} \\
            \hline
            Progress & #5/#4 steps \\
            Status & \textcolor{\pbcolor}{\uppercase{#6}} \\
            \textbf{Completion} & \textbf{\pgfmathprintnumber[precision=0]{\progress}\%} \\
        \end{tabular}
    };
}

% ============================================================================
% ADVANCED ANALYTICS AND PREDICTIVE MODELING
% ============================================================================

% Predictive threat forecast
% Usage: \drawThreatForecast{x}{y}{target}{current_risk}{predicted_risk}{timeframe}{confidence}
\newcommand{\drawThreatForecast}[7]{
    \pgfmathsetmacro{\predictedrisk}{#5}
    \ifthenelse{\lengthtest{\predictedrisk pt > 70 pt}}{
        \def\forecastcolor{threatCritical}
    }{
        \def\forecastcolor{orange}
    }

    \node[legend box, anchor=north west, minimum width=5.5cm,
          draw=\forecastcolor, line width=2pt] at (#1,#2) {
        \begin{tabular}{ll}
            \multicolumn{2}{c}{\textbf{Threat Forecast: #3}} \\
            \hline
            Current Risk & #4\% \\
            Predicted Risk & \textcolor{\forecastcolor}{#5\%} \\
            Timeframe & #6 \\
            Confidence & #7\% \\
        \end{tabular}
    };
}

% Attack surface analysis
% Usage: \drawAttackSurface{x}{y}{exposed_services}{vulnerabilities}{misconfigs}{total_risk}
\newcommand{\drawAttackSurface}[6]{
    \node[legend box, anchor=north west, minimum width=6cm,
          draw=orange!80, line width=2pt] at (#1,#2) {
        \begin{tabular}{lc}
            \multicolumn{2}{c}{\textbf{Attack Surface Analysis}} \\
            \hline
            Exposed Services & #3 \\
            Known Vulnerabilities & #4 \\
            Misconfigurations & #5 \\
            \hline
            \textbf{Total Risk Score} & \textcolor{threatHigh}{\textbf{#6/100}} \\
        \end{tabular}
    };
}

% Threat intelligence confidence scoring
% Usage: \drawConfidenceScore{x}{y}{intelligence}{sources}{corroboration}{age}{final_confidence}
\newcommand{\drawConfidenceScore}[7]{
    \pgfmathsetmacro{\confidence}{#7}
    \ifthenelse{\lengthtest{\confidence pt > 80 pt}}{
        \def\confcolor{green!60!black}
        \def\conflabel{HIGH CONFIDENCE}
    }{
    \ifthenelse{\lengthtest{\confidence pt > 50 pt}}{
        \def\confcolor{orange}
        \def\conflabel{MEDIUM CONFIDENCE}
    }{
        \def\confcolor{gray!60}
        \def\conflabel{LOW CONFIDENCE}
    }}

    \node[legend box, anchor=north west, minimum width=5.5cm,
          draw=\confcolor, line width=2pt] at (#1,#2) {
        \begin{tabular}{ll}
            \multicolumn{2}{c}{\textbf{Intelligence: #3}} \\
            \hline
            Sources & #4 \\
            Corroboration & #5\% \\
            Age & #6 days \\
            \hline
            \textbf{Confidence} & \textcolor{\confcolor}{\textbf{#7\% - \conflabel}} \\
        \end{tabular}
    };
}

% ============================================================================
% STIX/TAXII THREAT INTELLIGENCE SHARING
% ============================================================================

% STIX (Structured Threat Information Expression) indicator visualization
% Usage: \drawSTIXIndicator{x}{y}{indicator_type}{value}{confidence}{tlp_level}
% TLP levels: white, green, amber, red
\newcommand{\drawSTIXIndicator}[6]{
    \ifthenelse{\equal{#6}{red}}{
        \def\tlpcolor{red!80!black}
        \def\tlplabel{TLP:RED}
    }{
    \ifthenelse{\equal{#6}{amber}}{
        \def\tlpcolor{orange!90!black}
        \def\tlplabel{TLP:AMBER}
    }{
    \ifthenelse{\equal{#6}{green}}{
        \def\tlpcolor{green!60!black}
        \def\tlplabel{TLP:GREEN}
    }{
        \def\tlpcolor{gray!50}
        \def\tlplabel{TLP:WHITE}
    }}}

    \pgfmathsetmacro{\conf}{#5}
    \ifthenelse{\lengthtest{\conf pt > 75 pt}}{
        \def\confcolor{green!60!black}
    }{
    \ifthenelse{\lengthtest{\conf pt > 50 pt}}{
        \def\confcolor{orange}
    }{
        \def\confcolor{threatMedium}
    }}

    \node[legend box, anchor=north west, minimum width=6cm,
          draw=\tlpcolor, line width=2pt, fill=\tlpcolor!5] at (#1,#2) {
        \begin{tabular}{ll}
            \multicolumn{2}{c}{\textbf{\textcolor{\tlpcolor}{STIX Indicator}}} \\
            \hline
            \textbf{Type:} & #3 \\
            \textbf{Value:} & \texttt{\tiny #4} \\
            \textbf{Confidence:} & \textcolor{\confcolor}{#5\%} \\
            \hline
            \textbf{Sharing:} & \colorbox{\tlpcolor}{\textcolor{white}{\tiny\bfseries #6}} \\
        \end{tabular}
    };
}

% STIX object type visualizations
% Usage: \drawSTIXObject{x}{y}{object_type}{name}{description}
\newcommand{\drawSTIXObject}[5]{
    \ifthenelse{\equal{#3}{campaign}}{
        \def\stixcolor{purple!70}
    }{
    \ifthenelse{\equal{#3}{threat-actor}}{
        \def\stixcolor{red!70}
    }{
    \ifthenelse{\equal{#3}{malware}}{
        \def\stixcolor{threatCritical}
    }{
    \ifthenelse{\equal{#3}{attack-pattern}}{
        \def\stixcolor{orange!80}
    }{
        \def\stixcolor{blue!70}
    }}}}

    \node[legend box, anchor=north west, minimum width=5.5cm,
          draw=\stixcolor, line width=2pt] at (#1,#2) {
        \begin{tabular}{p{5cm}}
            \multicolumn{1}{c}{\textbf{\textcolor{\stixcolor}{STIX: \uppercase{#3}}}} \\
            \hline
            \textbf{Name:} #4 \\
            \hline
            #5 \\
        \end{tabular}
    };
}

% TAXII (Trusted Automated Exchange of Intelligence Information) feed status
% Usage: \drawTAXIIFeed{x}{y}{feed_name}{collection}{last_update}{object_count}{status}
\newcommand{\drawTAXIIFeed}[7]{
    \ifthenelse{\equal{#7}{active}}{
        \def\taxiicolor{green!60!black}
    }{
    \ifthenelse{\equal{#7}{stale}}{
        \def\taxiicolor{orange}
    }{
        \def\taxiicolor{red!70}
    }}

    \node[legend box, anchor=north west, minimum width=6cm,
          draw=\taxiicolor, line width=2pt] at (#1,#2) {
        \begin{tabular}{ll}
            \multicolumn{2}{c}{\textbf{TAXII Feed: #3}} \\
            \hline
            \textbf{Collection:} & #4 \\
            \textbf{Last Update:} & #5 \\
            \textbf{Objects:} & #6 \\
            \textbf{Status:} & \textcolor{\taxiicolor}{\uppercase{#7}} \\
        \end{tabular}
    };
}

% Multi-source threat intelligence aggregation
% Usage: \drawThreatIntelSharing{x}{y}{sources_count}{total_indicators}{high_confidence}{shared_iocs}
\newcommand{\drawThreatIntelSharing}[6]{
    \node[legend box, anchor=north west, minimum width=6.5cm,
          draw=purple!70, line width=2.5pt] at (#1,#2) {
        \begin{tabular}{lc}
            \multicolumn{2}{c}{\textbf{Threat Intelligence Sharing}} \\
            \hline
            Active Sources & #3 \\
            Total Indicators & #4 \\
            High Confidence & \textcolor{green!60!black}{#5} \\
            Shared IOCs & #6 \\
        \end{tabular}
    };
}

% STIX Bundle visualization (contains multiple objects)
% Usage: \drawSTIXBundle{x}{y}{bundle_name}{indicators}{campaigns}{actors}{patterns}
\newcommand{\drawSTIXBundle}[7]{
    \node[legend box, anchor=north west, minimum width=6cm,
          draw=blue!70, line width=2pt, fill=blue!3] at (#1,#2) {
        \begin{tabular}{lc}
            \multicolumn{2}{c}{\textbf{STIX Bundle: #3}} \\
            \hline
            Indicators & #4 \\
            Campaigns & #5 \\
            Threat Actors & #6 \\
            Attack Patterns & #7 \\
        \end{tabular}
    };
}

% Threat intelligence platform integration status
% Usage: \drawTIPIntegration{x}{y}{platform}{api_status}{last_sync}{iocs_imported}{errors}
\newcommand{\drawTIPIntegration}[7]{
    \ifthenelse{\equal{#4}{connected}}{
        \def\tipcolor{green!60!black}
    }{
        \def\tipcolor{red!70}
    }

    \node[legend box, anchor=north west, minimum width=6cm,
          draw=\tipcolor, line width=2pt] at (#1,#2) {
        \begin{tabular}{ll}
            \multicolumn{2}{c}{\textbf{TIP: #3}} \\
            \hline
            \textbf{API Status:} & \textcolor{\tipcolor}{\uppercase{#4}} \\
            \textbf{Last Sync:} & #5 \\
            \textbf{IOCs Imported:} & #6 \\
            \textbf{Errors:} & \textcolor{red!70}{#7} \\
        \end{tabular}
    };
}

% MISP (Malware Information Sharing Platform) integration
% Usage: \drawMISPIntegration{x}{y}{server}{events}{attributes}{tags}{last_pull}
\newcommand{\drawMISPIntegration}[7]{
    \node[legend box, anchor=north west, minimum width=6cm,
          draw=orange!80, line width=2pt] at (#1,#2) {
        \begin{tabular}{ll}
            \multicolumn{2}{c}{\textbf{MISP Integration}} \\
            \hline
            \textbf{Server:} & #3 \\
            \textbf{Events:} & #4 \\
            \textbf{Attributes:} & #5 \\
            \textbf{Tags:} & #6 \\
            \textbf{Last Pull:} & #7 \\
        \end{tabular}
    };
}

% OpenCTI (Open Cyber Threat Intelligence) platform
% Usage: \drawOpenCTI{x}{y}{observables}{indicators}{reports}{connectors_active}
\newcommand{\drawOpenCTI}[6]{
    \node[legend box, anchor=north west, minimum width=5.5cm,
          draw=purple!60, line width=2pt] at (#1,#2) {
        \begin{tabular}{lc}
            \multicolumn{2}{c}{\textbf{OpenCTI Platform}} \\
            \hline
            Observables & #3 \\
            Indicators & #4 \\
            Reports & #5 \\
            Connectors & \textcolor{green!60!black}{#6} \\
        \end{tabular}
    };
}

% Threat intelligence quality metrics
% Usage: \drawTIQualityMetrics{x}{y}{freshness_score}{accuracy}{coverage}{actionability}
\newcommand{\drawTIQualityMetrics}[6]{
    \node[legend box, anchor=north west, minimum width=6cm,
          draw=blue!60, line width=2pt] at (#1,#2) {
        \begin{tabular}{lc}
            \multicolumn{2}{c}{\textbf{TI Quality Metrics}} \\
            \hline
            Freshness & \drawQualityBar{#3} \\
            Accuracy & \drawQualityBar{#4} \\
            Coverage & \drawQualityBar{#5} \\
            Actionability & \drawQualityBar{#6} \\
        \end{tabular}
    };
}

% Helper for quality metric bars
\newcommand{\drawQualityBar}[1]{
    \pgfmathsetmacro{\quality}{#1}
    \ifthenelse{\lengthtest{\quality pt > 85 pt}}{
        \textcolor{green!60!black}{#1\% ●●●●●}
    }{
    \ifthenelse{\lengthtest{\quality pt > 70 pt}}{
        \textcolor{green!60!black}{#1\% ●●●●}\textcolor{gray!40}{●}
    }{
    \ifthenelse{\lengthtest{\quality pt > 50 pt}}{
        \textcolor{orange}{#1\% ●●●}\textcolor{gray!40}{●●}
    }{
        \textcolor{red!70}{#1\% ●●}\textcolor{gray!40}{●●●}
    }}}
}

% ============================================================================
% ADVANCED FORENSICS AND MEMORY ANALYSIS
% ============================================================================

% Memory dump analysis results
% Usage: \drawMemoryAnalysis{x}{y}{process_name}{pid}{malicious_indicators}{volatility_plugins}
\newcommand{\drawMemoryAnalysis}[6]{
    \pgfmathsetmacro{\indicators}{#5}
    \ifthenelse{\lengthtest{\indicators pt > 5 pt}}{
        \def\memcolor{threatCritical}
    }{
    \ifthenelse{\lengthtest{\indicators pt > 2 pt}}{
        \def\memcolor{threatHigh}
    }{
        \def\memcolor{orange}
    }}

    \node[legend box, anchor=north west, minimum width=6cm,
          draw=\memcolor, line width=2pt] at (#1,#2) {
        \begin{tabular}{ll}
            \multicolumn{2}{c}{\textbf{\textcolor{\memcolor}{Memory Analysis}}} \\
            \hline
            \textbf{Process:} & #3 \\
            \textbf{PID:} & #4 \\
            \textbf{Malicious Indicators:} & \textcolor{\memcolor}{#5} \\
            \hline
            \multicolumn{2}{l}{\textbf{Plugins Run:}} \\
            \multicolumn{2}{p{5.5cm}}{\tiny #6} \\
        \end{tabular}
    };
}

% Code injection detection
% Usage: \markCodeInjection{node}{technique}{target_process}{injected_code_hash}
\newcommand{\markCodeInjection}[4]{
    \node[draw=threatCritical, fill=red!30, text=black,
          font=\tiny\bfseries, rounded corners=2pt, inner sep=3pt,
          anchor=north] at (#1.north) [yshift=0.6cm] {
        CODE INJECTION DETECTED
    };
    \node[legend box, anchor=north, minimum width=4.5cm,
          draw=threatCritical, line width=2pt] at (#1.north) [yshift=0.1cm] {
        \begin{tabular}{ll}
            \textbf{Technique:} & #2 \\
            \textbf{Target:} & #3 \\
            \textbf{Hash:} & \texttt{\tiny #4} \\
        \end{tabular}
    };
}

% Process hollowing detection
% Usage: \markProcessHollowing{node}{process}{legitimate_path}{actual_code}
\newcommand{\markProcessHollowing}[4]{
    \node[draw=red!80!black, fill=yellow!40, text=black,
          font=\tiny\bfseries, rounded corners=2pt, inner sep=2pt,
          anchor=west] at (#1.west) [xshift=0.2cm] {
        PROCESS HOLLOWING
    };
    \node[below, font=\tiny, fill=white, draw=red!80!black,
          inner sep=2pt] at (#1.west) [xshift=0.8cm, yshift=-0.4cm] {
        Process: #2 | Path: #3
    };
}

% DLL hijacking detection
% Usage: \markDLLHijacking{node}{dll_name}{legitimate_path}{malicious_path}
\newcommand{\markDLLHijacking}[4]{
    \node[legend box, anchor=south west, minimum width=5cm,
          draw=threatHigh, line width=2pt] at (#1.south west) [yshift=-1cm] {
        \begin{tabular}{ll}
            \multicolumn{2}{c}{\textbf{\textcolor{threatHigh}{DLL Hijacking}}} \\
            \hline
            \textbf{DLL:} & #2 \\
            \textbf{Expected:} & \texttt{\tiny #3} \\
            \textbf{Actual:} & \textcolor{red!70}{\texttt{\tiny #4}} \\
        \end{tabular}
    };
}

% Registry persistence mechanism
% Usage: \markRegistryPersistence{node}{key_path}{value_name}{persistence_type}
\newcommand{\markRegistryPersistence}[4]{
    \node[legend box, anchor=east, minimum width=5.5cm,
          draw=threatHigh, line width=2pt] at (#1.east) [xshift=1cm] {
        \begin{tabular}{p{5cm}}
            \multicolumn{1}{c}{\textbf{\textcolor{threatHigh}{Registry Persistence}}} \\
            \hline
            \textbf{Type:} #4 \\
            \textbf{Key:} \texttt{\tiny #2} \\
            \textbf{Value:} \texttt{\tiny #3} \\
        \end{tabular}
    };
}

% Rootkit detection indicator
% Usage: \markRootkit{node}{rootkit_name}{type}{hidden_objects}
\newcommand{\markRootkit}[4]{
    \node[draw=black, fill=red!80!black, text=white,
          font=\small\bfseries, rounded corners=3pt, inner sep=4pt,
          anchor=north] at (#1.north) [yshift=0.9cm] {
        ⚠ ROOTKIT DETECTED ⚠
    };
    \node[legend box, anchor=north, minimum width=4.5cm,
          draw=black, line width=3pt, fill=red!10] at (#1.north) [yshift=0.4cm] {
        \begin{tabular}{ll}
            \textbf{Name:} & #2 \\
            \textbf{Type:} & #3 \\
            \textbf{Hidden Objects:} & #4 \\
        \end{tabular}
    };
}

% Timeline analysis results
% Usage: \drawForensicTimeline{x}{y}{start_time}{end_time}{key_events}
\newcommand{\drawForensicTimeline}[5]{
    \node[legend box, anchor=north west, minimum width=7cm,
          draw=blue!70, line width=2pt] at (#1,#2) {
        \begin{tabular}{p{6.5cm}}
            \multicolumn{1}{c}{\textbf{Forensic Timeline}} \\
            \hline
            \textbf{Period:} #3 to #4 \\
            \hline
            \textbf{Key Events:} \\
            #5 \\
        \end{tabular}
    };
}

% Disk forensics findings
% Usage: \drawDiskForensics{x}{y}{artifacts_found}{deleted_files}{encrypted_volumes}{suspicious_files}
\newcommand{\drawDiskForensics}[6]{
    \node[legend box, anchor=north west, minimum width=6cm,
          draw=purple!70, line width=2pt] at (#1,#2) {
        \begin{tabular}{lc}
            \multicolumn{2}{c}{\textbf{Disk Forensics}} \\
            \hline
            Artifacts Found & #3 \\
            Deleted Files Recovered & #4 \\
            Encrypted Volumes & #5 \\
            Suspicious Files & \textcolor{threatHigh}{#6} \\
        \end{tabular}
    };
}

% Network forensics capture analysis
% Usage: \drawNetworkForensics{x}{y}{packets}{malicious_traffic}{c2_beacons}{exfil_attempts}
\newcommand{\drawNetworkForensics}[6]{
    \node[legend box, anchor=north west, minimum width=6cm,
          draw=orange!80, line width=2pt] at (#1,#2) {
        \begin{tabular}{lc}
            \multicolumn{2}{c}{\textbf{Network Forensics}} \\
            \hline
            Packets Analyzed & #3 \\
            Malicious Traffic & \textcolor{threatHigh}{#4} \\
            C2 Beacons & \textcolor{threatCritical}{#5} \\
            Exfil Attempts & \textcolor{red!80}{#6} \\
        \end{tabular}
    };
}

% ============================================================================
% MACHINE LEARNING / AI DETECTION AND CONFIDENCE SCORING
% ============================================================================

% ML-based anomaly detection score
% Usage: \drawMLAnomalyScore{x}{y}{target}{anomaly_score}{confidence}{model_name}
\newcommand{\drawMLAnomalyScore}[6]{
    \pgfmathsetmacro{\anomalyscore}{#4}
    \ifthenelse{\lengthtest{\anomalyscore pt > 85 pt}}{
        \def\mlcolor{threatCritical}
        \def\mllabel{CRITICAL ANOMALY}
    }{
    \ifthenelse{\lengthtest{\anomalyscore pt > 70 pt}}{
        \def\mlcolor{threatHigh}
        \def\mllabel{HIGH ANOMALY}
    }{
    \ifthenelse{\lengthtest{\anomalyscore pt > 50 pt}}{
        \def\mlcolor{orange}
        \def\mllabel{MODERATE ANOMALY}
    }{
        \def\mlcolor{green!60!black}
        \def\mllabel{NORMAL}
    }}}

    \node[legend box, anchor=north west, minimum width=6cm,
          draw=\mlcolor, line width=2pt] at (#1,#2) {
        \begin{tabular}{ll}
            \multicolumn{2}{c}{\textbf{\textcolor{\mlcolor}{ML Anomaly Detection}}} \\
            \hline
            \textbf{Target:} & #3 \\
            \textbf{Anomaly Score:} & \textcolor{\mlcolor}{#4/100} \\
            \textbf{Confidence:} & #5\% \\
            \textbf{Model:} & #6 \\
            \hline
            \textbf{Classification:} & \textcolor{\mlcolor}{\mllabel} \\
        \end{tabular}
    };
}

% AI threat classification with confidence levels
% Usage: \drawAIThreatClassification{x}{y}{sample}{predicted_class}{confidence}{alternatives}
\newcommand{\drawAIThreatClassification}[6]{
    \pgfmathsetmacro{\conf}{#5}
    \ifthenelse{\lengthtest{\conf pt > 90 pt}}{
        \def\aicolor{green!60!black}
        \def\aiconfidlabel{VERY HIGH}
    }{
    \ifthenelse{\lengthtest{\conf pt > 75 pt}}{
        \def\aicolor{blue!70}
        \def\aiconfidlabel{HIGH}
    }{
    \ifthenelse{\lengthtest{\conf pt > 60 pt}}{
        \def\aicolor{orange}
        \def\aiconfidlabel{MODERATE}
    }{
        \def\aicolor{red!70}
        \def\aiconfidlabel{LOW}
    }}}

    \node[legend box, anchor=north west, minimum width=6.5cm,
          draw=\aicolor, line width=2pt] at (#1,#2) {
        \begin{tabular}{p{6cm}}
            \multicolumn{1}{c}{\textbf{AI Threat Classification}} \\
            \hline
            \textbf{Sample:} #3 \\
            \textbf{Predicted Class:} \textcolor{threatHigh}{#4} \\
            \textbf{Confidence:} \textcolor{\aicolor}{#5\% (\aiconfidlabel)} \\
            \hline
            \textbf{Alternative Classifications:} \\
            \tiny #6 \\
        \end{tabular}
    };
}

% Model performance metrics dashboard
% Usage: \drawMLModelMetrics{x}{y}{model_name}{accuracy}{precision}{recall}{f1_score}
\newcommand{\drawMLModelMetrics}[7]{
    \node[legend box, anchor=north west, minimum width=6cm,
          draw=blue!70, line width=2pt] at (#1,#2) {
        \begin{tabular}{lc}
            \multicolumn{2}{c}{\textbf{ML Model: #3}} \\
            \hline
            Accuracy & #4\% \\
            Precision & #5\% \\
            Recall & #6\% \\
            F1 Score & #7\% \\
        \end{tabular}
    };
}

% False positive/negative rate indicator
% Usage: \drawFPRates{x}{y}{model}{false_positives}{false_negatives}{total_predictions}
\newcommand{\drawFPRates}[6]{
    \pgfmathsetmacro{\fpr}{(#4 / #6) * 100}
    \pgfmathsetmacro{\fnr}{(#5 / #6) * 100}

    \node[legend box, anchor=north west, minimum width=5.5cm,
          draw=orange!70, line width=2pt] at (#1,#2) {
        \begin{tabular}{lc}
            \multicolumn{2}{c}{\textbf{Error Rates: #3}} \\
            \hline
            False Positives & #4 (\pgfmathprintnumber[precision=1]{\fpr}\%) \\
            False Negatives & #5 (\pgfmathprintnumber[precision=1]{\fnr}\%) \\
            Total Predictions & #6 \\
        \end{tabular}
    };
}

% Ensemble model prediction
% Usage: \drawEnsemblePrediction{x}{y}{target}{models_count}{consensus}{final_prediction}
\newcommand{\drawEnsemblePrediction}[6]{
    \pgfmathsetmacro{\consensus}{#5}
    \ifthenelse{\lengthtest{\consensus pt > 80 pt}}{
        \def\ensemblecolor{green!60!black}
    }{
    \ifthenelse{\lengthtest{\consensus pt > 60 pt}}{
        \def\ensemblecolor{blue!70}
    }{
        \def\ensemblecolor{orange}
    }}

    \node[legend box, anchor=north west, minimum width=6cm,
          draw=\ensemblecolor, line width=2pt] at (#1,#2) {
        \begin{tabular}{ll}
            \multicolumn{2}{c}{\textbf{Ensemble Prediction: #3}} \\
            \hline
            \textbf{Models Voting:} & #4 \\
            \textbf{Consensus:} & \textcolor{\ensemblecolor}{#5\%} \\
            \textbf{Final Prediction:} & #6 \\
        \end{tabular}
    };
}

% Feature importance visualization
% Usage: \drawFeatureImportance{x}{y}{model}{top_feature}{importance}{second_feature}{importance2}
\newcommand{\drawFeatureImportance}[7]{
    \node[legend box, anchor=north west, minimum width=6cm,
          draw=purple!60, line width=2pt] at (#1,#2) {
        \begin{tabular}{lc}
            \multicolumn{2}{c}{\textbf{Feature Importance: #3}} \\
            \hline
            #4 & \drawImportanceBar{#5} \\
            #6 & \drawImportanceBar{#7} \\
        \end{tabular}
    };
}

% Helper for importance bars
\newcommand{\drawImportanceBar}[1]{
    \pgfmathsetmacro{\importance}{#1}
    \ifthenelse{\lengthtest{\importance pt > 80 pt}}{
        \textcolor{blue!70}{#1\% ████████}
    }{
    \ifthenelse{\lengthtest{\importance pt > 60 pt}}{
        \textcolor{blue!60}{#1\% ██████}\textcolor{gray!30}{██}
    }{
    \ifthenelse{\lengthtest{\importance pt > 40 pt}}{
        \textcolor{blue!50}{#1\% ████}\textcolor{gray!30}{████}
    }{
        \textcolor{blue!40}{#1\% ██}\textcolor{gray!30}{██████}
    }}}
}

% Model drift detection
% Usage: \drawModelDrift{x}{y}{model}{baseline_accuracy}{current_accuracy}{drift_detected}
\newcommand{\drawModelDrift}[6]{
    \ifthenelse{\equal{#6}{yes}}{
        \def\driftcolor{red!70}
        \def\driftstatus{DRIFT DETECTED}
    }{
        \def\driftcolor{green!60!black}
        \def\driftstatus{STABLE}
    }

    \node[legend box, anchor=north west, minimum width=5.5cm,
          draw=\driftcolor, line width=2pt] at (#1,#2) {
        \begin{tabular}{ll}
            \multicolumn{2}{c}{\textbf{Model Drift: #3}} \\
            \hline
            Baseline Accuracy & #4\% \\
            Current Accuracy & #5\% \\
            \hline
            \textbf{Status} & \textcolor{\driftcolor}{\driftstatus} \\
        \end{tabular}
    };
}

% Automated retraining indicator
% Usage: \markAutoRetraining{node}{model}{trigger}{scheduled_time}
\newcommand{\markAutoRetraining}[4]{
    \node[draw=blue!70, fill=blue!20, text=black,
          font=\tiny\bfseries, rounded corners=2pt, inner sep=2pt,
          anchor=south] at (#1.south) [yshift=-0.5cm] {
        AUTO-RETRAINING SCHEDULED
    };
    \node[below, font=\tiny, fill=white, draw=blue!70,
          inner sep=2pt] at (#1.south) [yshift=-0.8cm] {
        Model: #2 | Trigger: #3 | ETA: #4
    };
}

% Deep learning threat detection
% Usage: \drawDeepLearningDetection{x}{y}{architecture}{layers}{parameters}{accuracy}
\newcommand{\drawDeepLearningDetection}[6]{
    \node[legend box, anchor=north west, minimum width=6cm,
          draw=purple!70, line width=2pt, fill=purple!5] at (#1,#2) {
        \begin{tabular}{ll}
            \multicolumn{2}{c}{\textbf{Deep Learning Model}} \\
            \hline
            \textbf{Architecture:} & #3 \\
            \textbf{Layers:} & #4 \\
            \textbf{Parameters:} & #5 \\
            \textbf{Accuracy:} & \textcolor{green!60!black}{#6\%} \\
        \end{tabular}
    };
}

% Neural network confidence heatmap
% Usage: \drawNNConfidence{node}{prediction}{confidence}{uncertainty}
\newcommand{\drawNNConfidence}[4]{
    \pgfmathsetmacro{\conf}{#3}
    \pgfmathsetmacro{\uncert}{#4}

    \ifthenelse{\lengthtest{\conf pt > 90 pt}}{
        \def\nncolor{green!60!black}
    }{
    \ifthenelse{\lengthtest{\conf pt > 75 pt}}{
        \def\nncolor{blue!70}
    }{
        \def\nncolor{orange}
    }}

    \node[draw=\nncolor, fill=\nncolor!20, font=\tiny\bfseries,
          rounded corners=2pt, inner sep=2pt, anchor=north]
        at (#1.north) [yshift=0.4cm] {
        NN: #2 (Conf: #3\%, Uncert: #4\%)
    };
}

% Explainable AI (XAI) insights
% Usage: \drawXAIExplanation{x}{y}{prediction}{key_factors}{shap_values}
\newcommand{\drawXAIExplanation}[5]{
    \node[legend box, anchor=north west, minimum width=6.5cm,
          draw=teal!70, line width=2pt] at (#1,#2) {
        \begin{tabular}{p{6cm}}
            \multicolumn{1}{c}{\textbf{XAI Explanation}} \\
            \hline
            \textbf{Prediction:} #3 \\
            \hline
            \textbf{Key Contributing Factors:} \\
            #4 \\
            \hline
            \textbf{SHAP Values:} \\
            \tiny #5 \\
        \end{tabular}
    };
}

% ============================================================================
% THREAT LANDSCAPE AND INDUSTRY TRACKING
% ============================================================================

% Global threat landscape overview
% Usage: \drawThreatLandscape{x}{y}{total_threats}{trending_up}{geographic_hotspots}{top_threat_type}
\newcommand{\drawThreatLandscape}[6]{
    \node[legend box, anchor=north west, minimum width=7cm,
          draw=red!70, line width=2.5pt] at (#1,#2) {
        \begin{tabular}{lc}
            \multicolumn{2}{c}{\textbf{Global Threat Landscape}} \\
            \hline
            Total Active Threats & \textcolor{threatHigh}{#3} \\
            Trending Upward & \textcolor{orange}{#4} \\
            Geographic Hotspots & #5 \\
            \hline
            \textbf{Top Threat Type} & \textcolor{threatCritical}{#6} \\
        \end{tabular}
    };
}

% Industry-specific threat tracking
% Usage: \drawIndustryThreats{x}{y}{industry}{specific_threats}{attack_frequency}{primary_targets}
\newcommand{\drawIndustryThreats}[6]{
    \node[legend box, anchor=north west, minimum width=6.5cm,
          draw=orange!80, line width=2pt] at (#1,#2) {
        \begin{tabular}{ll}
            \multicolumn{2}{c}{\textbf{Industry: #3}} \\
            \hline
            \textbf{Specific Threats:} & #4 \\
            \textbf{Attack Frequency:} & \textcolor{threatHigh}{#5/month} \\
            \textbf{Primary Targets:} & #6 \\
        \end{tabular}
    };
}

% Threat trend analysis
% Usage: \drawThreatTrends{x}{y}{threat_type}{last_month}{this_month}{trend}{prediction}
\newcommand{\drawThreatTrends}[7]{
    \ifthenelse{\equal{#6}{increasing}}{
        \def\trendcolor{red!70}
        \def\trendsymbol{↑}
    }{
    \ifthenelse{\equal{#6}{decreasing}}{
        \def\trendcolor{green!60!black}
        \def\trendsymbol{↓}
    }{
        \def\trendcolor{orange}
        \def\trendsymbol{→}
    }}

    \node[legend box, anchor=north west, minimum width=6cm,
          draw=\trendcolor, line width=2pt] at (#1,#2) {
        \begin{tabular}{ll}
            \multicolumn{2}{c}{\textbf{Trend: #3}} \\
            \hline
            Last Month & #4 incidents \\
            This Month & #5 incidents \\
            \textbf{Trend} & \textcolor{\trendcolor}{\trendsymbol\ \uppercase{#6}} \\
            \textbf{Prediction} & #7 \\
        \end{tabular}
    };
}

% Emerging threat radar
% Usage: \drawEmergingThreats{x}{y}{new_threats_count}{zero_days}{novel_ttps}
\newcommand{\drawEmergingThreats}[5]{
    \node[legend box, anchor=north west, minimum width=5.5cm,
          draw=purple!70, line width=2.5pt, fill=purple!5] at (#1,#2) {
        \begin{tabular}{lc}
            \multicolumn{2}{c}{\textbf{Emerging Threats Radar}} \\
            \hline
            New Threats (7d) & \textcolor{red!70}{#3} \\
            Zero-Days & \textcolor{threatCritical}{#4} \\
            Novel TTPs & \textcolor{orange}{#5} \\
        \end{tabular}
    };
}

% Threat actor activity heatmap
% Usage: \drawActorActivityMap{x}{y}{active_groups}{campaigns}{targets}{peak_activity}
\newcommand{\drawActorActivityMap}[6]{
    \node[legend box, anchor=north west, minimum width=6cm,
          draw=red!80, line width=2pt] at (#1,#2) {
        \begin{tabular}{lc}
            \multicolumn{2}{c}{\textbf{Threat Actor Activity}} \\
            \hline
            Active Groups & #3 \\
            Active Campaigns & #4 \\
            Targeted Sectors & #5 \\
            Peak Activity Time & #6 \\
        \end{tabular}
    };
}

% Seasonal threat patterns
% Usage: \drawSeasonalPatterns{x}{y}{season}{typical_threats}{expected_increase}{mitigation}
\newcommand{\drawSeasonalPatterns}[6]{
    \node[legend box, anchor=north west, minimum width=6.5cm,
          draw=blue!60, line width=2pt] at (#1,#2) {
        \begin{tabular}{p{6cm}}
            \multicolumn{1}{c}{\textbf{Seasonal Pattern: #3}} \\
            \hline
            \textbf{Typical Threats:} #4 \\
            \textbf{Expected Increase:} \textcolor{orange}{#5\%} \\
            \hline
            \textbf{Recommended Mitigation:} \\
            \tiny #6 \\
        \end{tabular}
    };
}

% ============================================================================
% RISK MATRICES AND THREAT HEAT MAPS
% ============================================================================

% Risk assessment matrix (Likelihood vs Impact)
% Usage: \drawRiskMatrix{x}{y}{likelihood}{impact}{risk_level}
\newcommand{\drawRiskMatrix}[5]{
    % Calculate risk score
    \pgfmathsetmacro{\riskscore}{(#3 * #4) / 100}

    \ifthenelse{\lengthtest{\riskscore pt > 7 pt}}{
        \def\riskcolor{threatCritical}
        \def\risklabel{CRITICAL}
    }{
    \ifthenelse{\lengthtest{\riskscore pt > 5 pt}}{
        \def\riskcolor{threatHigh}
        \def\risklabel{HIGH}
    }{
    \ifthenelse{\lengthtest{\riskscore pt > 3 pt}}{
        \def\riskcolor{orange}
        \def\risklabel{MEDIUM}
    }{
        \def\riskcolor{green!60!black}
        \def\risklabel{LOW}
    }}}

    \node[legend box, anchor=north west, minimum width=5.5cm,
          draw=\riskcolor, line width=2.5pt] at (#1,#2) {
        \begin{tabular}{lc}
            \multicolumn{2}{c}{\textbf{Risk Assessment Matrix}} \\
            \hline
            Likelihood & #3/10 \\
            Impact & #4/10 \\
            \hline
            \textbf{Risk Score} & \textcolor{\riskcolor}{\pgfmathprintnumber[precision=1]{\riskscore}/10} \\
            \textbf{Level} & \textcolor{\riskcolor}{\risklabel\ RISK} \\
        \end{tabular}
    };
}

% Comprehensive 5x5 risk matrix visualization
% Usage: \drawFullRiskMatrix{x}{y}{position_likelihood}{position_impact}
\newcommand{\drawFullRiskMatrix}[4]{
    \pgfmathsetmacro{\cellsize}{0.8}
    \pgfmathsetmacro{\matrixwidth}{5 * \cellsize}

    % Draw matrix grid
    \node[anchor=north west] at (#1,#2) {
        \begin{tabular}{|c|c|c|c|c|c|}
            \hline
            \diagbox{Impact}{Likelihood} & Rare & Unlikely & Possible & Likely & Certain \\
            \hline
            Severe & \cellcolor{orange} & \cellcolor{orange!80} & \cellcolor{red!60} & \cellcolor{red!80} & \cellcolor{red!90!black} \\
            \hline
            Major & \cellcolor{yellow!60} & \cellcolor{orange!60} & \cellcolor{orange!80} & \cellcolor{red!60} & \cellcolor{red!80} \\
            \hline
            Moderate & \cellcolor{green!60} & \cellcolor{yellow!60} & \cellcolor{orange!60} & \cellcolor{orange!80} & \cellcolor{red!60} \\
            \hline
            Minor & \cellcolor{green!70} & \cellcolor{green!60} & \cellcolor{yellow!60} & \cellcolor{orange!60} & \cellcolor{orange!80} \\
            \hline
            Insignificant & \cellcolor{green!80} & \cellcolor{green!70} & \cellcolor{green!60} & \cellcolor{yellow!60} & \cellcolor{orange!60} \\
            \hline
        \end{tabular}
    };
}

% Threat heat map with color gradient
% Usage: \drawThreatHeatMap{x}{y}{critical_zones}{high_zones}{medium_zones}{low_zones}
\newcommand{\drawThreatHeatMap}[6]{
    \node[legend box, anchor=north west, minimum width=6.5cm,
          draw=black, line width=2pt] at (#1,#2) {
        \begin{tabular}{lcc}
            \multicolumn{3}{c}{\textbf{Threat Heat Map}} \\
            \hline
            \textbf{Zone} & \textbf{Count} & \textbf{Intensity} \\
            \hline
            Critical & #3 & \cellcolor{red!90!black}\textcolor{white}{█████} \\
            High & #4 & \cellcolor{orange!80}████ \\
            Medium & #5 & \cellcolor{yellow!70}███ \\
            Low & #6 & \cellcolor{green!60}██ \\
        \end{tabular}
    };
}

% Asset criticality matrix
% Usage: \drawAssetCriticality{x}{y}{asset_name}{business_impact}{recovery_time}{criticality_score}
\newcommand{\drawAssetCriticality}[6]{
    \pgfmathsetmacro{\criticality}{#6}
    \ifthenelse{\lengthtest{\criticality pt > 85 pt}}{
        \def\critcolor{threatCritical}
        \def\critlabel{MISSION CRITICAL}
    }{
    \ifthenelse{\lengthtest{\criticality pt > 70 pt}}{
        \def\critcolor{orange}
        \def\critlabel{BUSINESS CRITICAL}
    }{
    \ifthenelse{\lengthtest{\criticality pt > 50 pt}}{
        \def\critcolor{yellow!80!black}
        \def\critlabel{IMPORTANT}
    }{
        \def\critcolor{green!60!black}
        \def\critlabel{NON-CRITICAL}
    }}}

    \node[legend box, anchor=north west, minimum width=6cm,
          draw=\critcolor, line width=2pt] at (#1,#2) {
        \begin{tabular}{ll}
            \multicolumn{2}{c}{\textbf{Asset: #3}} \\
            \hline
            Business Impact & #4/10 \\
            Recovery Time (RTO) & #5h \\
            \hline
            \textbf{Criticality} & \textcolor{\critcolor}{#6/100} \\
            \textbf{Classification} & \textcolor{\critcolor}{\critlabel} \\
        \end{tabular}
    };
}

% Vulnerability heat map for network zones
% Usage: \drawVulnerabilityHeatMap{x}{y}{dmz_score}{internal_score}{critical_score}{external_score}
\newcommand{\drawVulnerabilityHeatMap}[6]{
    \node[legend box, anchor=north west, minimum width=6.5cm,
          draw=black, line width=2pt] at (#1,#2) {
        \begin{tabular}{lc}
            \multicolumn{2}{c}{\textbf{Vulnerability Heat Map}} \\
            \hline
            \textbf{Zone} & \textbf{Vulnerability Score} \\
            \hline
            DMZ & \drawHeatBar{#3} \\
            Internal Network & \drawHeatBar{#4} \\
            Critical Assets & \drawHeatBar{#5} \\
            External Facing & \drawHeatBar{#6} \\
        \end{tabular}
    };
}

% Helper for heat map color bars
\newcommand{\drawHeatBar}[1]{
    \pgfmathsetmacro{\heatscore}{#1}
    \ifthenelse{\lengthtest{\heatscore pt > 80 pt}}{
        \cellcolor{red!90!black}\textcolor{white}{#1}
    }{
    \ifthenelse{\lengthtest{\heatscore pt > 60 pt}}{
        \cellcolor{orange!80}{#1}
    }{
    \ifthenelse{\lengthtest{\heatscore pt > 40 pt}}{
        \cellcolor{yellow!70}{#1}
    }{
        \cellcolor{green!60}{#1}
    }}}
}

% Geographic threat heat map
% Usage: \drawGeographicThreatMap{x}{y}{region}{threat_count}{severity}{trend}
\newcommand{\drawGeographicThreatMap}[6]{
    \pgfmathsetmacro{\severity}{#5}
    \ifthenelse{\lengthtest{\severity pt > 75 pt}}{
        \def\geocolor{red!80!black}
    }{
    \ifthenelse{\lengthtest{\severity pt > 50 pt}}{
        \def\geocolor{orange!80}
    }{
        \def\geocolor{yellow!70}
    }}

    \node[legend box, anchor=north west, minimum width=5.5cm,
          draw=\geocolor, line width=2pt, fill=\geocolor!10] at (#1,#2) {
        \begin{tabular}{ll}
            \multicolumn{2}{c}{\textbf{Region: #3}} \\
            \hline
            Threat Count & #4 \\
            Severity & \textcolor{\geocolor}{#5/100} \\
            Trend & #6 \\
        \end{tabular}
    };
}

% Time-based threat heat map
% Usage: \drawTemporalHeatMap{x}{y}{hour}{day}{week}{month}
\newcommand{\drawTemporalHeatMap}[6]{
    \node[legend box, anchor=north west, minimum width=6cm,
          draw=blue!70, line width=2pt] at (#1,#2) {
        \begin{tabular}{lc}
            \multicolumn{2}{c}{\textbf{Temporal Threat Pattern}} \\
            \hline
            \textbf{Time Period} & \textbf{Activity} \\
            \hline
            Past Hour & \drawThreatIntensity{#3} \\
            Past 24 Hours & \drawThreatIntensity{#4} \\
            Past Week & \drawThreatIntensity{#5} \\
            Past Month & \drawThreatIntensity{#6} \\
        \end{tabular}
    };
}

% Helper for threat intensity visualization
\newcommand{\drawThreatIntensity}[1]{
    \pgfmathsetmacro{\intensity}{#1}
    \ifthenelse{\lengthtest{\intensity pt > 80 pt}}{
        \textcolor{red!80!black}{#1 ████████}
    }{
    \ifthenelse{\lengthtest{\intensity pt > 60 pt}}{
        \textcolor{orange}{#1 ██████}\textcolor{gray!30}{██}
    }{
    \ifthenelse{\lengthtest{\intensity pt > 40 pt}}{
        \textcolor{yellow!80!black}{#1 ████}\textcolor{gray!30}{████}
    }{
        \textcolor{green!60!black}{#1 ██}\textcolor{gray!30}{██████}
    }}}
}

% Risk scoring grid for multiple assets
% Usage: \drawRiskScoringGrid{x}{y}{asset1}{score1}{asset2}{score2}{asset3}{score3}
\newcommand{\drawRiskScoringGrid}[8]{
    \node[legend box, anchor=north west, minimum width=6.5cm,
          draw=black, line width=2pt] at (#1,#2) {
        \begin{tabular}{lc}
            \multicolumn{2}{c}{\textbf{Risk Scoring Grid}} \\
            \hline
            \textbf{Asset} & \textbf{Risk Score} \\
            \hline
            #3 & \drawRiskScore{#4} \\
            #5 & \drawRiskScore{#6} \\
            #7 & \drawRiskScore{#8} \\
        \end{tabular}
    };
}

% Helper for risk score visualization
\newcommand{\drawRiskScore}[1]{
    \pgfmathsetmacro{\risk}{#1}
    \ifthenelse{\lengthtest{\risk pt > 80 pt}}{
        \cellcolor{red!90!black}\textcolor{white}{\textbf{#1}}
    }{
    \ifthenelse{\lengthtest{\risk pt > 60 pt}}{
        \cellcolor{orange!80}{#1}
    }{
    \ifthenelse{\lengthtest{\risk pt > 40 pt}}{
        \cellcolor{yellow!70}{#1}
    }{
        \cellcolor{green!60}{#1}
    }}}
}

% ============================================================================
% BREACH PROBABILITY AND IMPACT MODELING
% ============================================================================

% Breach probability calculator
% Usage: \drawBreachProbability{x}{y}{target}{vulnerabilities}{exposure}{controls}{probability}
\newcommand{\drawBreachProbability}[7]{
    \pgfmathsetmacro{\prob}{#7}
    \ifthenelse{\lengthtest{\prob pt > 75 pt}}{
        \def\probcolor{threatCritical}
        \def\problabel{VERY HIGH}
    }{
    \ifthenelse{\lengthtest{\prob pt > 50 pt}}{
        \def\probcolor{orange}
        \def\problabel{HIGH}
    }{
    \ifthenelse{\lengthtest{\prob pt > 25 pt}}{
        \def\probcolor{yellow!80!black}
        \def\problabel{MODERATE}
    }{
        \def\probcolor{green!60!black}
        \def\problabel{LOW}
    }}}

    \node[legend box, anchor=north west, minimum width=6.5cm,
          draw=\probcolor, line width=2.5pt] at (#1,#2) {
        \begin{tabular}{ll}
            \multicolumn{2}{c}{\textbf{Breach Probability: #3}} \\
            \hline
            Known Vulnerabilities & #4 \\
            Attack Surface Exposure & #5\% \\
            Security Controls & #6/10 \\
            \hline
            \textbf{Breach Probability} & \textcolor{\probcolor}{\textbf{#7\%}} \\
            \textbf{Risk Level} & \textcolor{\probcolor}{\problabel} \\
        \end{tabular}
    };
}

% Impact modeling dashboard
% Usage: \drawImpactModeling{x}{y}{target}{financial_impact}{operational_impact}{reputational_impact}{total_impact}
\newcommand{\drawImpactModeling}[7]{
    \node[legend box, anchor=north west, minimum width=7cm,
          draw=red!70, line width=2pt] at (#1,#2) {
        \begin{tabular}{lc}
            \multicolumn{2}{c}{\textbf{Impact Assessment: #3}} \\
            \hline
            \textbf{Impact Category} & \textbf{Score} \\
            \hline
            Financial Impact & \drawImpactBar{#4} \\
            Operational Impact & \drawImpactBar{#5} \\
            Reputational Impact & \drawImpactBar{#6} \\
            \hline
            \textbf{Total Impact Score} & \drawImpactBar{#7} \\
        \end{tabular}
    };
}

% Helper for impact visualization
\newcommand{\drawImpactBar}[1]{
    \pgfmathsetmacro{\impact}{#1}
    \ifthenelse{\lengthtest{\impact pt > 80 pt}}{
        \textcolor{red!80!black}{#1/100 ●●●●●}
    }{
    \ifthenelse{\lengthtest{\impact pt > 60 pt}}{
        \textcolor{orange}{#1/100 ●●●●}\textcolor{gray!40}{●}
    }{
    \ifthenelse{\lengthtest{\impact pt > 40 pt}}{
        \textcolor{yellow!80!black}{#1/100 ●●●}\textcolor{gray!40}{●●}
    }{
        \textcolor{green!60!black}{#1/100 ●●}\textcolor{gray!40}{●●●}
    }}}
}

% Mean Time to Breach (MTTB) prediction
% Usage: \drawMTTBPrediction{x}{y}{target}{current_controls}{predicted_mttb}{confidence}
\newcommand{\drawMTTBPrediction}[6]{
    \pgfmathsetmacro{\mttb}{#5}
    \ifthenelse{\lengthtest{\mttb pt < 30 pt}}{
        \def\mttbcolor{threatCritical}
        \def\mttblabel{CRITICAL}
    }{
    \ifthenelse{\lengthtest{\mttb pt < 90 pt}}{
        \def\mttbcolor{orange}
        \def\mttblabel{WARNING}
    }{
        \def\mttbcolor{green!60!black}
        \def\mttblabel{ACCEPTABLE}
    }}

    \node[legend box, anchor=north west, minimum width=6cm,
          draw=\mttbcolor, line width=2pt] at (#1,#2) {
        \begin{tabular}{ll}
            \multicolumn{2}{c}{\textbf{MTTB Prediction: #3}} \\
            \hline
            Current Controls & #4/10 \\
            Predicted MTTB & \textcolor{\mttbcolor}{#5 days} \\
            Confidence & #6\% \\
            \hline
            \textbf{Status} & \textcolor{\mttbcolor}{\mttblabel} \\
        \end{tabular}
    };
}

% Cost-benefit analysis for security investment
% Usage: \drawSecurityROI{x}{y}{investment}{risk_reduction}{annual_loss_expectancy}{roi}
\newcommand{\drawSecurityROI}[6]{
    \pgfmathsetmacro{\roi}{#6}
    \ifthenelse{\lengthtest{\roi pt > 100 pt}}{
        \def\roicolor{green!60!black}
        \def\roilabel{POSITIVE ROI}
    }{
        \def\roicolor{red!70}
        \def\roilabel{NEGATIVE ROI}
    }

    \node[legend box, anchor=north west, minimum width=6.5cm,
          draw=\roicolor, line width=2pt] at (#1,#2) {
        \begin{tabular}{ll}
            \multicolumn{2}{c}{\textbf{Security Investment ROI}} \\
            \hline
            Investment Cost & \$#3 \\
            Risk Reduction & #4\% \\
            Annual Loss Expectancy & \$#5 \\
            \hline
            \textbf{ROI} & \textcolor{\roicolor}{#6\%} \\
            \textbf{Assessment} & \textcolor{\roicolor}{\roilabel} \\
        \end{tabular}
    };
}

% Cyber insurance coverage analysis
% Usage: \drawCyberInsurance{x}{y}{coverage_limit}{deductible}{premium}{gap_analysis}
\newcommand{\drawCyberInsurance}[6]{
    \node[legend box, anchor=north west, minimum width=6cm,
          draw=blue!70, line width=2pt] at (#1,#2) {
        \begin{tabular}{ll}
            \multicolumn{2}{c}{\textbf{Cyber Insurance Coverage}} \\
            \hline
            Coverage Limit & \$#3 \\
            Deductible & \$#4 \\
            Annual Premium & \$#5 \\
            \hline
            \textbf{Coverage Gap} & \textcolor{orange}{#6\%} \\
        \end{tabular}
    };
}

% ============================================================================
% EXECUTIVE SUMMARY AND REPORTING TOOLS
% ============================================================================

% Executive security dashboard (C-level summary)
% Usage: \drawExecutiveDashboard{x}{y}{risk_score}{incidents}{compliance}{investment}
\newcommand{\drawExecutiveDashboard}[6]{
    \pgfmathsetmacro{\risk}{#3}
    \ifthenelse{\lengthtest{\risk pt > 70 pt}}{
        \def\execcolor{red!70}
    }{
    \ifthenelse{\lengthtest{\risk pt > 50 pt}}{
        \def\execcolor{orange}
    }{
        \def\execcolor{green!60!black}
    }}

    \node[legend box, anchor=north west, minimum width=8cm,
          draw=\execcolor, line width=3pt, fill=\execcolor!5] at (#1,#2) {
        \begin{tabular}{lc}
            \multicolumn{2}{c}{\Large\textbf{EXECUTIVE SECURITY DASHBOARD}} \\
            \hline
            \textbf{Overall Risk Score} & \textcolor{\execcolor}{\Large\textbf{#3/100}} \\
            \hline
            Active Incidents & \textcolor{red!70}{#4} \\
            Compliance Status & #5\% \\
            Security Investment & \$#6 \\
        \end{tabular}
    };
}

% Security posture briefing for board
% Usage: \drawBoardBriefing{x}{y}{quarter}{maturity_level}{major_incidents}{budget_utilization}
\newcommand{\drawBoardBriefing}[6]{
    \node[legend box, anchor=north west, minimum width=8.5cm,
          draw=blue!70, line width=2.5pt] at (#1,#2) {
        \begin{tabular}{p{8cm}}
            \multicolumn{1}{c}{\Large\textbf{BOARD SECURITY BRIEFING}} \\
            \multicolumn{1}{c}{\textit{#3}} \\
            \hline
            \\
            \textbf{Security Maturity Level:} #4/5 \\[0.2cm]
            \textbf{Major Incidents This Quarter:} \textcolor{orange}{#5} \\[0.2cm]
            \textbf{Security Budget Utilization:} #6\% \\[0.2cm]
        \end{tabular}
    };
}

% Incident response summary
% Usage: \drawIncidentSummary{x}{y}{incident_id}{severity}{status}{actions_taken}{next_steps}
\newcommand{\drawIncidentSummary}[7]{
    \ifthenelse{\equal{#4}{critical}}{
        \def\sumcolor{threatCritical}
    }{
    \ifthenelse{\equal{#4}{high}}{
        \def\sumcolor{orange}
    }{
        \def\sumcolor{yellow!80!black}
    }}

    \node[legend box, anchor=north west, minimum width=7.5cm,
          draw=\sumcolor, line width=2pt] at (#1,#2) {
        \begin{tabular}{p{7cm}}
            \multicolumn{1}{c}{\textbf{\textcolor{\sumcolor}{Incident Summary: #3}}} \\
            \hline
            \textbf{Severity:} \textcolor{\sumcolor}{\uppercase{#4}} \\
            \textbf{Status:} #5 \\
            \hline
            \textbf{Actions Taken:} \\
            \small #6 \\
            \hline
            \textbf{Next Steps:} \\
            \small #7 \\
        \end{tabular}
    };
}

% Compliance status report
% Usage: \drawComplianceReport{x}{y}{frameworks}{compliant}{gaps}{remediation_timeline}
\newcommand{\drawComplianceReport}[6]{
    \node[legend box, anchor=north west, minimum width=7cm,
          draw=blue!60, line width=2pt] at (#1,#2) {
        \begin{tabular}{lc}
            \multicolumn{2}{c}{\textbf{Compliance Status Report}} \\
            \hline
            Frameworks Assessed & #3 \\
            Compliant Controls & \textcolor{green!60!black}{#4} \\
            Identified Gaps & \textcolor{orange}{#5} \\
            \hline
            \textbf{Remediation Timeline} & \textbf{#6} \\
        \end{tabular}
    };
}

% Quarterly security metrics
% Usage: \drawQuarterlyMetrics{x}{y}{quarter}{threats_blocked}{incidents}{vulnerabilities_patched}{new_controls}
\newcommand{\drawQuarterlyMetrics}[7]{
    \node[legend box, anchor=north west, minimum width=7.5cm,
          draw=purple!60, line width=2pt] at (#1,#2) {
        \begin{tabular}{lc}
            \multicolumn{2}{c}{\textbf{Quarterly Security Metrics - #3}} \\
            \hline
            Threats Blocked & \textcolor{green!60!black}{#4} \\
            Security Incidents & \textcolor{orange}{#5} \\
            Vulnerabilities Patched & #6 \\
            New Controls Implemented & #7 \\
        \end{tabular}
    };
}

% Annual security report summary
% Usage: \drawAnnualReport{x}{y}{year}{total_incidents}{successful_defenses}{major_investments}{roi}
\newcommand{\drawAnnualReport}[7]{
    \node[legend box, anchor=north west, minimum width=8cm,
          draw=teal!70, line width=2.5pt, fill=teal!5] at (#1,#2) {
        \begin{tabular}{ll}
            \multicolumn{2}{c}{\Large\textbf{Annual Security Report #3}} \\
            \hline
            \textbf{Total Incidents:} & #4 \\
            \textbf{Successful Defenses:} & \textcolor{green!60!black}{#5} \\
            \textbf{Major Investments:} & \$#6 \\
            \textbf{Security ROI:} & \textcolor{green!60!black}{#7\%} \\
        \end{tabular}
    };
}

% Key Performance Indicators (KPI) dashboard
% Usage: \drawKPIDashboard{x}{y}{mttd}{mttr}{patch_rate}{uptime}
\newcommand{\drawKPIDashboard}[6]{
    \node[legend box, anchor=north west, minimum width=7cm,
          draw=blue!70, line width=2pt] at (#1,#2) {
        \begin{tabular}{lcc}
            \multicolumn{3}{c}{\textbf{Security KPI Dashboard}} \\
            \hline
            \textbf{Metric} & \textbf{Value} & \textbf{Target} \\
            \hline
            MTTD (Mean Time to Detect) & #3h & <4h \\
            MTTR (Mean Time to Respond) & #4h & <8h \\
            Patch Compliance Rate & #5\% & >95\% \\
            Security Uptime & #6\% & >99.9\% \\
        \end{tabular}
    };
}

% Security trend report
% Usage: \drawTrendReport{x}{y}{period}{trend_direction}{key_findings}
\newcommand{\drawTrendReport}[5]{
    \ifthenelse{\equal{#4}{improving}}{
        \def\trendcolor{green!60!black}
        \def\trendsymbol{↗}
    }{
    \ifthenelse{\equal{#4}{declining}}{
        \def\trendcolor{red!70}
        \def\trendsymbol{↘}
    }{
        \def\trendcolor{orange}
        \def\trendsymbol{→}
    }}

    \node[legend box, anchor=north west, minimum width=7.5cm,
          draw=\trendcolor, line width=2pt] at (#1,#2) {
        \begin{tabular}{p{7cm}}
            \multicolumn{1}{c}{\textbf{Security Trend Report}} \\
            \multicolumn{1}{c}{\textit{Period: #3}} \\
            \hline
            \textbf{Trend:} \textcolor{\trendcolor}{\Large\trendsymbol\ \uppercase{#4}} \\
            \hline
            \textbf{Key Findings:} \\
            \small #5 \\
        \end{tabular}
    };
}

% Recommendations and action items
% Usage: \drawRecommendations{x}{y}{priority}{recommendations}{owner}{deadline}
\newcommand{\drawRecommendations}[6]{
    \ifthenelse{\equal{#3}{critical}}{
        \def\reccolor{threatCritical}
    }{
    \ifthenelse{\equal{#3}{high}}{
        \def\reccolor{orange}
    }{
        \def\reccolor{blue!70}
    }}

    \node[legend box, anchor=north west, minimum width=7.5cm,
          draw=\reccolor, line width=2pt] at (#1,#2) {
        \begin{tabular}{p{7cm}}
            \multicolumn{1}{c}{\textbf{\textcolor{\reccolor}{Action Items}}} \\
            \multicolumn{1}{c}{\textit{Priority: \uppercase{#3}}} \\
            \hline
            \textbf{Recommendations:} \\
            \small #4 \\
            \hline
            \textbf{Owner:} #5 \\
            \textbf{Deadline:} #6 \\
        \end{tabular}
    };
}

% Risk register summary
% Usage: \drawRiskRegister{x}{y}{total_risks}{critical}{high}{medium}{low}
\newcommand{\drawRiskRegister}[7]{
    \node[legend box, anchor=north west, minimum width=7cm,
          draw=black, line width=2pt] at (#1,#2) {
        \begin{tabular}{lcc}
            \multicolumn{3}{c}{\textbf{Risk Register Summary}} \\
            \hline
            \textbf{Risk Level} & \textbf{Count} & \textbf{\%} \\
            \hline
            Critical & \textcolor{threatCritical}{#4} & \pgfmathparse{int((#4/#3)*100)}\pgfmathresult\% \\
            High & \textcolor{orange}{#5} & \pgfmathparse{int((#5/#3)*100)}\pgfmathresult\% \\
            Medium & \textcolor{yellow!80!black}{#6} & \pgfmathparse{int((#6/#3)*100)}\pgfmathresult\% \\
            Low & \textcolor{green!60!black}{#7} & \pgfmathparse{int((#7/#3)*100)}\pgfmathresult\% \\
            \hline
            \textbf{Total} & \textbf{#3} & \textbf{100\%} \\
        \end{tabular}
    };
}

% Security maturity assessment
% Usage: \drawMaturityAssessment{x}{y}{current_level}{target_level}{gap_analysis}
\newcommand{\drawMaturityAssessment}[5]{
    \node[legend box, anchor=north west, minimum width=7cm,
          draw=purple!70, line width=2pt] at (#1,#2) {
        \begin{tabular}{lc}
            \multicolumn{2}{c}{\textbf{Security Maturity Assessment}} \\
            \hline
            Current Maturity Level & #3/5 \\
            Target Maturity Level & #4/5 \\
            \hline
            \textbf{Gap} & \textcolor{orange}{#5} \\
        \end{tabular}
    };
}

% Budget allocation visualization
% Usage: \drawBudgetAllocation{x}{y}{total}{people}{process}{technology}{other}
\newcommand{\drawBudgetAllocation}[7]{
    \pgfmathsetmacro{\peoplepct}{(#4/#3)*100}
    \pgfmathsetmacro{\processpct}{(#5/#3)*100}
    \pgfmathsetmacro{\techpct}{(#6/#3)*100}
    \pgfmathsetmacro{\otherpct}{(#7/#3)*100}

    \node[legend box, anchor=north west, minimum width=7.5cm,
          draw=green!60!black, line width=2pt] at (#1,#2) {
        \begin{tabular}{lcc}
            \multicolumn{3}{c}{\textbf{Security Budget Allocation}} \\
            \hline
            \textbf{Category} & \textbf{Amount} & \textbf{\%} \\
            \hline
            People & \$#4 & \pgfmathprintnumber[precision=0]{\peoplepct}\% \\
            Process & \$#5 & \pgfmathprintnumber[precision=0]{\processpct}\% \\
            Technology & \$#6 & \pgfmathprintnumber[precision=0]{\techpct}\% \\
            Other & \$#7 & \pgfmathprintnumber[precision=0]{\otherpct}\% \\
            \hline
            \textbf{Total} & \textbf{\$#3} & \textbf{100\%} \\
        \end{tabular}
    };
}

% Executive action tracker
% Usage: \drawActionTracker{x}{y}{total_actions}{completed}{in_progress}{blocked}
\newcommand{\drawActionTracker}[6]{
    \pgfmathsetmacro{\completionrate}{(#4/#3)*100}

    \node[legend box, anchor=north west, minimum width=6.5cm,
          draw=blue!70, line width=2pt] at (#1,#2) {
        \begin{tabular}{lc}
            \multicolumn{2}{c}{\textbf{Executive Action Tracker}} \\
            \hline
            Total Actions & #3 \\
            Completed & \textcolor{green!60!black}{#4} \\
            In Progress & \textcolor{orange}{#5} \\
            Blocked & \textcolor{red!70}{#6} \\
            \hline
            \textbf{Completion Rate} & \textbf{\pgfmathprintnumber[precision=0]{\completionrate}\%} \\
        \end{tabular}
    };
}

% ============================================================================
% MAIN THREAT RENDERING ENGINE
% ============================================================================

\newcommand{\renderThreats}{
    % This will be populated by network_data.tex
    % Example structure:
    % \visualizeDDoS{attacker1,attacker2}{srv1}{critical}
    % \markVulnerability{srv2}{CVE-2024-1234}{9.8}
}

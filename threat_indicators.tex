% threat_indicators.tex - Security threat visualization and indicators
% This module handles threat detection, attack visualization, and security status

% ============================================================================
% THREAT LEVEL DEFINITIONS
% ============================================================================

\newcommand{\threatCriticalLevel}{5}
\newcommand{\threatHighLevel}{4}
\newcommand{\threatMediumLevel}{3}
\newcommand{\threatLowLevel}{2}
\newcommand{\threatInfoLevel}{1}

% TODO: Threat scoring system
% - CVSS score integration
% - Custom threat scoring algorithms
% - Risk = Likelihood × Impact calculations
% - Temporal scoring (degrading over time)
% - Environmental scoring based on context

% ============================================================================
% THREAT INDICATOR VISUALIZATION
% ============================================================================

% Draw threat indicator icon
% Usage: \drawThreatIndicator{x}{y}{level}{type}
\newcommand{\drawThreatIndicator}[4]{
    \ifthenelse{\equal{#3}{critical}}{
        \def\threatcolor{threatCritical}
        \def\threatsize{0.5}
    }{
    \ifthenelse{\equal{#3}{high}}{
        \def\threatcolor{threatHigh}
        \def\threatsize{0.4}
    }{
    \ifthenelse{\equal{#3}{medium}}{
        \def\threatcolor{threatMedium}
        \def\threatsize{0.35}
    }{
    \ifthenelse{\equal{#3}{low}}{
        \def\threatcolor{threatLow}
        \def\threatsize}{0.3}
    }{
        \def\threatcolor{threatInfo}
        \def\threatsize{0.25}
    }}}}
    
    \node[
        regular polygon,
        regular polygon sides=3,
        fill=\threatcolor,
        draw=\threatcolor!80,
        minimum size=\threatsize cm,
        inner sep=0pt
    ] at (#1,#2) {};
    
    \node[font=\tiny\bfseries\sffamily, text=white] at (#1,#2) {!};
    
    \node[below, font=\tiny\sffamily] at (#1,#2-0.15) {#4};
}

% TODO: Enhanced threat indicators
% - Animated pulsing for active threats
% - Different icon shapes for different threat types
% - Severity gradient visualization
% - Threat trend arrows (increasing/decreasing)
% - Time-to-remediation countdown

% ============================================================================
% ATTACK VISUALIZATION
% ============================================================================

% Visualize specific attack types
% Usage: \visualizeAttack{attacker}{target}{attack_type}{severity}

% DDoS Attack visualization
\newcommand{\visualizeDDoS}[3]{
    % #1 = attacker list (comma-separated)
    % #2 = target
    % #3 = severity
    \foreach \attacker in {#1} {
        \draw[attack conn, line width=2pt] (\attacker) -- (#2);
    }
    \node[draw=threatCritical, line width=3pt, circle, 
          minimum size=1.5cm, fill=threatCritical!20] at (#2) {};
    \node[above, font=\small\bfseries, text=threatCritical] 
        at (#2.north) [yshift=0.8cm] {DDoS ATTACK};
}

% SQL Injection visualization
\newcommand{\visualizeSQLi}[2]{
    % #1 = attacker
    % #2 = database server
    \draw[attack conn, line width=2pt] (#1) -- (#2)
        node[midway, threat label] {SQL Injection};
    \node[draw=threatHigh, star, star points=8, 
          minimum size=0.8cm, fill=threatHigh!30] at (#2.north east) {};
}

% Malware/Ransomware visualization
\newcommand{\visualizeMalware}[2]{
    % #1 = infected node
    % #2 = malware type
    \node[draw=threatCritical, line width=3pt, 
          rounded corners=3pt, inner sep=8pt,
          fill=threatCritical!15, dashed] at (#1) {};
    \node[above, font=\tiny\bfseries, text=threatCritical, 
          fill=white, inner sep=2pt] at (#1.north) {#2};
}

% Data exfiltration visualization
\newcommand{\visualizeExfiltration}[3]{
    % #1 = source (compromised node)
    % #2 = destination (attacker)
    % #3 = data amount
    \draw[attack conn, line width=3pt] (#1) -- (#2)
        node[midway, above, threat label] {Exfil: #3};
    \draw[draw=threatCritical, line width=2pt, dashed, 
          -{Stealth[length=5mm]}] (#1) -- (#2);
}

% TODO: Additional attack visualizations
% - Phishing attack chain
% - Privilege escalation path
% - Lateral movement tracking
% - Command & Control beaconing
% - Brute force attempts (with attempt counter)
% - Zero-day exploitation indicators
% - Supply chain attack visualization

% ============================================================================
% THREAT TIMELINE AND PROGRESSION
% ============================================================================

% Show attack kill chain progression
% Usage: \drawKillChain{stages}{current_stage}
\newcommand{\drawKillChain}[2]{
    % Reconnaissance, Weaponization, Delivery, Exploitation, 
    % Installation, Command & Control, Actions on Objectives
    % Draw progression with current stage highlighted
}

% TODO: Attack progression visualization
% - MITRE ATT&CK framework mapping
% - Kill chain stage indicators
% - Time-based attack timeline
% - Infection spread animation
% - Attack vector tree visualization

% ============================================================================
% SECURITY ZONES AND BOUNDARIES
% ============================================================================

% Mark security boundary breach
% Usage: \markBoundaryBreach{zone1}{zone2}{breach_point}
\newcommand{\markBoundaryBreach}[3]{
    \draw[draw=threatCritical, line width=3pt, 
          decoration={zigzag, segment length=4pt, amplitude=2pt},
          decorate] (#1) -- (#2);
    \node[circle, fill=threatCritical, minimum size=0.4cm] at (#3) {};
    \node[above, font=\tiny\bfseries, text=white] at (#3) {BREACH};
}

% Draw firewall bypass indicator
\newcommand{\showFirewallBypass}[2]{
    % #1 = firewall node
    % #2 = bypass method
    \node[draw=threatHigh, cross out, line width=2pt, 
          minimum size=1cm, inner sep=0pt] at (#1) {};
    \node[below, font=\tiny, text=threatHigh] at (#1.south) {Bypassed: #2};
}

% TODO: Security control visualization
% - IDS/IPS alert indicators
% - Failed authentication attempts
% - Access control violations
% - Encryption status (enabled/disabled/weak)
% - Patch status indicators

% ============================================================================
% VULNERABILITY INDICATORS
% ============================================================================

% Mark vulnerable node with CVE
% Usage: \markVulnerability{node}{cve}{cvss_score}
\newcommand{\markVulnerability}[3]{
    \pgfmathsetmacro{\severity}{#3/10}
    \ifthenelse{\lengthtest{\severity pt > 0.7 pt}}{
        \def\vulncolor{threatCritical}
    }{
    \ifthenelse{\lengthtest{\severity pt > 0.4 pt}}{
        \def\vulncolor{threatMedium}
    }{
        \def\vulncolor{threatLow}
    }}
    
    \node[draw=\vulncolor, line width=2pt, rectangle,
          rounded corners=2pt, inner sep=3pt, fill=\vulncolor!20,
          anchor=south west] at (#1.south west) {
        \tiny\ttfamily #2: #3
    };
}

% Show exploitable service
% Usage: \markExploitableService{node}{service}{port}
\newcommand{\markExploitableService}[3]{
    \node[draw=threatHigh, fill=threatHigh!20, 
          font=\tiny\ttfamily, anchor=north east] 
        at (#1.north east) {#2:#3};
}

% TODO: Vulnerability management
% - CVE database integration
% - CVSS scoring visualization
% - Exploit availability indicators
% - Patch availability status
% - Vulnerability age indicators
% - Risk prioritization markers

% ============================================================================
% THREAT ACTOR VISUALIZATION
% ============================================================================

% Mark threat actor with attribution
% Usage: \markThreatActor{node}{actor_name}{confidence}
\newcommand{\markThreatActor}[3]{
    \node[above, font=\small\bfseries, text=threatCritical,
          fill=white, draw=threatCritical, rounded corners=2pt,
          inner sep=3pt] at (#1.north) [yshift=0.5cm] {
        #2 (Confidence: #3\%)
    };
}

% Show threat intelligence indicators
% Usage: \markIOC{node}{ioc_type}{value}
\newcommand{\markIOC}[3]{
    % IOC = Indicator of Compromise
    \node[font=\tiny\ttfamily, fill=yellow!30, 
          draw=orange, inner sep=2pt, anchor=south] 
        at (#1.south) [yshift=-0.3cm] {
        #2: #3
    };
}

% TODO: Threat intelligence integration
% - IOC (Indicators of Compromise) visualization
% - TTP (Tactics, Techniques, Procedures) mapping
% - Threat actor attribution with confidence levels
% - Threat feed integration
% - Reputation scoring for IPs/domains

% ============================================================================
% SECURITY POSTURE OVERVIEW
% ============================================================================

% Draw security posture dashboard
% Usage: \drawSecurityDashboard{x}{y}
\newcommand{\drawSecurityDashboard}[2]{
    \node[legend box, anchor=north east, minimum width=4cm] at (#1,#2) {
        \begin{tabular}{ll}
            \multicolumn{2}{c}{\textbf{Security Posture}} \\
            \hline
            \textcolor{threatCritical}{● Critical} & 0 \\
            \textcolor{threatHigh}{● High} & 0 \\
            \textcolor{threatMedium}{● Medium} & 0 \\
            \textcolor{threatLow}{● Low} & 0 \\
            \hline
            \textbf{Risk Score} & \textbf{0/100} \\
        \end{tabular}
    };
}

% Show compliance status
% Usage: \drawComplianceStatus{x}{y}{framework}
\newcommand{\drawComplianceStatus}[3]{
    \node[legend box, anchor=north east] at (#1,#2) {
        \small\bfseries #3 Compliance \\
        \tiny Status: Checking...
    };
}

% TODO: Security dashboards
% - Overall network security score
% - Compliance framework status (NIST, CIS, PCI-DSS, etc.)
% - Security control effectiveness metrics
% - Mean time to detect (MTTD)
% - Mean time to respond (MTTR)
% - Security coverage heatmap

% ============================================================================
% MAIN THREAT RENDERING ENGINE
% ============================================================================

\newcommand{\renderThreats}{
    % This will be populated by network_data.tex
    % Example structure:
    % \visualizeDDoS{attacker1,attacker2}{srv1}{critical}
    % \markVulnerability{srv2}{CVE-2024-1234}{9.8}
}

% TODO: Intelligent threat rendering
% - Threat prioritization based on risk
% - Auto-layout threat indicators to avoid overlap
% - Threat correlation and grouping
% - Attack path visualization
% - Threat timeline with incident markers
% - Real-time threat feed updates

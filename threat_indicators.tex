% threat_indicators.tex - Security threat visualization and indicators
% This module handles threat detection, attack visualization, and security status

% ============================================================================
% THREAT LEVEL DEFINITIONS
% ============================================================================

\newcommand{\threatCriticalLevel}{5}
\newcommand{\threatHighLevel}{4}
\newcommand{\threatMediumLevel}{3}
\newcommand{\threatLowLevel}{2}
\newcommand{\threatInfoLevel}{1}

% TODO: Threat scoring system
% - CVSS score integration
% - Custom threat scoring algorithms
% - Risk = Likelihood × Impact calculations
% - Temporal scoring (degrading over time)
% - Environmental scoring based on context

% ============================================================================
% THREAT INDICATOR VISUALIZATION
% ============================================================================

% Draw threat indicator icon
% Usage: \drawThreatIndicator{x}{y}{level}{type}
\newcommand{\drawThreatIndicator}[4]{
    \ifthenelse{\equal{#3}{critical}}{
        \def\threatcolor{threatCritical}
        \def\threatsize{0.5}
    }{
    \ifthenelse{\equal{#3}{high}}{
        \def\threatcolor{threatHigh}
        \def\threatsize{0.4}
    }{
    \ifthenelse{\equal{#3}{medium}}{
        \def\threatcolor{threatMedium}
        \def\threatsize{0.35}
    }{
    \ifthenelse{\equal{#3}{low}}{
        \def\threatcolor{threatLow}
        \def\threatsize}{0.3}
    }{
        \def\threatcolor{threatInfo}
        \def\threatsize{0.25}
    }}}}
    
    \node[
        regular polygon,
        regular polygon sides=3,
        fill=\threatcolor,
        draw=\threatcolor!80,
        minimum size=\threatsize cm,
        inner sep=0pt
    ] at (#1,#2) {};
    
    \node[font=\tiny\bfseries\sffamily, text=white] at (#1,#2) {!};
    
    \node[below, font=\tiny\sffamily] at (#1,#2-0.15) {#4};
}

% TODO: Enhanced threat indicators
% - Animated pulsing for active threats
% - Different icon shapes for different threat types
% - Severity gradient visualization
% - Threat trend arrows (increasing/decreasing)
% - Time-to-remediation countdown

% ============================================================================
% ATTACK VISUALIZATION
% ============================================================================

% Visualize specific attack types
% Usage: \visualizeAttack{attacker}{target}{attack_type}{severity}

% DDoS Attack visualization
\newcommand{\visualizeDDoS}[3]{
    % #1 = attacker list (comma-separated)
    % #2 = target
    % #3 = severity
    \foreach \attacker in {#1} {
        \draw[attack conn, line width=2pt] (\attacker) -- (#2);
    }
    \node[draw=threatCritical, line width=3pt, circle, 
          minimum size=1.5cm, fill=threatCritical!20] at (#2) {};
    \node[above, font=\small\bfseries, text=threatCritical] 
        at (#2.north) [yshift=0.8cm] {DDoS ATTACK};
}

% SQL Injection visualization
\newcommand{\visualizeSQLi}[2]{
    % #1 = attacker
    % #2 = database server
    \draw[attack conn, line width=2pt] (#1) -- (#2)
        node[midway, threat label] {SQL Injection};
    \node[draw=threatHigh, star, star points=8, 
          minimum size=0.8cm, fill=threatHigh!30] at (#2.north east) {};
}

% Malware/Ransomware visualization
\newcommand{\visualizeMalware}[2]{
    % #1 = infected node
    % #2 = malware type
    \node[draw=threatCritical, line width=3pt, 
          rounded corners=3pt, inner sep=8pt,
          fill=threatCritical!15, dashed] at (#1) {};
    \node[above, font=\tiny\bfseries, text=threatCritical, 
          fill=white, inner sep=2pt] at (#1.north) {#2};
}

% Data exfiltration visualization
\newcommand{\visualizeExfiltration}[3]{
    % #1 = source (compromised node)
    % #2 = destination (attacker)
    % #3 = data amount
    \draw[attack conn, line width=3pt] (#1) -- (#2)
        node[midway, above, threat label] {Exfil: #3};
    \draw[draw=threatCritical, line width=2pt, dashed, 
          -{Stealth[length=5mm]}] (#1) -- (#2);
}

% TODO: Additional attack visualizations
% - Phishing attack chain
% - Privilege escalation path
% - Lateral movement tracking
% - Command & Control beaconing
% - Brute force attempts (with attempt counter)
% - Zero-day exploitation indicators
% - Supply chain attack visualization

% ============================================================================
% ATTACK KILL CHAIN PROGRESSION (LOCKHEED MARTIN)
% ============================================================================

% Show attack kill chain progression with 7 stages
% Usage: \drawKillChain{x}{y}{current_stage}{defenses}
% Stages: 1=Reconnaissance, 2=Weaponization, 3=Delivery, 4=Exploitation,
%         5=Installation, 6=Command & Control, 7=Actions on Objectives
% Defenses: comma-separated list of stage numbers with defensive controls
\newcommand{\drawKillChain}[4]{
    \pgfmathsetmacro{\boxwidth}{3.5}
    \pgfmathsetmacro{\boxheight}{0.6}
    \pgfmathsetmacro{\xpos}{#1}
    \pgfmathsetmacro{\ypos}{#2}

    \node[anchor=north west, font=\small\bfseries] at (\xpos, \ypos) {Cyber Kill Chain};
    \pgfmathsetmacro{\ypos}{\ypos - 0.5}

    % Stage 1: Reconnaissance
    \pgfmathsetmacro{\currentstage}{#3}
    \ifthenelse{\lengthtest{\currentstage pt > 0.9 pt}}{
        \def\stagecolor{threatCritical}
        \def\stagelabel{ACTIVE}
    }{\def\stagecolor{gray!30}\def\stagelabel{}}

    \node[draw=\stagecolor, fill=\stagecolor!20, minimum width=\boxwidth cm,
          minimum height=\boxheight cm, anchor=north west] at (\xpos, \ypos) {
        \small\textbf{1. Reconnaissance} \tiny\stagelabel
    };
    \pgfmathsetmacro{\ypos}{\ypos - \boxheight - 0.05}

    % Stage 2: Weaponization
    \ifthenelse{\lengthtest{\currentstage pt > 1.9 pt}}{
        \def\stagecolor{threatCritical}
        \def\stagelabel{ACTIVE}
    }{\def\stagecolor{gray!30}\def\stagelabel{}}

    \node[draw=\stagecolor, fill=\stagecolor!20, minimum width=\boxwidth cm,
          minimum height=\boxheight cm, anchor=north west] at (\xpos, \ypos) {
        \small\textbf{2. Weaponization} \tiny\stagelabel
    };
    \pgfmathsetmacro{\ypos}{\ypos - \boxheight - 0.05}

    % Stage 3: Delivery
    \ifthenelse{\lengthtest{\currentstage pt > 2.9 pt}}{
        \def\stagecolor{threatCritical}
        \def\stagelabel{ACTIVE}
    }{\def\stagecolor{gray!30}\def\stagelabel{}}

    \node[draw=\stagecolor, fill=\stagecolor!20, minimum width=\boxwidth cm,
          minimum height=\boxheight cm, anchor=north west] at (\xpos, \ypos) {
        \small\textbf{3. Delivery} \tiny\stagelabel
    };
    \pgfmathsetmacro{\ypos}{\ypos - \boxheight - 0.05}

    % Stage 4: Exploitation
    \ifthenelse{\lengthtest{\currentstage pt > 3.9 pt}}{
        \def\stagecolor{threatCritical}
        \def\stagelabel{ACTIVE}
    }{\def\stagecolor{gray!30}\def\stagelabel{}}

    \node[draw=\stagecolor, fill=\stagecolor!20, minimum width=\boxwidth cm,
          minimum height=\boxheight cm, anchor=north west] at (\xpos, \ypos) {
        \small\textbf{4. Exploitation} \tiny\stagelabel
    };
    \pgfmathsetmacro{\ypos}{\ypos - \boxheight - 0.05}

    % Stage 5: Installation
    \ifthenelse{\lengthtest{\currentstage pt > 4.9 pt}}{
        \def\stagecolor{threatCritical}
        \def\stagelabel{ACTIVE}
    }{\def\stagecolor{gray!30}\def\stagelabel{}}

    \node[draw=\stagecolor, fill=\stagecolor!20, minimum width=\boxwidth cm,
          minimum height=\boxheight cm, anchor=north west] at (\xpos, \ypos) {
        \small\textbf{5. Installation} \tiny\stagelabel
    };
    \pgfmathsetmacro{\ypos}{\ypos - \boxheight - 0.05}

    % Stage 6: Command & Control
    \ifthenelse{\lengthtest{\currentstage pt > 5.9 pt}}{
        \def\stagecolor{threatCritical}
        \def\stagelabel{ACTIVE}
    }{\def\stagecolor{gray!30}\def\stagelabel{}}

    \node[draw=\stagecolor, fill=\stagecolor!20, minimum width=\boxwidth cm,
          minimum height=\boxheight cm, anchor=north west] at (\xpos, \ypos) {
        \small\textbf{6. Command \& Control} \tiny\stagelabel
    };
    \pgfmathsetmacro{\ypos}{\ypos - \boxheight - 0.05}

    % Stage 7: Actions on Objectives
    \ifthenelse{\lengthtest{\currentstage pt > 6.9 pt}}{
        \def\stagecolor{threatCritical}
        \def\stagelabel{ACTIVE}
    }{\def\stagecolor{gray!30}\def\stagelabel{}}

    \node[draw=\stagecolor, fill=\stagecolor!20, minimum width=\boxwidth cm,
          minimum height=\boxheight cm, anchor=north west] at (\xpos, \ypos) {
        \small\textbf{7. Actions on Objectives} \tiny\stagelabel
    };
}

% Show defensive gaps in kill chain
% Usage: \drawDefensiveGaps{x}{y}{protected_stages}{vulnerable_stages}
\newcommand{\drawDefensiveGaps}[4]{
    \node[legend box, anchor=north west, minimum width=4cm,
          draw=threatMedium, line width=1.5pt] at (#1,#2) {
        \begin{tabular}{p{3.5cm}}
            \multicolumn{1}{c}{\textbf{Defensive Posture}} \\
            \hline
            \textcolor{green!60!black}{\textbf{Protected Stages:}} \\
            #3 \\
            \hline
            \textcolor{threatHigh}{\textbf{Vulnerable Stages:}} \\
            #4 \\
        \end{tabular}
    };
}

% NIST Cybersecurity Framework mapping
% Usage: \drawNISTMapping{x}{y}{identify}{protect}{detect}{respond}{recover}
\newcommand{\drawNISTMapping}[7]{
    \node[legend box, anchor=north west, minimum width=5cm,
          draw=blue!70, line width=2pt] at (#1,#2) {
        \begin{tabular}{lc}
            \multicolumn{2}{c}{\textbf{NIST CSF Coverage}} \\
            \hline
            \textbf{Identify} & #3\% \\
            \textbf{Protect} & #4\% \\
            \textbf{Detect} & #5\% \\
            \textbf{Respond} & #6\% \\
            \textbf{Recover} & #7\% \\
        \end{tabular}
    };
}

% Attack timeline visualization
% Usage: \drawAttackTimeline{x}{y}{start_time}{current_time}{events}
\newcommand{\drawAttackTimeline}[5]{
    \node[legend box, anchor=north west, minimum width=6cm,
          draw=threatHigh, line width=1.5pt] at (#1,#2) {
        \begin{tabular}{ll}
            \multicolumn{2}{c}{\textbf{Attack Timeline}} \\
            \hline
            \textbf{Attack Started:} & #3 \\
            \textbf{Current Time:} & #4 \\
            \hline
            \multicolumn{2}{l}{\textbf{Key Events:}} \\
            \multicolumn{2}{p{5.5cm}}{#5} \\
        \end{tabular}
    };
}

% Attack path visualization
% Usage: \drawAttackPath{source}{intermediate}{target}{stage}
\newcommand{\drawAttackPath}[4]{
    \draw[attack conn, line width=3pt, -{Stealth[length=5mm]}]
        (#1) -- (#2) node[midway, above, font=\tiny, fill=white, inner sep=1pt] {Stage #4};
    \draw[attack conn, line width=3pt, -{Stealth[length=5mm]}]
        (#2) -- (#3) node[midway, above, font=\tiny, fill=white, inner sep=1pt] {Stage #4};
}

% ============================================================================
% SECURITY ZONES AND BOUNDARIES
% ============================================================================

% Mark security boundary breach
% Usage: \markBoundaryBreach{zone1}{zone2}{breach_point}
\newcommand{\markBoundaryBreach}[3]{
    \draw[draw=threatCritical, line width=3pt, 
          decoration={zigzag, segment length=4pt, amplitude=2pt},
          decorate] (#1) -- (#2);
    \node[circle, fill=threatCritical, minimum size=0.4cm] at (#3) {};
    \node[above, font=\tiny\bfseries, text=white] at (#3) {BREACH};
}

% Draw firewall bypass indicator
\newcommand{\showFirewallBypass}[2]{
    % #1 = firewall node
    % #2 = bypass method
    \node[draw=threatHigh, cross out, line width=2pt, 
          minimum size=1cm, inner sep=0pt] at (#1) {};
    \node[below, font=\tiny, text=threatHigh] at (#1.south) {Bypassed: #2};
}

% TODO: Security control visualization
% - IDS/IPS alert indicators
% - Failed authentication attempts
% - Access control violations
% - Encryption status (enabled/disabled/weak)
% - Patch status indicators

% ============================================================================
% VULNERABILITY INDICATORS
% ============================================================================

% Mark vulnerable node with CVE
% Usage: \markVulnerability{node}{cve}{cvss_score}
\newcommand{\markVulnerability}[3]{
    \pgfmathsetmacro{\severity}{#3/10}
    \ifthenelse{\lengthtest{\severity pt > 0.7 pt}}{
        \def\vulncolor{threatCritical}
    }{
    \ifthenelse{\lengthtest{\severity pt > 0.4 pt}}{
        \def\vulncolor{threatMedium}
    }{
        \def\vulncolor{threatLow}
    }}
    
    \node[draw=\vulncolor, line width=2pt, rectangle,
          rounded corners=2pt, inner sep=3pt, fill=\vulncolor!20,
          anchor=south west] at (#1.south west) {
        \tiny\ttfamily #2: #3
    };
}

% Show exploitable service
% Usage: \markExploitableService{node}{service}{port}
\newcommand{\markExploitableService}[3]{
    \node[draw=threatHigh, fill=threatHigh!20, 
          font=\tiny\ttfamily, anchor=north east] 
        at (#1.north east) {#2:#3};
}

% ============================================================================
% ENHANCED CVSS SCORE VISUALIZATION
% ============================================================================

% Parse CVSS vector and display comprehensive score breakdown
% Usage: \drawCVSSScore{x}{y}{cve}{base_score}{temporal_score}{environmental_score}{vector}
\newcommand{\drawCVSSScore}[7]{
    % Determine severity level based on base score
    \pgfmathsetmacro{\basescore}{#4}
    \ifthenelse{\lengthtest{\basescore pt > 9.0 pt}}{
        \def\cvsscolor{threatCritical}
        \def\severitylabel{CRITICAL}
    }{
    \ifthenelse{\lengthtest{\basescore pt > 7.0 pt}}{
        \def\cvsscolor{threatHigh}
        \def\severitylabel{HIGH}
    }{
    \ifthenelse{\lengthtest{\basescore pt > 4.0 pt}}{
        \def\cvsscolor{threatMedium}
        \def\severitylabel{MEDIUM}
    }{
    \ifthenelse{\lengthtest{\basescore pt > 0.1 pt}}{
        \def\cvsscolor{threatLow}
        \def\severitylabel{LOW}
    }{
        \def\cvsscolor{threatInfo}
        \def\severitylabel{INFO}
    }}}}

    % Draw CVSS score box
    \node[legend box, anchor=north west, minimum width=3.5cm,
          draw=\cvsscolor, line width=2pt] at (#1,#2) {
        \begin{tabular}{ll}
            \multicolumn{2}{c}{\textbf{\textcolor{\cvsscolor}{#3}}} \\
            \hline
            \textbf{Severity:} & \textcolor{\cvsscolor}{\severitylabel} \\
            \textbf{Base:} & #4 \\
            \textbf{Temporal:} & #5 \\
            \textbf{Environmental:} & #6 \\
            \hline
            \multicolumn{2}{l}{\tiny\ttfamily #7} \\
        \end{tabular}
    };
}

% Compact CVSS badge for node overlay
% Usage: \cvssbadge{node}{cve}{score}
\newcommand{\cvssbadge}[3]{
    \pgfmathsetmacro{\score}{#3}
    \ifthenelse{\lengthtest{\score pt > 9.0 pt}}{
        \def\badgecolor{threatCritical}
    }{
    \ifthenelse{\lengthtest{\score pt > 7.0 pt}}{
        \def\badgecolor{threatHigh}
    }{
    \ifthenelse{\lengthtest{\score pt > 4.0 pt}}{
        \def\badgecolor{threatMedium}
    }{
        \def\badgecolor{threatLow}
    }}}

    \node[draw=\badgecolor, fill=\badgecolor, text=white,
          font=\tiny\bfseries, rounded corners=2pt, inner sep=2pt,
          anchor=north west] at (#1.north west) {#3};
    \node[draw=\badgecolor, fill=white, text=\badgecolor,
          font=\tiny\ttfamily, rounded corners=2pt, inner sep=2pt,
          anchor=south west] at (#1.south west) {#2};
}

% CVSS metrics breakdown visualization
% Usage: \drawCVSSMetrics{x}{y}{AV}{AC}{PR}{UI}{S}{C}{I}{A}
% AV=Attack Vector, AC=Attack Complexity, PR=Privileges Required,
% UI=User Interaction, S=Scope, C=Confidentiality, I=Integrity, A=Availability
\newcommand{\drawCVSSMetrics}[11]{
    \node[legend box, anchor=north west, minimum width=4cm] at (#1,#2) {
        \begin{tabular}{ll}
            \multicolumn{2}{c}{\textbf{CVSS v3.1 Metrics}} \\
            \hline
            \textbf{Attack Vector:} & #3 \\
            \textbf{Attack Complexity:} & #4 \\
            \textbf{Privileges Required:} & #5 \\
            \textbf{User Interaction:} & #6 \\
            \textbf{Scope:} & #7 \\
            \hline
            \textbf{Confidentiality:} & \textcolor{threatHigh}{#8} \\
            \textbf{Integrity:} & \textcolor{threatMedium}{#9} \\
            \textbf{Availability:} & \textcolor{threatMedium}{#{10}} \\
        \end{tabular}
    };
}

% ============================================================================
% MITRE ATT&CK FRAMEWORK MAPPING
% ============================================================================

% Display MITRE ATT&CK technique
% Usage: \drawMITREAttack{x}{y}{technique_id}{technique_name}{tactic}
\newcommand{\drawMITREAttack}[5]{
    \node[legend box, anchor=north west, minimum width=4.5cm,
          draw=threatHigh, line width=1.5pt] at (#1,#2) {
        \begin{tabular}{ll}
            \multicolumn{2}{c}{\textbf{\textcolor{threatHigh}{MITRE ATT\&CK}}} \\
            \hline
            \textbf{Technique:} & \texttt{#3} \\
            \textbf{Name:} & #4 \\
            \textbf{Tactic:} & \textcolor{threatMedium}{#5} \\
            \hline
            \multicolumn{2}{l}{\tiny attack.mitre.org/techniques/#3} \\
        \end{tabular}
    };
}

% MITRE ATT&CK badge for nodes
% Usage: \mitrebadge{node}{technique_id}
\newcommand{\mitrebadge}[2]{
    \node[draw=threatHigh, fill=threatHigh, text=white,
          font=\tiny\bfseries\ttfamily, rounded corners=1pt, inner sep=1pt,
          anchor=north east] at (#1.north east) [xshift=-0.1cm, yshift=-0.1cm] {#2};
}

% Draw MITRE ATT&CK kill chain progression
% Usage: \drawMITREKillChain{x}{y}{current_stage}
% Stages: Reconnaissance, Resource Development, Initial Access, Execution,
%         Persistence, Privilege Escalation, Defense Evasion, Credential Access,
%         Discovery, Lateral Movement, Collection, Command & Control, Exfiltration,
%         Impact
\newcommand{\drawMITREKillChain}[3]{
    \pgfmathsetmacro{\boxwidth}{1.8}
    \pgfmathsetmacro{\boxheight}{0.6}
    \pgfmathsetmacro{\ypos}{#2}

    \node[anchor=north west, font=\small\bfseries] at (#1, \ypos) {MITRE ATT\&CK Kill Chain};
    \pgfmathsetmacro{\ypos}{\ypos - 0.4}

    % Stage 1: Reconnaissance
    \ifthenelse{\equal{#3}{1} \OR \equal{#3}{Reconnaissance}}{
        \def\stagecolor{threatCritical}
    }{\def\stagecolor{gray!30}}
    \node[draw=\stagecolor, fill=\stagecolor!20, minimum width=\boxwidth cm,
          minimum height=\boxheight cm, font=\tiny, anchor=north west]
          at (#1, \ypos) {Reconnaissance};

    \pgfmathsetmacro{\ypos}{\ypos - \boxheight - 0.1}

    % Stage 2: Initial Access
    \ifthenelse{\equal{#3}{2} \OR \equal{#3}{Initial Access}}{
        \def\stagecolor{threatCritical}
    }{\def\stagecolor{gray!30}}
    \node[draw=\stagecolor, fill=\stagecolor!20, minimum width=\boxwidth cm,
          minimum height=\boxheight cm, font=\tiny, anchor=north west]
          at (#1, \ypos) {Initial Access};

    \pgfmathsetmacro{\ypos}{\ypos - \boxheight - 0.1}

    % Stage 3: Execution
    \ifthenelse{\equal{#3}{3} \OR \equal{#3}{Execution}}{
        \def\stagecolor{threatCritical}
    }{\def\stagecolor{gray!30}}
    \node[draw=\stagecolor, fill=\stagecolor!20, minimum width=\boxwidth cm,
          minimum height=\boxheight cm, font=\tiny, anchor=north west]
          at (#1, \ypos) {Execution};

    \pgfmathsetmacro{\ypos}{\ypos - \boxheight - 0.1}

    % Stage 4: Persistence
    \ifthenelse{\equal{#3}{4} \OR \equal{#3}{Persistence}}{
        \def\stagecolor{threatCritical}
    }{\def\stagecolor{gray!30}}
    \node[draw=\stagecolor, fill=\stagecolor!20, minimum width=\boxwidth cm,
          minimum height=\boxheight cm, font=\tiny, anchor=north west]
          at (#1, \ypos) {Persistence};

    \pgfmathsetmacro{\ypos}{\ypos - \boxheight - 0.1}

    % Stage 5: Privilege Escalation
    \ifthenelse{\equal{#3}{5} \OR \equal{#3}{Privilege Escalation}}{
        \def\stagecolor{threatCritical}
    }{\def\stagecolor{gray!30}}
    \node[draw=\stagecolor, fill=\stagecolor!20, minimum width=\boxwidth cm,
          minimum height=\boxheight cm, font=\tiny, anchor=north west]
          at (#1, \ypos) {Privilege Esc.};

    \pgfmathsetmacro{\ypos}{\ypos - \boxheight - 0.1}

    % Stage 6: Defense Evasion
    \ifthenelse{\equal{#3}{6} \OR \equal{#3}{Defense Evasion}}{
        \def\stagecolor{threatCritical}
    }{\def\stagecolor{gray!30}}
    \node[draw=\stagecolor, fill=\stagecolor!20, minimum width=\boxwidth cm,
          minimum height=\boxheight cm, font=\tiny, anchor=north west]
          at (#1, \ypos) {Defense Evasion};

    \pgfmathsetmacro{\ypos}{\ypos - \boxheight - 0.1}

    % Stage 7: Lateral Movement
    \ifthenelse{\equal{#3}{7} \OR \equal{#3}{Lateral Movement}}{
        \def\stagecolor{threatCritical}
    }{\def\stagecolor{gray!30}}
    \node[draw=\stagecolor, fill=\stagecolor!20, minimum width=\boxwidth cm,
          minimum height=\boxheight cm, font=\tiny, anchor=north west]
          at (#1, \ypos) {Lateral Movement};

    \pgfmathsetmacro{\ypos}{\ypos - \boxheight - 0.1}

    % Stage 8: Collection & Exfiltration
    \ifthenelse{\equal{#3}{8} \OR \equal{#3}{Exfiltration}}{
        \def\stagecolor{threatCritical}
    }{\def\stagecolor{gray!30}}
    \node[draw=\stagecolor, fill=\stagecolor!20, minimum width=\boxwidth cm,
          minimum height=\boxheight cm, font=\tiny, anchor=north west]
          at (#1, \ypos) {Exfiltration};

    \pgfmathsetmacro{\ypos}{\ypos - \boxheight - 0.1}

    % Stage 9: Impact
    \ifthenelse{\equal{#3}{9} \OR \equal{#3}{Impact}}{
        \def\stagecolor{threatCritical}
    }{\def\stagecolor{gray!30}}
    \node[draw=\stagecolor, fill=\stagecolor!20, minimum width=\boxwidth cm,
          minimum height=\boxheight cm, font=\tiny, anchor=north west]
          at (#1, \ypos) {Impact};
}

% Map technique to tactic with visual connection
% Usage: \mapTechniqueToTactic{node}{technique_id}{tactic_name}
\newcommand{\mapTechniqueToTactic}[3]{
    \node[above, font=\tiny\bfseries\ttfamily, text=threatHigh,
          fill=white, draw=threatHigh, rounded corners=1pt,
          inner sep=1pt] at (#1.north) [yshift=0.3cm] {
        #2: #3
    };
}

% Show multiple MITRE techniques for complex attack
% Usage: \drawMITRETTP{x}{y}{title}{techniques_list}
\newcommand{\drawMITRETTP}[4]{
    \node[legend box, anchor=north west, minimum width=5cm,
          draw=threatHigh, line width=1.5pt, fill=white] at (#1,#2) {
        \begin{tabular}{p{4.5cm}}
            \multicolumn{1}{c}{\textbf{\textcolor{threatHigh}{#3}}} \\
            \hline
            \textbf{Techniques Used:} \\
            #4 \\
        \end{tabular}
    };
}

% ============================================================================
% THREAT ACTOR VISUALIZATION
% ============================================================================

% Mark threat actor with attribution
% Usage: \markThreatActor{node}{actor_name}{confidence}
\newcommand{\markThreatActor}[3]{
    \node[above, font=\small\bfseries, text=threatCritical,
          fill=white, draw=threatCritical, rounded corners=2pt,
          inner sep=3pt] at (#1.north) [yshift=0.5cm] {
        #2 (Confidence: #3\%)
    };
}

% ============================================================================
% ENHANCED IOC (INDICATORS OF COMPROMISE) VISUALIZATION
% ============================================================================

% Show threat intelligence indicators with type-specific formatting
% Usage: \markIOC{node}{ioc_type}{value}{reputation_score}
\newcommand{\markIOC}[4]{
    % IOC = Indicator of Compromise
    % Determine color based on reputation score (0-100, higher is worse)
    \pgfmathsetmacro{\repscore}{#4}
    \ifthenelse{\lengthtest{\repscore pt > 80 pt}}{
        \def\ioccolor{threatCritical}
    }{
    \ifthenelse{\lengthtest{\repscore pt > 60 pt}}{
        \def\ioccolor{threatHigh}
    }{
    \ifthenelse{\lengthtest{\repscore pt > 40 pt}}{
        \def\ioccolor{threatMedium}
    }{
        \def\ioccolor{threatLow}
    }}}

    \node[font=\tiny\ttfamily, fill=\ioccolor!20,
          draw=\ioccolor, inner sep=2pt, anchor=south,
          line width=1pt]
        at (#1.south) [yshift=-0.3cm] {
        #2: #3 (Rep: #4)
    };
}

% Display malicious IP address indicator
% Usage: \markMaliciousIP{node}{ip_address}{reputation}{threat_feed}
\newcommand{\markMaliciousIP}[4]{
    \pgfmathsetmacro{\repscore}{#3}
    \ifthenelse{\lengthtest{\repscore pt > 80 pt}}{
        \def\ipcolor{threatCritical}
    }{\def\ipcolor{threatHigh}}

    \node[font=\tiny\ttfamily, fill=\ipcolor!30,
          draw=\ipcolor, line width=2pt, inner sep=2pt,
          anchor=south west, rounded corners=1pt]
        at (#1.south west) [yshift=-0.4cm] {
        \textbf{Malicious IP:} #2
    };
    \node[font=\tiny, fill=white, draw=\ipcolor,
          inner sep=1pt, anchor=north west]
        at (#1.south west) [yshift=-0.7cm] {
        Source: #4 | Rep: #3/100
    };
}

% Display malicious domain indicator
% Usage: \markMaliciousDomain{node}{domain}{reputation}{category}
\newcommand{\markMaliciousDomain}[4]{
    \node[font=\tiny\ttfamily, fill=threatHigh!30,
          draw=threatHigh, line width=1.5pt, inner sep=2pt,
          anchor=north, rounded corners=1pt]
        at (#1.north) [yshift=0.4cm] {
        \textbf{Malicious Domain:} #2
    };
    \node[font=\tiny, fill=white, draw=threatHigh,
          inner sep=1pt, anchor=north]
        at (#1.north) [yshift=0.1cm] {
        Category: #4 | Rep: #3/100
    };
}

% Display malware file hash indicator
% Usage: \markMalwareHash{node}{hash_type}{hash_value}{detection_rate}
\newcommand{\markMalwareHash}[4]{
    \node[legend box, anchor=north west, minimum width=5cm,
          draw=threatCritical, line width=2pt] at (#1) {
        \begin{tabular}{ll}
            \multicolumn{2}{c}{\textbf{\textcolor{threatCritical}{Malware Detected}}} \\
            \hline
            \textbf{Hash Type:} & #2 \\
            \textbf{Hash:} & \texttt{\tiny #3} \\
            \textbf{Detection:} & #4 AV engines \\
        \end{tabular}
    };
}

% IOC age/freshness indicator
% Usage: \markIOCFreshness{node}{age_days}{status}
\newcommand{\markIOCFreshness}[3]{
    \pgfmathsetmacro{\agedays}{#2}
    \ifthenelse{\lengthtest{\agedays pt < 7 pt}}{
        \def\freshnesscolor{threatCritical}
        \def\freshnesslabel{FRESH}
    }{
    \ifthenelse{\lengthtest{\agedays pt < 30 pt}}{
        \def\freshnesscolor{threatHigh}
        \def\freshnesslabel{RECENT}
    }{
    \ifthenelse{\lengthtest{\agedays pt < 90 pt}}{
        \def\freshnesscolor{threatMedium}
        \def\freshnesslabel{OLD}
    }{
        \def\freshnesscolor{threatLow}
        \def\freshnesslabel{STALE}
    }}}

    \node[fill=\freshnesscolor, text=white, font=\tiny\bfseries,
          inner sep=1pt, rounded corners=1pt, anchor=south east]
        at (#1.south east) {
        \freshnesslabel\ (#2d)
    };
}

% Comprehensive IOC dashboard
% Usage: \drawIOCDashboard{x}{y}{malicious_ips}{malicious_domains}{malware_hashes}{total_iocs}
\newcommand{\drawIOCDashboard}[6]{
    \node[legend box, anchor=north west, minimum width=4.5cm,
          draw=threatHigh, line width=2pt] at (#1,#2) {
        \begin{tabular}{lr}
            \multicolumn{2}{c}{\textbf{\textcolor{threatHigh}{IOC Summary}}} \\
            \hline
            Malicious IPs & #3 \\
            Malicious Domains & #4 \\
            Malware Hashes & #5 \\
            \hline
            \textbf{Total IOCs} & \textbf{#6} \\
        \end{tabular}
    };
}

% Threat feed integration indicator
% Usage: \markThreatFeed{x}{y}{feed_name}{last_update}{active_threats}
\newcommand{\markThreatFeed}[5]{
    \node[legend box, anchor=north west, minimum width=4cm,
          draw=threatMedium, line width=1.5pt] at (#1,#2) {
        \begin{tabular}{ll}
            \multicolumn{2}{c}{\textbf{Threat Feed: #3}} \\
            \hline
            \textbf{Last Update:} & #4 \\
            \textbf{Active Threats:} & \textcolor{threatHigh}{#5} \\
        \end{tabular}
    };
}

% ============================================================================
% SECURITY POSTURE OVERVIEW
% ============================================================================

% Draw security posture dashboard
% Usage: \drawSecurityDashboard{x}{y}
\newcommand{\drawSecurityDashboard}[2]{
    \node[legend box, anchor=north east, minimum width=4cm] at (#1,#2) {
        \begin{tabular}{ll}
            \multicolumn{2}{c}{\textbf{Security Posture}} \\
            \hline
            \textcolor{threatCritical}{● Critical} & 0 \\
            \textcolor{threatHigh}{● High} & 0 \\
            \textcolor{threatMedium}{● Medium} & 0 \\
            \textcolor{threatLow}{● Low} & 0 \\
            \hline
            \textbf{Risk Score} & \textbf{0/100} \\
        \end{tabular}
    };
}

% Show compliance status
% Usage: \drawComplianceStatus{x}{y}{framework}
\newcommand{\drawComplianceStatus}[3]{
    \node[legend box, anchor=north east] at (#1,#2) {
        \small\bfseries #3 Compliance \\
        \tiny Status: Checking...
    };
}

% TODO: Security dashboards
% - Overall network security score
% - Compliance framework status (NIST, CIS, PCI-DSS, etc.)
% - Security control effectiveness metrics
% - Mean time to detect (MTTD)
% - Mean time to respond (MTTR)
% - Security coverage heatmap

% ============================================================================
% MAIN THREAT RENDERING ENGINE
% ============================================================================

\newcommand{\renderThreats}{
    % This will be populated by network_data.tex
    % Example structure:
    % \visualizeDDoS{attacker1,attacker2}{srv1}{critical}
    % \markVulnerability{srv2}{CVE-2024-1234}{9.8}
}

% TODO: Intelligent threat rendering
% - Threat prioritization based on risk
% - Auto-layout threat indicators to avoid overlap
% - Threat correlation and grouping
% - Attack path visualization
% - Threat timeline with incident markers
% - Real-time threat feed updates

% threat_indicators.tex - Security threat visualization and indicators
% This module handles threat detection, attack visualization, and security status

% ============================================================================
% THREAT LEVEL DEFINITIONS
% ============================================================================

\newcommand{\threatCriticalLevel}{5}
\newcommand{\threatHighLevel}{4}
\newcommand{\threatMediumLevel}{3}
\newcommand{\threatLowLevel}{2}
\newcommand{\threatInfoLevel}{1}

% TODO: Threat scoring system
% - CVSS score integration
% - Custom threat scoring algorithms
% - Risk = Likelihood × Impact calculations
% - Temporal scoring (degrading over time)
% - Environmental scoring based on context

% ============================================================================
% THREAT INDICATOR VISUALIZATION
% ============================================================================

% Draw threat indicator icon
% Usage: \drawThreatIndicator{x}{y}{level}{type}
\newcommand{\drawThreatIndicator}[4]{
    \ifthenelse{\equal{#3}{critical}}{
        \def\threatcolor{threatCritical}
        \def\threatsize{0.5}
    }{
    \ifthenelse{\equal{#3}{high}}{
        \def\threatcolor{threatHigh}
        \def\threatsize{0.4}
    }{
    \ifthenelse{\equal{#3}{medium}}{
        \def\threatcolor{threatMedium}
        \def\threatsize{0.35}
    }{
    \ifthenelse{\equal{#3}{low}}{
        \def\threatcolor{threatLow}
        \def\threatsize}{0.3}
    }{
        \def\threatcolor{threatInfo}
        \def\threatsize{0.25}
    }}}}
    
    \node[
        regular polygon,
        regular polygon sides=3,
        fill=\threatcolor,
        draw=\threatcolor!80,
        minimum size=\threatsize cm,
        inner sep=0pt
    ] at (#1,#2) {};
    
    \node[font=\tiny\bfseries\sffamily, text=white] at (#1,#2) {!};
    
    \node[below, font=\tiny\sffamily] at (#1,#2-0.15) {#4};
}

% TODO: Enhanced threat indicators
% - Animated pulsing for active threats
% - Different icon shapes for different threat types
% - Severity gradient visualization
% - Threat trend arrows (increasing/decreasing)
% - Time-to-remediation countdown

% ============================================================================
% ATTACK VISUALIZATION
% ============================================================================

% Visualize specific attack types
% Usage: \visualizeAttack{attacker}{target}{attack_type}{severity}

% DDoS Attack visualization
\newcommand{\visualizeDDoS}[3]{
    % #1 = attacker list (comma-separated)
    % #2 = target
    % #3 = severity
    \foreach \attacker in {#1} {
        \draw[attack conn, line width=2pt] (\attacker) -- (#2);
    }
    \node[draw=threatCritical, line width=3pt, circle, 
          minimum size=1.5cm, fill=threatCritical!20] at (#2) {};
    \node[above, font=\small\bfseries, text=threatCritical] 
        at (#2.north) [yshift=0.8cm] {DDoS ATTACK};
}

% SQL Injection visualization
\newcommand{\visualizeSQLi}[2]{
    % #1 = attacker
    % #2 = database server
    \draw[attack conn, line width=2pt] (#1) -- (#2)
        node[midway, threat label] {SQL Injection};
    \node[draw=threatHigh, star, star points=8, 
          minimum size=0.8cm, fill=threatHigh!30] at (#2.north east) {};
}

% Malware/Ransomware visualization
\newcommand{\visualizeMalware}[2]{
    % #1 = infected node
    % #2 = malware type
    \node[draw=threatCritical, line width=3pt, 
          rounded corners=3pt, inner sep=8pt,
          fill=threatCritical!15, dashed] at (#1) {};
    \node[above, font=\tiny\bfseries, text=threatCritical, 
          fill=white, inner sep=2pt] at (#1.north) {#2};
}

% Data exfiltration visualization
\newcommand{\visualizeExfiltration}[3]{
    % #1 = source (compromised node)
    % #2 = destination (attacker)
    % #3 = data amount
    \draw[attack conn, line width=3pt] (#1) -- (#2)
        node[midway, above, threat label] {Exfil: #3};
    \draw[draw=threatCritical, line width=2pt, dashed, 
          -{Stealth[length=5mm]}] (#1) -- (#2);
}

% ============================================================================
% ADDITIONAL ATTACK VISUALIZATIONS
% ============================================================================

% Phishing attack visualization
\newcommand{\visualizePhishing}[3]{
    % #1 = attacker
    % #2 = victim
    % #3 = email subject/type
    \draw[attack conn, line width=1.5pt,
          decoration={snake, amplitude=2pt}, decorate]
        (#1) -- (#2)
        node[midway, threat label, fill=threatHigh] {Phishing: #3};
    \node[draw=threatHigh, isosceles triangle, minimum size=0.6cm,
          fill=threatHigh!30, anchor=south] at (#2.north) {};
}

% Privilege escalation visualization
\newcommand{\visualizePrivEsc}[3]{
    % #1 = compromised node
    % #2 = target privilege level
    % #3 = method
    \node[draw=threatHigh, line width=2pt,
          rounded corners=3pt, inner sep=5pt,
          fill=threatHigh!20] at (#1) {};
    \node[above, font=\tiny\bfseries, text=threatHigh,
          fill=yellow!30, inner sep=2pt] at (#1.north) [yshift=0.2cm] {
        PrivEsc: #2
    };
    \node[below, font=\tiny, text=black!70] at (#1.south) [yshift=-0.2cm] {
        #3
    };
}

% Lateral movement tracking
\newcommand{\visualizeLateralMovement}[2]{
    % #1 = path (comma-separated nodes)
    % #2 = technique
    \foreach \node [count=\i, remember=\node as \lastnode] in {#1} {
        \ifnum\i>1
            \draw[attack conn, line width=1.5pt,
                  decoration={markings,
                      mark=at position 0.5 with {\arrow{Stealth[length=2mm]}}},
                  postaction={decorate}]
                (\lastnode) -- (\node);
        \fi
    }
    \node[font=\tiny\bfseries, text=threatCritical,
          fill=white, draw=threatCritical, rounded corners=2pt,
          inner sep=2pt] at ($(current bounding box.north)$) [yshift=0.5cm] {
        Lateral Movement: #2
    };
}

% Command & Control beaconing
\newcommand{\visualizeC2Beacon}[3]{
    % #1 = compromised node
    % #2 = C2 server
    % #3 = interval
    \draw[draw=threatCritical, line width=1pt,
          dash pattern=on 3pt off 3pt,
          {Stealth[length=2mm]}-{Stealth[length=2mm]}]
        (#1) -- (#2);
    \node[font=\tiny\ttfamily, fill=threatCritical!20,
          draw=threatCritical, rounded corners=1pt]
        at ($(#1)!0.5!(#2)$) {C2: #3};
    % Beacon indicator
    \foreach \x in {0.2,0.4,0.6,0.8} {
        \node[circle, fill=threatCritical, inner sep=0.5pt]
            at ($(#1)!\x!(#2)$) {};
    }
}

% Brute force attack
\newcommand{\visualizeBruteForce}[4]{
    % #1 = attacker
    % #2 = target
    % #3 = service
    % #4 = attempt count
    \draw[attack conn, line width=2pt] (#1) -- (#2);
    \node[font=\small\bfseries, text=threatHigh,
          fill=white, draw=threatHigh, rounded corners=2pt,
          inner sep=3pt] at ($(#1)!0.5!(#2)$) [above=5pt] {
        Brute Force: #3
    };
    \node[font=\tiny, fill=threatHigh!20, inner sep=2pt]
        at ($(#1)!0.5!(#2)$) [below=3pt] {
        Attempts: #4
    };
}

% Zero-day exploitation
\newcommand{\visualizeZeroDay}[3]{
    % #1 = target node
    % #2 = vulnerability
    % #3 = impact
    \node[draw=threatCritical, line width=3pt,
          star, star points=8,
          minimum size=1.5cm,
          fill=threatCritical!30] at (#1) {};
    \node[above, font=\small\bfseries, text=white,
          fill=threatCritical, rounded corners=3pt,
          inner sep=3pt] at (#1.north) [yshift=0.8cm] {
        ZERO-DAY
    };
    \node[below, font=\tiny\ttfamily, text=threatCritical]
        at (#1.south) [yshift=-0.5cm] {
        #2 | Impact: #3
    };
}

% Supply chain attack
\newcommand{\visualizeSupplyChain}[3]{
    % #1 = compromised supplier
    % #2 = affected nodes (comma-separated)
    % #3 = vector
    \node[draw=threatCritical, line width=2pt,
          rounded corners=3pt, inner sep=6pt,
          fill=threatCritical!20] at (#1) {};
    \foreach \target in {#2} {
        \draw[attack conn, line width=1.5pt] (#1) -- (\target);
    }
    \node[above, font=\small\bfseries, text=threatCritical,
          fill=yellow!40, inner sep=3pt] at (#1.north) [yshift=0.5cm] {
        Supply Chain Attack: #3
    };
}

% ============================================================================
% THREAT TIMELINE AND PROGRESSION
% ============================================================================

% Show attack kill chain progression
% Usage: \drawKillChain{stages}{current_stage}
\newcommand{\drawKillChain}[2]{
    % Reconnaissance, Weaponization, Delivery, Exploitation, 
    % Installation, Command & Control, Actions on Objectives
    % Draw progression with current stage highlighted
}

% ============================================================================
% MITRE ATT&CK FRAMEWORK VISUALIZATION
% ============================================================================

% Draw MITRE ATT&CK tactic badge
% Usage: \addMITRETactic{nodename}{tactic}{technique_id}
\newcommand{\addMITRETactic}[3]{
    % Tactics: Initial Access, Execution, Persistence, Privilege Escalation,
    % Defense Evasion, Credential Access, Discovery, Lateral Movement,
    % Collection, Command and Control, Exfiltration, Impact
    \node[font=\tiny\ttfamily\bfseries, text=white,
          fill=threatCritical, rounded corners=2pt,
          inner sep=2pt, anchor=north west]
        at (#1.north west) [xshift=-0.2cm, yshift=0.2cm] {
        #2 | #3
    };
}

% Visualize kill chain progression
% Usage: \drawKillChainStage{x}{y}{current_stage}
\newcommand{\drawKillChainStage}[3]{
    \node[font=\small\bfseries, anchor=north west] at (#1,#2) {
        \begin{tabular}{l}
            \textbf{Cyber Kill Chain} \\
            \hline
            \textcolor{#3>=1?threatCritical:black!40}{1. Reconnaissance} \\
            \textcolor{#3>=2?threatCritical:black!40}{2. Weaponization} \\
            \textcolor{#3>=3?threatCritical:black!40}{3. Delivery} \\
            \textcolor{#3>=4?threatHigh:black!40}{4. Exploitation} \\
            \textcolor{#3>=5?threatHigh:black!40}{5. Installation} \\
            \textcolor{#3>=6?threatMedium:black!40}{6. C2} \\
            \textcolor{#3>=7?threatMedium:black!40}{7. Actions} \\
        \end{tabular}
    };
}

% Advanced kill chain with visual progress
% Usage: \visualizeKillChain{x}{y}{stages_completed}
\newcommand{\visualizeKillChain}[3]{
    \def\stages{{Recon, Weapon, Deliver, Exploit, Install, C2, Action}}
    \def\ypos{#2}

    \node[font=\small\bfseries] at (#1, \ypos + 0.5) {Attack Kill Chain};

    \foreach \stage [count=\i from 0] in {Recon, Weapon, Deliver, Exploit, Install, C2, Action} {
        \pgfmathsetmacro{\completed}{ifthenelse(\i<#3,1,0)}
        \pgfmathsetmacro{\yoffset}{\ypos - \i * 0.6}

        \ifnum\completed=1
            \def\stagecolor{threatCritical}
            \def\stagefill{threatCritical!30}
            \def\checkmark{✓}
        \else
            \def\stagecolor{black!30}
            \def\stagefill{white}
            \def\checkmark{}
        \fi

        \node[draw=\stagecolor, fill=\stagefill,
              rounded corners=2pt, minimum width=3cm,
              minimum height=0.5cm, font=\small]
            at (#1, \yoffset) {\i+1. \stage \checkmark};
    }
}

% MITRE ATT&CK technique mapping
% Usage: \mapMITRETechnique{nodename}{technique}{subtechnique}{description}
\newcommand{\mapMITRETechnique}[4]{
    \node[draw=threatHigh, fill=threatHigh!15,
          rounded corners=2pt, font=\tiny\ttfamily,
          inner sep=3pt, anchor=south]
        at (#1.south) [yshift=-0.8cm] {
        \begin{tabular}{l}
            \textbf{#2.#3} \\
            #4
        \end{tabular}
    };
}

% ============================================================================
% SECURITY ZONES AND BOUNDARIES
% ============================================================================

% Mark security boundary breach
% Usage: \markBoundaryBreach{zone1}{zone2}{breach_point}
\newcommand{\markBoundaryBreach}[3]{
    \draw[draw=threatCritical, line width=3pt, 
          decoration={zigzag, segment length=4pt, amplitude=2pt},
          decorate] (#1) -- (#2);
    \node[circle, fill=threatCritical, minimum size=0.4cm] at (#3) {};
    \node[above, font=\tiny\bfseries, text=white] at (#3) {BREACH};
}

% Draw firewall bypass indicator
\newcommand{\showFirewallBypass}[2]{
    % #1 = firewall node
    % #2 = bypass method
    \node[draw=threatHigh, cross out, line width=2pt, 
          minimum size=1cm, inner sep=0pt] at (#1) {};
    \node[below, font=\tiny, text=threatHigh] at (#1.south) {Bypassed: #2};
}

% TODO: Security control visualization
% - IDS/IPS alert indicators
% - Failed authentication attempts
% - Access control violations
% - Encryption status (enabled/disabled/weak)
% - Patch status indicators

% ============================================================================
% VULNERABILITY INDICATORS
% ============================================================================

% Mark vulnerable node with CVE
% Usage: \markVulnerability{node}{cve}{cvss_score}
\newcommand{\markVulnerability}[3]{
    \pgfmathsetmacro{\severity}{#3/10}
    \ifthenelse{\lengthtest{\severity pt > 0.7 pt}}{
        \def\vulncolor{threatCritical}
    }{
    \ifthenelse{\lengthtest{\severity pt > 0.4 pt}}{
        \def\vulncolor{threatMedium}
    }{
        \def\vulncolor{threatLow}
    }}
    
    \node[draw=\vulncolor, line width=2pt, rectangle,
          rounded corners=2pt, inner sep=3pt, fill=\vulncolor!20,
          anchor=south west] at (#1.south west) {
        \tiny\ttfamily #2: #3
    };
}

% Show exploitable service
% Usage: \markExploitableService{node}{service}{port}
\newcommand{\markExploitableService}[3]{
    \node[draw=threatHigh, fill=threatHigh!20, 
          font=\tiny\ttfamily, anchor=north east] 
        at (#1.north east) {#2:#3};
}

% ============================================================================
% ENHANCED VULNERABILITY MANAGEMENT
% ============================================================================

% Advanced CVE visualization with CVSS v3 scoring
% Usage: \markCVEAdvanced{node}{cve}{cvss_base}{cvss_temporal}{exploit_available}
\newcommand{\markCVEAdvanced}[5]{
    \pgfmathsetmacro{\cvsscolor}{ifthenelse(#3>=9,0,ifthenelse(#3>=7,1,ifthenelse(#3>=4,2,3)))}

    \ifnum\cvsscolor=0
        \def\severitycolor{threatCritical}
        \def\severitytext{CRITICAL}
    \else\ifnum\cvsscolor=1
        \def\severitycolor{threatHigh}
        \def\severitytext{HIGH}
    \else\ifnum\cvsscolor=2
        \def\severitycolor{threatMedium}
        \def\severitytext{MEDIUM}
    \else
        \def\severitycolor{threatLow}
        \def\severitytext{LOW}
    \fi\fi\fi

    \node[draw=\severitycolor, fill=\severitycolor!20,
          rounded corners=3pt, inner sep=4pt,
          font=\tiny\ttfamily, anchor=south west,
          minimum width=2.5cm]
        at (#1.south west) [yshift=-1.2cm] {
        \begin{tabular}{l}
            \textbf{#2} \\
            Base: #3 | Temp: #4 \\
            \textbf{\severitytext} \\
            \ifthenelse{\equal{#5}{yes}}{
                \textcolor{threatCritical}{⚠ Exploit Available}
            }{
                \textcolor{clientGreen}{No Public Exploit}
            }
        \end{tabular}
    };
}

% CVSS score visualization bar
% Usage: \drawCVSSBar{x}{y}{score}
\newcommand{\drawCVSSBar}[3]{
    \pgfmathsetmacro{\barwidth}{#3/10*3}
    \pgfmathsetmacro{\barcolor}{ifthenelse(#3>=9,0,ifthenelse(#3>=7,1,ifthenelse(#3>=4,2,3)))}

    \ifnum\barcolor=0
        \def\scorecolor{threatCritical}
    \else\ifnum\barcolor=1
        \def\scorecolor{threatHigh}
    \else\ifnum\barcolor=2
        \def\scorecolor{threatMedium}
    \else
        \def\scorecolor{threatLow}
    \fi\fi\fi

    % Background bar
    \draw[fill=black!10] (#1,#2) rectangle +(3,0.3);
    % Score bar
    \draw[fill=\scorecolor] (#1,#2) rectangle +(\barwidth,0.3);
    % Score text
    \node[font=\tiny\bfseries, anchor=west] at (#1+3.2,#2+0.15) {
        CVSS: #3
    };
}

% Vulnerability aging indicator
% Usage: \markVulnerabilityAge{node}{cve}{days_old}
\newcommand{\markVulnerabilityAge}[3]{
    \pgfmathsetmacro{\agecolor}{ifthenelse(#3>90,0,ifthenelse(#3>30,1,2))}

    \ifnum\agecolor=0
        \def\agealert{threatCritical}
        \def\agestatus{AGED}
    \else\ifnum\agecolor=1
        \def\agealert{threatMedium}
        \def\agestatus{OLD}
    \else
        \def\agealert{threatLow}
        \def\agestatus{NEW}
    \fi\fi

    \node[font=\tiny, fill=\agealert!20, draw=\agealert,
          rounded corners=1pt, inner sep=2pt]
        at (#1.north east) [xshift=0.3cm] {
        #2: #3d (\agestatus)
    };
}

% Patch availability indicator
% Usage: \markPatchStatus{node}{patch_available}{vendor_response}
\newcommand{\markPatchStatus}[3]{
    \ifthenelse{\equal{#2}{yes}}{
        \def\patchcolor{clientGreen}
        \def\patchtext{PATCH AVAILABLE}
    }{
    \ifthenelse{\equal{#2}{partial}}{
        \def\patchcolor{threatMedium}
        \def\patchtext{WORKAROUND ONLY}
    }{
        \def\patchcolor{threatCritical}
        \def\patchtext{NO PATCH}
    }}

    \node[font=\tiny\bfseries, fill=\patchcolor!20,
          draw=\patchcolor, rounded corners=2pt,
          inner sep=2pt, anchor=south east]
        at (#1.south east) {
        \patchtext
    };
    \node[font=\tiny, text=black!70, anchor=north east]
        at (#1.south east) [yshift=-0.3cm] {
        Vendor: #3
    };
}

% ============================================================================
% THREAT ACTOR VISUALIZATION
% ============================================================================

% Mark threat actor with attribution
% Usage: \markThreatActor{node}{actor_name}{confidence}
\newcommand{\markThreatActor}[3]{
    \node[above, font=\small\bfseries, text=threatCritical,
          fill=white, draw=threatCritical, rounded corners=2pt,
          inner sep=3pt] at (#1.north) [yshift=0.5cm] {
        #2 (Confidence: #3\%)
    };
}

% Show threat intelligence indicators
% Usage: \markIOC{node}{ioc_type}{value}
\newcommand{\markIOC}[3]{
    % IOC = Indicator of Compromise
    \node[font=\tiny\ttfamily, fill=yellow!30, 
          draw=orange, inner sep=2pt, anchor=south] 
        at (#1.south) [yshift=-0.3cm] {
        #2: #3
    };
}

% ============================================================================
% ENHANCED THREAT INTELLIGENCE & IOC TRACKING
% ============================================================================

% Advanced IOC visualization
% Usage: \markIOCAdvanced{node}{ioc_type}{value}{threat_score}{source}
% IOC types: ip, domain, hash, url, email
\newcommand{\markIOCAdvanced}[5]{
    \pgfmathsetmacro{\ioccolor}{ifthenelse(#4>=80,0,ifthenelse(#4>=50,1,2))}

    \ifnum\ioccolor=0
        \def\iocthreat{threatCritical}
        \def\iocrating{MALICIOUS}
    \else\ifnum\ioccolor=1
        \def\iocthreat{threatMedium}
        \def\iocrating{SUSPICIOUS}
    \else
        \def\iocthreat{threatLow}
        \def\iocrating{UNKNOWN}
    \fi\fi

    \node[draw=\iocthreat, fill=\iocthreat!15,
          rounded corners=2pt, font=\tiny\ttfamily,
          inner sep=3pt, anchor=south,
          minimum width=3cm]
        at (#1.south) [yshift=-1cm] {
        \begin{tabular}{l}
            \textbf{IOC: #2} \\
            #3 \\
            Score: #4/100 (\iocrating) \\
            Source: #5
        \end{tabular}
    };
}

% Threat feed indicator
% Usage: \addThreatFeed{node}{feed_name}{indicators_matched}
\newcommand{\addThreatFeed}[3]{
    \node[font=\tiny\bfseries, text=threatHigh,
          fill=yellow!40, draw=threatHigh,
          rounded corners=2pt, inner sep=2pt,
          anchor=north east]
        at (#1.north east) [xshift=0.5cm] {
        Feed: #2 | Matches: #3
    };
}

% TTP (Tactics, Techniques, Procedures) mapping
% Usage: \mapTTP{nodename}{tactic}{technique}{procedure}
\newcommand{\mapTTP}[4]{
    \node[draw=threatHigh, fill=threatHigh!10,
          rounded corners=3pt, font=\tiny,
          inner sep=4pt, anchor=west,
          minimum width=3.5cm]
        at (#1.east) [xshift=0.5cm] {
        \begin{tabular}{l}
            \textbf{TTP Analysis} \\
            \hline
            Tactic: #2 \\
            Technique: #3 \\
            Procedure: #4
        \end{tabular}
    };
}

% IP Reputation scoring
% Usage: \markIPReputation{node}{ip}{reputation}{category}
% Reputation: 0-100 (0=malicious, 100=clean)
\newcommand{\markIPReputation}[4]{
    \pgfmathsetmacro{\repcolor}{ifthenelse(#3<30,0,ifthenelse(#3<70,1,2))}

    \ifnum\repcolor=0
        \def\reputationcolor{threatCritical}
        \def\reputationstatus{MALICIOUS}
    \else\ifnum\repcolor=1
        \def\reputationcolor{threatMedium}
        \def\reputationstatus{SUSPICIOUS}
    \else
        \def\reputationcolor{clientGreen}
        \def\reputationstatus{CLEAN}
    \fi\fi

    \node[font=\tiny\ttfamily, fill=\reputationcolor!20,
          draw=\reputationcolor, rounded corners=2pt,
          inner sep=3pt]
        at (#1.south) [yshift=-0.6cm] {
        \begin{tabular}{c}
            #2 \\
            Rep: #3/100 \\
            \textbf{\reputationstatus} \\
            #4
        \end{tabular}
    };
}

% Enhanced threat actor attribution
% Usage: \markAdvancedThreatActor{node}{apt_name}{confidence}{motivation}{capabilities}
\newcommand{\markAdvancedThreatActor}[5]{
    \node[draw=threatCritical, fill=threatCritical!15,
          rounded corners=3pt, font=\tiny,
          inner sep=4pt, anchor=north,
          minimum width=4cm]
        at (#1.north) [yshift=1.5cm] {
        \begin{tabular}{ll}
            \multicolumn{2}{c}{\textbf{\large #2}} \\
            \hline
            Confidence: & #3\% \\
            Motivation: & #4 \\
            Capability: & #5 \\
        \end{tabular}
    };
}

% ============================================================================
% SECURITY POSTURE OVERVIEW
% ============================================================================

% Draw security posture dashboard
% Usage: \drawSecurityDashboard{x}{y}
\newcommand{\drawSecurityDashboard}[2]{
    \node[legend box, anchor=north east, minimum width=4cm] at (#1,#2) {
        \begin{tabular}{ll}
            \multicolumn{2}{c}{\textbf{Security Posture}} \\
            \hline
            \textcolor{threatCritical}{● Critical} & 0 \\
            \textcolor{threatHigh}{● High} & 0 \\
            \textcolor{threatMedium}{● Medium} & 0 \\
            \textcolor{threatLow}{● Low} & 0 \\
            \hline
            \textbf{Risk Score} & \textbf{0/100} \\
        \end{tabular}
    };
}

% Show compliance status
% Usage: \drawComplianceStatus{x}{y}{framework}
\newcommand{\drawComplianceStatus}[3]{
    \node[legend box, anchor=north east] at (#1,#2) {
        \small\bfseries #3 Compliance \\
        \tiny Status: Checking...
    };
}

% TODO: Security dashboards
% - Overall network security score
% - Compliance framework status (NIST, CIS, PCI-DSS, etc.)
% - Security control effectiveness metrics
% - Mean time to detect (MTTD)
% - Mean time to respond (MTTR)
% - Security coverage heatmap

% ============================================================================
% MAIN THREAT RENDERING ENGINE
% ============================================================================

\newcommand{\renderThreats}{
    % This will be populated by network_data.tex
    % Example structure:
    % \visualizeDDoS{attacker1,attacker2}{srv1}{critical}
    % \markVulnerability{srv2}{CVE-2024-1234}{9.8}
}

% TODO: Intelligent threat rendering
% - Threat prioritization based on risk
% - Auto-layout threat indicators to avoid overlap
% - Threat correlation and grouping
% - Attack path visualization
% - Threat timeline with incident markers
% - Real-time threat feed updates

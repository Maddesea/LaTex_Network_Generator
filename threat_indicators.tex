% threat_indicators.tex - Security threat visualization and indicators
% This module handles threat detection, attack visualization, and security status

% ============================================================================
% THREAT LEVEL DEFINITIONS
% ============================================================================

\newcommand{\threatCriticalLevel}{5}
\newcommand{\threatHighLevel}{4}
\newcommand{\threatMediumLevel}{3}
\newcommand{\threatLowLevel}{2}
\newcommand{\threatInfoLevel}{1}

% TODO: Threat scoring system
% - CVSS score integration
% - Custom threat scoring algorithms
% - Risk = Likelihood × Impact calculations
% - Temporal scoring (degrading over time)
% - Environmental scoring based on context

% ============================================================================
% THREAT INDICATOR VISUALIZATION
% ============================================================================

% Draw threat indicator icon
% Usage: \drawThreatIndicator{x}{y}{level}{type}
\newcommand{\drawThreatIndicator}[4]{
    \ifthenelse{\equal{#3}{critical}}{
        \def\threatcolor{threatCritical}
        \def\threatsize{0.5}
    }{
    \ifthenelse{\equal{#3}{high}}{
        \def\threatcolor{threatHigh}
        \def\threatsize{0.4}
    }{
    \ifthenelse{\equal{#3}{medium}}{
        \def\threatcolor{threatMedium}
        \def\threatsize{0.35}
    }{
    \ifthenelse{\equal{#3}{low}}{
        \def\threatcolor{threatLow}
        \def\threatsize}{0.3}
    }{
        \def\threatcolor{threatInfo}
        \def\threatsize{0.25}
    }}}}
    
    \node[
        regular polygon,
        regular polygon sides=3,
        fill=\threatcolor,
        draw=\threatcolor!80,
        minimum size=\threatsize cm,
        inner sep=0pt
    ] at (#1,#2) {};
    
    \node[font=\tiny\bfseries\sffamily, text=white] at (#1,#2) {!};
    
    \node[below, font=\tiny\sffamily] at (#1,#2-0.15) {#4};
}

% TODO: Enhanced threat indicators
% - Animated pulsing for active threats
% - Different icon shapes for different threat types
% - Severity gradient visualization
% - Threat trend arrows (increasing/decreasing)
% - Time-to-remediation countdown

% ============================================================================
% ATTACK VISUALIZATION
% ============================================================================

% Visualize specific attack types
% Usage: \visualizeAttack{attacker}{target}{attack_type}{severity}

% DDoS Attack visualization
\newcommand{\visualizeDDoS}[3]{
    % #1 = attacker list (comma-separated)
    % #2 = target
    % #3 = severity
    \foreach \attacker in {#1} {
        \draw[attack conn, line width=2pt] (\attacker) -- (#2);
    }
    \node[draw=threatCritical, line width=3pt, circle, 
          minimum size=1.5cm, fill=threatCritical!20] at (#2) {};
    \node[above, font=\small\bfseries, text=threatCritical] 
        at (#2.north) [yshift=0.8cm] {DDoS ATTACK};
}

% SQL Injection visualization
\newcommand{\visualizeSQLi}[2]{
    % #1 = attacker
    % #2 = database server
    \draw[attack conn, line width=2pt] (#1) -- (#2)
        node[midway, threat label] {SQL Injection};
    \node[draw=threatHigh, star, star points=8, 
          minimum size=0.8cm, fill=threatHigh!30] at (#2.north east) {};
}

% Malware/Ransomware visualization
\newcommand{\visualizeMalware}[2]{
    % #1 = infected node
    % #2 = malware type
    \node[draw=threatCritical, line width=3pt, 
          rounded corners=3pt, inner sep=8pt,
          fill=threatCritical!15, dashed] at (#1) {};
    \node[above, font=\tiny\bfseries, text=threatCritical, 
          fill=white, inner sep=2pt] at (#1.north) {#2};
}

% Data exfiltration visualization
\newcommand{\visualizeExfiltration}[3]{
    % #1 = source (compromised node)
    % #2 = destination (attacker)
    % #3 = data amount
    \draw[attack conn, line width=3pt] (#1) -- (#2)
        node[midway, above, threat label] {Exfil: #3};
    \draw[draw=threatCritical, line width=2pt, dashed, 
          -{Stealth[length=5mm]}] (#1) -- (#2);
}

% TODO: Additional attack visualizations
% - Phishing attack chain
% - Privilege escalation path
% - Lateral movement tracking
% - Command & Control beaconing
% - Brute force attempts (with attempt counter)
% - Zero-day exploitation indicators
% - Supply chain attack visualization

% ============================================================================
% ATTACK KILL CHAIN PROGRESSION (LOCKHEED MARTIN)
% ============================================================================

% Show attack kill chain progression with 7 stages
% Usage: \drawKillChain{x}{y}{current_stage}{defenses}
% Stages: 1=Reconnaissance, 2=Weaponization, 3=Delivery, 4=Exploitation,
%         5=Installation, 6=Command & Control, 7=Actions on Objectives
% Defenses: comma-separated list of stage numbers with defensive controls
\newcommand{\drawKillChain}[4]{
    \pgfmathsetmacro{\boxwidth}{3.5}
    \pgfmathsetmacro{\boxheight}{0.6}
    \pgfmathsetmacro{\xpos}{#1}
    \pgfmathsetmacro{\ypos}{#2}

    \node[anchor=north west, font=\small\bfseries] at (\xpos, \ypos) {Cyber Kill Chain};
    \pgfmathsetmacro{\ypos}{\ypos - 0.5}

    % Stage 1: Reconnaissance
    \pgfmathsetmacro{\currentstage}{#3}
    \ifthenelse{\lengthtest{\currentstage pt > 0.9 pt}}{
        \def\stagecolor{threatCritical}
        \def\stagelabel{ACTIVE}
    }{\def\stagecolor{gray!30}\def\stagelabel{}}

    \node[draw=\stagecolor, fill=\stagecolor!20, minimum width=\boxwidth cm,
          minimum height=\boxheight cm, anchor=north west] at (\xpos, \ypos) {
        \small\textbf{1. Reconnaissance} \tiny\stagelabel
    };
    \pgfmathsetmacro{\ypos}{\ypos - \boxheight - 0.05}

    % Stage 2: Weaponization
    \ifthenelse{\lengthtest{\currentstage pt > 1.9 pt}}{
        \def\stagecolor{threatCritical}
        \def\stagelabel{ACTIVE}
    }{\def\stagecolor{gray!30}\def\stagelabel{}}

    \node[draw=\stagecolor, fill=\stagecolor!20, minimum width=\boxwidth cm,
          minimum height=\boxheight cm, anchor=north west] at (\xpos, \ypos) {
        \small\textbf{2. Weaponization} \tiny\stagelabel
    };
    \pgfmathsetmacro{\ypos}{\ypos - \boxheight - 0.05}

    % Stage 3: Delivery
    \ifthenelse{\lengthtest{\currentstage pt > 2.9 pt}}{
        \def\stagecolor{threatCritical}
        \def\stagelabel{ACTIVE}
    }{\def\stagecolor{gray!30}\def\stagelabel{}}

    \node[draw=\stagecolor, fill=\stagecolor!20, minimum width=\boxwidth cm,
          minimum height=\boxheight cm, anchor=north west] at (\xpos, \ypos) {
        \small\textbf{3. Delivery} \tiny\stagelabel
    };
    \pgfmathsetmacro{\ypos}{\ypos - \boxheight - 0.05}

    % Stage 4: Exploitation
    \ifthenelse{\lengthtest{\currentstage pt > 3.9 pt}}{
        \def\stagecolor{threatCritical}
        \def\stagelabel{ACTIVE}
    }{\def\stagecolor{gray!30}\def\stagelabel{}}

    \node[draw=\stagecolor, fill=\stagecolor!20, minimum width=\boxwidth cm,
          minimum height=\boxheight cm, anchor=north west] at (\xpos, \ypos) {
        \small\textbf{4. Exploitation} \tiny\stagelabel
    };
    \pgfmathsetmacro{\ypos}{\ypos - \boxheight - 0.05}

    % Stage 5: Installation
    \ifthenelse{\lengthtest{\currentstage pt > 4.9 pt}}{
        \def\stagecolor{threatCritical}
        \def\stagelabel{ACTIVE}
    }{\def\stagecolor{gray!30}\def\stagelabel{}}

    \node[draw=\stagecolor, fill=\stagecolor!20, minimum width=\boxwidth cm,
          minimum height=\boxheight cm, anchor=north west] at (\xpos, \ypos) {
        \small\textbf{5. Installation} \tiny\stagelabel
    };
    \pgfmathsetmacro{\ypos}{\ypos - \boxheight - 0.05}

    % Stage 6: Command & Control
    \ifthenelse{\lengthtest{\currentstage pt > 5.9 pt}}{
        \def\stagecolor{threatCritical}
        \def\stagelabel{ACTIVE}
    }{\def\stagecolor{gray!30}\def\stagelabel{}}

    \node[draw=\stagecolor, fill=\stagecolor!20, minimum width=\boxwidth cm,
          minimum height=\boxheight cm, anchor=north west] at (\xpos, \ypos) {
        \small\textbf{6. Command \& Control} \tiny\stagelabel
    };
    \pgfmathsetmacro{\ypos}{\ypos - \boxheight - 0.05}

    % Stage 7: Actions on Objectives
    \ifthenelse{\lengthtest{\currentstage pt > 6.9 pt}}{
        \def\stagecolor{threatCritical}
        \def\stagelabel{ACTIVE}
    }{\def\stagecolor{gray!30}\def\stagelabel{}}

    \node[draw=\stagecolor, fill=\stagecolor!20, minimum width=\boxwidth cm,
          minimum height=\boxheight cm, anchor=north west] at (\xpos, \ypos) {
        \small\textbf{7. Actions on Objectives} \tiny\stagelabel
    };
}

% Show defensive gaps in kill chain
% Usage: \drawDefensiveGaps{x}{y}{protected_stages}{vulnerable_stages}
\newcommand{\drawDefensiveGaps}[4]{
    \node[legend box, anchor=north west, minimum width=4cm,
          draw=threatMedium, line width=1.5pt] at (#1,#2) {
        \begin{tabular}{p{3.5cm}}
            \multicolumn{1}{c}{\textbf{Defensive Posture}} \\
            \hline
            \textcolor{green!60!black}{\textbf{Protected Stages:}} \\
            #3 \\
            \hline
            \textcolor{threatHigh}{\textbf{Vulnerable Stages:}} \\
            #4 \\
        \end{tabular}
    };
}

% NIST Cybersecurity Framework mapping
% Usage: \drawNISTMapping{x}{y}{identify}{protect}{detect}{respond}{recover}
\newcommand{\drawNISTMapping}[7]{
    \node[legend box, anchor=north west, minimum width=5cm,
          draw=blue!70, line width=2pt] at (#1,#2) {
        \begin{tabular}{lc}
            \multicolumn{2}{c}{\textbf{NIST CSF Coverage}} \\
            \hline
            \textbf{Identify} & #3\% \\
            \textbf{Protect} & #4\% \\
            \textbf{Detect} & #5\% \\
            \textbf{Respond} & #6\% \\
            \textbf{Recover} & #7\% \\
        \end{tabular}
    };
}

% Attack timeline visualization
% Usage: \drawAttackTimeline{x}{y}{start_time}{current_time}{events}
\newcommand{\drawAttackTimeline}[5]{
    \node[legend box, anchor=north west, minimum width=6cm,
          draw=threatHigh, line width=1.5pt] at (#1,#2) {
        \begin{tabular}{ll}
            \multicolumn{2}{c}{\textbf{Attack Timeline}} \\
            \hline
            \textbf{Attack Started:} & #3 \\
            \textbf{Current Time:} & #4 \\
            \hline
            \multicolumn{2}{l}{\textbf{Key Events:}} \\
            \multicolumn{2}{p{5.5cm}}{#5} \\
        \end{tabular}
    };
}

% Attack path visualization
% Usage: \drawAttackPath{source}{intermediate}{target}{stage}
\newcommand{\drawAttackPath}[4]{
    \draw[attack conn, line width=3pt, -{Stealth[length=5mm]}]
        (#1) -- (#2) node[midway, above, font=\tiny, fill=white, inner sep=1pt] {Stage #4};
    \draw[attack conn, line width=3pt, -{Stealth[length=5mm]}]
        (#2) -- (#3) node[midway, above, font=\tiny, fill=white, inner sep=1pt] {Stage #4};
}

% ============================================================================
% SECURITY ZONES AND BOUNDARIES
% ============================================================================

% Mark security boundary breach
% Usage: \markBoundaryBreach{zone1}{zone2}{breach_point}
\newcommand{\markBoundaryBreach}[3]{
    \draw[draw=threatCritical, line width=3pt, 
          decoration={zigzag, segment length=4pt, amplitude=2pt},
          decorate] (#1) -- (#2);
    \node[circle, fill=threatCritical, minimum size=0.4cm] at (#3) {};
    \node[above, font=\tiny\bfseries, text=white] at (#3) {BREACH};
}

% Draw firewall bypass indicator
\newcommand{\showFirewallBypass}[2]{
    % #1 = firewall node
    % #2 = bypass method
    \node[draw=threatHigh, cross out, line width=2pt, 
          minimum size=1cm, inner sep=0pt] at (#1) {};
    \node[below, font=\tiny, text=threatHigh] at (#1.south) {Bypassed: #2};
}

% TODO: Security control visualization
% - IDS/IPS alert indicators
% - Failed authentication attempts
% - Access control violations
% - Encryption status (enabled/disabled/weak)
% - Patch status indicators

% ============================================================================
% VULNERABILITY INDICATORS
% ============================================================================

% Mark vulnerable node with CVE
% Usage: \markVulnerability{node}{cve}{cvss_score}
\newcommand{\markVulnerability}[3]{
    \pgfmathsetmacro{\severity}{#3/10}
    \ifthenelse{\lengthtest{\severity pt > 0.7 pt}}{
        \def\vulncolor{threatCritical}
    }{
    \ifthenelse{\lengthtest{\severity pt > 0.4 pt}}{
        \def\vulncolor{threatMedium}
    }{
        \def\vulncolor{threatLow}
    }}
    
    \node[draw=\vulncolor, line width=2pt, rectangle,
          rounded corners=2pt, inner sep=3pt, fill=\vulncolor!20,
          anchor=south west] at (#1.south west) {
        \tiny\ttfamily #2: #3
    };
}

% Show exploitable service
% Usage: \markExploitableService{node}{service}{port}
\newcommand{\markExploitableService}[3]{
    \node[draw=threatHigh, fill=threatHigh!20, 
          font=\tiny\ttfamily, anchor=north east] 
        at (#1.north east) {#2:#3};
}

% ============================================================================
% ENHANCED CVSS SCORE VISUALIZATION
% ============================================================================

% Parse CVSS vector and display comprehensive score breakdown
% Usage: \drawCVSSScore{x}{y}{cve}{base_score}{temporal_score}{environmental_score}{vector}
\newcommand{\drawCVSSScore}[7]{
    % Determine severity level based on base score
    \pgfmathsetmacro{\basescore}{#4}
    \ifthenelse{\lengthtest{\basescore pt > 9.0 pt}}{
        \def\cvsscolor{threatCritical}
        \def\severitylabel{CRITICAL}
    }{
    \ifthenelse{\lengthtest{\basescore pt > 7.0 pt}}{
        \def\cvsscolor{threatHigh}
        \def\severitylabel{HIGH}
    }{
    \ifthenelse{\lengthtest{\basescore pt > 4.0 pt}}{
        \def\cvsscolor{threatMedium}
        \def\severitylabel{MEDIUM}
    }{
    \ifthenelse{\lengthtest{\basescore pt > 0.1 pt}}{
        \def\cvsscolor{threatLow}
        \def\severitylabel{LOW}
    }{
        \def\cvsscolor{threatInfo}
        \def\severitylabel{INFO}
    }}}}

    % Draw CVSS score box
    \node[legend box, anchor=north west, minimum width=3.5cm,
          draw=\cvsscolor, line width=2pt] at (#1,#2) {
        \begin{tabular}{ll}
            \multicolumn{2}{c}{\textbf{\textcolor{\cvsscolor}{#3}}} \\
            \hline
            \textbf{Severity:} & \textcolor{\cvsscolor}{\severitylabel} \\
            \textbf{Base:} & #4 \\
            \textbf{Temporal:} & #5 \\
            \textbf{Environmental:} & #6 \\
            \hline
            \multicolumn{2}{l}{\tiny\ttfamily #7} \\
        \end{tabular}
    };
}

% Compact CVSS badge for node overlay
% Usage: \cvssbadge{node}{cve}{score}
\newcommand{\cvssbadge}[3]{
    \pgfmathsetmacro{\score}{#3}
    \ifthenelse{\lengthtest{\score pt > 9.0 pt}}{
        \def\badgecolor{threatCritical}
    }{
    \ifthenelse{\lengthtest{\score pt > 7.0 pt}}{
        \def\badgecolor{threatHigh}
    }{
    \ifthenelse{\lengthtest{\score pt > 4.0 pt}}{
        \def\badgecolor{threatMedium}
    }{
        \def\badgecolor{threatLow}
    }}}

    \node[draw=\badgecolor, fill=\badgecolor, text=white,
          font=\tiny\bfseries, rounded corners=2pt, inner sep=2pt,
          anchor=north west] at (#1.north west) {#3};
    \node[draw=\badgecolor, fill=white, text=\badgecolor,
          font=\tiny\ttfamily, rounded corners=2pt, inner sep=2pt,
          anchor=south west] at (#1.south west) {#2};
}

% CVSS metrics breakdown visualization
% Usage: \drawCVSSMetrics{x}{y}{AV}{AC}{PR}{UI}{S}{C}{I}{A}
% AV=Attack Vector, AC=Attack Complexity, PR=Privileges Required,
% UI=User Interaction, S=Scope, C=Confidentiality, I=Integrity, A=Availability
\newcommand{\drawCVSSMetrics}[11]{
    \node[legend box, anchor=north west, minimum width=4cm] at (#1,#2) {
        \begin{tabular}{ll}
            \multicolumn{2}{c}{\textbf{CVSS v3.1 Metrics}} \\
            \hline
            \textbf{Attack Vector:} & #3 \\
            \textbf{Attack Complexity:} & #4 \\
            \textbf{Privileges Required:} & #5 \\
            \textbf{User Interaction:} & #6 \\
            \textbf{Scope:} & #7 \\
            \hline
            \textbf{Confidentiality:} & \textcolor{threatHigh}{#8} \\
            \textbf{Integrity:} & \textcolor{threatMedium}{#9} \\
            \textbf{Availability:} & \textcolor{threatMedium}{#{10}} \\
        \end{tabular}
    };
}

% ============================================================================
% MITRE ATT&CK FRAMEWORK MAPPING
% ============================================================================

% Display MITRE ATT&CK technique
% Usage: \drawMITREAttack{x}{y}{technique_id}{technique_name}{tactic}
\newcommand{\drawMITREAttack}[5]{
    \node[legend box, anchor=north west, minimum width=4.5cm,
          draw=threatHigh, line width=1.5pt] at (#1,#2) {
        \begin{tabular}{ll}
            \multicolumn{2}{c}{\textbf{\textcolor{threatHigh}{MITRE ATT\&CK}}} \\
            \hline
            \textbf{Technique:} & \texttt{#3} \\
            \textbf{Name:} & #4 \\
            \textbf{Tactic:} & \textcolor{threatMedium}{#5} \\
            \hline
            \multicolumn{2}{l}{\tiny attack.mitre.org/techniques/#3} \\
        \end{tabular}
    };
}

% MITRE ATT&CK badge for nodes
% Usage: \mitrebadge{node}{technique_id}
\newcommand{\mitrebadge}[2]{
    \node[draw=threatHigh, fill=threatHigh, text=white,
          font=\tiny\bfseries\ttfamily, rounded corners=1pt, inner sep=1pt,
          anchor=north east] at (#1.north east) [xshift=-0.1cm, yshift=-0.1cm] {#2};
}

% Draw MITRE ATT&CK kill chain progression
% Usage: \drawMITREKillChain{x}{y}{current_stage}
% Stages: Reconnaissance, Resource Development, Initial Access, Execution,
%         Persistence, Privilege Escalation, Defense Evasion, Credential Access,
%         Discovery, Lateral Movement, Collection, Command & Control, Exfiltration,
%         Impact
\newcommand{\drawMITREKillChain}[3]{
    \pgfmathsetmacro{\boxwidth}{1.8}
    \pgfmathsetmacro{\boxheight}{0.6}
    \pgfmathsetmacro{\ypos}{#2}

    \node[anchor=north west, font=\small\bfseries] at (#1, \ypos) {MITRE ATT\&CK Kill Chain};
    \pgfmathsetmacro{\ypos}{\ypos - 0.4}

    % Stage 1: Reconnaissance
    \ifthenelse{\equal{#3}{1} \OR \equal{#3}{Reconnaissance}}{
        \def\stagecolor{threatCritical}
    }{\def\stagecolor{gray!30}}
    \node[draw=\stagecolor, fill=\stagecolor!20, minimum width=\boxwidth cm,
          minimum height=\boxheight cm, font=\tiny, anchor=north west]
          at (#1, \ypos) {Reconnaissance};

    \pgfmathsetmacro{\ypos}{\ypos - \boxheight - 0.1}

    % Stage 2: Initial Access
    \ifthenelse{\equal{#3}{2} \OR \equal{#3}{Initial Access}}{
        \def\stagecolor{threatCritical}
    }{\def\stagecolor{gray!30}}
    \node[draw=\stagecolor, fill=\stagecolor!20, minimum width=\boxwidth cm,
          minimum height=\boxheight cm, font=\tiny, anchor=north west]
          at (#1, \ypos) {Initial Access};

    \pgfmathsetmacro{\ypos}{\ypos - \boxheight - 0.1}

    % Stage 3: Execution
    \ifthenelse{\equal{#3}{3} \OR \equal{#3}{Execution}}{
        \def\stagecolor{threatCritical}
    }{\def\stagecolor{gray!30}}
    \node[draw=\stagecolor, fill=\stagecolor!20, minimum width=\boxwidth cm,
          minimum height=\boxheight cm, font=\tiny, anchor=north west]
          at (#1, \ypos) {Execution};

    \pgfmathsetmacro{\ypos}{\ypos - \boxheight - 0.1}

    % Stage 4: Persistence
    \ifthenelse{\equal{#3}{4} \OR \equal{#3}{Persistence}}{
        \def\stagecolor{threatCritical}
    }{\def\stagecolor{gray!30}}
    \node[draw=\stagecolor, fill=\stagecolor!20, minimum width=\boxwidth cm,
          minimum height=\boxheight cm, font=\tiny, anchor=north west]
          at (#1, \ypos) {Persistence};

    \pgfmathsetmacro{\ypos}{\ypos - \boxheight - 0.1}

    % Stage 5: Privilege Escalation
    \ifthenelse{\equal{#3}{5} \OR \equal{#3}{Privilege Escalation}}{
        \def\stagecolor{threatCritical}
    }{\def\stagecolor{gray!30}}
    \node[draw=\stagecolor, fill=\stagecolor!20, minimum width=\boxwidth cm,
          minimum height=\boxheight cm, font=\tiny, anchor=north west]
          at (#1, \ypos) {Privilege Esc.};

    \pgfmathsetmacro{\ypos}{\ypos - \boxheight - 0.1}

    % Stage 6: Defense Evasion
    \ifthenelse{\equal{#3}{6} \OR \equal{#3}{Defense Evasion}}{
        \def\stagecolor{threatCritical}
    }{\def\stagecolor{gray!30}}
    \node[draw=\stagecolor, fill=\stagecolor!20, minimum width=\boxwidth cm,
          minimum height=\boxheight cm, font=\tiny, anchor=north west]
          at (#1, \ypos) {Defense Evasion};

    \pgfmathsetmacro{\ypos}{\ypos - \boxheight - 0.1}

    % Stage 7: Lateral Movement
    \ifthenelse{\equal{#3}{7} \OR \equal{#3}{Lateral Movement}}{
        \def\stagecolor{threatCritical}
    }{\def\stagecolor{gray!30}}
    \node[draw=\stagecolor, fill=\stagecolor!20, minimum width=\boxwidth cm,
          minimum height=\boxheight cm, font=\tiny, anchor=north west]
          at (#1, \ypos) {Lateral Movement};

    \pgfmathsetmacro{\ypos}{\ypos - \boxheight - 0.1}

    % Stage 8: Collection & Exfiltration
    \ifthenelse{\equal{#3}{8} \OR \equal{#3}{Exfiltration}}{
        \def\stagecolor{threatCritical}
    }{\def\stagecolor{gray!30}}
    \node[draw=\stagecolor, fill=\stagecolor!20, minimum width=\boxwidth cm,
          minimum height=\boxheight cm, font=\tiny, anchor=north west]
          at (#1, \ypos) {Exfiltration};

    \pgfmathsetmacro{\ypos}{\ypos - \boxheight - 0.1}

    % Stage 9: Impact
    \ifthenelse{\equal{#3}{9} \OR \equal{#3}{Impact}}{
        \def\stagecolor{threatCritical}
    }{\def\stagecolor{gray!30}}
    \node[draw=\stagecolor, fill=\stagecolor!20, minimum width=\boxwidth cm,
          minimum height=\boxheight cm, font=\tiny, anchor=north west]
          at (#1, \ypos) {Impact};
}

% Map technique to tactic with visual connection
% Usage: \mapTechniqueToTactic{node}{technique_id}{tactic_name}
\newcommand{\mapTechniqueToTactic}[3]{
    \node[above, font=\tiny\bfseries\ttfamily, text=threatHigh,
          fill=white, draw=threatHigh, rounded corners=1pt,
          inner sep=1pt] at (#1.north) [yshift=0.3cm] {
        #2: #3
    };
}

% Show multiple MITRE techniques for complex attack
% Usage: \drawMITRETTP{x}{y}{title}{techniques_list}
\newcommand{\drawMITRETTP}[4]{
    \node[legend box, anchor=north west, minimum width=5cm,
          draw=threatHigh, line width=1.5pt, fill=white] at (#1,#2) {
        \begin{tabular}{p{4.5cm}}
            \multicolumn{1}{c}{\textbf{\textcolor{threatHigh}{#3}}} \\
            \hline
            \textbf{Techniques Used:} \\
            #4 \\
        \end{tabular}
    };
}

% ============================================================================
% ENHANCED THREAT ACTOR ATTRIBUTION
% ============================================================================

% Mark threat actor with attribution
% Usage: \markThreatActor{node}{actor_name}{confidence}
\newcommand{\markThreatActor}[3]{
    \node[above, font=\small\bfseries, text=threatCritical,
          fill=white, draw=threatCritical, rounded corners=2pt,
          inner sep=3pt] at (#1.north) [yshift=0.5cm] {
        #2 (Confidence: #3\%)
    };
}

% Comprehensive threat actor profile
% Usage: \drawThreatActorProfile{x}{y}{actor_name}{aka}{origin}{motivation}{confidence}
\newcommand{\drawThreatActorProfile}[7]{
    \pgfmathsetmacro{\confpct}{#7}
    \ifthenelse{\lengthtest{\confpct pt > 80 pt}}{
        \def\confcolor{threatCritical}
    }{
    \ifthenelse{\lengthtest{\confpct pt > 50 pt}}{
        \def\confcolor{threatHigh}
    }{
        \def\confcolor{threatMedium}
    }}

    \node[legend box, anchor=north west, minimum width=6cm,
          draw=\confcolor, line width=2pt] at (#1,#2) {
        \begin{tabular}{ll}
            \multicolumn{2}{c}{\textbf{\textcolor{\confcolor}{Threat Actor Profile}}} \\
            \hline
            \textbf{Primary Name:} & #3 \\
            \textbf{Also Known As:} & \textit{#4} \\
            \textbf{Origin:} & #5 \\
            \textbf{Motivation:} & #6 \\
            \hline
            \textbf{Confidence:} & \textcolor{\confcolor}{#7\%} \\
        \end{tabular}
    };
}

% TTP (Tactics, Techniques, Procedures) overlay
% Usage: \drawTTPProfile{x}{y}{actor_name}{preferred_techniques}{tools_used}
\newcommand{\drawTTPProfile}[5]{
    \node[legend box, anchor=north west, minimum width=6cm,
          draw=threatHigh, line width=1.5pt] at (#1,#2) {
        \begin{tabular}{p{5.5cm}}
            \multicolumn{1}{c}{\textbf{TTP Profile: #3}} \\
            \hline
            \textbf{Preferred Techniques:} \\
            #4 \\
            \hline
            \textbf{Known Tools:} \\
            #5 \\
        \end{tabular}
    };
}

% Campaign tracking visualization
% Usage: \drawCampaignTracker{x}{y}{campaign_name}{start_date}{targets}{status}
\newcommand{\drawCampaignTracker}[6]{
    \ifthenelse{\equal{#6}{active}}{
        \def\statuscolor{threatCritical}
    }{
    \ifthenelse{\equal{#6}{ongoing}}{
        \def\statuscolor{threatHigh}
    }{
        \def\statuscolor{gray!50}
    }}

    \node[legend box, anchor=north west, minimum width=5.5cm,
          draw=\statuscolor, line width=2pt] at (#1,#2) {
        \begin{tabular}{ll}
            \multicolumn{2}{c}{\textbf{\textcolor{\statuscolor}{Campaign: #3}}} \\
            \hline
            \textbf{Started:} & #4 \\
            \textbf{Targets:} & #5 \\
            \textbf{Status:} & \textcolor{\statuscolor}{\uppercase{#6}} \\
        \end{tabular}
    };
}

% Threat actor attribution badge with confidence indicator
% Usage: \actorBadge{node}{actor_name}{confidence}
\newcommand{\actorBadge}[3]{
    \pgfmathsetmacro{\conf}{#3}
    \ifthenelse{\lengthtest{\conf pt > 70 pt}}{
        \def\badgecolor{red!80!black}
    }{
    \ifthenelse{\lengthtest{\conf pt > 40 pt}}{
        \def\badgecolor{orange!80!black}
    }{
        \def\badgecolor{yellow!80!black}
    }}

    \node[draw=\badgecolor, fill=\badgecolor, text=white,
          font=\tiny\bfseries, rounded corners=2pt, inner sep=2pt,
          anchor=north east] at (#1.north east) [xshift=-0.1cm] {
        #2: #3\%
    };
}

% Multiple attribution (different analysts have different conclusions)
% Usage: \drawAttributionDebate{x}{y}{target}{analyst1_conclusion}{analyst2_conclusion}
\newcommand{\drawAttributionDebate}[5]{
    \node[legend box, anchor=north west, minimum width=5cm,
          draw=orange, line width=1.5pt] at (#1,#2) {
        \begin{tabular}{p{4.5cm}}
            \multicolumn{1}{c}{\textbf{Attribution Debate: #3}} \\
            \hline
            \textcolor{threatHigh}{Analyst A:} #4 \\
            \hline
            \textcolor{threatMedium}{Analyst B:} #5 \\
        \end{tabular}
    };
}

% ============================================================================
% ENHANCED IOC (INDICATORS OF COMPROMISE) VISUALIZATION
% ============================================================================

% Show threat intelligence indicators with type-specific formatting
% Usage: \markIOC{node}{ioc_type}{value}{reputation_score}
\newcommand{\markIOC}[4]{
    % IOC = Indicator of Compromise
    % Determine color based on reputation score (0-100, higher is worse)
    \pgfmathsetmacro{\repscore}{#4}
    \ifthenelse{\lengthtest{\repscore pt > 80 pt}}{
        \def\ioccolor{threatCritical}
    }{
    \ifthenelse{\lengthtest{\repscore pt > 60 pt}}{
        \def\ioccolor{threatHigh}
    }{
    \ifthenelse{\lengthtest{\repscore pt > 40 pt}}{
        \def\ioccolor{threatMedium}
    }{
        \def\ioccolor{threatLow}
    }}}

    \node[font=\tiny\ttfamily, fill=\ioccolor!20,
          draw=\ioccolor, inner sep=2pt, anchor=south,
          line width=1pt]
        at (#1.south) [yshift=-0.3cm] {
        #2: #3 (Rep: #4)
    };
}

% Display malicious IP address indicator
% Usage: \markMaliciousIP{node}{ip_address}{reputation}{threat_feed}
\newcommand{\markMaliciousIP}[4]{
    \pgfmathsetmacro{\repscore}{#3}
    \ifthenelse{\lengthtest{\repscore pt > 80 pt}}{
        \def\ipcolor{threatCritical}
    }{\def\ipcolor{threatHigh}}

    \node[font=\tiny\ttfamily, fill=\ipcolor!30,
          draw=\ipcolor, line width=2pt, inner sep=2pt,
          anchor=south west, rounded corners=1pt]
        at (#1.south west) [yshift=-0.4cm] {
        \textbf{Malicious IP:} #2
    };
    \node[font=\tiny, fill=white, draw=\ipcolor,
          inner sep=1pt, anchor=north west]
        at (#1.south west) [yshift=-0.7cm] {
        Source: #4 | Rep: #3/100
    };
}

% Display malicious domain indicator
% Usage: \markMaliciousDomain{node}{domain}{reputation}{category}
\newcommand{\markMaliciousDomain}[4]{
    \node[font=\tiny\ttfamily, fill=threatHigh!30,
          draw=threatHigh, line width=1.5pt, inner sep=2pt,
          anchor=north, rounded corners=1pt]
        at (#1.north) [yshift=0.4cm] {
        \textbf{Malicious Domain:} #2
    };
    \node[font=\tiny, fill=white, draw=threatHigh,
          inner sep=1pt, anchor=north]
        at (#1.north) [yshift=0.1cm] {
        Category: #4 | Rep: #3/100
    };
}

% Display malware file hash indicator
% Usage: \markMalwareHash{node}{hash_type}{hash_value}{detection_rate}
\newcommand{\markMalwareHash}[4]{
    \node[legend box, anchor=north west, minimum width=5cm,
          draw=threatCritical, line width=2pt] at (#1) {
        \begin{tabular}{ll}
            \multicolumn{2}{c}{\textbf{\textcolor{threatCritical}{Malware Detected}}} \\
            \hline
            \textbf{Hash Type:} & #2 \\
            \textbf{Hash:} & \texttt{\tiny #3} \\
            \textbf{Detection:} & #4 AV engines \\
        \end{tabular}
    };
}

% IOC age/freshness indicator
% Usage: \markIOCFreshness{node}{age_days}{status}
\newcommand{\markIOCFreshness}[3]{
    \pgfmathsetmacro{\agedays}{#2}
    \ifthenelse{\lengthtest{\agedays pt < 7 pt}}{
        \def\freshnesscolor{threatCritical}
        \def\freshnesslabel{FRESH}
    }{
    \ifthenelse{\lengthtest{\agedays pt < 30 pt}}{
        \def\freshnesscolor{threatHigh}
        \def\freshnesslabel{RECENT}
    }{
    \ifthenelse{\lengthtest{\agedays pt < 90 pt}}{
        \def\freshnesscolor{threatMedium}
        \def\freshnesslabel{OLD}
    }{
        \def\freshnesscolor{threatLow}
        \def\freshnesslabel{STALE}
    }}}

    \node[fill=\freshnesscolor, text=white, font=\tiny\bfseries,
          inner sep=1pt, rounded corners=1pt, anchor=south east]
        at (#1.south east) {
        \freshnesslabel\ (#2d)
    };
}

% Comprehensive IOC dashboard
% Usage: \drawIOCDashboard{x}{y}{malicious_ips}{malicious_domains}{malware_hashes}{total_iocs}
\newcommand{\drawIOCDashboard}[6]{
    \node[legend box, anchor=north west, minimum width=4.5cm,
          draw=threatHigh, line width=2pt] at (#1,#2) {
        \begin{tabular}{lr}
            \multicolumn{2}{c}{\textbf{\textcolor{threatHigh}{IOC Summary}}} \\
            \hline
            Malicious IPs & #3 \\
            Malicious Domains & #4 \\
            Malware Hashes & #5 \\
            \hline
            \textbf{Total IOCs} & \textbf{#6} \\
        \end{tabular}
    };
}

% Threat feed integration indicator
% Usage: \markThreatFeed{x}{y}{feed_name}{last_update}{active_threats}
\newcommand{\markThreatFeed}[5]{
    \node[legend box, anchor=north west, minimum width=4cm,
          draw=threatMedium, line width=1.5pt] at (#1,#2) {
        \begin{tabular}{ll}
            \multicolumn{2}{c}{\textbf{Threat Feed: #3}} \\
            \hline
            \textbf{Last Update:} & #4 \\
            \textbf{Active Threats:} & \textcolor{threatHigh}{#5} \\
        \end{tabular}
    };
}

% ============================================================================
% SECURITY POSTURE OVERVIEW
% ============================================================================

% Draw security posture dashboard
% Usage: \drawSecurityDashboard{x}{y}{critical}{high}{medium}{low}{risk_score}
\newcommand{\drawSecurityDashboard}[7]{
    \node[legend box, anchor=north east, minimum width=4cm] at (#1,#2) {
        \begin{tabular}{ll}
            \multicolumn{2}{c}{\textbf{Security Posture}} \\
            \hline
            \textcolor{threatCritical}{● Critical} & #3 \\
            \textcolor{threatHigh}{● High} & #4 \\
            \textcolor{threatMedium}{● Medium} & #5 \\
            \textcolor{threatLow}{● Low} & #6 \\
            \hline
            \textbf{Risk Score} & \textbf{#7/100} \\
        \end{tabular}
    };
}

% ============================================================================
% COMPLIANCE FRAMEWORK DASHBOARDS
% ============================================================================

% NIST Cybersecurity Framework detailed dashboard
% Usage: \drawNISTDashboard{x}{y}{identify}{protect}{detect}{respond}{recover}
\newcommand{\drawNISTDashboard}[7]{
    \node[legend box, anchor=north west, minimum width=6cm,
          draw=blue!70, line width=2pt] at (#1,#2) {
        \begin{tabular}{lcc}
            \multicolumn{3}{c}{\textbf{NIST CSF Assessment}} \\
            \hline
            \textbf{Function} & \textbf{Score} & \textbf{Status} \\
            \hline
            Identify & #3\% & \drawComplianceBar{#3} \\
            Protect & #4\% & \drawComplianceBar{#4} \\
            Detect & #5\% & \drawComplianceBar{#5} \\
            Respond & #6\% & \drawComplianceBar{#6} \\
            Recover & #7\% & \drawComplianceBar{#7} \\
        \end{tabular}
    };
}

% Helper for compliance status bar
\newcommand{\drawComplianceBar}[1]{
    \pgfmathsetmacro{\score}{#1}
    \ifthenelse{\lengthtest{\score pt > 80 pt}}{
        \textcolor{green!60!black}{●●●●●}
    }{
    \ifthenelse{\lengthtest{\score pt > 60 pt}}{
        \textcolor{green!60!black}{●●●●}\textcolor{gray!50}{●}
    }{
    \ifthenelse{\lengthtest{\score pt > 40 pt}}{
        \textcolor{orange}{●●●}\textcolor{gray!50}{●●}
    }{
    \ifthenelse{\lengthtest{\score pt > 20 pt}}{
        \textcolor{threatHigh}{●●}\textcolor{gray!50}{●●●}
    }{
        \textcolor{threatCritical}{●}\textcolor{gray!50}{●●●●}
    }}}}
}

% CIS Controls compliance dashboard
% Usage: \drawCISControls{x}{y}{basic}{foundational}{organizational}{overall}
\newcommand{\drawCISControls}[6]{
    \node[legend box, anchor=north west, minimum width=6cm,
          draw=blue!60, line width=2pt] at (#1,#2) {
        \begin{tabular}{lc}
            \multicolumn{2}{c}{\textbf{CIS Controls v8}} \\
            \hline
            \textbf{Control Category} & \textbf{Coverage} \\
            \hline
            Basic (IG1) & #3\% \\
            Foundational (IG2) & #4\% \\
            Organizational (IG3) & #5\% \\
            \hline
            \textbf{Overall Compliance} & \textbf{#6\%} \\
        \end{tabular}
    };
}

% PCI-DSS compliance dashboard
% Usage: \drawPCIDSS{x}{y}{network_security}{access_control}{monitoring}{info_security}{policies}{overall}
\newcommand{\drawPCIDSS}[8]{
    \pgfmathsetmacro{\overallscore}{#8}
    \ifthenelse{\lengthtest{\overallscore pt > 90 pt}}{
        \def\pcicolor{green!60!black}
        \def\pcistatus{COMPLIANT}
    }{
    \ifthenelse{\lengthtest{\overallscore pt > 70 pt}}{
        \def\pcicolor{orange}
        \def\pcistatus{REMEDIATION NEEDED}
    }{
        \def\pcicolor{threatCritical}
        \def\pcistatus{NON-COMPLIANT}
    }}

    \node[legend box, anchor=north west, minimum width=6.5cm,
          draw=\pcicolor, line width=2pt] at (#1,#2) {
        \begin{tabular}{lc}
            \multicolumn{2}{c}{\textbf{PCI-DSS v4.0 Compliance}} \\
            \hline
            Network Security & #3\% \\
            Access Control & #4\% \\
            Monitoring \& Testing & #5\% \\
            Information Security & #6\% \\
            Policies \& Procedures & #7\% \\
            \hline
            \textbf{Overall} & \textbf{#8\%} \\
            \textbf{Status} & \textcolor{\pcicolor}{\pcistatus} \\
        \end{tabular}
    };
}

% HIPAA compliance dashboard
% Usage: \drawHIPAA{x}{y}{admin}{physical}{technical}{overall}
\newcommand{\drawHIPAA}[6]{
    \pgfmathsetmacro{\hipaascore}{#6}
    \ifthenelse{\lengthtest{\hipaascore pt > 85 pt}}{
        \def\hipaacolor{green!60!black}
    }{
        \def\hipaacolor{threatCritical}
    }

    \node[legend box, anchor=north west, minimum width=5.5cm,
          draw=\hipaacolor, line width=2pt] at (#1,#2) {
        \begin{tabular}{lc}
            \multicolumn{2}{c}{\textbf{HIPAA Compliance}} \\
            \hline
            Administrative Safeguards & #3\% \\
            Physical Safeguards & #4\% \\
            Technical Safeguards & #5\% \\
            \hline
            \textbf{Overall Compliance} & \textbf{#6\%} \\
        \end{tabular}
    };
}

% SOC 2 trust services criteria
% Usage: \drawSOC2{x}{y}{security}{availability}{processing}{confidentiality}{privacy}
\newcommand{\drawSOC2}[7]{
    \node[legend box, anchor=north west, minimum width=6cm,
          draw=purple!70, line width=2pt] at (#1,#2) {
        \begin{tabular}{lc}
            \multicolumn{2}{c}{\textbf{SOC 2 Trust Services}} \\
            \hline
            Security & #3\% \\
            Availability & #4\% \\
            Processing Integrity & #5\% \\
            Confidentiality & #6\% \\
            Privacy & #7\% \\
        \end{tabular}
    };
}

% ISO 27001 compliance dashboard
% Usage: \drawISO27001{x}{y}{controls_implemented}{controls_total}{certification_status}
\newcommand{\drawISO27001}[5]{
    \ifthenelse{\equal{#5}{certified}}{
        \def\isocolor{green!60!black}
    }{
    \ifthenelse{\equal{#5}{in-progress}}{
        \def\isocolor{orange}
    }{
        \def\isocolor{gray!60}
    }}

    \node[legend box, anchor=north west, minimum width=5cm,
          draw=\isocolor, line width=2pt] at (#1,#2) {
        \begin{tabular}{lc}
            \multicolumn{2}{c}{\textbf{ISO 27001:2022}} \\
            \hline
            Controls Implemented & #3 \\
            Total Controls & #4 \\
            \hline
            \textbf{Status} & \textcolor{\isocolor}{\uppercase{#5}} \\
        \end{tabular}
    };
}

% Multi-framework compliance overview
% Usage: \drawMultiFrameworkCompliance{x}{y}{nist}{cis}{pci}{iso}
\newcommand{\drawMultiFrameworkCompliance}[6]{
    \node[legend box, anchor=north west, minimum width=5.5cm,
          draw=blue!80, line width=2.5pt] at (#1,#2) {
        \begin{tabular}{lc}
            \multicolumn{2}{c}{\textbf{Compliance Overview}} \\
            \hline
            NIST CSF & #3\% \\
            CIS Controls & #4\% \\
            PCI-DSS & #5\% \\
            ISO 27001 & #6\% \\
        \end{tabular}
    };
}

% ============================================================================
% VULNERABILITY DATABASE INTEGRATION
% ============================================================================

% Comprehensive vulnerability report for a node
% Usage: \drawVulnerabilityReport{x}{y}{node_name}{cve_count}{critical}{high}{medium}{low}
\newcommand{\drawVulnerabilityReport}[8]{
    \pgfmathsetmacro{\totalvulns}{#3 + #4 + #5 + #6}
    \node[legend box, anchor=north west, minimum width=5cm,
          draw=threatHigh, line width=2pt] at (#1,#2) {
        \begin{tabular}{lc}
            \multicolumn{2}{c}{\textbf{Vulnerability Report: #3}} \\
            \hline
            \textcolor{threatCritical}{Critical} & #5 \\
            \textcolor{threatHigh}{High} & #6 \\
            \textcolor{threatMedium}{Medium} & #7 \\
            \textcolor{threatLow}{Low} & #8 \\
            \hline
            \textbf{Total CVEs} & \textbf{#4} \\
        \end{tabular}
    };
}

% Exploit availability indicator
% Usage: \markExploitAvailable{node}{cve}{exploit_maturity}
% exploit_maturity: poc, functional, high (weaponized)
\newcommand{\markExploitAvailable}[3]{
    \ifthenelse{\equal{#3}{high}}{
        \def\exploitcolor{threatCritical}
        \def\exploitlabel{WEAPONIZED}
    }{
    \ifthenelse{\equal{#3}{functional}}{
        \def\exploitcolor{threatHigh}
        \def\exploitlabel{FUNCTIONAL}
    }{
        \def\exploitcolor{threatMedium}
        \def\exploitlabel{POC}
    }}

    \node[draw=\exploitcolor, fill=\exploitcolor!30, text=black,
          font=\tiny\bfseries, rounded corners=2pt, inner sep=2pt,
          anchor=east] at (#1.east) [xshift=-0.1cm] {
        EXPLOIT: \exploitlabel
    };
    \node[below, font=\tiny\ttfamily, fill=white, draw=\exploitcolor,
          inner sep=1pt] at (#1.east) [xshift=-0.5cm, yshift=-0.3cm] {
        #2
    };
}

% Patch availability status
% Usage: \markPatchStatus{node}{cve}{patch_status}{days_since_patch}
% patch_status: available, pending, none
\newcommand{\markPatchStatus}[4]{
    \ifthenelse{\equal{#3}{available}}{
        \def\patchcolor{green!60!black}
        \def\patchlabel{PATCH AVAILABLE}
    }{
    \ifthenelse{\equal{#3}{pending}}{
        \def\patchcolor{orange}
        \def\patchlabel{PATCH PENDING}
    }{
        \def\patchcolor{threatCritical}
        \def\patchlabel{NO PATCH}
    }}

    \node[fill=\patchcolor, text=white, font=\tiny\bfseries,
          inner sep=2pt, rounded corners=1pt, anchor=west]
        at (#1.west) [xshift=0.1cm] {
        \patchlabel\ (#4d)
    };
}

% Vulnerability age indicator
% Usage: \markVulnerabilityAge{node}{cve}{days_since_disclosure}
\newcommand{\markVulnerabilityAge}[3]{
    \pgfmathsetmacro{\agedays}{#3}
    \ifthenelse{\lengthtest{\agedays pt < 30 pt}}{
        \def\agecolor{threatCritical}
        \def\agelabel{NEW}
    }{
    \ifthenelse{\lengthtest{\agedays pt < 180 pt}}{
        \def\agecolor{threatHigh}
        \def\agelabel{RECENT}
    }{
    \ifthenelse{\lengthtest{\agedays pt < 365 pt}}{
        \def\agecolor{orange}
        \def\agelabel{AGING}
    }{
        \def\agecolor{gray!70}
        \def\agelabel{OLD}
    }}}

    \node[fill=\agecolor, text=white, font=\tiny,
          inner sep=1pt, rounded corners=1pt, anchor=south]
        at (#1.south) [yshift=-0.5cm] {
        \agelabel: #3 days
    };
}

% EPSS (Exploit Prediction Scoring System) indicator
% Usage: \markEPSS{node}{cve}{epss_score}{percentile}
\newcommand{\markEPSS}[4]{
    \pgfmathsetmacro{\epssscore}{#3 * 100}
    \ifthenelse{\lengthtest{\epssscore pt > 70 pt}}{
        \def\epsscolor{threatCritical}
    }{
    \ifthenelse{\lengthtest{\epssscore pt > 30 pt}}{
        \def\epsscolor{threatHigh}
    }{
        \def\epsscolor{threatMedium}
    }}

    \node[legend box, anchor=south west, minimum width=3cm,
          draw=\epsscolor, line width=1.5pt] at (#1.south west) [yshift=-1cm] {
        \begin{tabular}{ll}
            \multicolumn{2}{c}{\textbf{\textcolor{\epsscolor}{EPSS}}} \\
            \hline
            \textbf{CVE:} & \texttt{\tiny #2} \\
            \textbf{Score:} & #3 (#4th \%ile) \\
        \end{tabular}
    };
}

% Vulnerability priority score (combines CVSS, EPSS, exploit availability)
% Usage: \drawVulnPriority{x}{y}{cve}{cvss}{epss}{exploit}{priority_score}
\newcommand{\drawVulnPriority}[7]{
    \pgfmathsetmacro{\priority}{#7}
    \ifthenelse{\lengthtest{\priority pt > 90 pt}}{
        \def\prioritycolor{threatCritical}
        \def\prioritylabel{CRITICAL - PATCH NOW}
    }{
    \ifthenelse{\lengthtest{\priority pt > 70 pt}}{
        \def\prioritycolor{threatHigh}
        \def\prioritylabel{HIGH - PATCH URGENT}
    }{
    \ifthenelse{\lengthtest{\priority pt > 40 pt}}{
        \def\prioritycolor{threatMedium}
        \def\prioritylabel{MEDIUM - SCHEDULE}
    }{
        \def\prioritycolor{threatLow}
        \def\prioritylabel{LOW - MONITOR}
    }}}

    \node[legend box, anchor=north west, minimum width=5cm,
          draw=\prioritycolor, line width=2.5pt] at (#1,#2) {
        \begin{tabular}{ll}
            \multicolumn{2}{c}{\textbf{\textcolor{\prioritycolor}{#3}}} \\
            \hline
            CVSS Score & #4 \\
            EPSS Score & #5 \\
            Exploit Available & #6 \\
            \hline
            \textbf{Priority} & \textbf{#7/100} \\
            \multicolumn{2}{c}{\textcolor{\prioritycolor}{\prioritylabel}} \\
        \end{tabular}
    };
}

% Vulnerability scan results summary
% Usage: \drawScanResults{x}{y}{scanner_name}{scan_date}{vulns_found}{false_positives}
\newcommand{\drawScanResults}[6]{
    \node[legend box, anchor=north west, minimum width=5cm,
          draw=blue!60, line width=1.5pt] at (#1,#2) {
        \begin{tabular}{ll}
            \multicolumn{2}{c}{\textbf{Scan Results: #3}} \\
            \hline
            Scan Date & #4 \\
            Vulnerabilities Found & #5 \\
            False Positives & #6 \\
            \hline
            \textbf{Net Findings} & \textbf{\pgfmathparse{int(#5-#6)}\pgfmathresult} \\
        \end{tabular}
    };
}
% ============================================================================
% THREAT CORRELATION AND INCIDENT RECONSTRUCTION
% ============================================================================

% Correlate multiple IOCs to show relationship
% Usage: \drawIOCCorrelation{x}{y}{title}{ioc1}{ioc2}{ioc3}{correlation_confidence}
\newcommand{\drawIOCCorrelation}[7]{
    \pgfmathsetmacro{\correlation}{#7}
    \ifthenelse{\lengthtest{\correlation pt > 80 pt}}{
        \def\corrcolor{threatCritical}
        \def\corrlabel{HIGH CONFIDENCE}
    }{
    \ifthenelse{\lengthtest{\correlation pt > 50 pt}}{
        \def\corrcolor{threatHigh}
        \def\corrlabel{MEDIUM CONFIDENCE}
    }{
        \def\corrcolor{orange}
        \def\corrlabel{LOW CONFIDENCE}
    }}

    \node[legend box, anchor=north west, minimum width=6cm,
          draw=\corrcolor, line width=2pt] at (#1,#2) {
        \begin{tabular}{p{5.5cm}}
            \multicolumn{1}{c}{\textbf{\textcolor{\corrcolor}{#3}}} \\
            \hline
            \textbf{Correlated IOCs:} \\
            ● #4 \\
            ● #5 \\
            ● #6 \\
            \hline
            \textbf{Confidence:} \textcolor{\corrcolor}{#7\% - \corrlabel} \\
        \end{tabular}
    };
}

% Incident timeline reconstruction
% Usage: \drawIncidentTimeline{x}{y}{incident_name}{start}{end}{events}
\newcommand{\drawIncidentTimeline}[6]{
    \node[legend box, anchor=north west, minimum width=7cm,
          draw=threatCritical, line width=2pt] at (#1,#2) {
        \begin{tabular}{p{6.5cm}}
            \multicolumn{1}{c}{\textbf{\textcolor{threatCritical}{Incident: #3}}} \\
            \hline
            \textbf{Start:} #4 \\
            \textbf{End:} #5 \\
            \hline
            \textbf{Timeline:} \\
            #6 \\
        \end{tabular}
    };
}

% Lateral movement path visualization
% Usage: \drawLateralMovement{node1}{node2}{node3}{method}{timestamp}
\newcommand{\drawLateralMovement}[5]{
    \draw[draw=threatCritical, line width=2.5pt, dashed,
          -{Stealth[length=4mm]}] (#1) -- (#2)
        node[midway, above, font=\tiny, fill=yellow!30, inner sep=1pt] {
            #4 @ #5
        };
    \draw[draw=threatCritical, line width=2.5pt, dashed,
          -{Stealth[length=4mm]}] (#2) -- (#3)
        node[midway, above, font=\tiny, fill=yellow!30, inner sep=1pt] {
            #4 @ #5
        };
}

% Infection spread visualization
% Usage: \drawInfectionSpread{patient_zero}{infected_nodes_count}{containment_status}
\newcommand{\drawInfectionSpread}[3]{
    \node[circle, draw=threatCritical, line width=3pt, fill=threatCritical!20,
          minimum size=1.5cm] at (#1) {};
    \node[above, font=\small\bfseries, text=threatCritical, fill=white,
          inner sep=2pt] at (#1.north) [yshift=0.8cm] {
        PATIENT ZERO
    };
    \node[below, font=\tiny, fill=white, draw=threatCritical,
          inner sep=2pt] at (#1.south) [yshift=-0.5cm] {
        Infected: #2 | Status: #3
    };
}

% Command & Control (C2) beacon indicator
% Usage: \drawC2Beacon{compromised_node}{c2_server}{frequency}{protocol}
\newcommand{\drawC2Beacon}[4]{
    \draw[draw=threatCritical, line width=2pt, dotted,
          <->, shorten >=0.2cm, shorten <=0.2cm]
        (#1) -- (#2)
        node[midway, above, font=\tiny, fill=red!20, inner sep=2pt] {
            C2: #4 (#3)
        };
}

% Data exfiltration visualization with volume
% Usage: \drawExfiltrationPath{source}{intermediate}{destination}{data_volume}{detection}
\newcommand{\drawExfiltrationPath}[5]{
    \draw[draw=red!80!black, line width=3pt, -{Stealth[length=5mm]}]
        (#1) -- (#2)
        node[midway, below, font=\tiny, fill=red!20, inner sep=1pt] {
            #4
        };
    \draw[draw=red!80!black, line width=3pt, -{Stealth[length=5mm]}]
        (#2) -- (#3)
        node[midway, below, font=\tiny, fill=red!20, inner sep=1pt] {
            Detection: #5
        };
}

% Threat hunting query results
% Usage: \drawHuntingResults{x}{y}{query_name}{matches_found}{false_positives}{true_positives}
\newcommand{\drawHuntingResults}[6]{
    \pgfmathsetmacro{\precision}{(#6 / #4) * 100}
    \node[legend box, anchor=north west, minimum width=5.5cm,
          draw=purple!70, line width=2pt] at (#1,#2) {
        \begin{tabular}{ll}
            \multicolumn{2}{c}{\textbf{Hunt: #3}} \\
            \hline
            Matches Found & #4 \\
            False Positives & #5 \\
            True Positives & #6 \\
            \hline
            \textbf{Precision} & \textbf{\pgfmathprintnumber[precision=1]{\precision}\%} \\
        \end{tabular}
    };
}

% Incident severity meter
% Usage: \drawIncidentSeverity{x}{y}{incident_name}{severity_score}{impact}{urgency}
\newcommand{\drawIncidentSeverity}[6]{
    \pgfmathsetmacro{\severity}{#4}
    \ifthenelse{\lengthtest{\severity pt > 80 pt}}{
        \def\sevcolor{threatCritical}
        \def\sevlabel{CRITICAL INCIDENT}
    }{
    \ifthenelse{\lengthtest{\severity pt > 60 pt}}{
        \def\sevcolor{threatHigh}
        \def\sevlabel{HIGH SEVERITY}
    }{
    \ifthenelse{\lengthtest{\severity pt > 40 pt}}{
        \def\sevcolor{threatMedium}
        \def\sevlabel{MEDIUM SEVERITY}
    }{
        \def\sevcolor{threatLow}
        \def\sevlabel{LOW SEVERITY}
    }}}

    \node[legend box, anchor=north west, minimum width=5cm,
          draw=\sevcolor, line width=2.5pt] at (#1,#2) {
        \begin{tabular}{ll}
            \multicolumn{2}{c}{\textbf{\textcolor{\sevcolor}{#3}}} \\
            \hline
            Severity Score & #4/100 \\
            Impact & #5 \\
            Urgency & #6 \\
            \hline
            \multicolumn{2}{c}{\textcolor{\sevcolor}{\sevlabel}} \\
        \end{tabular}
    };
}

% ============================================================================
% SECURITY METRICS AND KPIs
% ============================================================================

% Mean Time metrics (MTTD, MTTR, MTTC)
% Usage: \drawMTTMetrics{x}{y}{mttd_hours}{mttr_hours}{mttc_hours}
\newcommand{\drawMTTMetrics}[5]{
    \node[legend box, anchor=north west, minimum width=5cm,
          draw=blue!70, line width=2pt] at (#1,#2) {
        \begin{tabular}{lc}
            \multicolumn{2}{c}{\textbf{Response Metrics}} \\
            \hline
            MTTD (Detect) & #3h \\
            MTTR (Respond) & #4h \\
            MTTC (Contain) & #5h \\
        \end{tabular}
    };
}

% Security coverage heatmap
% Usage: \drawCoverageHeatmap{x}{y}{endpoint}{network}{cloud}{identity}{data}
\newcommand{\drawCoverageHeatmap}[7]{
    \node[legend box, anchor=north west, minimum width=6cm,
          draw=green!60!black, line width=2pt] at (#1,#2) {
        \begin{tabular}{lc}
            \multicolumn{2}{c}{\textbf{Security Coverage}} \\
            \hline
            Endpoint Security & \drawCoverageBar{#3} \\
            Network Security & \drawCoverageBar{#4} \\
            Cloud Security & \drawCoverageBar{#5} \\
            Identity \& Access & \drawCoverageBar{#6} \\
            Data Protection & \drawCoverageBar{#7} \\
        \end{tabular}
    };
}

% Helper for coverage visualization
\newcommand{\drawCoverageBar}[1]{
    \pgfmathsetmacro{\cov}{#1}
    \ifthenelse{\lengthtest{\cov pt > 90 pt}}{
        \textcolor{green!60!black}{#1\%}
    }{
    \ifthenelse{\lengthtest{\cov pt > 70 pt}}{
        \textcolor{orange}{#1\%}
    }{
        \textcolor{threatHigh}{#1\%}
    }}
}

% Risk score trending
% Usage: \drawRiskTrend{x}{y}{current_score}{last_month}{trend}
% trend: increasing, decreasing, stable
\newcommand{\drawRiskTrend}[5]{
    \ifthenelse{\equal{#5}{increasing}}{
        \def\trendcolor{threatCritical}
        \def\trendlabel{↑ INCREASING}
    }{
    \ifthenelse{\equal{#5}{decreasing}}{
        \def\trendcolor{green!60!black}
        \def\trendlabel{↓ DECREASING}
    }{
        \def\trendcolor{orange}
        \def\trendlabel{→ STABLE}
    }}

    \node[legend box, anchor=north west, minimum width=4.5cm,
          draw=\trendcolor, line width=2pt] at (#1,#2) {
        \begin{tabular}{ll}
            \multicolumn{2}{c}{\textbf{Risk Score Trend}} \\
            \hline
            Current & #3/100 \\
            Last Month & #4/100 \\
            \hline
            \textbf{Trend} & \textcolor{\trendcolor}{\trendlabel} \\
        \end{tabular}
    };
}

% ============================================================================
% MAIN THREAT RENDERING ENGINE
% ============================================================================

\newcommand{\renderThreats}{
    % This will be populated by network_data.tex
    % Example structure:
    % \visualizeDDoS{attacker1,attacker2}{srv1}{critical}
    % \markVulnerability{srv2}{CVE-2024-1234}{9.8}
}

% TODO: Intelligent threat rendering
% - Threat prioritization based on risk
% - Auto-layout threat indicators to avoid overlap
% - Threat correlation and grouping
% - Attack path visualization
% - Threat timeline with incident markers
% - Real-time threat feed updates
